\documentclass[12pt,a4paper]{article}
\usepackage[utf8]{inputenc}
\usepackage{a4wide}
\usepackage{fancyhdr}
\usepackage{graphicx}
\usepackage{epsfig}
\usepackage{parskip}

\usepackage{amsmath}
\usepackage{amssymb}
\usepackage{bm}
\usepackage{tabularx}
\usepackage{booktabs}
\usepackage{array}
\usepackage{pdfpages}
\newcolumntype{M}[1]{>{\centering\arraybackslash}m{#1}}
\usepackage[free-standing-units=true]{siunitx} % for consistent handling of SI units
\usepackage[colorlinks=true, pdfstartview=FitV, linkcolor=blue, citecolor=blue, urlcolor=blue]{hyperref} % enable links

\setlength{\parindent}{0pt}

\newcommand{\m}[1]
{\mathrm{#1}}

\title{Experiment 16, Geometrische Optik}
\author{Cedric Renda, Fritz Kurz}
\date{\today }

\begin{document}

\title{Zusammenfassung Elektronik Basic}

\maketitle
\begin{itemize}
	\item Für die Übung mit dem Multimeter schätzen wir die Spannung über dem Spannungsteiler ab 
	$U_{R2}= U_0 \frac{R2}{R2+R1}$.
	Jedoch muss für die korrekte abschätzung der Innenwiderstand $R_{p}$ parallel zu R2 wie folgt gerechnet werden
	$R_i = \frac{R2R_p}{R2 + R_p}$
	$U_{R_i}= U_0 \frac{R_i}{R_i+R1}$.
	
	Daraus folgt der Innenwiderstand aus $\displaystyle R_i = U_0 \frac{-R1R2}{R1U_0+ R2(U_0-U_{R_i})}$ dieser muss sehr hochohmig sein. Danach messen wir den gleichen Spannungsteiler in einer Brückenschaltung damit werden die Innenwiderstände der Messgeräte weniger ins gewicht fallen.
	
\item	Danach wird auf ähdnliche weise der Innenwiederstand des Amperemeters gemessen. Dieser wird sehr nieder Ohmig sein. 
	
	
	
\item	Mit der Vierpunktmessung können sehr kleine Widerstände gemessen werden.
	Den Unterschied der 4 und Zweipunkt messung Zeigen da der strom einen längeren weg durch ein Kabel fliesst als bei anderer Methode.
	
	\item	Bei dem K.O werden funktionen ausprobiert und Dioden vermessen und verschiedene signlae illustriert
	
	
\end{itemize}





\end{document}