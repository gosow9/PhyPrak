\section{Results}


To verify how good Steiner's theorem is, we compare the measured moments of inertia with the values calculated using Steiner's theorem.
In figure \ref{fig::stein} the moments of inertia with different displacements are plotted against the displacement $d$ squared.
The measured values are laying on the linear line displaying the values calculated using the Steiner's theorem as seen in figure \ref{fig::stein}. 

\begin{figure}[ht]
	\begin{center}
		%% Creator: Matplotlib, PGF backend
%%
%% To include the figure in your LaTeX document, write
%%   \input{<filename>.pgf}
%%
%% Make sure the required packages are loaded in your preamble
%%   \usepackage{pgf}
%%
%% and, on pdftex
%%   \usepackage[utf8]{inputenc}\DeclareUnicodeCharacter{2212}{-}
%%
%% or, on luatex and xetex
%%   \usepackage{unicode-math}
%%
%% Figures using additional raster images can only be included by \input if
%% they are in the same directory as the main LaTeX file. For loading figures
%% from other directories you can use the `import` package
%%   \usepackage{import}
%%
%% and then include the figures with
%%   \import{<path to file>}{<filename>.pgf}
%%
%% Matplotlib used the following preamble
%%
\begingroup%
\makeatletter%
\begin{pgfpicture}%
\pgfpathrectangle{\pgfpointorigin}{\pgfqpoint{6.400000in}{4.000000in}}%
\pgfusepath{use as bounding box, clip}%
\begin{pgfscope}%
\pgfsetbuttcap%
\pgfsetmiterjoin%
\definecolor{currentfill}{rgb}{1.000000,1.000000,1.000000}%
\pgfsetfillcolor{currentfill}%
\pgfsetlinewidth{0.000000pt}%
\definecolor{currentstroke}{rgb}{1.000000,1.000000,1.000000}%
\pgfsetstrokecolor{currentstroke}%
\pgfsetdash{}{0pt}%
\pgfpathmoveto{\pgfqpoint{0.000000in}{0.000000in}}%
\pgfpathlineto{\pgfqpoint{6.400000in}{0.000000in}}%
\pgfpathlineto{\pgfqpoint{6.400000in}{4.000000in}}%
\pgfpathlineto{\pgfqpoint{0.000000in}{4.000000in}}%
\pgfpathclose%
\pgfusepath{fill}%
\end{pgfscope}%
\begin{pgfscope}%
\pgfsetbuttcap%
\pgfsetmiterjoin%
\definecolor{currentfill}{rgb}{1.000000,1.000000,1.000000}%
\pgfsetfillcolor{currentfill}%
\pgfsetlinewidth{0.000000pt}%
\definecolor{currentstroke}{rgb}{0.000000,0.000000,0.000000}%
\pgfsetstrokecolor{currentstroke}%
\pgfsetstrokeopacity{0.000000}%
\pgfsetdash{}{0pt}%
\pgfpathmoveto{\pgfqpoint{0.800000in}{0.440000in}}%
\pgfpathlineto{\pgfqpoint{5.760000in}{0.440000in}}%
\pgfpathlineto{\pgfqpoint{5.760000in}{3.520000in}}%
\pgfpathlineto{\pgfqpoint{0.800000in}{3.520000in}}%
\pgfpathclose%
\pgfusepath{fill}%
\end{pgfscope}%
\begin{pgfscope}%
\pgfpathrectangle{\pgfqpoint{0.800000in}{0.440000in}}{\pgfqpoint{4.960000in}{3.080000in}}%
\pgfusepath{clip}%
\pgfsetbuttcap%
\pgfsetroundjoin%
\definecolor{currentfill}{rgb}{1.000000,0.000000,0.000000}%
\pgfsetfillcolor{currentfill}%
\pgfsetfillopacity{0.100000}%
\pgfsetlinewidth{1.003750pt}%
\definecolor{currentstroke}{rgb}{1.000000,0.000000,0.000000}%
\pgfsetstrokecolor{currentstroke}%
\pgfsetstrokeopacity{0.100000}%
\pgfsetdash{}{0pt}%
\pgfsys@defobject{currentmarker}{\pgfqpoint{1.025455in}{0.580000in}}{\pgfqpoint{5.534545in}{1.067290in}}{%
\pgfpathmoveto{\pgfqpoint{1.025455in}{0.599922in}}%
\pgfpathlineto{\pgfqpoint{1.025455in}{0.580000in}}%
\pgfpathlineto{\pgfqpoint{1.025915in}{0.643439in}}%
\pgfpathlineto{\pgfqpoint{1.027295in}{0.699539in}}%
\pgfpathlineto{\pgfqpoint{1.029595in}{0.749081in}}%
\pgfpathlineto{\pgfqpoint{1.032816in}{0.792769in}}%
\pgfpathlineto{\pgfqpoint{1.036956in}{0.831233in}}%
\pgfpathlineto{\pgfqpoint{1.042017in}{0.865041in}}%
\pgfpathlineto{\pgfqpoint{1.047998in}{0.894699in}}%
\pgfpathlineto{\pgfqpoint{1.054899in}{0.920659in}}%
\pgfpathlineto{\pgfqpoint{1.062720in}{0.943324in}}%
\pgfpathlineto{\pgfqpoint{1.071461in}{0.963051in}}%
\pgfpathlineto{\pgfqpoint{1.081122in}{0.980157in}}%
\pgfpathlineto{\pgfqpoint{1.091704in}{0.994922in}}%
\pgfpathlineto{\pgfqpoint{1.103205in}{1.007596in}}%
\pgfpathlineto{\pgfqpoint{1.115627in}{1.018396in}}%
\pgfpathlineto{\pgfqpoint{1.128969in}{1.027516in}}%
\pgfpathlineto{\pgfqpoint{1.143231in}{1.035129in}}%
\pgfpathlineto{\pgfqpoint{1.158413in}{1.041384in}}%
\pgfpathlineto{\pgfqpoint{1.174515in}{1.046417in}}%
\pgfpathlineto{\pgfqpoint{1.191538in}{1.050345in}}%
\pgfpathlineto{\pgfqpoint{1.209480in}{1.053274in}}%
\pgfpathlineto{\pgfqpoint{1.228343in}{1.055300in}}%
\pgfpathlineto{\pgfqpoint{1.248126in}{1.056507in}}%
\pgfpathlineto{\pgfqpoint{1.268829in}{1.056972in}}%
\pgfpathlineto{\pgfqpoint{1.290452in}{1.056765in}}%
\pgfpathlineto{\pgfqpoint{1.312995in}{1.055949in}}%
\pgfpathlineto{\pgfqpoint{1.336458in}{1.054583in}}%
\pgfpathlineto{\pgfqpoint{1.360841in}{1.052721in}}%
\pgfpathlineto{\pgfqpoint{1.386145in}{1.050415in}}%
\pgfpathlineto{\pgfqpoint{1.412369in}{1.047711in}}%
\pgfpathlineto{\pgfqpoint{1.439512in}{1.044653in}}%
\pgfpathlineto{\pgfqpoint{1.467576in}{1.041285in}}%
\pgfpathlineto{\pgfqpoint{1.496560in}{1.037645in}}%
\pgfpathlineto{\pgfqpoint{1.526465in}{1.033771in}}%
\pgfpathlineto{\pgfqpoint{1.557289in}{1.029700in}}%
\pgfpathlineto{\pgfqpoint{1.589033in}{1.025463in}}%
\pgfpathlineto{\pgfqpoint{1.621698in}{1.021094in}}%
\pgfpathlineto{\pgfqpoint{1.655283in}{1.016620in}}%
\pgfpathlineto{\pgfqpoint{1.689787in}{1.012071in}}%
\pgfpathlineto{\pgfqpoint{1.725212in}{1.007471in}}%
\pgfpathlineto{\pgfqpoint{1.761558in}{1.002843in}}%
\pgfpathlineto{\pgfqpoint{1.798823in}{0.998208in}}%
\pgfpathlineto{\pgfqpoint{1.837008in}{0.993584in}}%
\pgfpathlineto{\pgfqpoint{1.876114in}{0.988989in}}%
\pgfpathlineto{\pgfqpoint{1.916139in}{0.984435in}}%
\pgfpathlineto{\pgfqpoint{1.957085in}{0.979933in}}%
\pgfpathlineto{\pgfqpoint{1.998951in}{0.975493in}}%
\pgfpathlineto{\pgfqpoint{2.041737in}{0.971121in}}%
\pgfpathlineto{\pgfqpoint{2.085443in}{0.966819in}}%
\pgfpathlineto{\pgfqpoint{2.130069in}{0.962589in}}%
\pgfpathlineto{\pgfqpoint{2.175615in}{0.958430in}}%
\pgfpathlineto{\pgfqpoint{2.222082in}{0.954338in}}%
\pgfpathlineto{\pgfqpoint{2.269469in}{0.950306in}}%
\pgfpathlineto{\pgfqpoint{2.317775in}{0.946327in}}%
\pgfpathlineto{\pgfqpoint{2.367002in}{0.942389in}}%
\pgfpathlineto{\pgfqpoint{2.417149in}{0.938481in}}%
\pgfpathlineto{\pgfqpoint{2.468216in}{0.934587in}}%
\pgfpathlineto{\pgfqpoint{2.520204in}{0.930692in}}%
\pgfpathlineto{\pgfqpoint{2.573111in}{0.926779in}}%
\pgfpathlineto{\pgfqpoint{2.626939in}{0.922827in}}%
\pgfpathlineto{\pgfqpoint{2.681686in}{0.918817in}}%
\pgfpathlineto{\pgfqpoint{2.737354in}{0.914727in}}%
\pgfpathlineto{\pgfqpoint{2.793942in}{0.910536in}}%
\pgfpathlineto{\pgfqpoint{2.851450in}{0.906220in}}%
\pgfpathlineto{\pgfqpoint{2.909878in}{0.901756in}}%
\pgfpathlineto{\pgfqpoint{2.969227in}{0.897122in}}%
\pgfpathlineto{\pgfqpoint{3.029495in}{0.892294in}}%
\pgfpathlineto{\pgfqpoint{3.090684in}{0.887248in}}%
\pgfpathlineto{\pgfqpoint{3.152792in}{0.881963in}}%
\pgfpathlineto{\pgfqpoint{3.215821in}{0.876417in}}%
\pgfpathlineto{\pgfqpoint{3.279770in}{0.870590in}}%
\pgfpathlineto{\pgfqpoint{3.344639in}{0.864463in}}%
\pgfpathlineto{\pgfqpoint{3.410428in}{0.858019in}}%
\pgfpathlineto{\pgfqpoint{3.477138in}{0.851244in}}%
\pgfpathlineto{\pgfqpoint{3.544767in}{0.844125in}}%
\pgfpathlineto{\pgfqpoint{3.613317in}{0.836654in}}%
\pgfpathlineto{\pgfqpoint{3.682786in}{0.828826in}}%
\pgfpathlineto{\pgfqpoint{3.753176in}{0.820640in}}%
\pgfpathlineto{\pgfqpoint{3.824486in}{0.812099in}}%
\pgfpathlineto{\pgfqpoint{3.896716in}{0.803214in}}%
\pgfpathlineto{\pgfqpoint{3.969867in}{0.794000in}}%
\pgfpathlineto{\pgfqpoint{4.043937in}{0.784480in}}%
\pgfpathlineto{\pgfqpoint{4.118927in}{0.774684in}}%
\pgfpathlineto{\pgfqpoint{4.194838in}{0.764652in}}%
\pgfpathlineto{\pgfqpoint{4.271669in}{0.754431in}}%
\pgfpathlineto{\pgfqpoint{4.349420in}{0.744082in}}%
\pgfpathlineto{\pgfqpoint{4.428091in}{0.733675in}}%
\pgfpathlineto{\pgfqpoint{4.507682in}{0.723292in}}%
\pgfpathlineto{\pgfqpoint{4.588193in}{0.713031in}}%
\pgfpathlineto{\pgfqpoint{4.669624in}{0.703001in}}%
\pgfpathlineto{\pgfqpoint{4.751976in}{0.693328in}}%
\pgfpathlineto{\pgfqpoint{4.835248in}{0.684152in}}%
\pgfpathlineto{\pgfqpoint{4.919439in}{0.675632in}}%
\pgfpathlineto{\pgfqpoint{5.004551in}{0.667939in}}%
\pgfpathlineto{\pgfqpoint{5.090583in}{0.661266in}}%
\pgfpathlineto{\pgfqpoint{5.177536in}{0.655817in}}%
\pgfpathlineto{\pgfqpoint{5.265408in}{0.651814in}}%
\pgfpathlineto{\pgfqpoint{5.354200in}{0.649494in}}%
\pgfpathlineto{\pgfqpoint{5.443913in}{0.649104in}}%
\pgfpathlineto{\pgfqpoint{5.534545in}{0.650903in}}%
\pgfpathlineto{\pgfqpoint{5.534545in}{0.669574in}}%
\pgfpathlineto{\pgfqpoint{5.534545in}{0.669574in}}%
\pgfpathlineto{\pgfqpoint{5.443913in}{0.667808in}}%
\pgfpathlineto{\pgfqpoint{5.354200in}{0.668191in}}%
\pgfpathlineto{\pgfqpoint{5.265408in}{0.670469in}}%
\pgfpathlineto{\pgfqpoint{5.177536in}{0.674400in}}%
\pgfpathlineto{\pgfqpoint{5.090583in}{0.679751in}}%
\pgfpathlineto{\pgfqpoint{5.004551in}{0.686305in}}%
\pgfpathlineto{\pgfqpoint{4.919439in}{0.693858in}}%
\pgfpathlineto{\pgfqpoint{4.835248in}{0.702224in}}%
\pgfpathlineto{\pgfqpoint{4.751976in}{0.711232in}}%
\pgfpathlineto{\pgfqpoint{4.669624in}{0.720729in}}%
\pgfpathlineto{\pgfqpoint{4.588193in}{0.730574in}}%
\pgfpathlineto{\pgfqpoint{4.507682in}{0.740646in}}%
\pgfpathlineto{\pgfqpoint{4.428091in}{0.750836in}}%
\pgfpathlineto{\pgfqpoint{4.349420in}{0.761049in}}%
\pgfpathlineto{\pgfqpoint{4.271669in}{0.771204in}}%
\pgfpathlineto{\pgfqpoint{4.194838in}{0.781231in}}%
\pgfpathlineto{\pgfqpoint{4.118927in}{0.791073in}}%
\pgfpathlineto{\pgfqpoint{4.043937in}{0.800682in}}%
\pgfpathlineto{\pgfqpoint{3.969867in}{0.810019in}}%
\pgfpathlineto{\pgfqpoint{3.896716in}{0.819055in}}%
\pgfpathlineto{\pgfqpoint{3.824486in}{0.827767in}}%
\pgfpathlineto{\pgfqpoint{3.753176in}{0.836141in}}%
\pgfpathlineto{\pgfqpoint{3.682786in}{0.844166in}}%
\pgfpathlineto{\pgfqpoint{3.613317in}{0.851839in}}%
\pgfpathlineto{\pgfqpoint{3.544767in}{0.859161in}}%
\pgfpathlineto{\pgfqpoint{3.477138in}{0.866138in}}%
\pgfpathlineto{\pgfqpoint{3.410428in}{0.872777in}}%
\pgfpathlineto{\pgfqpoint{3.344639in}{0.879090in}}%
\pgfpathlineto{\pgfqpoint{3.279770in}{0.885093in}}%
\pgfpathlineto{\pgfqpoint{3.215821in}{0.890801in}}%
\pgfpathlineto{\pgfqpoint{3.152792in}{0.896233in}}%
\pgfpathlineto{\pgfqpoint{3.090684in}{0.901409in}}%
\pgfpathlineto{\pgfqpoint{3.029495in}{0.906351in}}%
\pgfpathlineto{\pgfqpoint{2.969227in}{0.911079in}}%
\pgfpathlineto{\pgfqpoint{2.909878in}{0.915616in}}%
\pgfpathlineto{\pgfqpoint{2.851450in}{0.919986in}}%
\pgfpathlineto{\pgfqpoint{2.793942in}{0.924212in}}%
\pgfpathlineto{\pgfqpoint{2.737354in}{0.928314in}}%
\pgfpathlineto{\pgfqpoint{2.681686in}{0.932317in}}%
\pgfpathlineto{\pgfqpoint{2.626939in}{0.936242in}}%
\pgfpathlineto{\pgfqpoint{2.573111in}{0.940110in}}%
\pgfpathlineto{\pgfqpoint{2.520204in}{0.943939in}}%
\pgfpathlineto{\pgfqpoint{2.468216in}{0.947750in}}%
\pgfpathlineto{\pgfqpoint{2.417149in}{0.951560in}}%
\pgfpathlineto{\pgfqpoint{2.367002in}{0.955383in}}%
\pgfpathlineto{\pgfqpoint{2.317775in}{0.959235in}}%
\pgfpathlineto{\pgfqpoint{2.269469in}{0.963128in}}%
\pgfpathlineto{\pgfqpoint{2.222082in}{0.967071in}}%
\pgfpathlineto{\pgfqpoint{2.175615in}{0.971073in}}%
\pgfpathlineto{\pgfqpoint{2.130069in}{0.975139in}}%
\pgfpathlineto{\pgfqpoint{2.085443in}{0.979275in}}%
\pgfpathlineto{\pgfqpoint{2.041737in}{0.983480in}}%
\pgfpathlineto{\pgfqpoint{1.998951in}{0.987755in}}%
\pgfpathlineto{\pgfqpoint{1.957085in}{0.992094in}}%
\pgfpathlineto{\pgfqpoint{1.916139in}{0.996493in}}%
\pgfpathlineto{\pgfqpoint{1.876114in}{1.000943in}}%
\pgfpathlineto{\pgfqpoint{1.837008in}{1.005433in}}%
\pgfpathlineto{\pgfqpoint{1.798823in}{1.009949in}}%
\pgfpathlineto{\pgfqpoint{1.761558in}{1.014476in}}%
\pgfpathlineto{\pgfqpoint{1.725212in}{1.018996in}}%
\pgfpathlineto{\pgfqpoint{1.689787in}{1.023488in}}%
\pgfpathlineto{\pgfqpoint{1.655283in}{1.027929in}}%
\pgfpathlineto{\pgfqpoint{1.621698in}{1.032296in}}%
\pgfpathlineto{\pgfqpoint{1.589033in}{1.036560in}}%
\pgfpathlineto{\pgfqpoint{1.557289in}{1.040694in}}%
\pgfpathlineto{\pgfqpoint{1.526465in}{1.044667in}}%
\pgfpathlineto{\pgfqpoint{1.496560in}{1.048446in}}%
\pgfpathlineto{\pgfqpoint{1.467576in}{1.051996in}}%
\pgfpathlineto{\pgfqpoint{1.439512in}{1.055280in}}%
\pgfpathlineto{\pgfqpoint{1.412369in}{1.058262in}}%
\pgfpathlineto{\pgfqpoint{1.386145in}{1.060898in}}%
\pgfpathlineto{\pgfqpoint{1.360841in}{1.063147in}}%
\pgfpathlineto{\pgfqpoint{1.336458in}{1.064961in}}%
\pgfpathlineto{\pgfqpoint{1.312995in}{1.066293in}}%
\pgfpathlineto{\pgfqpoint{1.290452in}{1.067088in}}%
\pgfpathlineto{\pgfqpoint{1.268829in}{1.067290in}}%
\pgfpathlineto{\pgfqpoint{1.248126in}{1.066837in}}%
\pgfpathlineto{\pgfqpoint{1.228343in}{1.065660in}}%
\pgfpathlineto{\pgfqpoint{1.209480in}{1.063686in}}%
\pgfpathlineto{\pgfqpoint{1.191538in}{1.060830in}}%
\pgfpathlineto{\pgfqpoint{1.174515in}{1.057000in}}%
\pgfpathlineto{\pgfqpoint{1.158413in}{1.052093in}}%
\pgfpathlineto{\pgfqpoint{1.143231in}{1.045991in}}%
\pgfpathlineto{\pgfqpoint{1.128969in}{1.038564in}}%
\pgfpathlineto{\pgfqpoint{1.115627in}{1.029662in}}%
\pgfpathlineto{\pgfqpoint{1.103205in}{1.019118in}}%
\pgfpathlineto{\pgfqpoint{1.091704in}{1.006740in}}%
\pgfpathlineto{\pgfqpoint{1.081122in}{0.992313in}}%
\pgfpathlineto{\pgfqpoint{1.071461in}{0.975591in}}%
\pgfpathlineto{\pgfqpoint{1.062720in}{0.956298in}}%
\pgfpathlineto{\pgfqpoint{1.054899in}{0.934121in}}%
\pgfpathlineto{\pgfqpoint{1.047998in}{0.908706in}}%
\pgfpathlineto{\pgfqpoint{1.042017in}{0.879656in}}%
\pgfpathlineto{\pgfqpoint{1.036956in}{0.846525in}}%
\pgfpathlineto{\pgfqpoint{1.032816in}{0.808811in}}%
\pgfpathlineto{\pgfqpoint{1.029595in}{0.765955in}}%
\pgfpathlineto{\pgfqpoint{1.027295in}{0.717331in}}%
\pgfpathlineto{\pgfqpoint{1.025915in}{0.662244in}}%
\pgfpathlineto{\pgfqpoint{1.025455in}{0.599922in}}%
\pgfpathclose%
\pgfusepath{stroke,fill}%
}%
\begin{pgfscope}%
\pgfsys@transformshift{0.000000in}{0.000000in}%
\pgfsys@useobject{currentmarker}{}%
\end{pgfscope}%
\end{pgfscope}%
\begin{pgfscope}%
\pgfsetbuttcap%
\pgfsetroundjoin%
\definecolor{currentfill}{rgb}{0.000000,0.000000,0.000000}%
\pgfsetfillcolor{currentfill}%
\pgfsetlinewidth{0.803000pt}%
\definecolor{currentstroke}{rgb}{0.000000,0.000000,0.000000}%
\pgfsetstrokecolor{currentstroke}%
\pgfsetdash{}{0pt}%
\pgfsys@defobject{currentmarker}{\pgfqpoint{0.000000in}{-0.048611in}}{\pgfqpoint{0.000000in}{0.000000in}}{%
\pgfpathmoveto{\pgfqpoint{0.000000in}{0.000000in}}%
\pgfpathlineto{\pgfqpoint{0.000000in}{-0.048611in}}%
\pgfusepath{stroke,fill}%
}%
\begin{pgfscope}%
\pgfsys@transformshift{1.025455in}{0.440000in}%
\pgfsys@useobject{currentmarker}{}%
\end{pgfscope}%
\end{pgfscope}%
\begin{pgfscope}%
\definecolor{textcolor}{rgb}{0.000000,0.000000,0.000000}%
\pgfsetstrokecolor{textcolor}%
\pgfsetfillcolor{textcolor}%
\pgftext[x=1.025455in,y=0.342778in,,top]{\color{textcolor}\rmfamily\fontsize{10.000000}{12.000000}\selectfont \(\displaystyle {0.00}\)}%
\end{pgfscope}%
\begin{pgfscope}%
\pgfsetbuttcap%
\pgfsetroundjoin%
\definecolor{currentfill}{rgb}{0.000000,0.000000,0.000000}%
\pgfsetfillcolor{currentfill}%
\pgfsetlinewidth{0.803000pt}%
\definecolor{currentstroke}{rgb}{0.000000,0.000000,0.000000}%
\pgfsetstrokecolor{currentstroke}%
\pgfsetdash{}{0pt}%
\pgfsys@defobject{currentmarker}{\pgfqpoint{0.000000in}{-0.048611in}}{\pgfqpoint{0.000000in}{0.000000in}}{%
\pgfpathmoveto{\pgfqpoint{0.000000in}{0.000000in}}%
\pgfpathlineto{\pgfqpoint{0.000000in}{-0.048611in}}%
\pgfusepath{stroke,fill}%
}%
\begin{pgfscope}%
\pgfsys@transformshift{1.746909in}{0.440000in}%
\pgfsys@useobject{currentmarker}{}%
\end{pgfscope}%
\end{pgfscope}%
\begin{pgfscope}%
\definecolor{textcolor}{rgb}{0.000000,0.000000,0.000000}%
\pgfsetstrokecolor{textcolor}%
\pgfsetfillcolor{textcolor}%
\pgftext[x=1.746909in,y=0.342778in,,top]{\color{textcolor}\rmfamily\fontsize{10.000000}{12.000000}\selectfont \(\displaystyle {0.01}\)}%
\end{pgfscope}%
\begin{pgfscope}%
\pgfsetbuttcap%
\pgfsetroundjoin%
\definecolor{currentfill}{rgb}{0.000000,0.000000,0.000000}%
\pgfsetfillcolor{currentfill}%
\pgfsetlinewidth{0.803000pt}%
\definecolor{currentstroke}{rgb}{0.000000,0.000000,0.000000}%
\pgfsetstrokecolor{currentstroke}%
\pgfsetdash{}{0pt}%
\pgfsys@defobject{currentmarker}{\pgfqpoint{0.000000in}{-0.048611in}}{\pgfqpoint{0.000000in}{0.000000in}}{%
\pgfpathmoveto{\pgfqpoint{0.000000in}{0.000000in}}%
\pgfpathlineto{\pgfqpoint{0.000000in}{-0.048611in}}%
\pgfusepath{stroke,fill}%
}%
\begin{pgfscope}%
\pgfsys@transformshift{2.468364in}{0.440000in}%
\pgfsys@useobject{currentmarker}{}%
\end{pgfscope}%
\end{pgfscope}%
\begin{pgfscope}%
\definecolor{textcolor}{rgb}{0.000000,0.000000,0.000000}%
\pgfsetstrokecolor{textcolor}%
\pgfsetfillcolor{textcolor}%
\pgftext[x=2.468364in,y=0.342778in,,top]{\color{textcolor}\rmfamily\fontsize{10.000000}{12.000000}\selectfont \(\displaystyle {0.02}\)}%
\end{pgfscope}%
\begin{pgfscope}%
\pgfsetbuttcap%
\pgfsetroundjoin%
\definecolor{currentfill}{rgb}{0.000000,0.000000,0.000000}%
\pgfsetfillcolor{currentfill}%
\pgfsetlinewidth{0.803000pt}%
\definecolor{currentstroke}{rgb}{0.000000,0.000000,0.000000}%
\pgfsetstrokecolor{currentstroke}%
\pgfsetdash{}{0pt}%
\pgfsys@defobject{currentmarker}{\pgfqpoint{0.000000in}{-0.048611in}}{\pgfqpoint{0.000000in}{0.000000in}}{%
\pgfpathmoveto{\pgfqpoint{0.000000in}{0.000000in}}%
\pgfpathlineto{\pgfqpoint{0.000000in}{-0.048611in}}%
\pgfusepath{stroke,fill}%
}%
\begin{pgfscope}%
\pgfsys@transformshift{3.189818in}{0.440000in}%
\pgfsys@useobject{currentmarker}{}%
\end{pgfscope}%
\end{pgfscope}%
\begin{pgfscope}%
\definecolor{textcolor}{rgb}{0.000000,0.000000,0.000000}%
\pgfsetstrokecolor{textcolor}%
\pgfsetfillcolor{textcolor}%
\pgftext[x=3.189818in,y=0.342778in,,top]{\color{textcolor}\rmfamily\fontsize{10.000000}{12.000000}\selectfont \(\displaystyle {0.03}\)}%
\end{pgfscope}%
\begin{pgfscope}%
\pgfsetbuttcap%
\pgfsetroundjoin%
\definecolor{currentfill}{rgb}{0.000000,0.000000,0.000000}%
\pgfsetfillcolor{currentfill}%
\pgfsetlinewidth{0.803000pt}%
\definecolor{currentstroke}{rgb}{0.000000,0.000000,0.000000}%
\pgfsetstrokecolor{currentstroke}%
\pgfsetdash{}{0pt}%
\pgfsys@defobject{currentmarker}{\pgfqpoint{0.000000in}{-0.048611in}}{\pgfqpoint{0.000000in}{0.000000in}}{%
\pgfpathmoveto{\pgfqpoint{0.000000in}{0.000000in}}%
\pgfpathlineto{\pgfqpoint{0.000000in}{-0.048611in}}%
\pgfusepath{stroke,fill}%
}%
\begin{pgfscope}%
\pgfsys@transformshift{3.911273in}{0.440000in}%
\pgfsys@useobject{currentmarker}{}%
\end{pgfscope}%
\end{pgfscope}%
\begin{pgfscope}%
\definecolor{textcolor}{rgb}{0.000000,0.000000,0.000000}%
\pgfsetstrokecolor{textcolor}%
\pgfsetfillcolor{textcolor}%
\pgftext[x=3.911273in,y=0.342778in,,top]{\color{textcolor}\rmfamily\fontsize{10.000000}{12.000000}\selectfont \(\displaystyle {0.04}\)}%
\end{pgfscope}%
\begin{pgfscope}%
\pgfsetbuttcap%
\pgfsetroundjoin%
\definecolor{currentfill}{rgb}{0.000000,0.000000,0.000000}%
\pgfsetfillcolor{currentfill}%
\pgfsetlinewidth{0.803000pt}%
\definecolor{currentstroke}{rgb}{0.000000,0.000000,0.000000}%
\pgfsetstrokecolor{currentstroke}%
\pgfsetdash{}{0pt}%
\pgfsys@defobject{currentmarker}{\pgfqpoint{0.000000in}{-0.048611in}}{\pgfqpoint{0.000000in}{0.000000in}}{%
\pgfpathmoveto{\pgfqpoint{0.000000in}{0.000000in}}%
\pgfpathlineto{\pgfqpoint{0.000000in}{-0.048611in}}%
\pgfusepath{stroke,fill}%
}%
\begin{pgfscope}%
\pgfsys@transformshift{4.632727in}{0.440000in}%
\pgfsys@useobject{currentmarker}{}%
\end{pgfscope}%
\end{pgfscope}%
\begin{pgfscope}%
\definecolor{textcolor}{rgb}{0.000000,0.000000,0.000000}%
\pgfsetstrokecolor{textcolor}%
\pgfsetfillcolor{textcolor}%
\pgftext[x=4.632727in,y=0.342778in,,top]{\color{textcolor}\rmfamily\fontsize{10.000000}{12.000000}\selectfont \(\displaystyle {0.05}\)}%
\end{pgfscope}%
\begin{pgfscope}%
\pgfsetbuttcap%
\pgfsetroundjoin%
\definecolor{currentfill}{rgb}{0.000000,0.000000,0.000000}%
\pgfsetfillcolor{currentfill}%
\pgfsetlinewidth{0.803000pt}%
\definecolor{currentstroke}{rgb}{0.000000,0.000000,0.000000}%
\pgfsetstrokecolor{currentstroke}%
\pgfsetdash{}{0pt}%
\pgfsys@defobject{currentmarker}{\pgfqpoint{0.000000in}{-0.048611in}}{\pgfqpoint{0.000000in}{0.000000in}}{%
\pgfpathmoveto{\pgfqpoint{0.000000in}{0.000000in}}%
\pgfpathlineto{\pgfqpoint{0.000000in}{-0.048611in}}%
\pgfusepath{stroke,fill}%
}%
\begin{pgfscope}%
\pgfsys@transformshift{5.354182in}{0.440000in}%
\pgfsys@useobject{currentmarker}{}%
\end{pgfscope}%
\end{pgfscope}%
\begin{pgfscope}%
\definecolor{textcolor}{rgb}{0.000000,0.000000,0.000000}%
\pgfsetstrokecolor{textcolor}%
\pgfsetfillcolor{textcolor}%
\pgftext[x=5.354182in,y=0.342778in,,top]{\color{textcolor}\rmfamily\fontsize{10.000000}{12.000000}\selectfont \(\displaystyle {0.06}\)}%
\end{pgfscope}%
\begin{pgfscope}%
\definecolor{textcolor}{rgb}{0.000000,0.000000,0.000000}%
\pgfsetstrokecolor{textcolor}%
\pgfsetfillcolor{textcolor}%
\pgftext[x=3.280000in,y=0.163766in,,top]{\color{textcolor}\rmfamily\fontsize{10.000000}{12.000000}\selectfont Squared displacement of bars \(\displaystyle d^2\) in \(\displaystyle m^2\)}%
\end{pgfscope}%

	\end{center}
	\caption{Plotting the measured moments of inertia against the displacement $d$ squared with the corresponding error bars, in comparison to the values calculated with Steiner's theorem presented in a continuous linear line.}\label{fig::stein}
\end{figure}


The results of measuring the oscillation period of a dumbbell in different axes of rotation on a plane, are plotted against the cosine squared of corresponding angle.
This is done to determine, if there is a linear behaviour. 

\begin{figure}[ht]
	\begin{center}
		%% Creator: Matplotlib, PGF backend
%%
%% To include the figure in your LaTeX document, write
%%   \input{<filename>.pgf}
%%
%% Make sure the required packages are loaded in your preamble
%%   \usepackage{pgf}
%%
%% and, on pdftex
%%   \usepackage[utf8]{inputenc}\DeclareUnicodeCharacter{2212}{-}
%%
%% or, on luatex and xetex
%%   \usepackage{unicode-math}
%%
%% Figures using additional raster images can only be included by \input if
%% they are in the same directory as the main LaTeX file. For loading figures
%% from other directories you can use the `import` package
%%   \usepackage{import}
%%
%% and then include the figures with
%%   \import{<path to file>}{<filename>.pgf}
%%
%% Matplotlib used the following preamble
%%
\begingroup%
\makeatletter%
\begin{pgfpicture}%
\pgfpathrectangle{\pgfpointorigin}{\pgfqpoint{6.400000in}{4.000000in}}%
\pgfusepath{use as bounding box, clip}%
\begin{pgfscope}%
\pgfsetbuttcap%
\pgfsetmiterjoin%
\definecolor{currentfill}{rgb}{1.000000,1.000000,1.000000}%
\pgfsetfillcolor{currentfill}%
\pgfsetlinewidth{0.000000pt}%
\definecolor{currentstroke}{rgb}{1.000000,1.000000,1.000000}%
\pgfsetstrokecolor{currentstroke}%
\pgfsetdash{}{0pt}%
\pgfpathmoveto{\pgfqpoint{0.000000in}{0.000000in}}%
\pgfpathlineto{\pgfqpoint{6.400000in}{0.000000in}}%
\pgfpathlineto{\pgfqpoint{6.400000in}{4.000000in}}%
\pgfpathlineto{\pgfqpoint{0.000000in}{4.000000in}}%
\pgfpathclose%
\pgfusepath{fill}%
\end{pgfscope}%
\begin{pgfscope}%
\pgfsetbuttcap%
\pgfsetmiterjoin%
\definecolor{currentfill}{rgb}{1.000000,1.000000,1.000000}%
\pgfsetfillcolor{currentfill}%
\pgfsetlinewidth{0.000000pt}%
\definecolor{currentstroke}{rgb}{0.000000,0.000000,0.000000}%
\pgfsetstrokecolor{currentstroke}%
\pgfsetstrokeopacity{0.000000}%
\pgfsetdash{}{0pt}%
\pgfpathmoveto{\pgfqpoint{0.800000in}{0.440000in}}%
\pgfpathlineto{\pgfqpoint{5.760000in}{0.440000in}}%
\pgfpathlineto{\pgfqpoint{5.760000in}{3.520000in}}%
\pgfpathlineto{\pgfqpoint{0.800000in}{3.520000in}}%
\pgfpathclose%
\pgfusepath{fill}%
\end{pgfscope}%
\begin{pgfscope}%
\pgfsetbuttcap%
\pgfsetroundjoin%
\definecolor{currentfill}{rgb}{0.000000,0.000000,0.000000}%
\pgfsetfillcolor{currentfill}%
\pgfsetlinewidth{0.803000pt}%
\definecolor{currentstroke}{rgb}{0.000000,0.000000,0.000000}%
\pgfsetstrokecolor{currentstroke}%
\pgfsetdash{}{0pt}%
\pgfsys@defobject{currentmarker}{\pgfqpoint{0.000000in}{-0.048611in}}{\pgfqpoint{0.000000in}{0.000000in}}{%
\pgfpathmoveto{\pgfqpoint{0.000000in}{0.000000in}}%
\pgfpathlineto{\pgfqpoint{0.000000in}{-0.048611in}}%
\pgfusepath{stroke,fill}%
}%
\begin{pgfscope}%
\pgfsys@transformshift{1.025455in}{0.440000in}%
\pgfsys@useobject{currentmarker}{}%
\end{pgfscope}%
\end{pgfscope}%
\begin{pgfscope}%
\definecolor{textcolor}{rgb}{0.000000,0.000000,0.000000}%
\pgfsetstrokecolor{textcolor}%
\pgfsetfillcolor{textcolor}%
\pgftext[x=1.025455in,y=0.342778in,,top]{\color{textcolor}\rmfamily\fontsize{10.000000}{12.000000}\selectfont \(\displaystyle {0.0}\)}%
\end{pgfscope}%
\begin{pgfscope}%
\pgfsetbuttcap%
\pgfsetroundjoin%
\definecolor{currentfill}{rgb}{0.000000,0.000000,0.000000}%
\pgfsetfillcolor{currentfill}%
\pgfsetlinewidth{0.803000pt}%
\definecolor{currentstroke}{rgb}{0.000000,0.000000,0.000000}%
\pgfsetstrokecolor{currentstroke}%
\pgfsetdash{}{0pt}%
\pgfsys@defobject{currentmarker}{\pgfqpoint{0.000000in}{-0.048611in}}{\pgfqpoint{0.000000in}{0.000000in}}{%
\pgfpathmoveto{\pgfqpoint{0.000000in}{0.000000in}}%
\pgfpathlineto{\pgfqpoint{0.000000in}{-0.048611in}}%
\pgfusepath{stroke,fill}%
}%
\begin{pgfscope}%
\pgfsys@transformshift{1.927273in}{0.440000in}%
\pgfsys@useobject{currentmarker}{}%
\end{pgfscope}%
\end{pgfscope}%
\begin{pgfscope}%
\definecolor{textcolor}{rgb}{0.000000,0.000000,0.000000}%
\pgfsetstrokecolor{textcolor}%
\pgfsetfillcolor{textcolor}%
\pgftext[x=1.927273in,y=0.342778in,,top]{\color{textcolor}\rmfamily\fontsize{10.000000}{12.000000}\selectfont \(\displaystyle {0.2}\)}%
\end{pgfscope}%
\begin{pgfscope}%
\pgfsetbuttcap%
\pgfsetroundjoin%
\definecolor{currentfill}{rgb}{0.000000,0.000000,0.000000}%
\pgfsetfillcolor{currentfill}%
\pgfsetlinewidth{0.803000pt}%
\definecolor{currentstroke}{rgb}{0.000000,0.000000,0.000000}%
\pgfsetstrokecolor{currentstroke}%
\pgfsetdash{}{0pt}%
\pgfsys@defobject{currentmarker}{\pgfqpoint{0.000000in}{-0.048611in}}{\pgfqpoint{0.000000in}{0.000000in}}{%
\pgfpathmoveto{\pgfqpoint{0.000000in}{0.000000in}}%
\pgfpathlineto{\pgfqpoint{0.000000in}{-0.048611in}}%
\pgfusepath{stroke,fill}%
}%
\begin{pgfscope}%
\pgfsys@transformshift{2.829091in}{0.440000in}%
\pgfsys@useobject{currentmarker}{}%
\end{pgfscope}%
\end{pgfscope}%
\begin{pgfscope}%
\definecolor{textcolor}{rgb}{0.000000,0.000000,0.000000}%
\pgfsetstrokecolor{textcolor}%
\pgfsetfillcolor{textcolor}%
\pgftext[x=2.829091in,y=0.342778in,,top]{\color{textcolor}\rmfamily\fontsize{10.000000}{12.000000}\selectfont \(\displaystyle {0.4}\)}%
\end{pgfscope}%
\begin{pgfscope}%
\pgfsetbuttcap%
\pgfsetroundjoin%
\definecolor{currentfill}{rgb}{0.000000,0.000000,0.000000}%
\pgfsetfillcolor{currentfill}%
\pgfsetlinewidth{0.803000pt}%
\definecolor{currentstroke}{rgb}{0.000000,0.000000,0.000000}%
\pgfsetstrokecolor{currentstroke}%
\pgfsetdash{}{0pt}%
\pgfsys@defobject{currentmarker}{\pgfqpoint{0.000000in}{-0.048611in}}{\pgfqpoint{0.000000in}{0.000000in}}{%
\pgfpathmoveto{\pgfqpoint{0.000000in}{0.000000in}}%
\pgfpathlineto{\pgfqpoint{0.000000in}{-0.048611in}}%
\pgfusepath{stroke,fill}%
}%
\begin{pgfscope}%
\pgfsys@transformshift{3.730909in}{0.440000in}%
\pgfsys@useobject{currentmarker}{}%
\end{pgfscope}%
\end{pgfscope}%
\begin{pgfscope}%
\definecolor{textcolor}{rgb}{0.000000,0.000000,0.000000}%
\pgfsetstrokecolor{textcolor}%
\pgfsetfillcolor{textcolor}%
\pgftext[x=3.730909in,y=0.342778in,,top]{\color{textcolor}\rmfamily\fontsize{10.000000}{12.000000}\selectfont \(\displaystyle {0.6}\)}%
\end{pgfscope}%
\begin{pgfscope}%
\pgfsetbuttcap%
\pgfsetroundjoin%
\definecolor{currentfill}{rgb}{0.000000,0.000000,0.000000}%
\pgfsetfillcolor{currentfill}%
\pgfsetlinewidth{0.803000pt}%
\definecolor{currentstroke}{rgb}{0.000000,0.000000,0.000000}%
\pgfsetstrokecolor{currentstroke}%
\pgfsetdash{}{0pt}%
\pgfsys@defobject{currentmarker}{\pgfqpoint{0.000000in}{-0.048611in}}{\pgfqpoint{0.000000in}{0.000000in}}{%
\pgfpathmoveto{\pgfqpoint{0.000000in}{0.000000in}}%
\pgfpathlineto{\pgfqpoint{0.000000in}{-0.048611in}}%
\pgfusepath{stroke,fill}%
}%
\begin{pgfscope}%
\pgfsys@transformshift{4.632727in}{0.440000in}%
\pgfsys@useobject{currentmarker}{}%
\end{pgfscope}%
\end{pgfscope}%
\begin{pgfscope}%
\definecolor{textcolor}{rgb}{0.000000,0.000000,0.000000}%
\pgfsetstrokecolor{textcolor}%
\pgfsetfillcolor{textcolor}%
\pgftext[x=4.632727in,y=0.342778in,,top]{\color{textcolor}\rmfamily\fontsize{10.000000}{12.000000}\selectfont \(\displaystyle {0.8}\)}%
\end{pgfscope}%
\begin{pgfscope}%
\pgfsetbuttcap%
\pgfsetroundjoin%
\definecolor{currentfill}{rgb}{0.000000,0.000000,0.000000}%
\pgfsetfillcolor{currentfill}%
\pgfsetlinewidth{0.803000pt}%
\definecolor{currentstroke}{rgb}{0.000000,0.000000,0.000000}%
\pgfsetstrokecolor{currentstroke}%
\pgfsetdash{}{0pt}%
\pgfsys@defobject{currentmarker}{\pgfqpoint{0.000000in}{-0.048611in}}{\pgfqpoint{0.000000in}{0.000000in}}{%
\pgfpathmoveto{\pgfqpoint{0.000000in}{0.000000in}}%
\pgfpathlineto{\pgfqpoint{0.000000in}{-0.048611in}}%
\pgfusepath{stroke,fill}%
}%
\begin{pgfscope}%
\pgfsys@transformshift{5.534545in}{0.440000in}%
\pgfsys@useobject{currentmarker}{}%
\end{pgfscope}%
\end{pgfscope}%
\begin{pgfscope}%
\definecolor{textcolor}{rgb}{0.000000,0.000000,0.000000}%
\pgfsetstrokecolor{textcolor}%
\pgfsetfillcolor{textcolor}%
\pgftext[x=5.534545in,y=0.342778in,,top]{\color{textcolor}\rmfamily\fontsize{10.000000}{12.000000}\selectfont \(\displaystyle {1.0}\)}%
\end{pgfscope}%
\begin{pgfscope}%
\definecolor{textcolor}{rgb}{0.000000,0.000000,0.000000}%
\pgfsetstrokecolor{textcolor}%
\pgfsetfillcolor{textcolor}%
\pgftext[x=3.280000in,y=0.163766in,,top]{\color{textcolor}\rmfamily\fontsize{10.000000}{12.000000}\selectfont Alignement of the angle \(\displaystyle \cos^2(\varphi)\) }%
\end{pgfscope}%
\begin{pgfscope}%
\pgfsetbuttcap%
\pgfsetroundjoin%
\definecolor{currentfill}{rgb}{0.000000,0.000000,0.000000}%
\pgfsetfillcolor{currentfill}%
\pgfsetlinewidth{0.803000pt}%
\definecolor{currentstroke}{rgb}{0.000000,0.000000,0.000000}%
\pgfsetstrokecolor{currentstroke}%
\pgfsetdash{}{0pt}%
\pgfsys@defobject{currentmarker}{\pgfqpoint{-0.048611in}{0.000000in}}{\pgfqpoint{-0.000000in}{0.000000in}}{%
\pgfpathmoveto{\pgfqpoint{-0.000000in}{0.000000in}}%
\pgfpathlineto{\pgfqpoint{-0.048611in}{0.000000in}}%
\pgfusepath{stroke,fill}%
}%
\begin{pgfscope}%
\pgfsys@transformshift{0.800000in}{0.652672in}%
\pgfsys@useobject{currentmarker}{}%
\end{pgfscope}%
\end{pgfscope}%
\begin{pgfscope}%
\definecolor{textcolor}{rgb}{0.000000,0.000000,0.000000}%
\pgfsetstrokecolor{textcolor}%
\pgfsetfillcolor{textcolor}%
\pgftext[x=0.563888in, y=0.604447in, left, base]{\color{textcolor}\rmfamily\fontsize{10.000000}{12.000000}\selectfont \(\displaystyle {12}\)}%
\end{pgfscope}%
\begin{pgfscope}%
\pgfsetbuttcap%
\pgfsetroundjoin%
\definecolor{currentfill}{rgb}{0.000000,0.000000,0.000000}%
\pgfsetfillcolor{currentfill}%
\pgfsetlinewidth{0.803000pt}%
\definecolor{currentstroke}{rgb}{0.000000,0.000000,0.000000}%
\pgfsetstrokecolor{currentstroke}%
\pgfsetdash{}{0pt}%
\pgfsys@defobject{currentmarker}{\pgfqpoint{-0.048611in}{0.000000in}}{\pgfqpoint{-0.000000in}{0.000000in}}{%
\pgfpathmoveto{\pgfqpoint{-0.000000in}{0.000000in}}%
\pgfpathlineto{\pgfqpoint{-0.048611in}{0.000000in}}%
\pgfusepath{stroke,fill}%
}%
\begin{pgfscope}%
\pgfsys@transformshift{0.800000in}{1.126077in}%
\pgfsys@useobject{currentmarker}{}%
\end{pgfscope}%
\end{pgfscope}%
\begin{pgfscope}%
\definecolor{textcolor}{rgb}{0.000000,0.000000,0.000000}%
\pgfsetstrokecolor{textcolor}%
\pgfsetfillcolor{textcolor}%
\pgftext[x=0.563888in, y=1.077851in, left, base]{\color{textcolor}\rmfamily\fontsize{10.000000}{12.000000}\selectfont \(\displaystyle {14}\)}%
\end{pgfscope}%
\begin{pgfscope}%
\pgfsetbuttcap%
\pgfsetroundjoin%
\definecolor{currentfill}{rgb}{0.000000,0.000000,0.000000}%
\pgfsetfillcolor{currentfill}%
\pgfsetlinewidth{0.803000pt}%
\definecolor{currentstroke}{rgb}{0.000000,0.000000,0.000000}%
\pgfsetstrokecolor{currentstroke}%
\pgfsetdash{}{0pt}%
\pgfsys@defobject{currentmarker}{\pgfqpoint{-0.048611in}{0.000000in}}{\pgfqpoint{-0.000000in}{0.000000in}}{%
\pgfpathmoveto{\pgfqpoint{-0.000000in}{0.000000in}}%
\pgfpathlineto{\pgfqpoint{-0.048611in}{0.000000in}}%
\pgfusepath{stroke,fill}%
}%
\begin{pgfscope}%
\pgfsys@transformshift{0.800000in}{1.599481in}%
\pgfsys@useobject{currentmarker}{}%
\end{pgfscope}%
\end{pgfscope}%
\begin{pgfscope}%
\definecolor{textcolor}{rgb}{0.000000,0.000000,0.000000}%
\pgfsetstrokecolor{textcolor}%
\pgfsetfillcolor{textcolor}%
\pgftext[x=0.563888in, y=1.551256in, left, base]{\color{textcolor}\rmfamily\fontsize{10.000000}{12.000000}\selectfont \(\displaystyle {16}\)}%
\end{pgfscope}%
\begin{pgfscope}%
\pgfsetbuttcap%
\pgfsetroundjoin%
\definecolor{currentfill}{rgb}{0.000000,0.000000,0.000000}%
\pgfsetfillcolor{currentfill}%
\pgfsetlinewidth{0.803000pt}%
\definecolor{currentstroke}{rgb}{0.000000,0.000000,0.000000}%
\pgfsetstrokecolor{currentstroke}%
\pgfsetdash{}{0pt}%
\pgfsys@defobject{currentmarker}{\pgfqpoint{-0.048611in}{0.000000in}}{\pgfqpoint{-0.000000in}{0.000000in}}{%
\pgfpathmoveto{\pgfqpoint{-0.000000in}{0.000000in}}%
\pgfpathlineto{\pgfqpoint{-0.048611in}{0.000000in}}%
\pgfusepath{stroke,fill}%
}%
\begin{pgfscope}%
\pgfsys@transformshift{0.800000in}{2.072885in}%
\pgfsys@useobject{currentmarker}{}%
\end{pgfscope}%
\end{pgfscope}%
\begin{pgfscope}%
\definecolor{textcolor}{rgb}{0.000000,0.000000,0.000000}%
\pgfsetstrokecolor{textcolor}%
\pgfsetfillcolor{textcolor}%
\pgftext[x=0.563888in, y=2.024660in, left, base]{\color{textcolor}\rmfamily\fontsize{10.000000}{12.000000}\selectfont \(\displaystyle {18}\)}%
\end{pgfscope}%
\begin{pgfscope}%
\pgfsetbuttcap%
\pgfsetroundjoin%
\definecolor{currentfill}{rgb}{0.000000,0.000000,0.000000}%
\pgfsetfillcolor{currentfill}%
\pgfsetlinewidth{0.803000pt}%
\definecolor{currentstroke}{rgb}{0.000000,0.000000,0.000000}%
\pgfsetstrokecolor{currentstroke}%
\pgfsetdash{}{0pt}%
\pgfsys@defobject{currentmarker}{\pgfqpoint{-0.048611in}{0.000000in}}{\pgfqpoint{-0.000000in}{0.000000in}}{%
\pgfpathmoveto{\pgfqpoint{-0.000000in}{0.000000in}}%
\pgfpathlineto{\pgfqpoint{-0.048611in}{0.000000in}}%
\pgfusepath{stroke,fill}%
}%
\begin{pgfscope}%
\pgfsys@transformshift{0.800000in}{2.546290in}%
\pgfsys@useobject{currentmarker}{}%
\end{pgfscope}%
\end{pgfscope}%
\begin{pgfscope}%
\definecolor{textcolor}{rgb}{0.000000,0.000000,0.000000}%
\pgfsetstrokecolor{textcolor}%
\pgfsetfillcolor{textcolor}%
\pgftext[x=0.563888in, y=2.498064in, left, base]{\color{textcolor}\rmfamily\fontsize{10.000000}{12.000000}\selectfont \(\displaystyle {20}\)}%
\end{pgfscope}%
\begin{pgfscope}%
\pgfsetbuttcap%
\pgfsetroundjoin%
\definecolor{currentfill}{rgb}{0.000000,0.000000,0.000000}%
\pgfsetfillcolor{currentfill}%
\pgfsetlinewidth{0.803000pt}%
\definecolor{currentstroke}{rgb}{0.000000,0.000000,0.000000}%
\pgfsetstrokecolor{currentstroke}%
\pgfsetdash{}{0pt}%
\pgfsys@defobject{currentmarker}{\pgfqpoint{-0.048611in}{0.000000in}}{\pgfqpoint{-0.000000in}{0.000000in}}{%
\pgfpathmoveto{\pgfqpoint{-0.000000in}{0.000000in}}%
\pgfpathlineto{\pgfqpoint{-0.048611in}{0.000000in}}%
\pgfusepath{stroke,fill}%
}%
\begin{pgfscope}%
\pgfsys@transformshift{0.800000in}{3.019694in}%
\pgfsys@useobject{currentmarker}{}%
\end{pgfscope}%
\end{pgfscope}%
\begin{pgfscope}%
\definecolor{textcolor}{rgb}{0.000000,0.000000,0.000000}%
\pgfsetstrokecolor{textcolor}%
\pgfsetfillcolor{textcolor}%
\pgftext[x=0.563888in, y=2.971469in, left, base]{\color{textcolor}\rmfamily\fontsize{10.000000}{12.000000}\selectfont \(\displaystyle {22}\)}%
\end{pgfscope}%
\begin{pgfscope}%
\pgfsetbuttcap%
\pgfsetroundjoin%
\definecolor{currentfill}{rgb}{0.000000,0.000000,0.000000}%
\pgfsetfillcolor{currentfill}%
\pgfsetlinewidth{0.803000pt}%
\definecolor{currentstroke}{rgb}{0.000000,0.000000,0.000000}%
\pgfsetstrokecolor{currentstroke}%
\pgfsetdash{}{0pt}%
\pgfsys@defobject{currentmarker}{\pgfqpoint{-0.048611in}{0.000000in}}{\pgfqpoint{-0.000000in}{0.000000in}}{%
\pgfpathmoveto{\pgfqpoint{-0.000000in}{0.000000in}}%
\pgfpathlineto{\pgfqpoint{-0.048611in}{0.000000in}}%
\pgfusepath{stroke,fill}%
}%
\begin{pgfscope}%
\pgfsys@transformshift{0.800000in}{3.493099in}%
\pgfsys@useobject{currentmarker}{}%
\end{pgfscope}%
\end{pgfscope}%
\begin{pgfscope}%
\definecolor{textcolor}{rgb}{0.000000,0.000000,0.000000}%
\pgfsetstrokecolor{textcolor}%
\pgfsetfillcolor{textcolor}%
\pgftext[x=0.563888in, y=3.444873in, left, base]{\color{textcolor}\rmfamily\fontsize{10.000000}{12.000000}\selectfont \(\displaystyle {24}\)}%
\end{pgfscope}%
\begin{pgfscope}%
\definecolor{textcolor}{rgb}{0.000000,0.000000,0.000000}%
\pgfsetstrokecolor{textcolor}%
\pgfsetfillcolor{textcolor}%
\pgftext[x=0.508333in,y=1.980000in,,bottom,rotate=90.000000]{\color{textcolor}\rmfamily\fontsize{10.000000}{12.000000}\selectfont Time for one period (s)}%
\end{pgfscope}%
\begin{pgfscope}%
\pgfpathrectangle{\pgfqpoint{0.800000in}{0.440000in}}{\pgfqpoint{4.960000in}{3.080000in}}%
\pgfusepath{clip}%
\pgfsetbuttcap%
\pgfsetroundjoin%
\pgfsetlinewidth{1.505625pt}%
\definecolor{currentstroke}{rgb}{1.000000,0.000000,0.000000}%
\pgfsetstrokecolor{currentstroke}%
\pgfsetdash{}{0pt}%
\pgfpathmoveto{\pgfqpoint{5.534545in}{0.655786in}}%
\pgfpathlineto{\pgfqpoint{5.534545in}{0.655786in}}%
\pgfusepath{stroke}%
\end{pgfscope}%
\begin{pgfscope}%
\pgfpathrectangle{\pgfqpoint{0.800000in}{0.440000in}}{\pgfqpoint{4.960000in}{3.080000in}}%
\pgfusepath{clip}%
\pgfsetbuttcap%
\pgfsetroundjoin%
\pgfsetlinewidth{1.505625pt}%
\definecolor{currentstroke}{rgb}{1.000000,0.000000,0.000000}%
\pgfsetstrokecolor{currentstroke}%
\pgfsetdash{}{0pt}%
\pgfpathmoveto{\pgfqpoint{5.385121in}{0.760654in}}%
\pgfpathlineto{\pgfqpoint{5.412038in}{0.760654in}}%
\pgfusepath{stroke}%
\end{pgfscope}%
\begin{pgfscope}%
\pgfpathrectangle{\pgfqpoint{0.800000in}{0.440000in}}{\pgfqpoint{4.960000in}{3.080000in}}%
\pgfusepath{clip}%
\pgfsetbuttcap%
\pgfsetroundjoin%
\pgfsetlinewidth{1.505625pt}%
\definecolor{currentstroke}{rgb}{1.000000,0.000000,0.000000}%
\pgfsetstrokecolor{currentstroke}%
\pgfsetdash{}{0pt}%
\pgfpathmoveto{\pgfqpoint{4.981789in}{0.976087in}}%
\pgfpathlineto{\pgfqpoint{5.032375in}{0.976087in}}%
\pgfusepath{stroke}%
\end{pgfscope}%
\begin{pgfscope}%
\pgfpathrectangle{\pgfqpoint{0.800000in}{0.440000in}}{\pgfqpoint{4.960000in}{3.080000in}}%
\pgfusepath{clip}%
\pgfsetbuttcap%
\pgfsetroundjoin%
\pgfsetlinewidth{1.505625pt}%
\definecolor{currentstroke}{rgb}{1.000000,0.000000,0.000000}%
\pgfsetstrokecolor{currentstroke}%
\pgfsetdash{}{0pt}%
\pgfpathmoveto{\pgfqpoint{4.373195in}{1.339014in}}%
\pgfpathlineto{\pgfqpoint{4.441350in}{1.339014in}}%
\pgfusepath{stroke}%
\end{pgfscope}%
\begin{pgfscope}%
\pgfpathrectangle{\pgfqpoint{0.800000in}{0.440000in}}{\pgfqpoint{4.960000in}{3.080000in}}%
\pgfusepath{clip}%
\pgfsetbuttcap%
\pgfsetroundjoin%
\pgfsetlinewidth{1.505625pt}%
\definecolor{currentstroke}{rgb}{1.000000,0.000000,0.000000}%
\pgfsetstrokecolor{currentstroke}%
\pgfsetdash{}{0pt}%
\pgfpathmoveto{\pgfqpoint{3.632746in}{1.766662in}}%
\pgfpathlineto{\pgfqpoint{3.710249in}{1.766662in}}%
\pgfusepath{stroke}%
\end{pgfscope}%
\begin{pgfscope}%
\pgfpathrectangle{\pgfqpoint{0.800000in}{0.440000in}}{\pgfqpoint{4.960000in}{3.080000in}}%
\pgfusepath{clip}%
\pgfsetbuttcap%
\pgfsetroundjoin%
\pgfsetlinewidth{1.505625pt}%
\definecolor{currentstroke}{rgb}{1.000000,0.000000,0.000000}%
\pgfsetstrokecolor{currentstroke}%
\pgfsetdash{}{0pt}%
\pgfpathmoveto{\pgfqpoint{2.849751in}{2.217415in}}%
\pgfpathlineto{\pgfqpoint{2.927254in}{2.217415in}}%
\pgfusepath{stroke}%
\end{pgfscope}%
\begin{pgfscope}%
\pgfpathrectangle{\pgfqpoint{0.800000in}{0.440000in}}{\pgfqpoint{4.960000in}{3.080000in}}%
\pgfusepath{clip}%
\pgfsetbuttcap%
\pgfsetroundjoin%
\pgfsetlinewidth{1.505625pt}%
\definecolor{currentstroke}{rgb}{1.000000,0.000000,0.000000}%
\pgfsetstrokecolor{currentstroke}%
\pgfsetdash{}{0pt}%
\pgfpathmoveto{\pgfqpoint{2.118650in}{2.609727in}}%
\pgfpathlineto{\pgfqpoint{2.186805in}{2.609727in}}%
\pgfusepath{stroke}%
\end{pgfscope}%
\begin{pgfscope}%
\pgfpathrectangle{\pgfqpoint{0.800000in}{0.440000in}}{\pgfqpoint{4.960000in}{3.080000in}}%
\pgfusepath{clip}%
\pgfsetbuttcap%
\pgfsetroundjoin%
\pgfsetlinewidth{1.505625pt}%
\definecolor{currentstroke}{rgb}{1.000000,0.000000,0.000000}%
\pgfsetstrokecolor{currentstroke}%
\pgfsetdash{}{0pt}%
\pgfpathmoveto{\pgfqpoint{1.527625in}{2.962676in}}%
\pgfpathlineto{\pgfqpoint{1.578211in}{2.962676in}}%
\pgfusepath{stroke}%
\end{pgfscope}%
\begin{pgfscope}%
\pgfpathrectangle{\pgfqpoint{0.800000in}{0.440000in}}{\pgfqpoint{4.960000in}{3.080000in}}%
\pgfusepath{clip}%
\pgfsetbuttcap%
\pgfsetroundjoin%
\pgfsetlinewidth{1.505625pt}%
\definecolor{currentstroke}{rgb}{1.000000,0.000000,0.000000}%
\pgfsetstrokecolor{currentstroke}%
\pgfsetdash{}{0pt}%
\pgfpathmoveto{\pgfqpoint{1.147962in}{3.214480in}}%
\pgfpathlineto{\pgfqpoint{1.174879in}{3.214480in}}%
\pgfusepath{stroke}%
\end{pgfscope}%
\begin{pgfscope}%
\pgfpathrectangle{\pgfqpoint{0.800000in}{0.440000in}}{\pgfqpoint{4.960000in}{3.080000in}}%
\pgfusepath{clip}%
\pgfsetbuttcap%
\pgfsetroundjoin%
\pgfsetlinewidth{1.505625pt}%
\definecolor{currentstroke}{rgb}{1.000000,0.000000,0.000000}%
\pgfsetstrokecolor{currentstroke}%
\pgfsetdash{}{0pt}%
\pgfpathmoveto{\pgfqpoint{1.025455in}{3.274958in}}%
\pgfpathlineto{\pgfqpoint{1.025455in}{3.274958in}}%
\pgfusepath{stroke}%
\end{pgfscope}%
\begin{pgfscope}%
\pgfpathrectangle{\pgfqpoint{0.800000in}{0.440000in}}{\pgfqpoint{4.960000in}{3.080000in}}%
\pgfusepath{clip}%
\pgfsetbuttcap%
\pgfsetroundjoin%
\pgfsetlinewidth{1.505625pt}%
\definecolor{currentstroke}{rgb}{1.000000,0.000000,0.000000}%
\pgfsetstrokecolor{currentstroke}%
\pgfsetdash{}{0pt}%
\pgfpathmoveto{\pgfqpoint{5.534545in}{0.580000in}}%
\pgfpathlineto{\pgfqpoint{5.534545in}{0.731572in}}%
\pgfusepath{stroke}%
\end{pgfscope}%
\begin{pgfscope}%
\pgfpathrectangle{\pgfqpoint{0.800000in}{0.440000in}}{\pgfqpoint{4.960000in}{3.080000in}}%
\pgfusepath{clip}%
\pgfsetbuttcap%
\pgfsetroundjoin%
\pgfsetlinewidth{1.505625pt}%
\definecolor{currentstroke}{rgb}{1.000000,0.000000,0.000000}%
\pgfsetstrokecolor{currentstroke}%
\pgfsetdash{}{0pt}%
\pgfpathmoveto{\pgfqpoint{5.398580in}{0.683483in}}%
\pgfpathlineto{\pgfqpoint{5.398580in}{0.837825in}}%
\pgfusepath{stroke}%
\end{pgfscope}%
\begin{pgfscope}%
\pgfpathrectangle{\pgfqpoint{0.800000in}{0.440000in}}{\pgfqpoint{4.960000in}{3.080000in}}%
\pgfusepath{clip}%
\pgfsetbuttcap%
\pgfsetroundjoin%
\pgfsetlinewidth{1.505625pt}%
\definecolor{currentstroke}{rgb}{1.000000,0.000000,0.000000}%
\pgfsetstrokecolor{currentstroke}%
\pgfsetdash{}{0pt}%
\pgfpathmoveto{\pgfqpoint{5.007082in}{0.896146in}}%
\pgfpathlineto{\pgfqpoint{5.007082in}{1.056028in}}%
\pgfusepath{stroke}%
\end{pgfscope}%
\begin{pgfscope}%
\pgfpathrectangle{\pgfqpoint{0.800000in}{0.440000in}}{\pgfqpoint{4.960000in}{3.080000in}}%
\pgfusepath{clip}%
\pgfsetbuttcap%
\pgfsetroundjoin%
\pgfsetlinewidth{1.505625pt}%
\definecolor{currentstroke}{rgb}{1.000000,0.000000,0.000000}%
\pgfsetstrokecolor{currentstroke}%
\pgfsetdash{}{0pt}%
\pgfpathmoveto{\pgfqpoint{4.407273in}{1.254613in}}%
\pgfpathlineto{\pgfqpoint{4.407273in}{1.423415in}}%
\pgfusepath{stroke}%
\end{pgfscope}%
\begin{pgfscope}%
\pgfpathrectangle{\pgfqpoint{0.800000in}{0.440000in}}{\pgfqpoint{4.960000in}{3.080000in}}%
\pgfusepath{clip}%
\pgfsetbuttcap%
\pgfsetroundjoin%
\pgfsetlinewidth{1.505625pt}%
\definecolor{currentstroke}{rgb}{1.000000,0.000000,0.000000}%
\pgfsetstrokecolor{currentstroke}%
\pgfsetdash{}{0pt}%
\pgfpathmoveto{\pgfqpoint{3.671498in}{1.677290in}}%
\pgfpathlineto{\pgfqpoint{3.671498in}{1.856034in}}%
\pgfusepath{stroke}%
\end{pgfscope}%
\begin{pgfscope}%
\pgfpathrectangle{\pgfqpoint{0.800000in}{0.440000in}}{\pgfqpoint{4.960000in}{3.080000in}}%
\pgfusepath{clip}%
\pgfsetbuttcap%
\pgfsetroundjoin%
\pgfsetlinewidth{1.505625pt}%
\definecolor{currentstroke}{rgb}{1.000000,0.000000,0.000000}%
\pgfsetstrokecolor{currentstroke}%
\pgfsetdash{}{0pt}%
\pgfpathmoveto{\pgfqpoint{2.888502in}{2.123087in}}%
\pgfpathlineto{\pgfqpoint{2.888502in}{2.311743in}}%
\pgfusepath{stroke}%
\end{pgfscope}%
\begin{pgfscope}%
\pgfpathrectangle{\pgfqpoint{0.800000in}{0.440000in}}{\pgfqpoint{4.960000in}{3.080000in}}%
\pgfusepath{clip}%
\pgfsetbuttcap%
\pgfsetroundjoin%
\pgfsetlinewidth{1.505625pt}%
\definecolor{currentstroke}{rgb}{1.000000,0.000000,0.000000}%
\pgfsetstrokecolor{currentstroke}%
\pgfsetdash{}{0pt}%
\pgfpathmoveto{\pgfqpoint{2.152727in}{2.511288in}}%
\pgfpathlineto{\pgfqpoint{2.152727in}{2.708166in}}%
\pgfusepath{stroke}%
\end{pgfscope}%
\begin{pgfscope}%
\pgfpathrectangle{\pgfqpoint{0.800000in}{0.440000in}}{\pgfqpoint{4.960000in}{3.080000in}}%
\pgfusepath{clip}%
\pgfsetbuttcap%
\pgfsetroundjoin%
\pgfsetlinewidth{1.505625pt}%
\definecolor{currentstroke}{rgb}{1.000000,0.000000,0.000000}%
\pgfsetstrokecolor{currentstroke}%
\pgfsetdash{}{0pt}%
\pgfpathmoveto{\pgfqpoint{1.552918in}{2.860680in}}%
\pgfpathlineto{\pgfqpoint{1.552918in}{3.064672in}}%
\pgfusepath{stroke}%
\end{pgfscope}%
\begin{pgfscope}%
\pgfpathrectangle{\pgfqpoint{0.800000in}{0.440000in}}{\pgfqpoint{4.960000in}{3.080000in}}%
\pgfusepath{clip}%
\pgfsetbuttcap%
\pgfsetroundjoin%
\pgfsetlinewidth{1.505625pt}%
\definecolor{currentstroke}{rgb}{1.000000,0.000000,0.000000}%
\pgfsetstrokecolor{currentstroke}%
\pgfsetdash{}{0pt}%
\pgfpathmoveto{\pgfqpoint{1.161420in}{3.110020in}}%
\pgfpathlineto{\pgfqpoint{1.161420in}{3.318939in}}%
\pgfusepath{stroke}%
\end{pgfscope}%
\begin{pgfscope}%
\pgfpathrectangle{\pgfqpoint{0.800000in}{0.440000in}}{\pgfqpoint{4.960000in}{3.080000in}}%
\pgfusepath{clip}%
\pgfsetbuttcap%
\pgfsetroundjoin%
\pgfsetlinewidth{1.505625pt}%
\definecolor{currentstroke}{rgb}{1.000000,0.000000,0.000000}%
\pgfsetstrokecolor{currentstroke}%
\pgfsetdash{}{0pt}%
\pgfpathmoveto{\pgfqpoint{1.025455in}{3.169915in}}%
\pgfpathlineto{\pgfqpoint{1.025455in}{3.380000in}}%
\pgfusepath{stroke}%
\end{pgfscope}%
\begin{pgfscope}%
\pgfpathrectangle{\pgfqpoint{0.800000in}{0.440000in}}{\pgfqpoint{4.960000in}{3.080000in}}%
\pgfusepath{clip}%
\pgfsetrectcap%
\pgfsetroundjoin%
\pgfsetlinewidth{1.505625pt}%
\definecolor{currentstroke}{rgb}{0.121569,0.466667,0.705882}%
\pgfsetstrokecolor{currentstroke}%
\pgfsetdash{}{0pt}%
\pgfpathmoveto{\pgfqpoint{5.534545in}{0.675844in}}%
\pgfpathlineto{\pgfqpoint{5.533410in}{0.676500in}}%
\pgfpathlineto{\pgfqpoint{5.530006in}{0.678466in}}%
\pgfpathlineto{\pgfqpoint{5.524337in}{0.681740in}}%
\pgfpathlineto{\pgfqpoint{5.516407in}{0.686319in}}%
\pgfpathlineto{\pgfqpoint{5.506226in}{0.692198in}}%
\pgfpathlineto{\pgfqpoint{5.493803in}{0.699372in}}%
\pgfpathlineto{\pgfqpoint{5.479151in}{0.707832in}}%
\pgfpathlineto{\pgfqpoint{5.462284in}{0.717572in}}%
\pgfpathlineto{\pgfqpoint{5.443221in}{0.728581in}}%
\pgfpathlineto{\pgfqpoint{5.421979in}{0.740847in}}%
\pgfpathlineto{\pgfqpoint{5.398580in}{0.754359in}}%
\pgfpathlineto{\pgfqpoint{5.373048in}{0.769102in}}%
\pgfpathlineto{\pgfqpoint{5.345408in}{0.785063in}}%
\pgfpathlineto{\pgfqpoint{5.315689in}{0.802225in}}%
\pgfpathlineto{\pgfqpoint{5.283920in}{0.820570in}}%
\pgfpathlineto{\pgfqpoint{5.250133in}{0.840080in}}%
\pgfpathlineto{\pgfqpoint{5.214363in}{0.860736in}}%
\pgfpathlineto{\pgfqpoint{5.176644in}{0.882517in}}%
\pgfpathlineto{\pgfqpoint{5.137016in}{0.905400in}}%
\pgfpathlineto{\pgfqpoint{5.095518in}{0.929363in}}%
\pgfpathlineto{\pgfqpoint{5.052192in}{0.954382in}}%
\pgfpathlineto{\pgfqpoint{5.007082in}{0.980432in}}%
\pgfpathlineto{\pgfqpoint{4.960233in}{1.007485in}}%
\pgfpathlineto{\pgfqpoint{4.911691in}{1.035516in}}%
\pgfpathlineto{\pgfqpoint{4.861507in}{1.064495in}}%
\pgfpathlineto{\pgfqpoint{4.809730in}{1.094394in}}%
\pgfpathlineto{\pgfqpoint{4.756413in}{1.125182in}}%
\pgfpathlineto{\pgfqpoint{4.701610in}{1.156828in}}%
\pgfpathlineto{\pgfqpoint{4.645375in}{1.189302in}}%
\pgfpathlineto{\pgfqpoint{4.587765in}{1.222569in}}%
\pgfpathlineto{\pgfqpoint{4.528838in}{1.256597in}}%
\pgfpathlineto{\pgfqpoint{4.468654in}{1.291350in}}%
\pgfpathlineto{\pgfqpoint{4.407273in}{1.326795in}}%
\pgfpathlineto{\pgfqpoint{4.344757in}{1.362895in}}%
\pgfpathlineto{\pgfqpoint{4.281168in}{1.399615in}}%
\pgfpathlineto{\pgfqpoint{4.216572in}{1.436916in}}%
\pgfpathlineto{\pgfqpoint{4.151033in}{1.474763in}}%
\pgfpathlineto{\pgfqpoint{4.084616in}{1.513115in}}%
\pgfpathlineto{\pgfqpoint{4.017390in}{1.551936in}}%
\pgfpathlineto{\pgfqpoint{3.949420in}{1.591185in}}%
\pgfpathlineto{\pgfqpoint{3.880777in}{1.630823in}}%
\pgfpathlineto{\pgfqpoint{3.811529in}{1.670811in}}%
\pgfpathlineto{\pgfqpoint{3.741746in}{1.711108in}}%
\pgfpathlineto{\pgfqpoint{3.671498in}{1.751673in}}%
\pgfpathlineto{\pgfqpoint{3.600855in}{1.792466in}}%
\pgfpathlineto{\pgfqpoint{3.529890in}{1.833445in}}%
\pgfpathlineto{\pgfqpoint{3.458673in}{1.874570in}}%
\pgfpathlineto{\pgfqpoint{3.387276in}{1.915799in}}%
\pgfpathlineto{\pgfqpoint{3.315771in}{1.957090in}}%
\pgfpathlineto{\pgfqpoint{3.244229in}{1.998402in}}%
\pgfpathlineto{\pgfqpoint{3.172724in}{2.039693in}}%
\pgfpathlineto{\pgfqpoint{3.101327in}{2.080921in}}%
\pgfpathlineto{\pgfqpoint{3.030110in}{2.122046in}}%
\pgfpathlineto{\pgfqpoint{2.959145in}{2.163026in}}%
\pgfpathlineto{\pgfqpoint{2.888502in}{2.203819in}}%
\pgfpathlineto{\pgfqpoint{2.818254in}{2.244384in}}%
\pgfpathlineto{\pgfqpoint{2.748471in}{2.284681in}}%
\pgfpathlineto{\pgfqpoint{2.679223in}{2.324668in}}%
\pgfpathlineto{\pgfqpoint{2.610580in}{2.364307in}}%
\pgfpathlineto{\pgfqpoint{2.542610in}{2.403556in}}%
\pgfpathlineto{\pgfqpoint{2.475384in}{2.442377in}}%
\pgfpathlineto{\pgfqpoint{2.408967in}{2.480729in}}%
\pgfpathlineto{\pgfqpoint{2.343428in}{2.518575in}}%
\pgfpathlineto{\pgfqpoint{2.278832in}{2.555877in}}%
\pgfpathlineto{\pgfqpoint{2.215243in}{2.592596in}}%
\pgfpathlineto{\pgfqpoint{2.152727in}{2.628697in}}%
\pgfpathlineto{\pgfqpoint{2.091346in}{2.664141in}}%
\pgfpathlineto{\pgfqpoint{2.031162in}{2.698895in}}%
\pgfpathlineto{\pgfqpoint{1.972235in}{2.732923in}}%
\pgfpathlineto{\pgfqpoint{1.914625in}{2.766190in}}%
\pgfpathlineto{\pgfqpoint{1.858390in}{2.798663in}}%
\pgfpathlineto{\pgfqpoint{1.803587in}{2.830310in}}%
\pgfpathlineto{\pgfqpoint{1.750270in}{2.861098in}}%
\pgfpathlineto{\pgfqpoint{1.698493in}{2.890997in}}%
\pgfpathlineto{\pgfqpoint{1.648309in}{2.919976in}}%
\pgfpathlineto{\pgfqpoint{1.599767in}{2.948007in}}%
\pgfpathlineto{\pgfqpoint{1.552918in}{2.975060in}}%
\pgfpathlineto{\pgfqpoint{1.507808in}{3.001109in}}%
\pgfpathlineto{\pgfqpoint{1.464482in}{3.026128in}}%
\pgfpathlineto{\pgfqpoint{1.422984in}{3.050092in}}%
\pgfpathlineto{\pgfqpoint{1.383356in}{3.072975in}}%
\pgfpathlineto{\pgfqpoint{1.345637in}{3.094756in}}%
\pgfpathlineto{\pgfqpoint{1.309867in}{3.115412in}}%
\pgfpathlineto{\pgfqpoint{1.276080in}{3.134922in}}%
\pgfpathlineto{\pgfqpoint{1.244311in}{3.153267in}}%
\pgfpathlineto{\pgfqpoint{1.214592in}{3.170429in}}%
\pgfpathlineto{\pgfqpoint{1.186952in}{3.186389in}}%
\pgfpathlineto{\pgfqpoint{1.161420in}{3.201133in}}%
\pgfpathlineto{\pgfqpoint{1.138021in}{3.214645in}}%
\pgfpathlineto{\pgfqpoint{1.116779in}{3.226911in}}%
\pgfpathlineto{\pgfqpoint{1.097716in}{3.237920in}}%
\pgfpathlineto{\pgfqpoint{1.080849in}{3.247659in}}%
\pgfpathlineto{\pgfqpoint{1.066197in}{3.256120in}}%
\pgfpathlineto{\pgfqpoint{1.053774in}{3.263294in}}%
\pgfpathlineto{\pgfqpoint{1.043593in}{3.269173in}}%
\pgfpathlineto{\pgfqpoint{1.035663in}{3.273752in}}%
\pgfpathlineto{\pgfqpoint{1.029994in}{3.277026in}}%
\pgfpathlineto{\pgfqpoint{1.026590in}{3.278992in}}%
\pgfpathlineto{\pgfqpoint{1.025455in}{3.279647in}}%
\pgfusepath{stroke}%
\end{pgfscope}%
\begin{pgfscope}%
\pgfpathrectangle{\pgfqpoint{0.800000in}{0.440000in}}{\pgfqpoint{4.960000in}{3.080000in}}%
\pgfusepath{clip}%
\pgfsetbuttcap%
\pgfsetroundjoin%
\definecolor{currentfill}{rgb}{1.000000,0.000000,0.000000}%
\pgfsetfillcolor{currentfill}%
\pgfsetlinewidth{1.003750pt}%
\definecolor{currentstroke}{rgb}{1.000000,0.000000,0.000000}%
\pgfsetstrokecolor{currentstroke}%
\pgfsetdash{}{0pt}%
\pgfsys@defobject{currentmarker}{\pgfqpoint{0.000000in}{-0.041667in}}{\pgfqpoint{0.000000in}{0.041667in}}{%
\pgfpathmoveto{\pgfqpoint{0.000000in}{-0.041667in}}%
\pgfpathlineto{\pgfqpoint{0.000000in}{0.041667in}}%
\pgfusepath{stroke,fill}%
}%
\begin{pgfscope}%
\pgfsys@transformshift{5.534545in}{0.655786in}%
\pgfsys@useobject{currentmarker}{}%
\end{pgfscope}%
\begin{pgfscope}%
\pgfsys@transformshift{5.385121in}{0.760654in}%
\pgfsys@useobject{currentmarker}{}%
\end{pgfscope}%
\begin{pgfscope}%
\pgfsys@transformshift{4.981789in}{0.976087in}%
\pgfsys@useobject{currentmarker}{}%
\end{pgfscope}%
\begin{pgfscope}%
\pgfsys@transformshift{4.373195in}{1.339014in}%
\pgfsys@useobject{currentmarker}{}%
\end{pgfscope}%
\begin{pgfscope}%
\pgfsys@transformshift{3.632746in}{1.766662in}%
\pgfsys@useobject{currentmarker}{}%
\end{pgfscope}%
\begin{pgfscope}%
\pgfsys@transformshift{2.849751in}{2.217415in}%
\pgfsys@useobject{currentmarker}{}%
\end{pgfscope}%
\begin{pgfscope}%
\pgfsys@transformshift{2.118650in}{2.609727in}%
\pgfsys@useobject{currentmarker}{}%
\end{pgfscope}%
\begin{pgfscope}%
\pgfsys@transformshift{1.527625in}{2.962676in}%
\pgfsys@useobject{currentmarker}{}%
\end{pgfscope}%
\begin{pgfscope}%
\pgfsys@transformshift{1.147962in}{3.214480in}%
\pgfsys@useobject{currentmarker}{}%
\end{pgfscope}%
\begin{pgfscope}%
\pgfsys@transformshift{1.025455in}{3.274958in}%
\pgfsys@useobject{currentmarker}{}%
\end{pgfscope}%
\end{pgfscope}%
\begin{pgfscope}%
\pgfpathrectangle{\pgfqpoint{0.800000in}{0.440000in}}{\pgfqpoint{4.960000in}{3.080000in}}%
\pgfusepath{clip}%
\pgfsetbuttcap%
\pgfsetroundjoin%
\definecolor{currentfill}{rgb}{1.000000,0.000000,0.000000}%
\pgfsetfillcolor{currentfill}%
\pgfsetlinewidth{1.003750pt}%
\definecolor{currentstroke}{rgb}{1.000000,0.000000,0.000000}%
\pgfsetstrokecolor{currentstroke}%
\pgfsetdash{}{0pt}%
\pgfsys@defobject{currentmarker}{\pgfqpoint{0.000000in}{-0.041667in}}{\pgfqpoint{0.000000in}{0.041667in}}{%
\pgfpathmoveto{\pgfqpoint{0.000000in}{-0.041667in}}%
\pgfpathlineto{\pgfqpoint{0.000000in}{0.041667in}}%
\pgfusepath{stroke,fill}%
}%
\begin{pgfscope}%
\pgfsys@transformshift{5.534545in}{0.655786in}%
\pgfsys@useobject{currentmarker}{}%
\end{pgfscope}%
\begin{pgfscope}%
\pgfsys@transformshift{5.412038in}{0.760654in}%
\pgfsys@useobject{currentmarker}{}%
\end{pgfscope}%
\begin{pgfscope}%
\pgfsys@transformshift{5.032375in}{0.976087in}%
\pgfsys@useobject{currentmarker}{}%
\end{pgfscope}%
\begin{pgfscope}%
\pgfsys@transformshift{4.441350in}{1.339014in}%
\pgfsys@useobject{currentmarker}{}%
\end{pgfscope}%
\begin{pgfscope}%
\pgfsys@transformshift{3.710249in}{1.766662in}%
\pgfsys@useobject{currentmarker}{}%
\end{pgfscope}%
\begin{pgfscope}%
\pgfsys@transformshift{2.927254in}{2.217415in}%
\pgfsys@useobject{currentmarker}{}%
\end{pgfscope}%
\begin{pgfscope}%
\pgfsys@transformshift{2.186805in}{2.609727in}%
\pgfsys@useobject{currentmarker}{}%
\end{pgfscope}%
\begin{pgfscope}%
\pgfsys@transformshift{1.578211in}{2.962676in}%
\pgfsys@useobject{currentmarker}{}%
\end{pgfscope}%
\begin{pgfscope}%
\pgfsys@transformshift{1.174879in}{3.214480in}%
\pgfsys@useobject{currentmarker}{}%
\end{pgfscope}%
\begin{pgfscope}%
\pgfsys@transformshift{1.025455in}{3.274958in}%
\pgfsys@useobject{currentmarker}{}%
\end{pgfscope}%
\end{pgfscope}%
\begin{pgfscope}%
\pgfpathrectangle{\pgfqpoint{0.800000in}{0.440000in}}{\pgfqpoint{4.960000in}{3.080000in}}%
\pgfusepath{clip}%
\pgfsetbuttcap%
\pgfsetroundjoin%
\definecolor{currentfill}{rgb}{1.000000,0.000000,0.000000}%
\pgfsetfillcolor{currentfill}%
\pgfsetlinewidth{1.003750pt}%
\definecolor{currentstroke}{rgb}{1.000000,0.000000,0.000000}%
\pgfsetstrokecolor{currentstroke}%
\pgfsetdash{}{0pt}%
\pgfsys@defobject{currentmarker}{\pgfqpoint{-0.041667in}{-0.000000in}}{\pgfqpoint{0.041667in}{0.000000in}}{%
\pgfpathmoveto{\pgfqpoint{0.041667in}{-0.000000in}}%
\pgfpathlineto{\pgfqpoint{-0.041667in}{0.000000in}}%
\pgfusepath{stroke,fill}%
}%
\begin{pgfscope}%
\pgfsys@transformshift{5.534545in}{0.580000in}%
\pgfsys@useobject{currentmarker}{}%
\end{pgfscope}%
\begin{pgfscope}%
\pgfsys@transformshift{5.398580in}{0.683483in}%
\pgfsys@useobject{currentmarker}{}%
\end{pgfscope}%
\begin{pgfscope}%
\pgfsys@transformshift{5.007082in}{0.896146in}%
\pgfsys@useobject{currentmarker}{}%
\end{pgfscope}%
\begin{pgfscope}%
\pgfsys@transformshift{4.407273in}{1.254613in}%
\pgfsys@useobject{currentmarker}{}%
\end{pgfscope}%
\begin{pgfscope}%
\pgfsys@transformshift{3.671498in}{1.677290in}%
\pgfsys@useobject{currentmarker}{}%
\end{pgfscope}%
\begin{pgfscope}%
\pgfsys@transformshift{2.888502in}{2.123087in}%
\pgfsys@useobject{currentmarker}{}%
\end{pgfscope}%
\begin{pgfscope}%
\pgfsys@transformshift{2.152727in}{2.511288in}%
\pgfsys@useobject{currentmarker}{}%
\end{pgfscope}%
\begin{pgfscope}%
\pgfsys@transformshift{1.552918in}{2.860680in}%
\pgfsys@useobject{currentmarker}{}%
\end{pgfscope}%
\begin{pgfscope}%
\pgfsys@transformshift{1.161420in}{3.110020in}%
\pgfsys@useobject{currentmarker}{}%
\end{pgfscope}%
\begin{pgfscope}%
\pgfsys@transformshift{1.025455in}{3.169915in}%
\pgfsys@useobject{currentmarker}{}%
\end{pgfscope}%
\end{pgfscope}%
\begin{pgfscope}%
\pgfpathrectangle{\pgfqpoint{0.800000in}{0.440000in}}{\pgfqpoint{4.960000in}{3.080000in}}%
\pgfusepath{clip}%
\pgfsetbuttcap%
\pgfsetroundjoin%
\definecolor{currentfill}{rgb}{1.000000,0.000000,0.000000}%
\pgfsetfillcolor{currentfill}%
\pgfsetlinewidth{1.003750pt}%
\definecolor{currentstroke}{rgb}{1.000000,0.000000,0.000000}%
\pgfsetstrokecolor{currentstroke}%
\pgfsetdash{}{0pt}%
\pgfsys@defobject{currentmarker}{\pgfqpoint{-0.041667in}{-0.000000in}}{\pgfqpoint{0.041667in}{0.000000in}}{%
\pgfpathmoveto{\pgfqpoint{0.041667in}{-0.000000in}}%
\pgfpathlineto{\pgfqpoint{-0.041667in}{0.000000in}}%
\pgfusepath{stroke,fill}%
}%
\begin{pgfscope}%
\pgfsys@transformshift{5.534545in}{0.731572in}%
\pgfsys@useobject{currentmarker}{}%
\end{pgfscope}%
\begin{pgfscope}%
\pgfsys@transformshift{5.398580in}{0.837825in}%
\pgfsys@useobject{currentmarker}{}%
\end{pgfscope}%
\begin{pgfscope}%
\pgfsys@transformshift{5.007082in}{1.056028in}%
\pgfsys@useobject{currentmarker}{}%
\end{pgfscope}%
\begin{pgfscope}%
\pgfsys@transformshift{4.407273in}{1.423415in}%
\pgfsys@useobject{currentmarker}{}%
\end{pgfscope}%
\begin{pgfscope}%
\pgfsys@transformshift{3.671498in}{1.856034in}%
\pgfsys@useobject{currentmarker}{}%
\end{pgfscope}%
\begin{pgfscope}%
\pgfsys@transformshift{2.888502in}{2.311743in}%
\pgfsys@useobject{currentmarker}{}%
\end{pgfscope}%
\begin{pgfscope}%
\pgfsys@transformshift{2.152727in}{2.708166in}%
\pgfsys@useobject{currentmarker}{}%
\end{pgfscope}%
\begin{pgfscope}%
\pgfsys@transformshift{1.552918in}{3.064672in}%
\pgfsys@useobject{currentmarker}{}%
\end{pgfscope}%
\begin{pgfscope}%
\pgfsys@transformshift{1.161420in}{3.318939in}%
\pgfsys@useobject{currentmarker}{}%
\end{pgfscope}%
\begin{pgfscope}%
\pgfsys@transformshift{1.025455in}{3.380000in}%
\pgfsys@useobject{currentmarker}{}%
\end{pgfscope}%
\end{pgfscope}%
\begin{pgfscope}%
\pgfpathrectangle{\pgfqpoint{0.800000in}{0.440000in}}{\pgfqpoint{4.960000in}{3.080000in}}%
\pgfusepath{clip}%
\pgfsetbuttcap%
\pgfsetroundjoin%
\definecolor{currentfill}{rgb}{1.000000,0.000000,0.000000}%
\pgfsetfillcolor{currentfill}%
\pgfsetlinewidth{1.003750pt}%
\definecolor{currentstroke}{rgb}{1.000000,0.000000,0.000000}%
\pgfsetstrokecolor{currentstroke}%
\pgfsetdash{}{0pt}%
\pgfsys@defobject{currentmarker}{\pgfqpoint{-0.020833in}{-0.020833in}}{\pgfqpoint{0.020833in}{0.020833in}}{%
\pgfpathmoveto{\pgfqpoint{0.000000in}{-0.020833in}}%
\pgfpathcurveto{\pgfqpoint{0.005525in}{-0.020833in}}{\pgfqpoint{0.010825in}{-0.018638in}}{\pgfqpoint{0.014731in}{-0.014731in}}%
\pgfpathcurveto{\pgfqpoint{0.018638in}{-0.010825in}}{\pgfqpoint{0.020833in}{-0.005525in}}{\pgfqpoint{0.020833in}{0.000000in}}%
\pgfpathcurveto{\pgfqpoint{0.020833in}{0.005525in}}{\pgfqpoint{0.018638in}{0.010825in}}{\pgfqpoint{0.014731in}{0.014731in}}%
\pgfpathcurveto{\pgfqpoint{0.010825in}{0.018638in}}{\pgfqpoint{0.005525in}{0.020833in}}{\pgfqpoint{0.000000in}{0.020833in}}%
\pgfpathcurveto{\pgfqpoint{-0.005525in}{0.020833in}}{\pgfqpoint{-0.010825in}{0.018638in}}{\pgfqpoint{-0.014731in}{0.014731in}}%
\pgfpathcurveto{\pgfqpoint{-0.018638in}{0.010825in}}{\pgfqpoint{-0.020833in}{0.005525in}}{\pgfqpoint{-0.020833in}{0.000000in}}%
\pgfpathcurveto{\pgfqpoint{-0.020833in}{-0.005525in}}{\pgfqpoint{-0.018638in}{-0.010825in}}{\pgfqpoint{-0.014731in}{-0.014731in}}%
\pgfpathcurveto{\pgfqpoint{-0.010825in}{-0.018638in}}{\pgfqpoint{-0.005525in}{-0.020833in}}{\pgfqpoint{0.000000in}{-0.020833in}}%
\pgfpathclose%
\pgfusepath{stroke,fill}%
}%
\begin{pgfscope}%
\pgfsys@transformshift{5.534545in}{0.655786in}%
\pgfsys@useobject{currentmarker}{}%
\end{pgfscope}%
\begin{pgfscope}%
\pgfsys@transformshift{5.398580in}{0.760654in}%
\pgfsys@useobject{currentmarker}{}%
\end{pgfscope}%
\begin{pgfscope}%
\pgfsys@transformshift{5.007082in}{0.976087in}%
\pgfsys@useobject{currentmarker}{}%
\end{pgfscope}%
\begin{pgfscope}%
\pgfsys@transformshift{4.407273in}{1.339014in}%
\pgfsys@useobject{currentmarker}{}%
\end{pgfscope}%
\begin{pgfscope}%
\pgfsys@transformshift{3.671498in}{1.766662in}%
\pgfsys@useobject{currentmarker}{}%
\end{pgfscope}%
\begin{pgfscope}%
\pgfsys@transformshift{2.888502in}{2.217415in}%
\pgfsys@useobject{currentmarker}{}%
\end{pgfscope}%
\begin{pgfscope}%
\pgfsys@transformshift{2.152727in}{2.609727in}%
\pgfsys@useobject{currentmarker}{}%
\end{pgfscope}%
\begin{pgfscope}%
\pgfsys@transformshift{1.552918in}{2.962676in}%
\pgfsys@useobject{currentmarker}{}%
\end{pgfscope}%
\begin{pgfscope}%
\pgfsys@transformshift{1.161420in}{3.214480in}%
\pgfsys@useobject{currentmarker}{}%
\end{pgfscope}%
\begin{pgfscope}%
\pgfsys@transformshift{1.025455in}{3.274958in}%
\pgfsys@useobject{currentmarker}{}%
\end{pgfscope}%
\end{pgfscope}%
\begin{pgfscope}%
\pgfsetrectcap%
\pgfsetmiterjoin%
\pgfsetlinewidth{0.803000pt}%
\definecolor{currentstroke}{rgb}{0.000000,0.000000,0.000000}%
\pgfsetstrokecolor{currentstroke}%
\pgfsetdash{}{0pt}%
\pgfpathmoveto{\pgfqpoint{0.800000in}{0.440000in}}%
\pgfpathlineto{\pgfqpoint{0.800000in}{3.520000in}}%
\pgfusepath{stroke}%
\end{pgfscope}%
\begin{pgfscope}%
\pgfsetrectcap%
\pgfsetmiterjoin%
\pgfsetlinewidth{0.803000pt}%
\definecolor{currentstroke}{rgb}{0.000000,0.000000,0.000000}%
\pgfsetstrokecolor{currentstroke}%
\pgfsetdash{}{0pt}%
\pgfpathmoveto{\pgfqpoint{5.760000in}{0.440000in}}%
\pgfpathlineto{\pgfqpoint{5.760000in}{3.520000in}}%
\pgfusepath{stroke}%
\end{pgfscope}%
\begin{pgfscope}%
\pgfsetrectcap%
\pgfsetmiterjoin%
\pgfsetlinewidth{0.803000pt}%
\definecolor{currentstroke}{rgb}{0.000000,0.000000,0.000000}%
\pgfsetstrokecolor{currentstroke}%
\pgfsetdash{}{0pt}%
\pgfpathmoveto{\pgfqpoint{0.800000in}{0.440000in}}%
\pgfpathlineto{\pgfqpoint{5.760000in}{0.440000in}}%
\pgfusepath{stroke}%
\end{pgfscope}%
\begin{pgfscope}%
\pgfsetrectcap%
\pgfsetmiterjoin%
\pgfsetlinewidth{0.803000pt}%
\definecolor{currentstroke}{rgb}{0.000000,0.000000,0.000000}%
\pgfsetstrokecolor{currentstroke}%
\pgfsetdash{}{0pt}%
\pgfpathmoveto{\pgfqpoint{0.800000in}{3.520000in}}%
\pgfpathlineto{\pgfqpoint{5.760000in}{3.520000in}}%
\pgfusepath{stroke}%
\end{pgfscope}%
\begin{pgfscope}%
\pgfsetbuttcap%
\pgfsetmiterjoin%
\definecolor{currentfill}{rgb}{1.000000,1.000000,1.000000}%
\pgfsetfillcolor{currentfill}%
\pgfsetfillopacity{0.800000}%
\pgfsetlinewidth{1.003750pt}%
\definecolor{currentstroke}{rgb}{0.800000,0.800000,0.800000}%
\pgfsetstrokecolor{currentstroke}%
\pgfsetstrokeopacity{0.800000}%
\pgfsetdash{}{0pt}%
\pgfpathmoveto{\pgfqpoint{3.550891in}{3.021543in}}%
\pgfpathlineto{\pgfqpoint{5.662778in}{3.021543in}}%
\pgfpathquadraticcurveto{\pgfqpoint{5.690556in}{3.021543in}}{\pgfqpoint{5.690556in}{3.049321in}}%
\pgfpathlineto{\pgfqpoint{5.690556in}{3.422778in}}%
\pgfpathquadraticcurveto{\pgfqpoint{5.690556in}{3.450556in}}{\pgfqpoint{5.662778in}{3.450556in}}%
\pgfpathlineto{\pgfqpoint{3.550891in}{3.450556in}}%
\pgfpathquadraticcurveto{\pgfqpoint{3.523113in}{3.450556in}}{\pgfqpoint{3.523113in}{3.422778in}}%
\pgfpathlineto{\pgfqpoint{3.523113in}{3.049321in}}%
\pgfpathquadraticcurveto{\pgfqpoint{3.523113in}{3.021543in}}{\pgfqpoint{3.550891in}{3.021543in}}%
\pgfpathclose%
\pgfusepath{stroke,fill}%
\end{pgfscope}%
\begin{pgfscope}%
\pgfsetrectcap%
\pgfsetroundjoin%
\pgfsetlinewidth{1.505625pt}%
\definecolor{currentstroke}{rgb}{0.121569,0.466667,0.705882}%
\pgfsetstrokecolor{currentstroke}%
\pgfsetdash{}{0pt}%
\pgfpathmoveto{\pgfqpoint{3.578669in}{3.346389in}}%
\pgfpathlineto{\pgfqpoint{3.856447in}{3.346389in}}%
\pgfusepath{stroke}%
\end{pgfscope}%
\begin{pgfscope}%
\definecolor{textcolor}{rgb}{0.000000,0.000000,0.000000}%
\pgfsetstrokecolor{textcolor}%
\pgfsetfillcolor{textcolor}%
\pgftext[x=3.967558in,y=3.297778in,left,base]{\color{textcolor}\rmfamily\fontsize{10.000000}{12.000000}\selectfont Approximated linear model}%
\end{pgfscope}%
\begin{pgfscope}%
\pgfsetbuttcap%
\pgfsetroundjoin%
\pgfsetlinewidth{1.505625pt}%
\definecolor{currentstroke}{rgb}{1.000000,0.000000,0.000000}%
\pgfsetstrokecolor{currentstroke}%
\pgfsetdash{}{0pt}%
\pgfpathmoveto{\pgfqpoint{3.648113in}{3.152716in}}%
\pgfpathlineto{\pgfqpoint{3.787002in}{3.152716in}}%
\pgfusepath{stroke}%
\end{pgfscope}%
\begin{pgfscope}%
\pgfsetbuttcap%
\pgfsetroundjoin%
\pgfsetlinewidth{1.505625pt}%
\definecolor{currentstroke}{rgb}{1.000000,0.000000,0.000000}%
\pgfsetstrokecolor{currentstroke}%
\pgfsetdash{}{0pt}%
\pgfpathmoveto{\pgfqpoint{3.717558in}{3.083272in}}%
\pgfpathlineto{\pgfqpoint{3.717558in}{3.222161in}}%
\pgfusepath{stroke}%
\end{pgfscope}%
\begin{pgfscope}%
\pgfsetbuttcap%
\pgfsetroundjoin%
\definecolor{currentfill}{rgb}{1.000000,0.000000,0.000000}%
\pgfsetfillcolor{currentfill}%
\pgfsetlinewidth{1.003750pt}%
\definecolor{currentstroke}{rgb}{1.000000,0.000000,0.000000}%
\pgfsetstrokecolor{currentstroke}%
\pgfsetdash{}{0pt}%
\pgfsys@defobject{currentmarker}{\pgfqpoint{0.000000in}{-0.041667in}}{\pgfqpoint{0.000000in}{0.041667in}}{%
\pgfpathmoveto{\pgfqpoint{0.000000in}{-0.041667in}}%
\pgfpathlineto{\pgfqpoint{0.000000in}{0.041667in}}%
\pgfusepath{stroke,fill}%
}%
\begin{pgfscope}%
\pgfsys@transformshift{3.648113in}{3.152716in}%
\pgfsys@useobject{currentmarker}{}%
\end{pgfscope}%
\end{pgfscope}%
\begin{pgfscope}%
\pgfsetbuttcap%
\pgfsetroundjoin%
\definecolor{currentfill}{rgb}{1.000000,0.000000,0.000000}%
\pgfsetfillcolor{currentfill}%
\pgfsetlinewidth{1.003750pt}%
\definecolor{currentstroke}{rgb}{1.000000,0.000000,0.000000}%
\pgfsetstrokecolor{currentstroke}%
\pgfsetdash{}{0pt}%
\pgfsys@defobject{currentmarker}{\pgfqpoint{0.000000in}{-0.041667in}}{\pgfqpoint{0.000000in}{0.041667in}}{%
\pgfpathmoveto{\pgfqpoint{0.000000in}{-0.041667in}}%
\pgfpathlineto{\pgfqpoint{0.000000in}{0.041667in}}%
\pgfusepath{stroke,fill}%
}%
\begin{pgfscope}%
\pgfsys@transformshift{3.787002in}{3.152716in}%
\pgfsys@useobject{currentmarker}{}%
\end{pgfscope}%
\end{pgfscope}%
\begin{pgfscope}%
\pgfsetbuttcap%
\pgfsetroundjoin%
\definecolor{currentfill}{rgb}{1.000000,0.000000,0.000000}%
\pgfsetfillcolor{currentfill}%
\pgfsetlinewidth{1.003750pt}%
\definecolor{currentstroke}{rgb}{1.000000,0.000000,0.000000}%
\pgfsetstrokecolor{currentstroke}%
\pgfsetdash{}{0pt}%
\pgfsys@defobject{currentmarker}{\pgfqpoint{-0.041667in}{-0.000000in}}{\pgfqpoint{0.041667in}{0.000000in}}{%
\pgfpathmoveto{\pgfqpoint{0.041667in}{-0.000000in}}%
\pgfpathlineto{\pgfqpoint{-0.041667in}{0.000000in}}%
\pgfusepath{stroke,fill}%
}%
\begin{pgfscope}%
\pgfsys@transformshift{3.717558in}{3.083272in}%
\pgfsys@useobject{currentmarker}{}%
\end{pgfscope}%
\end{pgfscope}%
\begin{pgfscope}%
\pgfsetbuttcap%
\pgfsetroundjoin%
\definecolor{currentfill}{rgb}{1.000000,0.000000,0.000000}%
\pgfsetfillcolor{currentfill}%
\pgfsetlinewidth{1.003750pt}%
\definecolor{currentstroke}{rgb}{1.000000,0.000000,0.000000}%
\pgfsetstrokecolor{currentstroke}%
\pgfsetdash{}{0pt}%
\pgfsys@defobject{currentmarker}{\pgfqpoint{-0.041667in}{-0.000000in}}{\pgfqpoint{0.041667in}{0.000000in}}{%
\pgfpathmoveto{\pgfqpoint{0.041667in}{-0.000000in}}%
\pgfpathlineto{\pgfqpoint{-0.041667in}{0.000000in}}%
\pgfusepath{stroke,fill}%
}%
\begin{pgfscope}%
\pgfsys@transformshift{3.717558in}{3.222161in}%
\pgfsys@useobject{currentmarker}{}%
\end{pgfscope}%
\end{pgfscope}%
\begin{pgfscope}%
\pgfsetbuttcap%
\pgfsetroundjoin%
\definecolor{currentfill}{rgb}{1.000000,0.000000,0.000000}%
\pgfsetfillcolor{currentfill}%
\pgfsetlinewidth{1.003750pt}%
\definecolor{currentstroke}{rgb}{1.000000,0.000000,0.000000}%
\pgfsetstrokecolor{currentstroke}%
\pgfsetdash{}{0pt}%
\pgfsys@defobject{currentmarker}{\pgfqpoint{-0.020833in}{-0.020833in}}{\pgfqpoint{0.020833in}{0.020833in}}{%
\pgfpathmoveto{\pgfqpoint{0.000000in}{-0.020833in}}%
\pgfpathcurveto{\pgfqpoint{0.005525in}{-0.020833in}}{\pgfqpoint{0.010825in}{-0.018638in}}{\pgfqpoint{0.014731in}{-0.014731in}}%
\pgfpathcurveto{\pgfqpoint{0.018638in}{-0.010825in}}{\pgfqpoint{0.020833in}{-0.005525in}}{\pgfqpoint{0.020833in}{0.000000in}}%
\pgfpathcurveto{\pgfqpoint{0.020833in}{0.005525in}}{\pgfqpoint{0.018638in}{0.010825in}}{\pgfqpoint{0.014731in}{0.014731in}}%
\pgfpathcurveto{\pgfqpoint{0.010825in}{0.018638in}}{\pgfqpoint{0.005525in}{0.020833in}}{\pgfqpoint{0.000000in}{0.020833in}}%
\pgfpathcurveto{\pgfqpoint{-0.005525in}{0.020833in}}{\pgfqpoint{-0.010825in}{0.018638in}}{\pgfqpoint{-0.014731in}{0.014731in}}%
\pgfpathcurveto{\pgfqpoint{-0.018638in}{0.010825in}}{\pgfqpoint{-0.020833in}{0.005525in}}{\pgfqpoint{-0.020833in}{0.000000in}}%
\pgfpathcurveto{\pgfqpoint{-0.020833in}{-0.005525in}}{\pgfqpoint{-0.018638in}{-0.010825in}}{\pgfqpoint{-0.014731in}{-0.014731in}}%
\pgfpathcurveto{\pgfqpoint{-0.010825in}{-0.018638in}}{\pgfqpoint{-0.005525in}{-0.020833in}}{\pgfqpoint{0.000000in}{-0.020833in}}%
\pgfpathclose%
\pgfusepath{stroke,fill}%
}%
\begin{pgfscope}%
\pgfsys@transformshift{3.717558in}{3.152716in}%
\pgfsys@useobject{currentmarker}{}%
\end{pgfscope}%
\end{pgfscope}%
\begin{pgfscope}%
\definecolor{textcolor}{rgb}{0.000000,0.000000,0.000000}%
\pgfsetstrokecolor{textcolor}%
\pgfsetfillcolor{textcolor}%
\pgftext[x=3.967558in,y=3.104105in,left,base]{\color{textcolor}\rmfamily\fontsize{10.000000}{12.000000}\selectfont Measured period times \(\displaystyle T_m\)}%
\end{pgfscope}%
\end{pgfpicture}%
\makeatother%
\endgroup%

	\end{center}
	\caption{Plotting the measured period times against the cosine squared of the angle of the weight. }\label{fig::ellips}
\end{figure}

As we can see in figure \ref{fig::ellips} a clear linear behaviour is given.
The calculated values of an fitted linear model, are always inside the error bar of both axis.
Implying that the moments of inertia of a body rotated in a plane indeed behave elliptical.
