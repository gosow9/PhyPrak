\section{Discussion}

Absolute zero is known with pretty good precision.
The exact result is inside our uncertainty interval, which means that our measurements are possible.
In the case of the temperature of liquid nitrogen, we know that the boiling point at room pressure is around \SI{-195.8}{\degreeCelsius}.
We do not know the exact temperature of the liquid we measured, but it has to be below the boiling point.
The nitrogen is exposed to room temperature, which is much warmer, so it will not be a lot below said boiling point.
As our range covers temperatures right below the boiling point, this measurement also seems possible.

The results we have got with our experiment have pretty big uncertainties.
So if we would want to redo the experiment, we would have to improve on that.
We see the biggest problem in the sensor used.
The device is meant to operate at voltages from \SI{0}{\milli\volt} upwards.
When calibrating, we had to measure voltages below that, which obviously is not optimal.
So we would try to find a device more suited to its purpose.