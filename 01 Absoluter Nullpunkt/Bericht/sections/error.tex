\section{Error Calculation}
First we have to take account of the possible error of the pressure sensor. From the experiment manual we get that there is an absolute error of $\pm\SI{0.075}{\milli\volt}$ Due to different temperatures when the sensor was originally calibrated and the time we do the experiment, there is also an error of $\pm\SI{0.08}{\milli\volt}$, also according to the manual. In addition to that, the slope of the curve can vary by $\pm\SI{0.1}{\percent}$. We can now calculate our error for voltage measurments $\Delta U = 0.075 + 0.08 + 0.001 \times U$. With our measurments we get $U_L = 132.21 \pm \SI{0.29}{\milli\volt}$ and $U_t = -3.72 \pm \SI{0.16}{\milli\volt}$

We then need to know the errors of $p_L$ and $p_t$. From the manual we know that $p_t = 10 \pm \SI{10}{\pascal}$.
The measurment of the barometer is very exact, so we assume that there is no relevant error in that. 

We want to know the error of the pressure measurments we make during the experiment. According to formula (1), we need to know the errors $\Delta p_0$, $\Delta C$, and $\Delta U$. The last one we know already.
First we determine $\Delta C$, the error of the slope of the characteristic curve of the sensor. The formula for that is $C = \frac{p_T - p_l}{U_T - u_l}$. With Gaussian error propagation we can evaluate $\Delta C = 1.7$, so $C = 704.1 \pm 1.7$
Second we want to know $\Delta p_0$. The formula for $p_0$ is $p_0 = p_t - C U_t$ We use Gaussian error propagation again and find that $\Delta p_0 = \SI{109.8}{\pascal}$

Now we have everything to determine the error of a pressure measurment. The formula for that is given as $p = p_0 + C U$. So again with the same method we find the function 
\begin{align}
	\Delta p(U) = \sqrt{\Delta p_0^2 + (U \Delta C)^2 + (C \Delta U)^2}
\end{align}
which we must use for every pressure measurment we make. Our measurments give us $p_e = 70345 \pm \SI{264}{\pascal}$ and $p_k = 95240 \pm \SI{320}{\pascal}$

Now we are finally able to calculate $\Delta t_0$, again by the Gaussian method. Unlike befor, we don't want to take derivatives by hand, so for approximating $\frac{\partial t_0}{\partial x_i}(x_1, ..., x_n)$, we use the differential quotient $\frac{t_0(x_1,..., x_i+h,...,x_n) - t_0(x_1,..., x_i,...,x_n)}{h}$ and use a small $h = 10^{-5}$. By doing that for all parameters $x_i, i = 1,...,n$

(in formula xxx)
and then using the Gaussian method, we get to $t_0 = -267.67 \pm \SI{5.02}{\degreeCelsius}$.