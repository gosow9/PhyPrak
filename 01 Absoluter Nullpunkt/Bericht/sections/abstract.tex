\begin{abstract}
Classical thermodynamics states, that there is an absoulte zero temperature.
Many experiments in the past show, that it is at $-273.150 \pm \SI{0.005}{\degreeCelsius}$.
In this experiment, we want to verify said temperature.
In order to do this, we measure the pressure of a constant amount and volume of gas at different temperatures.
This lets us then calculate absolute zero.
With the precision we were able to produce, we got absolute zero at $-269.6 \pm \SI{5.1}{\degreeCelsius}$, which matches the literature value.
\end{abstract}