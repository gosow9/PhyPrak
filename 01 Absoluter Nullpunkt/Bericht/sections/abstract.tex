\begin{abstract}
From the universal gas equation we know that for a given volume, filled with a given amount of gas, temperature is linear to pressure. This indicates, that there is an absolute zero for temperature, because otherwise there would have to be negative pressure. In this experiment, (after calibrating our measuring device) we want to find said absolute zero by measuring the pressure of a constant amount of gas in a constant volume at two known temperatures (in our case ice water and the steam of boiling water) and then calculation absolute zero from that. Our measurments were off by more than the calculated error suggested, so we had to try and find the reason for that. We suspect that the error could come from the limitation of the sensor used to measure the pressure. 


\end{abstract}