\subsection{Results}

\begin{table}[ht]
	\begin{tabularx}{\textwidth}{XM{4cm}M{4cm}M{4cm}}%{XXXXXX}{M{1.7cm}M{1.5cm}M{1.5cm}M{2.5cm}M{2cm}}
		\toprule 
		\textbf{Colour}& \textbf{Observed $\lambda$ order 1}  [\si{\nano\m}]    & \textbf{$\lambda$-Lit. order 2} [\si{\nano\m}] & \textbf{Deviation}   \\
		\hline
		&&&\\[-5pt]
		Purple  	& $380 \pm 4$	& $382$ &  \\[5pt]
		Blue		& $441 \pm 4$	& $447$ &  \\[5pt]
		Turquoise 	& $457 \pm 4$	& $471$ &  \\[5pt]
		Green   	& $487 \pm 4$ 	& $492$ &  \\[5pt]
		Green		& $494 \pm 4$  	& $501$ &  \\[5pt]
		Yellow		& $574 \pm 4$  	& $587$	&  \\[5pt]
		Red			& $656 \pm 4$  	& $656$	&  \\[5pt]
		Red			& $695 \pm 4$  	& $706$	&  \\[5pt]
		\bottomrule 
	\end{tabularx}
	\caption{Measured period times with corresponding standard deviation (std) and estimated error.}
	\label{tab::period}
\end{table}

This paragraph is where you present the results from your
experiment. This could be in the form of a table (if only very few
parameters where measured) or as a figure. In the text you should
essentially describe what can be seen in the figures, i.e. explain
your axis and how the dependent variable changes as a function of
the independent variable. Discuss trends of the data as well as the
magnitude, origin and nature of the experimental uncertainties. Keep
in mind that the measurement results are always correct! They might
just not be the answer to the question you had in mind.

Depending on the experiment the discussion of the results can
also be a part of the data analysis section.