\section{Conclusion}
The first method is not very good to evaluate exact values. 
But it is suitable to get an rough estimate of the searched spectrum and its wavelengths.

If we want to make this method more exact the distances and positioning as well as the scales have to be more exact. 
In conclusion it is a very easy setup which leads in very rough approximations on the searched values.



For the method used to get the Hg spectrum it would be important to get a better estimation of sources of error.
This means that the calibration of the apparatus have to be done multiple times to see how big the effects are on the results. 
Furthermore it would help to measure the starting and ending point of an light line to see where the middle lays.
This would remove another error source coming from measuring the wrong angle.

On the whole there is potential in doing more precise measurements with this second method. 
It just have to be done very carefully and multiple times to remove as many errors as possible.
