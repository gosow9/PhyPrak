\subsection{Data Analysis}

Since the vernier we use has a scale of one angular minute and we have an error of $0,5$ angular minutes for measurement.
To calculate the wavelength we use equation \ref{eq::wavelength} with $2\varphi = A_1-A_2$. 


Using the Gaussian method (\ref{eq::gauss}) to calculate the error propagated to the angle $\varphi$ and later to the wavelength, we get a rounded error of $\pm0,2$\si{\nano\m} for the first order values. 
For the second order the rounded error is $\pm0,1$\si{\nano\m} for all colours.

These errors are not including the biggest uncertainty we had measuring the spectrum. 
Since the lines have different thicknesses and brightnesses it was very hard to determine if the cross hair is aligned with the line.
Also for the thicker lines we try to set the cross hair in the middle of the line which can vary strongly.

This uncertainty is not included in the analysis since we are not able quantify it in a reasonable matter. Thus the small errors are only representing the uncertainty of the vernier scale.