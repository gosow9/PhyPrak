\subsection{Data Analysis}

We start with a look trough the telescope and decide which lines of the spectrum are the brightest and best visible in the first and second order.
After deciding which lines we take, we aligned the lines with the cross hair of the scope and read the angle of the vernier.
This is done for each colour line.


Since the vernier we used has an scale of one angular minute and we have an error of $0,5$ angular minutes for measurement.
To calculate the wavelength we use equation \ref{eq::wavelength} with $2\varphi = A_1-A_2$. 


Using the Gaussian method to calculate the error propagated to the angle $\varphi$ and later ot the wavelength, we get an rounded error of $\pm0,2$\si{\nano\m} for the first order values. 
For the second order the rounded error is $\pm0,1$\si{\nano\m} for all colours.

This errors are not including the biggest uncertainty we had measuring the spectrum. 
Since the lines had a different thickness and brightness it was very hard to determine if the cross hair is aligned with the line.
Also for the thicker lines we tried to set the cross hair in the middle of the line which can vary strongly.

This uncertainty was not included in the analysis since we could not quantify it in a reasonable matter. Thus the small errors are only presenting the uncertainty of the vernier scale.

 