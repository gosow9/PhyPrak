\section{Questions for Physics Students}

\begin{itemize}
	\item \textbf{What is a grating?}
	An (optical) grating is an object, which has certain properties that enable it to separate different wavelengths of incoming light.
	In order to achieve that, there have to be periodic optical constants.
	The most simple grating probably is just some transparent material with wires regularly spread across the material.
	Usually, the smaller the distance between those wires, the better the grating.
	
	\item \textbf{How does a Hg lamp work?}
	The tube is filled with liquid mercury and argon gas. 
	The argon will ionize when there is enough voltage.
	The heat of the ionized argon will further vaporize and then ionize the mercury.
	Then the lamp will radiate.
	
	
	\item \textbf{Can a brightness maximum be assigned to a
		wavelength?}
	According to Wiens Law, there is a wavelength with maximum intensity emitted by a star (around 289.9 nm).
	
	\item \textbf{How do you derive Braggs Law from Equation \ref{eq::wavelength}?}
	As Braggs Law describes a similar effect to equation \ref{eq::wavelength}, but for reflection and not transmission, the wavelength for a given maximum is divided by two, which leads us to $p\lambda = 2 g \sin(\varphi)$.
	
	\item \textbf{Is a wavelength measurement using a spectrometer more accurate than a direct observation? Why?}
	Yes, in praxis the spectrometer lets us measure much more precise than direct observation. 
	First, we can determine the angle much better.
	Second, we can adjust the width of the incoming light, which leads to a more precise measurement of the spectral line.
	
	\item \textbf{What is the function of the collimator in the
		spectrometer setup?}
	The collimator aligns the directions of the incoming waves, which leads to more parallel rays.
\end{itemize}