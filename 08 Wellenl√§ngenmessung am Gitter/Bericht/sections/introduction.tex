\section{Introduction}

Determining the wavelength of light can be helpful in various parts of physics.
For example it is very common to analyse gases of astronomic objects by analysing the wavelengths that reach the earth.
This works, because different gases emit different wavelengths.
Huygens principle says that when light is bent at a grating, we can assume it diffracts like a spherical wave at every point.
This leads to constructive or destructive interference, depending on the wavelength.
We can use those so called interference patterns do determine the initial wavelength of the source.
A maximum occurs, if the following formula is fulfilled:
\begin{align}
	\lambda = \frac{g \sin(\varphi)}{p}
	\label{eq::wavelength}
\end{align}
Here, $g$ is the grating constant, $\varphi$ is the angle between the incoming light and the maximum and $p$ is an integer.
We call $p$ the order number.
For every order, the refraction pattern repeats itself, because for every wavelength there is one line for every order.
