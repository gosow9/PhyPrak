\section{Discussion}


\paragraph{He spectrum}
In this part of the experiment the scales used are relative big, as the corresponding errors that lead from it.
If we compare the measured values with the literature we see a correlation. 
But only about half of the literature values are inside of the confidence interval.
This shows us that the method used is not very good to evaluate very exact values.
It can be a good starting point to see in which area our searched can lay nonetheless.

 

\paragraph{Hg spectrum}

The method used in this part of the experiment is more exact.
Having an uncertainty from the scale of the measuring device of $\pm0,1$\si{\nano\m} and $\pm0,2$\si{\nano\m} is very good.
But the values calculated from the first and second order did not match once. 
So we got an error in our measurement which is not from reading the scale of our device.
The biggest difference of our values we got $\pm 8$\si{\nano\m} difference between the first and second order. 
This is way bigger than the expected error which shows us that we have an big error somewhere.
Even with this big difference in expected error and the difference we observed, we still have less than 2\% difference between the first and second order measurements for the wavelength.

If we compare the values with the literature we see that there is a spectrum line in between these two values.
This shows us that our measurements are not very accurate but still in the right magnitude and not way off.

Thus there is an error source which is not included in our analysis part.



