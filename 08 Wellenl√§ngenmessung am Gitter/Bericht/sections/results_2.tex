\subsection{Results}
With a grating of 15000 lines per inch, we measure nine colours of the spectrum of the mercury lamp.
We observe the first and second order of the spectrum. 
The wavelengths calculated out of our measurements are shown in the table \ref{tab::Hg}.
For comparison the we take the values given in the NIST atomic spectra database lines data \cite{nist} and put them next to our measurements.
As in the first part, we look at the relative intensities to determine which literature values we have to look at.

\begin{table}[ht]
	\begin{tabularx}{\textwidth}{XM{4cm}M{4cm}M{4cm}}%{XXXXXX}{M{1.7cm}M{1.5cm}M{1.5cm}M{2.5cm}M{2cm}}
		\toprule 
		\textbf{Colour}& \textbf{$\lambda$ order 1} \qquad\qquad [\si{\nano\m}]  &  \textbf{$\lambda$ order 2}\qquad\qquad [\si{\nano\m}]  & \textbf{$\lambda$-Literatur}\qquad\qquad [\si{\nano\m}]  \\
		\hline
		&&&\\[-5pt]
		Violet		& 411 $\pm 0,2$  & 404 $\pm 0,1$ & 405	\\[5pt]
		Blue		& 443 $\pm 0,2$  & 435 $\pm 0,1$ & 440	\\[5pt]
		Turquoise	& 496 $\pm 0,2$  & 491 $\pm 0,1$ & 496	\\[5pt]
		Turquoise	& 500 $\pm 0,2$  & 495 $\pm 0,1$ & 498	\\[5pt]
		Green		& 552 $\pm 0,2$  & 545 $\pm 0,1$ & 542  \\[5pt]
		Yellow		& 582 $\pm 0,2$  & 576 $\pm 0,1$ & 579	\\[5pt]
		Yellow		& 584 $\pm 0,2$  & 578 $\pm 0,1$ & 585	\\[5pt]
		Red			& 628 $\pm 0,2$  & 623 $\pm 0,1$ & 624	\\[5pt]
		Red			& 695 $\pm 0,2$  & 690 $\pm 0,1$ & 690	\\[5pt]
		\bottomrule 
	\end{tabularx}
	\caption{Measured wavelengths for each observed colour in the experiment with their corresponding uncertainties given by the scale of the used vernier. The reference values are from the NIST ASD library \cite{nist}.}
	\label{tab::Hg}
\end{table}