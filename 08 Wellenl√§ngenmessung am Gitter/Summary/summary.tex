\documentclass[12pt,a4paper]{article}

\usepackage{a4wide}
\usepackage{fancyhdr}
\usepackage{graphicx}
\usepackage{epsfig}
\usepackage{parskip}
\usepackage[ansinew]{inputenc}
\usepackage{amsmath}
\usepackage{amssymb}
\usepackage{bm}
\usepackage[free-standing-units=true]{siunitx} % for consistent handling of SI units
\usepackage[colorlinks=true, pdfstartview=FitV, linkcolor=blue, citecolor=blue, urlcolor=blue]{hyperref} % enable links

\setlength{\parindent}{0pt}

\newcommand{\m}[1]
{\mathrm{#1}}

\title{Experiment 15, Wavelength measurement with grating}
\author{Cedric Renda, Fritz Kurz}
\date{\today }

\begin{document}
	
	\maketitle
	\section{Summary}
	The goal of this experiment is to measure the wavelengths of first order for the helium spectrum and the first and second order for mercury spectrum.
	The formula for the wavelength of order p with $g$ as the grating constant and $\varphi$ the angle at which we are looking at it is given as follows:
	\begin{align*}
		\lambda = \frac{g \sin(\varphi)}{p}
	\end{align*}
	
	The first measurement will be less precise and mostly done by eye. 
	The grating is a distance $a$ away from the scale.
	By looking through the grating, we can see the diffraction pattern on the scale, where we can measure the distance $b$ between the centre and the maxima. 
	This lets us compute the angle $\varphi$. 
	
	The second part will be much more precise as we will be using a grating spectrometer.
	In order to use the device, we first have to set it up properly.
	We first need to focus the telescope to infinity using the crosshairs of the telescope.
	Next we need to align the telescope with the spectrometer correctly. 
	Then we adjust the collimator and the grating.
	Having done all that we are now able to observe the first and second order maxima and the calculate the wavelength using the formula shown above. 

	
	
\end{document}