\documentclass[12pt,a4paper]{article}



\usepackage{a4wide}
\usepackage{fancyhdr}
\usepackage{graphicx}
\usepackage{epsfig}
\usepackage{parskip}
\usepackage[ansinew]{inputenc}
\usepackage{amsmath}
\usepackage{amssymb}
\usepackage{bm}
\usepackage{tikz}
\usepackage{graphicx}
\usepackage{pgf}
\usepackage{pgfplots}
\usepackage{pgfplotstable}
\usepackage{tabularx}
\usepackage{booktabs}
\usepackage{array}
\usepackage{pdfpages}
\newcolumntype{M}[1]{>{\centering\arraybackslash}m{#1}}
\usepackage[free-standing-units=true]{siunitx} % for consistent handling of SI units
\usepackage[colorlinks=true, pdfstartview=FitV, linkcolor=blue, citecolor=blue, urlcolor=blue]{hyperref} % enable links


\setlength{\parindent}{0pt}

\newcommand{\m}[1]
{\mathrm{#1}}

\title{02 Traegheitsmoment}
\author{Cedric Renda, Fritz Kurz}
\date{\today }

\begin{document}
	
\maketitle
\paragraph{Summary}
In this experiment we want to determine the moments of inertia for a dumbbell for different axes of rotation lying on a plane.

\subparagraph{Arranging Momentum}
To do this, we first have to determine the arranging moment of the torsion axis.
We can do that by measuring the oscillation period $T$ of two setups:
\begin{itemize}
	\item $T_1$: Two quadratic beams placed on the torsion axle so that the axle is perpendicular to the principle axis of the two bars.
	\item $T_2$: The same setup as before, but we add a circular disk with mass $M_S$ and radius $R_S$.
\end{itemize}
With $\theta_S = 1/2 M_S R_S^2$ the torsion coefficient $D$ is given by:
\begin{align*}
	D = \frac{4 \pi^2 \theta_S}{T_2^2 - T_1^2}
\end{align*}

\subparagraph{Steiner's Theorem}
Next we want to verify Steiner's theorem.
We use the same setup as in the beginning but shift the two bars step by step. We then compare with the values given by Steiner's theorem.

\subparagraph{Ellipse of inertia}
To show that our body of interest has a combined moment of inertia which resembles a ellipse, we have to measure the inertia in 10 different angles of the body.
If the measured Periods $T$ and $\cos^2 (\phi)$, with the tilt angle $\phi$, is now plotted against each other we expect a straight line.
With this we can show the linear relationship between $T_i ^2$ and $\cos^2(\phi)$ as in 
\begin{align*}
T_i^2 = A(1-B \cos^2 \phi_i)
\end{align*}

	
\end{document}