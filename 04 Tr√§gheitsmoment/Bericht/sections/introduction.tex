\section{Introduction}

When working with any rotational movement, knowing the moment of inertia is crucial.
For simple geometric bodies it is relatively easy to calculate its value.
But also when looking at more complex bodies, we can often get good approximations if we use the values for similar geometric ones.

\paragraph{Steiner's theorem}
Talking about rotation only makes sense when also giving an axis of rotation.
The easiest way to rotate a body is along an axis that passes through its centre of mass. 
But what if we are interested in a different axis?
For any axis we can find a parallel one which is passing through the centre of mass.
Steiner's theorem tells us how to get from the moment of inertia regarding an axis through the centre of mass to any other axis parallel to it. 

\paragraph{Ellipsoid of Inertia}
There is a relatively simple way to describe the moment of inertia of a body geometrically.
When looking at different axes passing through the centre of mass of a given body, we can think about the following graphic representation:
For a given axis, we plot the value of the moment of inertia regarding this axis as its length.
It is possible to show, that this will always result in an ellipsoid \cite{manual}.
If we rotate our body in a plane, we should get a two-dimensional ellipsoid, i.e. an ellipse. 
This is what we verify in the second part of this experiment.
