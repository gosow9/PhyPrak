\subsection{Questions for Physics Students}
\begin{enumerate}
	\item \textbf{How is the moment of inertia defined?}
	
	The moment of inertia $I$ of a rigid body is specified by the torque needed to accelerate it to a certain angular acceleration. This is expressed by the formula
	\begin{align}
		I = \int_{V} (\vec{r}_\perp)^2 \rho (\vec{r}) dV.
		\label{eq::inertia}
	\end{align}
	Here, $\vec{r}_\perp$ is the part of $\vec{r}$ perpendicular to the axis of rotation and $\rho$ is the mass distribution.
	
	\begin{enumerate}
		\item 
		\textbf{Specify its value for a cylinder when the axis of rotation coincides with the cylinder axis itself.}
		
	 	With $R_2 > R_1$ being the radii, $L$ the length and $m$ the mass of the cylinder and assuming a constant mass distribution, formula (\ref{eq::inertia}) results in
		\begin{align}
			I_{cyl}{}
			&=\int_{0}^{2 \pi} \int_{R_1}^{R_2} \int_{0}^{L} r^3 \frac{m}{\pi L ({R_2}^2 - {R_1}^2)} dl dr d\theta \\
			&= \frac{m}{\pi L ({R_2}^2 - {R_1}^2)} 2 \pi L \left[ \frac{r^4}{4}\right]_{R_1}^{R_2} \\
			&= \frac{m (R_2^4-R_1^4)}{2 ({R_2}^2 - {R_1}^2)} \\
			&= \frac{1}{2}m({R_2}^2 + {R_1}^2).
			\label{proof::inertia}
		\end{align}

	 	
 		\item
 		\textbf{What happens to the value of the torque when the cylinder is not infinitely thin, but has a finite radius?}
 		
 		We know that torque $M$ is given as $M = \dot{L} = I \frac{d^2 \phi}{d t^2}$ with $L$ being the angular momentum.
 		
 		Inserting our result from (\ref{proof::inertia}), this results in
 		\begin{align*}
 			M = \frac{1}{2} m (R_2^2 + R_1^2) \frac{d^2 \phi}{d t^2}.
 		\end{align*}
 		If we let $R_1 \rightarrow R_2$, we get
 		\begin{align*}
 			M = m R_2^2 \frac{d^2 \phi}{d t^2}.
 		\end{align*}
 		
	\end{enumerate}
	\item
	\textbf{The laws of rotation and translation are often simplified by change of variable.}
	\begin{enumerate}
		\item 
		\textbf{For the equation of the motion of a rigid body [$M = \theta \ddot{\varphi}$] give the equivalent translational equation.}
		
		When changing between rotation and translation, we can use the following relationships:
		\begin{itemize}
			\label{rottrans}
			\item $M \leftrightarrow F$
			\item $\theta \leftrightarrow m$
			\item $\varphi \leftrightarrow x$
		\end{itemize}
		Here, $F$ is force, $m$ is mass and $x$ is position.
		This lets us translate the given equation to $F = m \ddot{x}$, which we recognize as Newton's second law.
		\item
		\textbf{Which equation can be found if one rewrites the equation of motion of a spring $K = -fx$ as the rotational equivalent?}
		
		The relationships from (\ref{rottrans}) let us transform the translational spring equation to the rotational equivalent:
		\begin{align*}
			K = -fx \Longrightarrow M = -f \varphi
		\end{align*}
		
	\end{enumerate}
	\item
	\textbf{What are the main axes of an Inertia Ellipsoid?}

	\begin{enumerate}
		\item 
		\textbf{How are the axis parts connected to the moment of Inertia?}
		
		Where the main axes of an Inertia Ellipsoid intersect with the body, the components of the moment of inertia reach their maxima.
		
		\item
		\textbf{How many components has such an ellipsoid at most?}
		
		As the formula for an ellipsoid has three components, the maximal components of such an ellipsoid is also three.
		For example, this is the case for any cuboids with distinct side lengths.
		
		\item
		\textbf{Enter the number of components in the case of a sphere and cube.}
		
		For the sphere, 
		
		\item
		\textbf{In which points the major axes intersect the body?}
	\end{enumerate}
	
\end{enumerate}