\subsection{Introduction}

The counter tube we are using for this experiment has to be operated at a specific voltage.
If the voltage is too low, either no or not all particles are detected.
If on the other hand the voltage is too high, self-discharging can occur, which leads to too many counts.
The range, at which the device operates as we want is called the Geiger plateau.
So first we will have to find this plateau, so that we can do further measurements.

The activity, the number of decays in a given time, of a radioactive source is measured in Becquerel (\SI{}{\becquerel}).
One Becquerel stands for one decay per second. 
In order to use our source properly, we need to know its activity.
Because even without a radioactive source there is some radiation in our atmosphere, so we have to take account to that.
As our device only measures at a limited place, but the source decays in all directions, we have to factor that in to know the real activity of the source.
The fraction of the covered directions is called the acceptance $\varepsilon$ of the detector.
