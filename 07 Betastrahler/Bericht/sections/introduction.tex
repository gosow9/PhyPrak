\subsection{Introduction}
Decays and radiation play a big part in physics.
Radiation happens at every scales of our universe, from particle physics to astrophysics it has an important role.
For us humans radiation can be very dangerous, as multiple disastrous examples in our history teach us.
We also use decays to calculate the age of a given substance.
If a particle like a neutron or a positron decays there are some other particles emitted.
This is what we call radiation.

There are three types of decay in the nucleus.
We call them $\alpha$-, $\beta$- and $\gamma$-radiation.
They differ from each other in the process and which particles are emitted.
In this experiment we will look at $\beta$-decays.

There are two different types of $\beta$-decays.
In a $\beta^+$-decay, a proton becomes a neutron while emitting a positron.
On the other hand, in a $\beta^-$-decay, a neutron becomes a proton and emits an electron.
The second type is the one we will be looking at here.

