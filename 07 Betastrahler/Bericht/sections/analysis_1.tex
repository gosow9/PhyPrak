\subsection{Data analysis}
\subsubsection{Counter Tube Characteristics}
As shown in Fig. \ref{fig::plateau}, the Geiger plateau of the characteristic curve starts at around \SI{500}{\volt} and goes up to our last measurement point at \SI{980}{\volt}. 
This gives us a value of $V_{Ideal} = \SI{740}{\volt}$ as middle and by that operating voltage of our detector.
To know the quality of the detector, we look at its voltage dependency, so the slope of the plateau at $V_{Ideal}$.
We fit a linear approximation with the method of least squares to the measured points, resulting the blue line in Fig. \ref{fig::plateau}, and a slope of $9.5 \cdot 10^{-2} \si{\volt^{-1}}$.
So for every variation of the voltage by \SI{100}{\volt}, the number of counts differs by $9.5$.
As for high number counts this is way smaller than the uncertainty we get from the Poisson-distribution, we can ignore that additional error in further calculations.
In general, the smaller the gradient of the plateau, the better the detector.

\subsubsection{Activity of the Source}
To calculate the effective number of particles $N_{eff}$ detected by the counter tube, we have to subtract $N_{BG}$ from $N$.
The error propagation method we will use most in this report is the Gaussian one. 
Therefore we introduce it here in a general form and reference to it later:
For a function $R(A, B, \dots)$, where $A \pm \Delta A, B \pm \Delta B, \dots$ are measured values, it is as follows.
\begin{align}
	\Delta R = \sqrt{\left(\frac{\partial R}{\partial A} \Delta A\right)^2 + \left(\frac{\partial R}{\partial B} \Delta B\right)^2 + \dots}
	\label{eq::gauss}
\end{align}

Using this method gives us an effective decay of $N_{eff} = N - N_{BG} = 7403 \pm 87$ decays in $\SI{90}{\second}$.

As we measured the diameter but not the radius of the aperture, we have to divide the value in two.
Using Gauss again, we get a value of $r = 9.025 \pm \SI{0.0125}{\milli\meter}$.
To determine the acceptance $\varepsilon$ of the detector, we need to know the half angle $\theta$.
Given our measurements, and using basic trigonometry, $\theta$ is given as $\theta (r, d) = \arctan(r/d)$.
Once again we use Gauss to determine the error propagation, leaving us with $\theta = 5.11 \pm \SI{0.03}{\deg}$.
Now, equation (\ref{eq::acceptance}) finally gives us $\varepsilon = 1.98 \pm 0.02 \cdot 10^{-3}$.
The total activity of the source is now given as $A_{tot} = \frac{N_{eff}}{t \cdot \varepsilon} = 41454 \pm \SI{632}{\becquerel}$, using Gauss.

As the source is surrounded by a thin layer of steel which will prevent some radiation from getting to the source.
For $^{90}$Sr decay, that is around 55\%, for $^{90}$Y decay, that is around 10\%.
Using the fact that in this case around half of the decays are $^{90}$Sr and the other half $^{90}$Y, we can calculate the activity of the source without the steel layer as ${\tilde{A}_{tot}} = \frac{A_{tot} (45 + 90)}{2*(100)} = 69090 \pm \SI{1011}{\becquerel}$.

\subsection{Activity and Dose Calculations}
Here we want to calculate the lifetime $\tau$ of Sr and Y. 
If $n_0$ is the amount of atoms at the start, there are $n_0/e$ atoms are left after $\tau$.
We can calculate $\tau$ using the half-live of Sr and Y.
The half-live of Sr is $h_{Sr} = 28.8$y, while the half-live of Y is $h_Y = 64$h \cite{manual}.
The number of atoms remaining is given by $n(t) = n_0 e^{-t/\tau}$.
The formula for getting the lifetime is then $\tau = h/\log(2)$, which leaves us with $\tau_{Sr} = 42$y and $\tau_Y = 92$h.

As the activity of the source is also described as $A(t) =  \tilde{A}_{tot} e^{-t/\tau}$, we can calculate, when the activity is down to $A = 10000 \si{\becquerel}$, which leads us to $t = -\tau \log(A/\tilde{A}_{tot}) = 80$y.
So in $80$ years, the activity of our source will be around $\SI{10000}{\becquerel}$.

The average radiation people in Switzerland are exposed to in a year is $S = \SI{5.5}{\milli\sievert}$. Knowing the activity of the source we can calculate how long it would take to reach that dose if someone incorporated it.
We can describe the dose with $D = t A (d_{Sr} + d{Y})$, where $d_{Sr} = 2.80 \cdot 10^{-8} \si{\sievert \becquerel^{-1}}$ and $d_Y = 2.70 \cdot 10^{-9} \si{\sievert \becquerel^{-1}}$ are the dose coefficients for Sr and Y \cite{manual}.
To know the time $t_{Dose}$ it would take, we can rearrange this formula to
\begin{align*}
	t_{Dose} = \frac{D}{A(d_{Sr} + d_Y)} = \SI{2.6}{\second}.
\end{align*}
After that time, the average annual dose would be reached, so it is better to not incorporate the source!








