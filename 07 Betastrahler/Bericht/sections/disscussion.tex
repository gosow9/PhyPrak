\section{Conclusion and Discussion}

In the first part of the experiments we looked at the detector an how well its performance is.
Having an gradient which is negligible small on the Geiger plateau is a good sign that we can trust the values measured by the detector.


The found activity of around 69kBq with an error of 1.5\% the value lays in between the reference values given in the manual.
Also the calculated time needed to reduce the activity of the source to 10kBq is 80 years with a half-life time of 28,8 years in Strontium, this results seem plausible.
We can also see, that even with a small source like this, it would only need 2,6 seconds to get the same radiation as annually reached in Switzerland.
This shows how bad it can be if somebody swallows radio active materials like this.

In the second part we looked at the energies.
If we compare our literature value for Yttrium 2.282\si{\mega\electronvolt} with our results, we can see that the reference value lays in the confidence interval of all the results.
While the expected value of method 1 and 3 are a little further away method 2 got close for the expected value. 


For all methods we had to use some logarithmic tables to read values of it.
This lead to some big errors on the results.
Method 1 is the worst performing method in our experiments.
It produced a very big uncertainty value, bigger than in the other two methods while having an expected value which is not better than method 2 or 3.


There a many ways to improve the results if the experiment was redone.
On way would be to measure for longer time periods, which would result in a smaller error on the counted activity.
Measuring the same part of the experiment would increase the confidence in the numbers and improve the expected values.

Especially in the absorption curve, it would be a good thing to have more data in the interesting parts.
With more different strengths of aluminium we could improve the $x_{max}$ value.
Improving overall the energy measurement results.

The last part which could drastically improve the error, would be to use the function or dataset used to generate the logarithmic reference plots used in the manual.
This way a lot of values can be calculated instead of read out of an plot wich can lead to high errors.

