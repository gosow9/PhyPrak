\subsection{Questions for physics students}

\begin{itemize}
	\item \textbf{Describe the various ranges of operation of a gas
		detector. Which voltage range should be chosen
		to operate the detector as ionisation chamber,
		proportional chamber or Geiger-Mueller counter tube?}
	
	If the voltage of the detector is very low, most of the ions do not reach the electrodes of the tube.
	When we want to use the detector as an ionisation chamber, we have to increase the voltage, so that the ions ar able to reach the electrodes, but not so high that the ion current can be detected. 
	
	As we increase the voltage further, the electric field gets bigger.
	Then, the gas atoms in the detector get ionized by the incoming electrons, which then inoize further gas atoms and so on.
	This leads to the so-called gas amplification \cite{manual}.
	In the right range, the charge is then proportional to the amount of incoming ions in the beginning.
	This is called the proportionality range.
	
	If we want to use it as a Geiger-Mueller counter tube, as we do in this experiment, we have to operate it at said Geiger-Plateau.
	This is described in the report above. 
	
	\item \textbf{What determines the energy resolution of the
		detector? A particularly good energy resolution
		can be achieved with semiconductor detectors.
		Why? How do scintillation detector work?}
	
	A detector with a higher energy resolution detects decays with lower energy.
	A detector with a low energy resolution does not detect all decays, while a better one detects more.
	While in a gas detector ions are measured, the semiconductor detector measures the amount of electrons.
	The energy required to produce electrons is far lower than for ions, so the number of undetected decays is much smaller in a semiconductor detector than in a gas detector.
	
	A scintillating material reacts to passing radiation by emitting light.
	This light gets amplified by a photomultiplier, so that it can be measured and analysed.
	
	\item \textbf{What other important aspects of detectors can
		you think of?}
	The detector should not produce any radiation itself, it should not react to outer influences (for example light in case of a scintillator).
	
	\item \textbf{How can spatial resolution be achieved with
		counting detectors?}
	We could install multiple detectors spread across the area to get spatial resolution.
	
	\item \textbf{Beta particles are either negatively charged
		(=electrons) or positively charged (=positrons).
		How are positrons used nowadays to investigate
		and examine materials?}
	
	\item \textbf{Are neutrons also subject to $\beta$-decays?}

	\item \textbf{What is the difference between $\gamma$-rays and X-rays?}
	

\end{itemize}
