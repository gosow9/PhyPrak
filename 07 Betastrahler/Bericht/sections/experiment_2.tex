\subsection{Experiment}

The goal of this task is to find the maximum energy $E_{max}$ of the beta particles leafing the source.
We know that higher energy particles can penetrate thicker layers of material.
With this knowledge, we measure the amount of particles which successfully penetrate our testing material to get an absorption curve of the material.
From this curve we use three methods to get our $E_{max}$.


To get this absorption curve, we lay aluminium plates of twelve different sizes in $x$-axis in the steel slider of our testing apparatus and measure the count rate $N$ over a time of 90 seconds.
The aluminium plates used are in a range from 0mm to 4mm in thickness.
After placing each plate in the measuring apparatus we let the automatic counter count all electrons reaching the sensor for 90 seconds. 
The counted electrons then get logarithmically plotted against the thickness of the plates.


From the plotted absorption curve we then can make our calculations of $E_{max}$.
\paragraph{Method 1} we look at the absorption curve and find the maximum thickness $x_{max}$ of the aluminium, where the radiation is starting to fail to penetrate the aluminium.
This thickness is then used to read out the maximal energy from a given figure \cite{manual} witch shows the range of the electrons in aluminium as a function of the distance-density product $\rho x$.
On top of this value we add the energy lost $E_{loss}$ which the electrons lose to penetrate the thin stainless steel foil placed over the source.
These two values added describe the initial energy of source  $E_{maxsource}$.
Like above the we get this value from a plot given in the manual \cite{manual}, which shows the energy loss of electrons for 0.1\si{\mm} stainless steel, as a function of electron energy.

\paragraph{Method 2} Here we use all the measured counts with a activity above our background noise $N_{BG}$ ($x<x_{max}$), to make a fit on a giving model from the manual \cite{manual}.
The model used is an exponential which when plotted in a logarithmic scale gives us an straight line. 
Using the slope $\mu$ of the straight line we can calculate the $E_{max}$ from a given equation.

\paragraph{Method 3} in the last method we first correct our background noise.
Then we plot the corrected counts $N_{eff}$ against $E_{max}-E(x)$ in a double logarithmic scale.
From this plot we calculate the slope $n$ to draw a new plot were we plot the $n$-th root of the corrected counts $N_{eff}$ against the energy $E$.
From this we get an extrapolated intersecting line were the counts are zero giving us a new 
$E_{max}$ value.






  