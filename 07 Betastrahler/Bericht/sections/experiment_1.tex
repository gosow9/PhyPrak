\subsection{Experiment}


\subsubsection{Counter Tube Characteristics}
To perform latter measurements, we first need to know at which voltage we want to operate the counter tube.
In order to do that, we need to find the so called Geiger plateau.
The Geiger plateau is the plateau section of the curve where the number of counts is plotted against the voltage.
In order to do that, we start with a voltage of 

This section is where you describe what you actually did in the lab.
You explain what data you measured and how. Here, you might want to
provide a sketch of the experimental setup, if this is applicable.
This sketch could be taken from the experiment manual. In particular,
you want to define all parameters that you use or measure in this
experiment.

In general the information you provide in this section should be
sufficient for the reader to reproduce your results.