\begin{abstract}
In this experiment we examine a the $\beta^-$-decay of an strontium source. 
When a $\beta^-$-decay happens an electron is emitted with a given energy.
Measuring the activity with an detector close to the source and the geometric properties of the setup, allowed us to determine the activity of the source itself.
Further we measured the absorption curve giving us the ability to calculate the maximal initial energy of the electrons emitted from the decay of Yttrium in three different ways.
From two methods we got the very close results to each other while being a bit of to the literature value. 
The third method compares reasonably good with the literature. 

\end{abstract}