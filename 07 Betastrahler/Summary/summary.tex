\documentclass[12pt,a4paper]{article}


\usepackage{a4wide}
\usepackage{fancyhdr}
\usepackage{graphicx}
\usepackage{epsfig}
\usepackage{parskip}
\usepackage[ansinew]{inputenc}
\usepackage{amsmath}
\usepackage{amssymb}
\usepackage{bm}
\usepackage{tikz}
\usepackage{graphicx}
\usepackage{pgf}
\usepackage{pgfplots}
\usepackage{pgfplotstable}
\usepackage{tabularx}
\usepackage{booktabs}
\usepackage{array}
\usepackage{pdfpages}
\usepackage{printlen}
\newcolumntype{M}[1]{>{\centering\arraybackslash}m{#1}}
\usepackage[free-standing-units=true]{siunitx} % for consistent handling of SI units
\usepackage[colorlinks=true, pdfstartview=FitV, linkcolor=blue, citecolor=blue, urlcolor=blue]{hyperref} % enable links


\setlength{\parindent}{0pt}

\newcommand{\m}[1]
{\mathrm{#1}}



\title{Experiment 41 - Beta Decay}
\author{Cedric Renda, Fritz Kurz}
\date{\today }

\begin{document}
	\maketitle
	
	\section{Summary}
	In this experiment we look at the $\beta$-decay of $^{90}Sr$ to $^{90}Y$.
	This will further decay to $^{90}Zr$.
	We use a standard couting tube as a  detector to count the number of electrons and positrons that are emitted in this process per second.
	
	Even without any radioactive material, the detector will count some electrons/positrons.
	To eliminate this background noise, we will perform a measurement without the radioactive source; if we subtract this value from the one measured with the source, we get the effective decay.
	
	In our case, we have to consider that it is a double decay with $^{90}Y$, so on average, half of the $\beta $-particles come from $^{90}Y$-decay.
	For safety reasons, there is a steel foil around the radioactive material which absorbs around $55 \%$ of the radiation.
	As those are averages and we only can make a finite amount of measurements, this is a possible source of error of the experiment.

	In a second part we want to determine $E_{max}$, the maximum energy of the beta particles in different ways.
	


	
	
	
	
	
	
	
\end{document}