\documentclass[12pt,a4paper]{article}

\usepackage{a4wide}
\usepackage{fancyhdr}
\usepackage{graphicx}
\usepackage{epsfig}
\usepackage{parskip}
\usepackage[ansinew]{inputenc}
\usepackage{amsmath}
\usepackage{amssymb}
\usepackage{bm}
\usepackage{tabularx}
\usepackage{booktabs}
\usepackage{array}
\usepackage{pdfpages}
\newcolumntype{M}[1]{>{\centering\arraybackslash}m{#1}}
\usepackage[free-standing-units=true]{siunitx} % for consistent handling of SI units
\usepackage[colorlinks=true, pdfstartview=FitV, linkcolor=blue, citecolor=blue, urlcolor=blue]{hyperref} % enable links

\setlength{\parindent}{0pt}

\newcommand{\m}[1]
{\mathrm{#1}}

\title{Experiment 61, Analogue Electronics}
\author{Cedric Renda, Fritz Kurz}
\date{\today }

\begin{document}
\maketitle
\paragraph{Summary}
In this experiment we want to get to know the op-amp.
An op-amp can be used for various tasks.
For example comparators, to measure voltages, amplifying signals (positive or negative), inverters or filters.

First we have to find the level-control limits $U_{out, min}$ and $U_{out, max}$ of the given op-amp.
We feed a signal to the op-amp and look at the output.

Next, we want to build a comparator and analyse the signal path of $U_{in}$ versus $U_{out}$.

We then want to test the inverting and the non-inverting amplifier, and for the non-inverting we compare the value to the one we calculated in the homework.

Next we want to look at the correlation between the gain and the frequency of the incoming signal. 

If we still have time we can perform one of the exercises given in paragraph 3 of the manual. 

	
	
\end{document}