
\section{Dos and Don'ts}

\begin{itemize}

\item Be honest with yourself and with the reader. Try to find possible loopholes in your conclusion and explicitly mention them.

\item Be aware that \textbf{scientific fraud} is an important topic that we (and the entire ETH) take very seriously. There are many forms of fraud, from copying text without referencing it to forging data. If unsure, ask your assistants about specific issues.

\item Good writing is largely a question of practice and of experience. Why not read some scientific papers to study how professionals write? We are happy to recommend some literature. 

\item A good practice is to begin each paragraph with a `topic sentence' that conveys the main message of the paragraph. As an example, the experimental section might start with ``We performed experiments with a mechanical resonator inside a vacuum chamber.'' From this short sentence, the reader gathers immediately that the paragraph is about the experimental setup.

\item Avoid using passive voice for extended paragraphs. The use of active language can make the text more interesting to read and is by default preferred by many English writers. For instance, instead of writing ``The data points were measured over the course of 405 seconds'', you can write ``We measured data points over the course of 405 seconds'' or simply ``The data acquisition lasted 405 seconds''. Of course, sometimes it may be better to use passive voice in order to describe basic processes, e.g. ``Samples were cleaned for 3 minutes in acetone''. The choice is yours - try out what fits better in specific cases.

\item Write in short sentences. Always put clarity before artistic form. Whenever possible, avoid interrupting your sentences with brackets, formulas, or complex mathematical signs. Your text is much easier to read when you group such additional information at the end of a sentence, in a table, or in the caption of a figure. Remember that your readers might need their full attention for the physics involved (and do not want to decipher complex sentences).

\item Avoid slang and terms that might not be known to the reader. One of the most difficult tasks is to explain something very complicated in simple terms that newspaper readers might understand. If you have to use specialized terms, try to explain them when they first appear.

\item When you use abbreviations like `AFM', make sure you use the full term once. ``Nanoscale surfaces can be characterized with an atomic force microscope (AFM).''

\item As a rule, use ``cannot'' instead of ``can't'',  ``will not'' instead of ``won't'', ``do not'' instead of ``don't'' and so on (the title of this section is an exception).

\item Graphs should not be overloaded with information. Make the essential features stand out. Presenting scientific data is an art!

\item Graphs should be drawn with the help of a software. In any case, graphs have to fulfill all relevant criteria of good scientific practice, such as well-scaled and labeled axes (including units), and the data points must be clearly
visible and contain error bars where applicable.

\item Fitting parameters only need to be provided for actual physical models, not for a ``guide to the eye''. Measured and derived values should be given with error bars (confidence intervals) and an appropriate number of significant digits.

\item Figure captions are an important part of a figure. Ideally, a reader that is familiar with the field should understand your results by merely looking at your figures and reading the captions. Figure captions are an ideal place to give specific numbers that are not absolutely required in the main text (such as `applied voltage' or `laser power').

\item The reports should be written with a text processing software (e.g. Latex).

\end{itemize}

