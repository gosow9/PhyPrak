\section{Discussion}
While the tasks themselves were not that complicated, we noticed, some measurements were difficult to perform.
Specially when adjusting the distances between objects to get a picture as sharp as possible, we noticed that our eyes are not that useful.
Often there was quite a big range, where we saw a sharp image.
That's why we often used an error of $\pm \SI{10}{\milli\meter}$.
Compared to that, almost all other errors are negligible.
The most effective way to improve the experiment would be to find an objective method to measure "sharpness".
Another would be to always opt for the biggest distances possible between objects, as this lets the error shrink.
This obviously is only true if the error margin is considered constant.

Regarding Abbe's imaging theory, our experiment matches the calculated value quite well. 
This is expected, as his theory already takes account of diffraction effects \cite{manual}.