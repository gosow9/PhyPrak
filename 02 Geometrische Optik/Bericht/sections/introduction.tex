\section{Introduction}
Optics is omnipresent in today's technical world.
Since solving the wave equation of light, with respect to all boundary conditions, for every simple problem is unpractical.
The simplified method of geometrical optics offers itself to be used whenever possible.

Because the geometric optics interpretation of light does not consider its wave characteristics it is only suitable in situations were the wave phenomenon is negligible.
This includes situations, were the wavelength of the light is far smaller than the linear dimensions of the aperture it passes through. 
In our experiment we use a setup like this to determine the focal length of three lenses. 
To obtain a focal length $f$ two different approaches can be used.
One is called the Bessel method\cite{manual}
\begin{equation}
f = \frac{d^2 - e^2}{4d}
\label{eq::bessel}
\end{equation}
were $d$ is a fixed distance between the object and the image.
For the displacement $e$, the distance between the demagnified and the magnified image of the object is taken.
The more known method is trough the lens equation\cite{manual} 
\begin{equation}
\frac{1}{f} = \frac{1}{a} + \frac{1}{b}
\label{eq::lens}
\end{equation}

with $a$ as the distance between object and lens with $b$ giving the distance on the other side between lens and image.

In the last part the boundaries of this simplification is clearly visible. 
The goal is to examine if Abbe's imaging theory\cite{manual}
\begin{equation}
d \approx f\frac{\lambda}{g},
\label{eq::diff}
\end{equation}
which simplifies the influence of diffraction effects is useful. 
Light diffraction effects can not be explained without the wave characteristic of light.
Thus the need of an complementary theory.
