\section{Data analysis}

All but one measurements were done on the same measuring rail as seen in figure \ref{fig::rail}, with a included \si{\milli \m}-scale. 
\begin{figure}[h!]
	\centering
	\includegraphics[width=\textwidth]{img/railcut.jpeg}
	\caption{Photo of the measurement rail, light source (left) and the setup from the critical slit width experiment from chapter \ref{chap::slit}.}
	\label{fig::rail}
\end{figure}

Since the later described error of focusing $err_f$ is way bigger, the following smaller sources are included in the focusing error.
In a setup with a \si{\milli \m} measuring rail like in figure\ref{fig::rail}, the best reading accuracy is $0.5 \pm 0.25$\si{\milli \m}. 
If well done the parallax error can be included in this error margin, otherwise the parallax error can be added on top.
Depending on rail length and material a certain temperature error has to be included. 
Since the coefficient of length expansion for most metals used in length measuring devices is in the magnitude of $10^{-5} \dots 10^{-6}$ per Kelvin.
The errors in a measurement like this errors are negligible. 

\paragraph{Lens equation}
Measuring the focal length of the different lenses were done according to the description in chapter \ref{chap::lens}. 
First we had to find a position for our lens to project a sharp image on the screen.
This task itself is very error prone since to determine sharpness from bare eye is not very exact.
This method generated an error of $\pm 5$\si{\milli\m}, were it was hard to decide if the image was still sharp or not.
This error is so big that it already includes all to errors mentioned above.

Since the error is dependent on the distances $a$ and $b$ we are adding the error to the value which leads to an bigger confidence interval.
Which was always $a$.
The error propagation from the measured results to each focal length $f_i$ is calculated like
\[
  \displaystyle	\Delta f_i = \sqrt{\left(\frac{\partial f}{\partial a} \cdot \Delta a \right)^2 +\left(\frac{\partial f}{\partial b} \cdot \Delta b\right)^2 } = \sqrt{\left(\frac{b^2}{(a+b)^2}\cdot \Delta a\right)^2}
\]
with an assumed $\Delta b = 0$. 
Since we made multiple measurements for one focal length the standard error of the mean is
\[
\displaystyle	\Delta \overline{f} =\frac{1}{n} \sqrt{\sum_{i=1}^{n}\Delta f_i^2} = \frac{1}{2}\sqrt{\Delta f_1^2 + \Delta f_2^2 }.
\]

With this we get the following average focal length and standard error for the lens one and two with two measurements.

	\begin{tabularx}{\textwidth}{XM{5cm}M{5cm}}%{XXXXXX}{M{1.7cm}M{1.5cm}M{1.5cm}M{2.5cm}M{2cm}}
		\toprule 
		Lens nr.& mean focal length $\overline{f}$ & error $\Delta \overline{f}$\\
		\hline
		&&\\[-5pt]
		Lens 1	& 58 \si{\milli \m} & $\pm 2,7$ \si{\milli \m}	\\
		Lens 2	& 149 \si{\milli \m} & $\pm 3,2$ \si{\milli \m}	\\
		

		\bottomrule 
	\end{tabularx}

\paragraph{Bessel}
The measurements results used to calculate the focal length via the Bessel method \ref{eq::bessel} $e$ and $d$ were obtained as described in chapter \ref{chap::bessel}.
Calculating an error $\Delta f$ with the Gaussian method
\[
\displaystyle	\Delta f = \sqrt{\left(\frac{\partial f}{\partial e} \cdot \Delta e \right)^2 } = \frac{e}{2d}\cdot \Delta e
\]

we get the following focal lengths and errors for our lenses

\begin{tabularx}{\textwidth}{XM{5cm}M{5cm}}%{XXXXXX}{M{1.7cm}M{1.5cm}M{1.5cm}M{2.5cm}M{2cm}}
	\toprule 
	Lens nr.&focal length $f$ & error $\Delta f$\\
	\hline
	&&\\[-5pt]
	Lens 1	& 55 \si{\milli \m} & $\pm 2,8$ \si{\milli \m}	\\
	Lens 2	& 149 \si{\milli \m} & $\pm 1,4$ \si{\milli \m}	\\	
	\bottomrule 
\end{tabularx}

\paragraph{Diverging lens}
To measure the diverging lens we used lens number two with the bigger focal length and the method described in chapter \ref{chap::div}.








