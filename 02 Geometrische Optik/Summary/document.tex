\documentclass[12pt,a4paper]{article}



\usepackage{a4wide}
\usepackage{fancyhdr}
\usepackage{graphicx}
\usepackage{epsfig}
\usepackage{parskip}
\usepackage[ansinew]{inputenc}
\usepackage{amsmath}
\usepackage{amssymb}
\usepackage{bm}
\usepackage{tikz}
\usepackage{graphicx}
\usepackage{pgf}
\usepackage{pgfplots}
\usepackage{pgfplotstable}
\usepackage[free-standing-units=true]{siunitx} % for consistent handling of SI units
\usepackage[colorlinks=true, pdfstartview=FitV, linkcolor=blue, citecolor=blue, urlcolor=blue]{hyperref} % enable links


\setlength{\parindent}{0pt}

\newcommand{\m}[1]
{\mathrm{#1}}

\title{16 Geometrical Optics, Summary}
\author{Cedric Renda, Fritz Kurz}
\date{\today }

\begin{document}
In this experiment the first goal is to measure the focal length of two converging lenses with two different methods:
One time, using the fine wire mesh, using the object and distances method.
The second time, using the coarse wire mesh, by applying Bessel's method.
\begin{align}
	f = \frac{a b}{a+b} = \frac{d^2 - e^2}{4 d}
\end{align}
$f_{conv} > 0$


Next we want to measure the focal length of a diverging lens, once again by using the object and distances method, by inserting the long-focus converging lens.
\begin{align}
	f_{div} = \frac{f_{conv} f}{f_{conv}-f}
\end{align}
where f is the measured focal length of the system. $f_{div} < 0$

Then we want to determine the grating constant $g$ of two wire nets.
Using the magnification scale $v = b/a = g'/g$ and measuring the image size $g'$, we can determine $g$.

Last we want to verify Abbe's imaging theory, which states:
\begin{align}
	d \approx f \frac{\lambda}{g}
\end{align}
We image the two wire nets we measured previously using the long-focus converging lense.
We put in a slit, which we close until the vertical bars just disappear. 
Then, using the method from before, we measure the slit with using the short-focus lense.
We are now able to compare the two values, thus we know the accuracy of Abbe's theory
\end{document}