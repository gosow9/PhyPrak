\section{Questions for students}

\begin{itemize}
	\item \textbf{ What is a harmonic oscillator, what is its Eigenfrequency?}
	A system which experiences a restoring force proportional to the displacement it experiences.
	Usually this is done by converting one form of energy into another,like kinetic energy into potential energy as it happens with a pendulum.
	The Eigenfrequency is the oscillation frequency of the system once activated.
	
	\item \textbf{How does an electromagnetic brake work?}
	If a conductive material is moving past a magnet a current gets induced. This current is called Eddicurrent and is closed in itself in small loops. 
	The current loops builds up a magnetic field which opposes the inducing field.
	Causing a force in the opposite direction of the moving motion of the conductor.
	Trough this force the conductor experience a de acceleration. 
	The faster the object moves trough the magnetic field to bigger is the breaking force.
	And if the magnetic force from outside is bigger also the breaking force gets bigger.
	Making this break bad to stop very slow movements but good at high velocity movements.
	
	\item \textbf{What is a moment of inertia?}
	The moment of inertia $\Theta$ is the torque $M$ needed for a desired angular acceleration $\ddot{\varphi}$ around itself in a given axis. 
	\[
	\Theta = \frac{M}{\ddot{\varphi}}
	\]
	  
	
	
	
	\item\textbf{ What is the differential equation of a damped
	harmonic oscillator? How does one differentiate
	the three cases of damped harmonic motion?}
The equation of an damped harmonic oscillator is
\[
\frac{\partial^2x}{\partial t^2}+2K\omega_0\frac{\partial x}{\partial t} + \omega_0^2 x =0.
\] With $K$ as the damping ratio and $\omega_0$ the undamped angular frequency.
For the three cases only the damping ratio K is important.

$K>1$ is called over damped. The system decays exponential to steady state.
$K=1$ is called critically damped. Resulting in the system returning to steady state as quickly as possible.
$K<1$ is called under damped. The system oscillates with an exponential decaying rate 


	
	\item \textbf{Which frequency dominates in the stationary
	state of a forced oscillation (driven oscillator)?}
	Until the natural oscillation is decayed an superposition of two different oscillations is present.
	After the decay only the forced oscillation remains.
	
	\item \textbf{How does the phase shift between driving moment
	and forced oscillation behave, if the driving
	frequency is much larger than the Eigenfrequency?
	(This case arises for the scattering
	of X-ray radiation)} 

\textbf{??}


	
	\item \textbf{Resonance curves are characterized by their
	quality (or Q factor). What is the Q factor in
	the case of a forced oscillation?}
The Q factor is defined as 
\[
Q = \frac{E_{stored}}{E_{lost}}.
\]
In a forced oscillation the Q factor is seen in the resonance curve in the height of the peak. 
The bigger Q is the bigger the peak in the curve gets around de resonance frequency. 
It also can be seen if the phase shift is plotted against the frequency. 
The bigger Q is the steeper and quicker the phase shift gets.

	
\end{itemize}