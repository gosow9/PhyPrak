\subsection{Results}

The amplitudes $A$ measured in the first experiment can be seen in figure \ref{fig::log}.
There is an estimated error bar of $\pm 2  \degree$  for all values other than the first one.
The first value was fixed at an amplitude of $A_0=110 \degree$  with an $1 \degree$ scale, resulting in a $\pm 0.5 \degree$ error.
In figure \ref{fig::log} we see an linear behaviour of the measured values if plotted in a logarithmic scale getting close to the expected exponential decay.


\begin{figure} [ht]

	\resizebox {\textwidth} {!}{%% Creator: Matplotlib, PGF backend
%%
%% To include the figure in your LaTeX document, write
%%   \input{<filename>.pgf}
%%
%% Make sure the required packages are loaded in your preamble
%%   \usepackage{pgf}
%%
%% and, on pdftex
%%   \usepackage[utf8]{inputenc}\DeclareUnicodeCharacter{2212}{-}
%%
%% or, on luatex and xetex
%%   \usepackage{unicode-math}
%%
%% Figures using additional raster images can only be included by \input if
%% they are in the same directory as the main LaTeX file. For loading figures
%% from other directories you can use the `import` package
%%   \usepackage{import}
%%
%% and then include the figures with
%%   \import{<path to file>}{<filename>.pgf}
%%
%% Matplotlib used the following preamble
%%
\begingroup%
\makeatletter%
\begin{pgfpicture}%
\pgfpathrectangle{\pgfpointorigin}{\pgfqpoint{6.400000in}{4.000000in}}%
\pgfusepath{use as bounding box, clip}%
\begin{pgfscope}%
\pgfsetbuttcap%
\pgfsetmiterjoin%
\definecolor{currentfill}{rgb}{1.000000,1.000000,1.000000}%
\pgfsetfillcolor{currentfill}%
\pgfsetlinewidth{0.000000pt}%
\definecolor{currentstroke}{rgb}{1.000000,1.000000,1.000000}%
\pgfsetstrokecolor{currentstroke}%
\pgfsetdash{}{0pt}%
\pgfpathmoveto{\pgfqpoint{0.000000in}{0.000000in}}%
\pgfpathlineto{\pgfqpoint{6.400000in}{0.000000in}}%
\pgfpathlineto{\pgfqpoint{6.400000in}{4.000000in}}%
\pgfpathlineto{\pgfqpoint{0.000000in}{4.000000in}}%
\pgfpathclose%
\pgfusepath{fill}%
\end{pgfscope}%
\begin{pgfscope}%
\pgfsetbuttcap%
\pgfsetmiterjoin%
\definecolor{currentfill}{rgb}{1.000000,1.000000,1.000000}%
\pgfsetfillcolor{currentfill}%
\pgfsetlinewidth{0.000000pt}%
\definecolor{currentstroke}{rgb}{0.000000,0.000000,0.000000}%
\pgfsetstrokecolor{currentstroke}%
\pgfsetstrokeopacity{0.000000}%
\pgfsetdash{}{0pt}%
\pgfpathmoveto{\pgfqpoint{0.800000in}{0.440000in}}%
\pgfpathlineto{\pgfqpoint{5.760000in}{0.440000in}}%
\pgfpathlineto{\pgfqpoint{5.760000in}{3.520000in}}%
\pgfpathlineto{\pgfqpoint{0.800000in}{3.520000in}}%
\pgfpathclose%
\pgfusepath{fill}%
\end{pgfscope}%
\begin{pgfscope}%
\pgfpathrectangle{\pgfqpoint{0.800000in}{0.440000in}}{\pgfqpoint{4.960000in}{3.080000in}}%
\pgfusepath{clip}%
\pgfsetbuttcap%
\pgfsetroundjoin%
\definecolor{currentfill}{rgb}{1.000000,0.000000,0.000000}%
\pgfsetfillcolor{currentfill}%
\pgfsetfillopacity{0.100000}%
\pgfsetlinewidth{1.003750pt}%
\definecolor{currentstroke}{rgb}{1.000000,0.000000,0.000000}%
\pgfsetstrokecolor{currentstroke}%
\pgfsetstrokeopacity{0.100000}%
\pgfsetdash{}{0pt}%
\pgfsys@defobject{currentmarker}{\pgfqpoint{1.025455in}{1.890405in}}{\pgfqpoint{5.534545in}{3.377648in}}{%
\pgfpathmoveto{\pgfqpoint{1.025455in}{3.377648in}}%
\pgfpathlineto{\pgfqpoint{1.025455in}{3.377648in}}%
\pgfpathlineto{\pgfqpoint{1.071001in}{3.362625in}}%
\pgfpathlineto{\pgfqpoint{1.116547in}{3.347602in}}%
\pgfpathlineto{\pgfqpoint{1.162094in}{3.332580in}}%
\pgfpathlineto{\pgfqpoint{1.207640in}{3.317557in}}%
\pgfpathlineto{\pgfqpoint{1.253186in}{3.302534in}}%
\pgfpathlineto{\pgfqpoint{1.298733in}{3.287512in}}%
\pgfpathlineto{\pgfqpoint{1.344279in}{3.272489in}}%
\pgfpathlineto{\pgfqpoint{1.389826in}{3.257466in}}%
\pgfpathlineto{\pgfqpoint{1.435372in}{3.242444in}}%
\pgfpathlineto{\pgfqpoint{1.480918in}{3.227421in}}%
\pgfpathlineto{\pgfqpoint{1.526465in}{3.212398in}}%
\pgfpathlineto{\pgfqpoint{1.572011in}{3.197376in}}%
\pgfpathlineto{\pgfqpoint{1.617557in}{3.182353in}}%
\pgfpathlineto{\pgfqpoint{1.663104in}{3.167330in}}%
\pgfpathlineto{\pgfqpoint{1.708650in}{3.152308in}}%
\pgfpathlineto{\pgfqpoint{1.754197in}{3.137285in}}%
\pgfpathlineto{\pgfqpoint{1.799743in}{3.122263in}}%
\pgfpathlineto{\pgfqpoint{1.845289in}{3.107240in}}%
\pgfpathlineto{\pgfqpoint{1.890836in}{3.092217in}}%
\pgfpathlineto{\pgfqpoint{1.936382in}{3.077195in}}%
\pgfpathlineto{\pgfqpoint{1.981928in}{3.062172in}}%
\pgfpathlineto{\pgfqpoint{2.027475in}{3.047149in}}%
\pgfpathlineto{\pgfqpoint{2.073021in}{3.032127in}}%
\pgfpathlineto{\pgfqpoint{2.118567in}{3.017104in}}%
\pgfpathlineto{\pgfqpoint{2.164114in}{3.002081in}}%
\pgfpathlineto{\pgfqpoint{2.209660in}{2.987059in}}%
\pgfpathlineto{\pgfqpoint{2.255207in}{2.972036in}}%
\pgfpathlineto{\pgfqpoint{2.300753in}{2.957013in}}%
\pgfpathlineto{\pgfqpoint{2.346299in}{2.941991in}}%
\pgfpathlineto{\pgfqpoint{2.391846in}{2.926968in}}%
\pgfpathlineto{\pgfqpoint{2.437392in}{2.911945in}}%
\pgfpathlineto{\pgfqpoint{2.482938in}{2.896923in}}%
\pgfpathlineto{\pgfqpoint{2.528485in}{2.881900in}}%
\pgfpathlineto{\pgfqpoint{2.574031in}{2.866877in}}%
\pgfpathlineto{\pgfqpoint{2.619578in}{2.851855in}}%
\pgfpathlineto{\pgfqpoint{2.665124in}{2.836832in}}%
\pgfpathlineto{\pgfqpoint{2.710670in}{2.821809in}}%
\pgfpathlineto{\pgfqpoint{2.756217in}{2.806787in}}%
\pgfpathlineto{\pgfqpoint{2.801763in}{2.791764in}}%
\pgfpathlineto{\pgfqpoint{2.847309in}{2.776741in}}%
\pgfpathlineto{\pgfqpoint{2.892856in}{2.761719in}}%
\pgfpathlineto{\pgfqpoint{2.938402in}{2.746696in}}%
\pgfpathlineto{\pgfqpoint{2.983949in}{2.731674in}}%
\pgfpathlineto{\pgfqpoint{3.029495in}{2.716651in}}%
\pgfpathlineto{\pgfqpoint{3.075041in}{2.701628in}}%
\pgfpathlineto{\pgfqpoint{3.120588in}{2.686606in}}%
\pgfpathlineto{\pgfqpoint{3.166134in}{2.671583in}}%
\pgfpathlineto{\pgfqpoint{3.211680in}{2.656560in}}%
\pgfpathlineto{\pgfqpoint{3.257227in}{2.641538in}}%
\pgfpathlineto{\pgfqpoint{3.302773in}{2.626515in}}%
\pgfpathlineto{\pgfqpoint{3.348320in}{2.611492in}}%
\pgfpathlineto{\pgfqpoint{3.393866in}{2.596470in}}%
\pgfpathlineto{\pgfqpoint{3.439412in}{2.581447in}}%
\pgfpathlineto{\pgfqpoint{3.484959in}{2.566424in}}%
\pgfpathlineto{\pgfqpoint{3.530505in}{2.551402in}}%
\pgfpathlineto{\pgfqpoint{3.576051in}{2.536379in}}%
\pgfpathlineto{\pgfqpoint{3.621598in}{2.521356in}}%
\pgfpathlineto{\pgfqpoint{3.667144in}{2.506334in}}%
\pgfpathlineto{\pgfqpoint{3.712691in}{2.491311in}}%
\pgfpathlineto{\pgfqpoint{3.758237in}{2.476288in}}%
\pgfpathlineto{\pgfqpoint{3.803783in}{2.461266in}}%
\pgfpathlineto{\pgfqpoint{3.849330in}{2.446243in}}%
\pgfpathlineto{\pgfqpoint{3.894876in}{2.431220in}}%
\pgfpathlineto{\pgfqpoint{3.940422in}{2.416198in}}%
\pgfpathlineto{\pgfqpoint{3.985969in}{2.401175in}}%
\pgfpathlineto{\pgfqpoint{4.031515in}{2.386152in}}%
\pgfpathlineto{\pgfqpoint{4.077062in}{2.371130in}}%
\pgfpathlineto{\pgfqpoint{4.122608in}{2.356107in}}%
\pgfpathlineto{\pgfqpoint{4.168154in}{2.341085in}}%
\pgfpathlineto{\pgfqpoint{4.213701in}{2.326062in}}%
\pgfpathlineto{\pgfqpoint{4.259247in}{2.311039in}}%
\pgfpathlineto{\pgfqpoint{4.304793in}{2.296017in}}%
\pgfpathlineto{\pgfqpoint{4.350340in}{2.280994in}}%
\pgfpathlineto{\pgfqpoint{4.395886in}{2.265971in}}%
\pgfpathlineto{\pgfqpoint{4.441433in}{2.250949in}}%
\pgfpathlineto{\pgfqpoint{4.486979in}{2.235926in}}%
\pgfpathlineto{\pgfqpoint{4.532525in}{2.220903in}}%
\pgfpathlineto{\pgfqpoint{4.578072in}{2.205881in}}%
\pgfpathlineto{\pgfqpoint{4.623618in}{2.190858in}}%
\pgfpathlineto{\pgfqpoint{4.669164in}{2.175835in}}%
\pgfpathlineto{\pgfqpoint{4.714711in}{2.160813in}}%
\pgfpathlineto{\pgfqpoint{4.760257in}{2.145790in}}%
\pgfpathlineto{\pgfqpoint{4.805803in}{2.130767in}}%
\pgfpathlineto{\pgfqpoint{4.851350in}{2.115745in}}%
\pgfpathlineto{\pgfqpoint{4.896896in}{2.100722in}}%
\pgfpathlineto{\pgfqpoint{4.942443in}{2.085699in}}%
\pgfpathlineto{\pgfqpoint{4.987989in}{2.070677in}}%
\pgfpathlineto{\pgfqpoint{5.033535in}{2.055654in}}%
\pgfpathlineto{\pgfqpoint{5.079082in}{2.040631in}}%
\pgfpathlineto{\pgfqpoint{5.124628in}{2.025609in}}%
\pgfpathlineto{\pgfqpoint{5.170174in}{2.010586in}}%
\pgfpathlineto{\pgfqpoint{5.215721in}{1.995563in}}%
\pgfpathlineto{\pgfqpoint{5.261267in}{1.980541in}}%
\pgfpathlineto{\pgfqpoint{5.306814in}{1.965518in}}%
\pgfpathlineto{\pgfqpoint{5.352360in}{1.950496in}}%
\pgfpathlineto{\pgfqpoint{5.397906in}{1.935473in}}%
\pgfpathlineto{\pgfqpoint{5.443453in}{1.920450in}}%
\pgfpathlineto{\pgfqpoint{5.488999in}{1.905428in}}%
\pgfpathlineto{\pgfqpoint{5.534545in}{1.890405in}}%
\pgfpathlineto{\pgfqpoint{5.534545in}{2.050301in}}%
\pgfpathlineto{\pgfqpoint{5.534545in}{2.050301in}}%
\pgfpathlineto{\pgfqpoint{5.488999in}{2.063708in}}%
\pgfpathlineto{\pgfqpoint{5.443453in}{2.077116in}}%
\pgfpathlineto{\pgfqpoint{5.397906in}{2.090524in}}%
\pgfpathlineto{\pgfqpoint{5.352360in}{2.103931in}}%
\pgfpathlineto{\pgfqpoint{5.306814in}{2.117339in}}%
\pgfpathlineto{\pgfqpoint{5.261267in}{2.130746in}}%
\pgfpathlineto{\pgfqpoint{5.215721in}{2.144154in}}%
\pgfpathlineto{\pgfqpoint{5.170174in}{2.157561in}}%
\pgfpathlineto{\pgfqpoint{5.124628in}{2.170969in}}%
\pgfpathlineto{\pgfqpoint{5.079082in}{2.184376in}}%
\pgfpathlineto{\pgfqpoint{5.033535in}{2.197784in}}%
\pgfpathlineto{\pgfqpoint{4.987989in}{2.211191in}}%
\pgfpathlineto{\pgfqpoint{4.942443in}{2.224599in}}%
\pgfpathlineto{\pgfqpoint{4.896896in}{2.238006in}}%
\pgfpathlineto{\pgfqpoint{4.851350in}{2.251414in}}%
\pgfpathlineto{\pgfqpoint{4.805803in}{2.264822in}}%
\pgfpathlineto{\pgfqpoint{4.760257in}{2.278229in}}%
\pgfpathlineto{\pgfqpoint{4.714711in}{2.291637in}}%
\pgfpathlineto{\pgfqpoint{4.669164in}{2.305044in}}%
\pgfpathlineto{\pgfqpoint{4.623618in}{2.318452in}}%
\pgfpathlineto{\pgfqpoint{4.578072in}{2.331859in}}%
\pgfpathlineto{\pgfqpoint{4.532525in}{2.345267in}}%
\pgfpathlineto{\pgfqpoint{4.486979in}{2.358674in}}%
\pgfpathlineto{\pgfqpoint{4.441433in}{2.372082in}}%
\pgfpathlineto{\pgfqpoint{4.395886in}{2.385489in}}%
\pgfpathlineto{\pgfqpoint{4.350340in}{2.398897in}}%
\pgfpathlineto{\pgfqpoint{4.304793in}{2.412305in}}%
\pgfpathlineto{\pgfqpoint{4.259247in}{2.425712in}}%
\pgfpathlineto{\pgfqpoint{4.213701in}{2.439120in}}%
\pgfpathlineto{\pgfqpoint{4.168154in}{2.452527in}}%
\pgfpathlineto{\pgfqpoint{4.122608in}{2.465935in}}%
\pgfpathlineto{\pgfqpoint{4.077062in}{2.479342in}}%
\pgfpathlineto{\pgfqpoint{4.031515in}{2.492750in}}%
\pgfpathlineto{\pgfqpoint{3.985969in}{2.506157in}}%
\pgfpathlineto{\pgfqpoint{3.940422in}{2.519565in}}%
\pgfpathlineto{\pgfqpoint{3.894876in}{2.532972in}}%
\pgfpathlineto{\pgfqpoint{3.849330in}{2.546380in}}%
\pgfpathlineto{\pgfqpoint{3.803783in}{2.559788in}}%
\pgfpathlineto{\pgfqpoint{3.758237in}{2.573195in}}%
\pgfpathlineto{\pgfqpoint{3.712691in}{2.586603in}}%
\pgfpathlineto{\pgfqpoint{3.667144in}{2.600010in}}%
\pgfpathlineto{\pgfqpoint{3.621598in}{2.613418in}}%
\pgfpathlineto{\pgfqpoint{3.576051in}{2.626825in}}%
\pgfpathlineto{\pgfqpoint{3.530505in}{2.640233in}}%
\pgfpathlineto{\pgfqpoint{3.484959in}{2.653640in}}%
\pgfpathlineto{\pgfqpoint{3.439412in}{2.667048in}}%
\pgfpathlineto{\pgfqpoint{3.393866in}{2.680455in}}%
\pgfpathlineto{\pgfqpoint{3.348320in}{2.693863in}}%
\pgfpathlineto{\pgfqpoint{3.302773in}{2.707270in}}%
\pgfpathlineto{\pgfqpoint{3.257227in}{2.720678in}}%
\pgfpathlineto{\pgfqpoint{3.211680in}{2.734086in}}%
\pgfpathlineto{\pgfqpoint{3.166134in}{2.747493in}}%
\pgfpathlineto{\pgfqpoint{3.120588in}{2.760901in}}%
\pgfpathlineto{\pgfqpoint{3.075041in}{2.774308in}}%
\pgfpathlineto{\pgfqpoint{3.029495in}{2.787716in}}%
\pgfpathlineto{\pgfqpoint{2.983949in}{2.801123in}}%
\pgfpathlineto{\pgfqpoint{2.938402in}{2.814531in}}%
\pgfpathlineto{\pgfqpoint{2.892856in}{2.827938in}}%
\pgfpathlineto{\pgfqpoint{2.847309in}{2.841346in}}%
\pgfpathlineto{\pgfqpoint{2.801763in}{2.854753in}}%
\pgfpathlineto{\pgfqpoint{2.756217in}{2.868161in}}%
\pgfpathlineto{\pgfqpoint{2.710670in}{2.881569in}}%
\pgfpathlineto{\pgfqpoint{2.665124in}{2.894976in}}%
\pgfpathlineto{\pgfqpoint{2.619578in}{2.908384in}}%
\pgfpathlineto{\pgfqpoint{2.574031in}{2.921791in}}%
\pgfpathlineto{\pgfqpoint{2.528485in}{2.935199in}}%
\pgfpathlineto{\pgfqpoint{2.482938in}{2.948606in}}%
\pgfpathlineto{\pgfqpoint{2.437392in}{2.962014in}}%
\pgfpathlineto{\pgfqpoint{2.391846in}{2.975421in}}%
\pgfpathlineto{\pgfqpoint{2.346299in}{2.988829in}}%
\pgfpathlineto{\pgfqpoint{2.300753in}{3.002236in}}%
\pgfpathlineto{\pgfqpoint{2.255207in}{3.015644in}}%
\pgfpathlineto{\pgfqpoint{2.209660in}{3.029052in}}%
\pgfpathlineto{\pgfqpoint{2.164114in}{3.042459in}}%
\pgfpathlineto{\pgfqpoint{2.118567in}{3.055867in}}%
\pgfpathlineto{\pgfqpoint{2.073021in}{3.069274in}}%
\pgfpathlineto{\pgfqpoint{2.027475in}{3.082682in}}%
\pgfpathlineto{\pgfqpoint{1.981928in}{3.096089in}}%
\pgfpathlineto{\pgfqpoint{1.936382in}{3.109497in}}%
\pgfpathlineto{\pgfqpoint{1.890836in}{3.122904in}}%
\pgfpathlineto{\pgfqpoint{1.845289in}{3.136312in}}%
\pgfpathlineto{\pgfqpoint{1.799743in}{3.149719in}}%
\pgfpathlineto{\pgfqpoint{1.754197in}{3.163127in}}%
\pgfpathlineto{\pgfqpoint{1.708650in}{3.176534in}}%
\pgfpathlineto{\pgfqpoint{1.663104in}{3.189942in}}%
\pgfpathlineto{\pgfqpoint{1.617557in}{3.203350in}}%
\pgfpathlineto{\pgfqpoint{1.572011in}{3.216757in}}%
\pgfpathlineto{\pgfqpoint{1.526465in}{3.230165in}}%
\pgfpathlineto{\pgfqpoint{1.480918in}{3.243572in}}%
\pgfpathlineto{\pgfqpoint{1.435372in}{3.256980in}}%
\pgfpathlineto{\pgfqpoint{1.389826in}{3.270387in}}%
\pgfpathlineto{\pgfqpoint{1.344279in}{3.283795in}}%
\pgfpathlineto{\pgfqpoint{1.298733in}{3.297202in}}%
\pgfpathlineto{\pgfqpoint{1.253186in}{3.310610in}}%
\pgfpathlineto{\pgfqpoint{1.207640in}{3.324017in}}%
\pgfpathlineto{\pgfqpoint{1.162094in}{3.337425in}}%
\pgfpathlineto{\pgfqpoint{1.116547in}{3.350833in}}%
\pgfpathlineto{\pgfqpoint{1.071001in}{3.364240in}}%
\pgfpathlineto{\pgfqpoint{1.025455in}{3.377648in}}%
\pgfpathclose%
\pgfusepath{stroke,fill}%
}%
\begin{pgfscope}%
\pgfsys@transformshift{0.000000in}{0.000000in}%
\pgfsys@useobject{currentmarker}{}%
\end{pgfscope}%
\end{pgfscope}%
\begin{pgfscope}%
\pgfpathrectangle{\pgfqpoint{0.800000in}{0.440000in}}{\pgfqpoint{4.960000in}{3.080000in}}%
\pgfusepath{clip}%
\pgfsetbuttcap%
\pgfsetroundjoin%
\definecolor{currentfill}{rgb}{0.000000,0.000000,1.000000}%
\pgfsetfillcolor{currentfill}%
\pgfsetfillopacity{0.100000}%
\pgfsetlinewidth{1.003750pt}%
\definecolor{currentstroke}{rgb}{0.000000,0.000000,1.000000}%
\pgfsetstrokecolor{currentstroke}%
\pgfsetstrokeopacity{0.100000}%
\pgfsetdash{}{0pt}%
\pgfsys@defobject{currentmarker}{\pgfqpoint{1.025455in}{1.993874in}}{\pgfqpoint{3.650447in}{3.377648in}}{%
\pgfpathmoveto{\pgfqpoint{1.025455in}{3.377648in}}%
\pgfpathlineto{\pgfqpoint{1.025455in}{3.377648in}}%
\pgfpathlineto{\pgfqpoint{1.051970in}{3.363670in}}%
\pgfpathlineto{\pgfqpoint{1.078485in}{3.349693in}}%
\pgfpathlineto{\pgfqpoint{1.105000in}{3.335715in}}%
\pgfpathlineto{\pgfqpoint{1.131515in}{3.321738in}}%
\pgfpathlineto{\pgfqpoint{1.158030in}{3.307760in}}%
\pgfpathlineto{\pgfqpoint{1.184545in}{3.293783in}}%
\pgfpathlineto{\pgfqpoint{1.211060in}{3.279805in}}%
\pgfpathlineto{\pgfqpoint{1.237575in}{3.265828in}}%
\pgfpathlineto{\pgfqpoint{1.264090in}{3.251850in}}%
\pgfpathlineto{\pgfqpoint{1.290605in}{3.237873in}}%
\pgfpathlineto{\pgfqpoint{1.317120in}{3.223895in}}%
\pgfpathlineto{\pgfqpoint{1.343635in}{3.209918in}}%
\pgfpathlineto{\pgfqpoint{1.370151in}{3.195940in}}%
\pgfpathlineto{\pgfqpoint{1.396666in}{3.181962in}}%
\pgfpathlineto{\pgfqpoint{1.423181in}{3.167985in}}%
\pgfpathlineto{\pgfqpoint{1.449696in}{3.154007in}}%
\pgfpathlineto{\pgfqpoint{1.476211in}{3.140030in}}%
\pgfpathlineto{\pgfqpoint{1.502726in}{3.126052in}}%
\pgfpathlineto{\pgfqpoint{1.529241in}{3.112075in}}%
\pgfpathlineto{\pgfqpoint{1.555756in}{3.098097in}}%
\pgfpathlineto{\pgfqpoint{1.582271in}{3.084120in}}%
\pgfpathlineto{\pgfqpoint{1.608786in}{3.070142in}}%
\pgfpathlineto{\pgfqpoint{1.635301in}{3.056165in}}%
\pgfpathlineto{\pgfqpoint{1.661816in}{3.042187in}}%
\pgfpathlineto{\pgfqpoint{1.688332in}{3.028210in}}%
\pgfpathlineto{\pgfqpoint{1.714847in}{3.014232in}}%
\pgfpathlineto{\pgfqpoint{1.741362in}{3.000255in}}%
\pgfpathlineto{\pgfqpoint{1.767877in}{2.986277in}}%
\pgfpathlineto{\pgfqpoint{1.794392in}{2.972300in}}%
\pgfpathlineto{\pgfqpoint{1.820907in}{2.958322in}}%
\pgfpathlineto{\pgfqpoint{1.847422in}{2.944345in}}%
\pgfpathlineto{\pgfqpoint{1.873937in}{2.930367in}}%
\pgfpathlineto{\pgfqpoint{1.900452in}{2.916390in}}%
\pgfpathlineto{\pgfqpoint{1.926967in}{2.902412in}}%
\pgfpathlineto{\pgfqpoint{1.953482in}{2.888435in}}%
\pgfpathlineto{\pgfqpoint{1.979997in}{2.874457in}}%
\pgfpathlineto{\pgfqpoint{2.006512in}{2.860480in}}%
\pgfpathlineto{\pgfqpoint{2.033028in}{2.846502in}}%
\pgfpathlineto{\pgfqpoint{2.059543in}{2.832525in}}%
\pgfpathlineto{\pgfqpoint{2.086058in}{2.818547in}}%
\pgfpathlineto{\pgfqpoint{2.112573in}{2.804570in}}%
\pgfpathlineto{\pgfqpoint{2.139088in}{2.790592in}}%
\pgfpathlineto{\pgfqpoint{2.165603in}{2.776615in}}%
\pgfpathlineto{\pgfqpoint{2.192118in}{2.762637in}}%
\pgfpathlineto{\pgfqpoint{2.218633in}{2.748660in}}%
\pgfpathlineto{\pgfqpoint{2.245148in}{2.734682in}}%
\pgfpathlineto{\pgfqpoint{2.271663in}{2.720705in}}%
\pgfpathlineto{\pgfqpoint{2.298178in}{2.706727in}}%
\pgfpathlineto{\pgfqpoint{2.324693in}{2.692750in}}%
\pgfpathlineto{\pgfqpoint{2.351208in}{2.678772in}}%
\pgfpathlineto{\pgfqpoint{2.377724in}{2.664795in}}%
\pgfpathlineto{\pgfqpoint{2.404239in}{2.650817in}}%
\pgfpathlineto{\pgfqpoint{2.430754in}{2.636840in}}%
\pgfpathlineto{\pgfqpoint{2.457269in}{2.622862in}}%
\pgfpathlineto{\pgfqpoint{2.483784in}{2.608885in}}%
\pgfpathlineto{\pgfqpoint{2.510299in}{2.594907in}}%
\pgfpathlineto{\pgfqpoint{2.536814in}{2.580930in}}%
\pgfpathlineto{\pgfqpoint{2.563329in}{2.566952in}}%
\pgfpathlineto{\pgfqpoint{2.589844in}{2.552975in}}%
\pgfpathlineto{\pgfqpoint{2.616359in}{2.538997in}}%
\pgfpathlineto{\pgfqpoint{2.642874in}{2.525019in}}%
\pgfpathlineto{\pgfqpoint{2.669389in}{2.511042in}}%
\pgfpathlineto{\pgfqpoint{2.695905in}{2.497064in}}%
\pgfpathlineto{\pgfqpoint{2.722420in}{2.483087in}}%
\pgfpathlineto{\pgfqpoint{2.748935in}{2.469109in}}%
\pgfpathlineto{\pgfqpoint{2.775450in}{2.455132in}}%
\pgfpathlineto{\pgfqpoint{2.801965in}{2.441154in}}%
\pgfpathlineto{\pgfqpoint{2.828480in}{2.427177in}}%
\pgfpathlineto{\pgfqpoint{2.854995in}{2.413199in}}%
\pgfpathlineto{\pgfqpoint{2.881510in}{2.399222in}}%
\pgfpathlineto{\pgfqpoint{2.908025in}{2.385244in}}%
\pgfpathlineto{\pgfqpoint{2.934540in}{2.371267in}}%
\pgfpathlineto{\pgfqpoint{2.961055in}{2.357289in}}%
\pgfpathlineto{\pgfqpoint{2.987570in}{2.343312in}}%
\pgfpathlineto{\pgfqpoint{3.014085in}{2.329334in}}%
\pgfpathlineto{\pgfqpoint{3.040601in}{2.315357in}}%
\pgfpathlineto{\pgfqpoint{3.067116in}{2.301379in}}%
\pgfpathlineto{\pgfqpoint{3.093631in}{2.287402in}}%
\pgfpathlineto{\pgfqpoint{3.120146in}{2.273424in}}%
\pgfpathlineto{\pgfqpoint{3.146661in}{2.259447in}}%
\pgfpathlineto{\pgfqpoint{3.173176in}{2.245469in}}%
\pgfpathlineto{\pgfqpoint{3.199691in}{2.231492in}}%
\pgfpathlineto{\pgfqpoint{3.226206in}{2.217514in}}%
\pgfpathlineto{\pgfqpoint{3.252721in}{2.203537in}}%
\pgfpathlineto{\pgfqpoint{3.279236in}{2.189559in}}%
\pgfpathlineto{\pgfqpoint{3.305751in}{2.175582in}}%
\pgfpathlineto{\pgfqpoint{3.332266in}{2.161604in}}%
\pgfpathlineto{\pgfqpoint{3.358781in}{2.147627in}}%
\pgfpathlineto{\pgfqpoint{3.385297in}{2.133649in}}%
\pgfpathlineto{\pgfqpoint{3.411812in}{2.119672in}}%
\pgfpathlineto{\pgfqpoint{3.438327in}{2.105694in}}%
\pgfpathlineto{\pgfqpoint{3.464842in}{2.091717in}}%
\pgfpathlineto{\pgfqpoint{3.491357in}{2.077739in}}%
\pgfpathlineto{\pgfqpoint{3.517872in}{2.063762in}}%
\pgfpathlineto{\pgfqpoint{3.544387in}{2.049784in}}%
\pgfpathlineto{\pgfqpoint{3.570902in}{2.035807in}}%
\pgfpathlineto{\pgfqpoint{3.597417in}{2.021829in}}%
\pgfpathlineto{\pgfqpoint{3.623932in}{2.007852in}}%
\pgfpathlineto{\pgfqpoint{3.650447in}{1.993874in}}%
\pgfpathlineto{\pgfqpoint{3.650447in}{2.146470in}}%
\pgfpathlineto{\pgfqpoint{3.650447in}{2.146470in}}%
\pgfpathlineto{\pgfqpoint{3.623932in}{2.158906in}}%
\pgfpathlineto{\pgfqpoint{3.597417in}{2.171342in}}%
\pgfpathlineto{\pgfqpoint{3.570902in}{2.183778in}}%
\pgfpathlineto{\pgfqpoint{3.544387in}{2.196215in}}%
\pgfpathlineto{\pgfqpoint{3.517872in}{2.208651in}}%
\pgfpathlineto{\pgfqpoint{3.491357in}{2.221087in}}%
\pgfpathlineto{\pgfqpoint{3.464842in}{2.233523in}}%
\pgfpathlineto{\pgfqpoint{3.438327in}{2.245959in}}%
\pgfpathlineto{\pgfqpoint{3.411812in}{2.258395in}}%
\pgfpathlineto{\pgfqpoint{3.385297in}{2.270831in}}%
\pgfpathlineto{\pgfqpoint{3.358781in}{2.283268in}}%
\pgfpathlineto{\pgfqpoint{3.332266in}{2.295704in}}%
\pgfpathlineto{\pgfqpoint{3.305751in}{2.308140in}}%
\pgfpathlineto{\pgfqpoint{3.279236in}{2.320576in}}%
\pgfpathlineto{\pgfqpoint{3.252721in}{2.333012in}}%
\pgfpathlineto{\pgfqpoint{3.226206in}{2.345448in}}%
\pgfpathlineto{\pgfqpoint{3.199691in}{2.357884in}}%
\pgfpathlineto{\pgfqpoint{3.173176in}{2.370321in}}%
\pgfpathlineto{\pgfqpoint{3.146661in}{2.382757in}}%
\pgfpathlineto{\pgfqpoint{3.120146in}{2.395193in}}%
\pgfpathlineto{\pgfqpoint{3.093631in}{2.407629in}}%
\pgfpathlineto{\pgfqpoint{3.067116in}{2.420065in}}%
\pgfpathlineto{\pgfqpoint{3.040601in}{2.432501in}}%
\pgfpathlineto{\pgfqpoint{3.014085in}{2.444937in}}%
\pgfpathlineto{\pgfqpoint{2.987570in}{2.457373in}}%
\pgfpathlineto{\pgfqpoint{2.961055in}{2.469810in}}%
\pgfpathlineto{\pgfqpoint{2.934540in}{2.482246in}}%
\pgfpathlineto{\pgfqpoint{2.908025in}{2.494682in}}%
\pgfpathlineto{\pgfqpoint{2.881510in}{2.507118in}}%
\pgfpathlineto{\pgfqpoint{2.854995in}{2.519554in}}%
\pgfpathlineto{\pgfqpoint{2.828480in}{2.531990in}}%
\pgfpathlineto{\pgfqpoint{2.801965in}{2.544426in}}%
\pgfpathlineto{\pgfqpoint{2.775450in}{2.556863in}}%
\pgfpathlineto{\pgfqpoint{2.748935in}{2.569299in}}%
\pgfpathlineto{\pgfqpoint{2.722420in}{2.581735in}}%
\pgfpathlineto{\pgfqpoint{2.695905in}{2.594171in}}%
\pgfpathlineto{\pgfqpoint{2.669389in}{2.606607in}}%
\pgfpathlineto{\pgfqpoint{2.642874in}{2.619043in}}%
\pgfpathlineto{\pgfqpoint{2.616359in}{2.631479in}}%
\pgfpathlineto{\pgfqpoint{2.589844in}{2.643916in}}%
\pgfpathlineto{\pgfqpoint{2.563329in}{2.656352in}}%
\pgfpathlineto{\pgfqpoint{2.536814in}{2.668788in}}%
\pgfpathlineto{\pgfqpoint{2.510299in}{2.681224in}}%
\pgfpathlineto{\pgfqpoint{2.483784in}{2.693660in}}%
\pgfpathlineto{\pgfqpoint{2.457269in}{2.706096in}}%
\pgfpathlineto{\pgfqpoint{2.430754in}{2.718532in}}%
\pgfpathlineto{\pgfqpoint{2.404239in}{2.730968in}}%
\pgfpathlineto{\pgfqpoint{2.377724in}{2.743405in}}%
\pgfpathlineto{\pgfqpoint{2.351208in}{2.755841in}}%
\pgfpathlineto{\pgfqpoint{2.324693in}{2.768277in}}%
\pgfpathlineto{\pgfqpoint{2.298178in}{2.780713in}}%
\pgfpathlineto{\pgfqpoint{2.271663in}{2.793149in}}%
\pgfpathlineto{\pgfqpoint{2.245148in}{2.805585in}}%
\pgfpathlineto{\pgfqpoint{2.218633in}{2.818021in}}%
\pgfpathlineto{\pgfqpoint{2.192118in}{2.830458in}}%
\pgfpathlineto{\pgfqpoint{2.165603in}{2.842894in}}%
\pgfpathlineto{\pgfqpoint{2.139088in}{2.855330in}}%
\pgfpathlineto{\pgfqpoint{2.112573in}{2.867766in}}%
\pgfpathlineto{\pgfqpoint{2.086058in}{2.880202in}}%
\pgfpathlineto{\pgfqpoint{2.059543in}{2.892638in}}%
\pgfpathlineto{\pgfqpoint{2.033028in}{2.905074in}}%
\pgfpathlineto{\pgfqpoint{2.006512in}{2.917511in}}%
\pgfpathlineto{\pgfqpoint{1.979997in}{2.929947in}}%
\pgfpathlineto{\pgfqpoint{1.953482in}{2.942383in}}%
\pgfpathlineto{\pgfqpoint{1.926967in}{2.954819in}}%
\pgfpathlineto{\pgfqpoint{1.900452in}{2.967255in}}%
\pgfpathlineto{\pgfqpoint{1.873937in}{2.979691in}}%
\pgfpathlineto{\pgfqpoint{1.847422in}{2.992127in}}%
\pgfpathlineto{\pgfqpoint{1.820907in}{3.004564in}}%
\pgfpathlineto{\pgfqpoint{1.794392in}{3.017000in}}%
\pgfpathlineto{\pgfqpoint{1.767877in}{3.029436in}}%
\pgfpathlineto{\pgfqpoint{1.741362in}{3.041872in}}%
\pgfpathlineto{\pgfqpoint{1.714847in}{3.054308in}}%
\pgfpathlineto{\pgfqpoint{1.688332in}{3.066744in}}%
\pgfpathlineto{\pgfqpoint{1.661816in}{3.079180in}}%
\pgfpathlineto{\pgfqpoint{1.635301in}{3.091616in}}%
\pgfpathlineto{\pgfqpoint{1.608786in}{3.104053in}}%
\pgfpathlineto{\pgfqpoint{1.582271in}{3.116489in}}%
\pgfpathlineto{\pgfqpoint{1.555756in}{3.128925in}}%
\pgfpathlineto{\pgfqpoint{1.529241in}{3.141361in}}%
\pgfpathlineto{\pgfqpoint{1.502726in}{3.153797in}}%
\pgfpathlineto{\pgfqpoint{1.476211in}{3.166233in}}%
\pgfpathlineto{\pgfqpoint{1.449696in}{3.178669in}}%
\pgfpathlineto{\pgfqpoint{1.423181in}{3.191106in}}%
\pgfpathlineto{\pgfqpoint{1.396666in}{3.203542in}}%
\pgfpathlineto{\pgfqpoint{1.370151in}{3.215978in}}%
\pgfpathlineto{\pgfqpoint{1.343635in}{3.228414in}}%
\pgfpathlineto{\pgfqpoint{1.317120in}{3.240850in}}%
\pgfpathlineto{\pgfqpoint{1.290605in}{3.253286in}}%
\pgfpathlineto{\pgfqpoint{1.264090in}{3.265722in}}%
\pgfpathlineto{\pgfqpoint{1.237575in}{3.278159in}}%
\pgfpathlineto{\pgfqpoint{1.211060in}{3.290595in}}%
\pgfpathlineto{\pgfqpoint{1.184545in}{3.303031in}}%
\pgfpathlineto{\pgfqpoint{1.158030in}{3.315467in}}%
\pgfpathlineto{\pgfqpoint{1.131515in}{3.327903in}}%
\pgfpathlineto{\pgfqpoint{1.105000in}{3.340339in}}%
\pgfpathlineto{\pgfqpoint{1.078485in}{3.352775in}}%
\pgfpathlineto{\pgfqpoint{1.051970in}{3.365212in}}%
\pgfpathlineto{\pgfqpoint{1.025455in}{3.377648in}}%
\pgfpathclose%
\pgfusepath{stroke,fill}%
}%
\begin{pgfscope}%
\pgfsys@transformshift{0.000000in}{0.000000in}%
\pgfsys@useobject{currentmarker}{}%
\end{pgfscope}%
\end{pgfscope}%
\begin{pgfscope}%
\pgfpathrectangle{\pgfqpoint{0.800000in}{0.440000in}}{\pgfqpoint{4.960000in}{3.080000in}}%
\pgfusepath{clip}%
\pgfsetbuttcap%
\pgfsetroundjoin%
\definecolor{currentfill}{rgb}{0.000000,0.501961,0.000000}%
\pgfsetfillcolor{currentfill}%
\pgfsetfillopacity{0.100000}%
\pgfsetlinewidth{1.003750pt}%
\definecolor{currentstroke}{rgb}{0.000000,0.501961,0.000000}%
\pgfsetstrokecolor{currentstroke}%
\pgfsetstrokeopacity{0.100000}%
\pgfsetdash{}{0pt}%
\pgfsys@defobject{currentmarker}{\pgfqpoint{1.025455in}{2.137435in}}{\pgfqpoint{2.544336in}{3.377648in}}{%
\pgfpathmoveto{\pgfqpoint{1.025455in}{3.377648in}}%
\pgfpathlineto{\pgfqpoint{1.025455in}{3.377648in}}%
\pgfpathlineto{\pgfqpoint{1.040797in}{3.365120in}}%
\pgfpathlineto{\pgfqpoint{1.056139in}{3.352593in}}%
\pgfpathlineto{\pgfqpoint{1.071481in}{3.340065in}}%
\pgfpathlineto{\pgfqpoint{1.086824in}{3.327538in}}%
\pgfpathlineto{\pgfqpoint{1.102166in}{3.315011in}}%
\pgfpathlineto{\pgfqpoint{1.117508in}{3.302483in}}%
\pgfpathlineto{\pgfqpoint{1.132850in}{3.289956in}}%
\pgfpathlineto{\pgfqpoint{1.148192in}{3.277428in}}%
\pgfpathlineto{\pgfqpoint{1.163535in}{3.264901in}}%
\pgfpathlineto{\pgfqpoint{1.178877in}{3.252374in}}%
\pgfpathlineto{\pgfqpoint{1.194219in}{3.239846in}}%
\pgfpathlineto{\pgfqpoint{1.209561in}{3.227319in}}%
\pgfpathlineto{\pgfqpoint{1.224904in}{3.214791in}}%
\pgfpathlineto{\pgfqpoint{1.240246in}{3.202264in}}%
\pgfpathlineto{\pgfqpoint{1.255588in}{3.189737in}}%
\pgfpathlineto{\pgfqpoint{1.270930in}{3.177209in}}%
\pgfpathlineto{\pgfqpoint{1.286273in}{3.164682in}}%
\pgfpathlineto{\pgfqpoint{1.301615in}{3.152155in}}%
\pgfpathlineto{\pgfqpoint{1.316957in}{3.139627in}}%
\pgfpathlineto{\pgfqpoint{1.332299in}{3.127100in}}%
\pgfpathlineto{\pgfqpoint{1.347642in}{3.114572in}}%
\pgfpathlineto{\pgfqpoint{1.362984in}{3.102045in}}%
\pgfpathlineto{\pgfqpoint{1.378326in}{3.089518in}}%
\pgfpathlineto{\pgfqpoint{1.393668in}{3.076990in}}%
\pgfpathlineto{\pgfqpoint{1.409011in}{3.064463in}}%
\pgfpathlineto{\pgfqpoint{1.424353in}{3.051935in}}%
\pgfpathlineto{\pgfqpoint{1.439695in}{3.039408in}}%
\pgfpathlineto{\pgfqpoint{1.455037in}{3.026881in}}%
\pgfpathlineto{\pgfqpoint{1.470379in}{3.014353in}}%
\pgfpathlineto{\pgfqpoint{1.485722in}{3.001826in}}%
\pgfpathlineto{\pgfqpoint{1.501064in}{2.989298in}}%
\pgfpathlineto{\pgfqpoint{1.516406in}{2.976771in}}%
\pgfpathlineto{\pgfqpoint{1.531748in}{2.964244in}}%
\pgfpathlineto{\pgfqpoint{1.547091in}{2.951716in}}%
\pgfpathlineto{\pgfqpoint{1.562433in}{2.939189in}}%
\pgfpathlineto{\pgfqpoint{1.577775in}{2.926661in}}%
\pgfpathlineto{\pgfqpoint{1.593117in}{2.914134in}}%
\pgfpathlineto{\pgfqpoint{1.608460in}{2.901607in}}%
\pgfpathlineto{\pgfqpoint{1.623802in}{2.889079in}}%
\pgfpathlineto{\pgfqpoint{1.639144in}{2.876552in}}%
\pgfpathlineto{\pgfqpoint{1.654486in}{2.864024in}}%
\pgfpathlineto{\pgfqpoint{1.669829in}{2.851497in}}%
\pgfpathlineto{\pgfqpoint{1.685171in}{2.838970in}}%
\pgfpathlineto{\pgfqpoint{1.700513in}{2.826442in}}%
\pgfpathlineto{\pgfqpoint{1.715855in}{2.813915in}}%
\pgfpathlineto{\pgfqpoint{1.731198in}{2.801387in}}%
\pgfpathlineto{\pgfqpoint{1.746540in}{2.788860in}}%
\pgfpathlineto{\pgfqpoint{1.761882in}{2.776333in}}%
\pgfpathlineto{\pgfqpoint{1.777224in}{2.763805in}}%
\pgfpathlineto{\pgfqpoint{1.792567in}{2.751278in}}%
\pgfpathlineto{\pgfqpoint{1.807909in}{2.738750in}}%
\pgfpathlineto{\pgfqpoint{1.823251in}{2.726223in}}%
\pgfpathlineto{\pgfqpoint{1.838593in}{2.713696in}}%
\pgfpathlineto{\pgfqpoint{1.853935in}{2.701168in}}%
\pgfpathlineto{\pgfqpoint{1.869278in}{2.688641in}}%
\pgfpathlineto{\pgfqpoint{1.884620in}{2.676113in}}%
\pgfpathlineto{\pgfqpoint{1.899962in}{2.663586in}}%
\pgfpathlineto{\pgfqpoint{1.915304in}{2.651059in}}%
\pgfpathlineto{\pgfqpoint{1.930647in}{2.638531in}}%
\pgfpathlineto{\pgfqpoint{1.945989in}{2.626004in}}%
\pgfpathlineto{\pgfqpoint{1.961331in}{2.613476in}}%
\pgfpathlineto{\pgfqpoint{1.976673in}{2.600949in}}%
\pgfpathlineto{\pgfqpoint{1.992016in}{2.588422in}}%
\pgfpathlineto{\pgfqpoint{2.007358in}{2.575894in}}%
\pgfpathlineto{\pgfqpoint{2.022700in}{2.563367in}}%
\pgfpathlineto{\pgfqpoint{2.038042in}{2.550839in}}%
\pgfpathlineto{\pgfqpoint{2.053385in}{2.538312in}}%
\pgfpathlineto{\pgfqpoint{2.068727in}{2.525785in}}%
\pgfpathlineto{\pgfqpoint{2.084069in}{2.513257in}}%
\pgfpathlineto{\pgfqpoint{2.099411in}{2.500730in}}%
\pgfpathlineto{\pgfqpoint{2.114754in}{2.488202in}}%
\pgfpathlineto{\pgfqpoint{2.130096in}{2.475675in}}%
\pgfpathlineto{\pgfqpoint{2.145438in}{2.463148in}}%
\pgfpathlineto{\pgfqpoint{2.160780in}{2.450620in}}%
\pgfpathlineto{\pgfqpoint{2.176123in}{2.438093in}}%
\pgfpathlineto{\pgfqpoint{2.191465in}{2.425566in}}%
\pgfpathlineto{\pgfqpoint{2.206807in}{2.413038in}}%
\pgfpathlineto{\pgfqpoint{2.222149in}{2.400511in}}%
\pgfpathlineto{\pgfqpoint{2.237491in}{2.387983in}}%
\pgfpathlineto{\pgfqpoint{2.252834in}{2.375456in}}%
\pgfpathlineto{\pgfqpoint{2.268176in}{2.362929in}}%
\pgfpathlineto{\pgfqpoint{2.283518in}{2.350401in}}%
\pgfpathlineto{\pgfqpoint{2.298860in}{2.337874in}}%
\pgfpathlineto{\pgfqpoint{2.314203in}{2.325346in}}%
\pgfpathlineto{\pgfqpoint{2.329545in}{2.312819in}}%
\pgfpathlineto{\pgfqpoint{2.344887in}{2.300292in}}%
\pgfpathlineto{\pgfqpoint{2.360229in}{2.287764in}}%
\pgfpathlineto{\pgfqpoint{2.375572in}{2.275237in}}%
\pgfpathlineto{\pgfqpoint{2.390914in}{2.262709in}}%
\pgfpathlineto{\pgfqpoint{2.406256in}{2.250182in}}%
\pgfpathlineto{\pgfqpoint{2.421598in}{2.237655in}}%
\pgfpathlineto{\pgfqpoint{2.436941in}{2.225127in}}%
\pgfpathlineto{\pgfqpoint{2.452283in}{2.212600in}}%
\pgfpathlineto{\pgfqpoint{2.467625in}{2.200072in}}%
\pgfpathlineto{\pgfqpoint{2.482967in}{2.187545in}}%
\pgfpathlineto{\pgfqpoint{2.498310in}{2.175018in}}%
\pgfpathlineto{\pgfqpoint{2.513652in}{2.162490in}}%
\pgfpathlineto{\pgfqpoint{2.528994in}{2.149963in}}%
\pgfpathlineto{\pgfqpoint{2.544336in}{2.137435in}}%
\pgfpathlineto{\pgfqpoint{2.544336in}{2.270943in}}%
\pgfpathlineto{\pgfqpoint{2.544336in}{2.270943in}}%
\pgfpathlineto{\pgfqpoint{2.528994in}{2.282122in}}%
\pgfpathlineto{\pgfqpoint{2.513652in}{2.293301in}}%
\pgfpathlineto{\pgfqpoint{2.498310in}{2.304479in}}%
\pgfpathlineto{\pgfqpoint{2.482967in}{2.315658in}}%
\pgfpathlineto{\pgfqpoint{2.467625in}{2.326837in}}%
\pgfpathlineto{\pgfqpoint{2.452283in}{2.338016in}}%
\pgfpathlineto{\pgfqpoint{2.436941in}{2.349195in}}%
\pgfpathlineto{\pgfqpoint{2.421598in}{2.360374in}}%
\pgfpathlineto{\pgfqpoint{2.406256in}{2.371552in}}%
\pgfpathlineto{\pgfqpoint{2.390914in}{2.382731in}}%
\pgfpathlineto{\pgfqpoint{2.375572in}{2.393910in}}%
\pgfpathlineto{\pgfqpoint{2.360229in}{2.405089in}}%
\pgfpathlineto{\pgfqpoint{2.344887in}{2.416268in}}%
\pgfpathlineto{\pgfqpoint{2.329545in}{2.427447in}}%
\pgfpathlineto{\pgfqpoint{2.314203in}{2.438625in}}%
\pgfpathlineto{\pgfqpoint{2.298860in}{2.449804in}}%
\pgfpathlineto{\pgfqpoint{2.283518in}{2.460983in}}%
\pgfpathlineto{\pgfqpoint{2.268176in}{2.472162in}}%
\pgfpathlineto{\pgfqpoint{2.252834in}{2.483341in}}%
\pgfpathlineto{\pgfqpoint{2.237491in}{2.494520in}}%
\pgfpathlineto{\pgfqpoint{2.222149in}{2.505698in}}%
\pgfpathlineto{\pgfqpoint{2.206807in}{2.516877in}}%
\pgfpathlineto{\pgfqpoint{2.191465in}{2.528056in}}%
\pgfpathlineto{\pgfqpoint{2.176123in}{2.539235in}}%
\pgfpathlineto{\pgfqpoint{2.160780in}{2.550414in}}%
\pgfpathlineto{\pgfqpoint{2.145438in}{2.561593in}}%
\pgfpathlineto{\pgfqpoint{2.130096in}{2.572771in}}%
\pgfpathlineto{\pgfqpoint{2.114754in}{2.583950in}}%
\pgfpathlineto{\pgfqpoint{2.099411in}{2.595129in}}%
\pgfpathlineto{\pgfqpoint{2.084069in}{2.606308in}}%
\pgfpathlineto{\pgfqpoint{2.068727in}{2.617487in}}%
\pgfpathlineto{\pgfqpoint{2.053385in}{2.628666in}}%
\pgfpathlineto{\pgfqpoint{2.038042in}{2.639844in}}%
\pgfpathlineto{\pgfqpoint{2.022700in}{2.651023in}}%
\pgfpathlineto{\pgfqpoint{2.007358in}{2.662202in}}%
\pgfpathlineto{\pgfqpoint{1.992016in}{2.673381in}}%
\pgfpathlineto{\pgfqpoint{1.976673in}{2.684560in}}%
\pgfpathlineto{\pgfqpoint{1.961331in}{2.695739in}}%
\pgfpathlineto{\pgfqpoint{1.945989in}{2.706917in}}%
\pgfpathlineto{\pgfqpoint{1.930647in}{2.718096in}}%
\pgfpathlineto{\pgfqpoint{1.915304in}{2.729275in}}%
\pgfpathlineto{\pgfqpoint{1.899962in}{2.740454in}}%
\pgfpathlineto{\pgfqpoint{1.884620in}{2.751633in}}%
\pgfpathlineto{\pgfqpoint{1.869278in}{2.762812in}}%
\pgfpathlineto{\pgfqpoint{1.853935in}{2.773990in}}%
\pgfpathlineto{\pgfqpoint{1.838593in}{2.785169in}}%
\pgfpathlineto{\pgfqpoint{1.823251in}{2.796348in}}%
\pgfpathlineto{\pgfqpoint{1.807909in}{2.807527in}}%
\pgfpathlineto{\pgfqpoint{1.792567in}{2.818706in}}%
\pgfpathlineto{\pgfqpoint{1.777224in}{2.829885in}}%
\pgfpathlineto{\pgfqpoint{1.761882in}{2.841064in}}%
\pgfpathlineto{\pgfqpoint{1.746540in}{2.852242in}}%
\pgfpathlineto{\pgfqpoint{1.731198in}{2.863421in}}%
\pgfpathlineto{\pgfqpoint{1.715855in}{2.874600in}}%
\pgfpathlineto{\pgfqpoint{1.700513in}{2.885779in}}%
\pgfpathlineto{\pgfqpoint{1.685171in}{2.896958in}}%
\pgfpathlineto{\pgfqpoint{1.669829in}{2.908137in}}%
\pgfpathlineto{\pgfqpoint{1.654486in}{2.919315in}}%
\pgfpathlineto{\pgfqpoint{1.639144in}{2.930494in}}%
\pgfpathlineto{\pgfqpoint{1.623802in}{2.941673in}}%
\pgfpathlineto{\pgfqpoint{1.608460in}{2.952852in}}%
\pgfpathlineto{\pgfqpoint{1.593117in}{2.964031in}}%
\pgfpathlineto{\pgfqpoint{1.577775in}{2.975210in}}%
\pgfpathlineto{\pgfqpoint{1.562433in}{2.986388in}}%
\pgfpathlineto{\pgfqpoint{1.547091in}{2.997567in}}%
\pgfpathlineto{\pgfqpoint{1.531748in}{3.008746in}}%
\pgfpathlineto{\pgfqpoint{1.516406in}{3.019925in}}%
\pgfpathlineto{\pgfqpoint{1.501064in}{3.031104in}}%
\pgfpathlineto{\pgfqpoint{1.485722in}{3.042283in}}%
\pgfpathlineto{\pgfqpoint{1.470379in}{3.053461in}}%
\pgfpathlineto{\pgfqpoint{1.455037in}{3.064640in}}%
\pgfpathlineto{\pgfqpoint{1.439695in}{3.075819in}}%
\pgfpathlineto{\pgfqpoint{1.424353in}{3.086998in}}%
\pgfpathlineto{\pgfqpoint{1.409011in}{3.098177in}}%
\pgfpathlineto{\pgfqpoint{1.393668in}{3.109356in}}%
\pgfpathlineto{\pgfqpoint{1.378326in}{3.120534in}}%
\pgfpathlineto{\pgfqpoint{1.362984in}{3.131713in}}%
\pgfpathlineto{\pgfqpoint{1.347642in}{3.142892in}}%
\pgfpathlineto{\pgfqpoint{1.332299in}{3.154071in}}%
\pgfpathlineto{\pgfqpoint{1.316957in}{3.165250in}}%
\pgfpathlineto{\pgfqpoint{1.301615in}{3.176429in}}%
\pgfpathlineto{\pgfqpoint{1.286273in}{3.187607in}}%
\pgfpathlineto{\pgfqpoint{1.270930in}{3.198786in}}%
\pgfpathlineto{\pgfqpoint{1.255588in}{3.209965in}}%
\pgfpathlineto{\pgfqpoint{1.240246in}{3.221144in}}%
\pgfpathlineto{\pgfqpoint{1.224904in}{3.232323in}}%
\pgfpathlineto{\pgfqpoint{1.209561in}{3.243502in}}%
\pgfpathlineto{\pgfqpoint{1.194219in}{3.254680in}}%
\pgfpathlineto{\pgfqpoint{1.178877in}{3.265859in}}%
\pgfpathlineto{\pgfqpoint{1.163535in}{3.277038in}}%
\pgfpathlineto{\pgfqpoint{1.148192in}{3.288217in}}%
\pgfpathlineto{\pgfqpoint{1.132850in}{3.299396in}}%
\pgfpathlineto{\pgfqpoint{1.117508in}{3.310575in}}%
\pgfpathlineto{\pgfqpoint{1.102166in}{3.321753in}}%
\pgfpathlineto{\pgfqpoint{1.086824in}{3.332932in}}%
\pgfpathlineto{\pgfqpoint{1.071481in}{3.344111in}}%
\pgfpathlineto{\pgfqpoint{1.056139in}{3.355290in}}%
\pgfpathlineto{\pgfqpoint{1.040797in}{3.366469in}}%
\pgfpathlineto{\pgfqpoint{1.025455in}{3.377648in}}%
\pgfpathclose%
\pgfusepath{stroke,fill}%
}%
\begin{pgfscope}%
\pgfsys@transformshift{0.000000in}{0.000000in}%
\pgfsys@useobject{currentmarker}{}%
\end{pgfscope}%
\end{pgfscope}%
\begin{pgfscope}%
\pgfsetbuttcap%
\pgfsetroundjoin%
\definecolor{currentfill}{rgb}{0.000000,0.000000,0.000000}%
\pgfsetfillcolor{currentfill}%
\pgfsetlinewidth{0.803000pt}%
\definecolor{currentstroke}{rgb}{0.000000,0.000000,0.000000}%
\pgfsetstrokecolor{currentstroke}%
\pgfsetdash{}{0pt}%
\pgfsys@defobject{currentmarker}{\pgfqpoint{0.000000in}{-0.048611in}}{\pgfqpoint{0.000000in}{0.000000in}}{%
\pgfpathmoveto{\pgfqpoint{0.000000in}{0.000000in}}%
\pgfpathlineto{\pgfqpoint{0.000000in}{-0.048611in}}%
\pgfusepath{stroke,fill}%
}%
\begin{pgfscope}%
\pgfsys@transformshift{1.025455in}{0.440000in}%
\pgfsys@useobject{currentmarker}{}%
\end{pgfscope}%
\end{pgfscope}%
\begin{pgfscope}%
\definecolor{textcolor}{rgb}{0.000000,0.000000,0.000000}%
\pgfsetstrokecolor{textcolor}%
\pgfsetfillcolor{textcolor}%
\pgftext[x=1.025455in,y=0.342778in,,top]{\color{textcolor}\rmfamily\fontsize{10.000000}{12.000000}\selectfont \(\displaystyle {0}\)}%
\end{pgfscope}%
\begin{pgfscope}%
\pgfsetbuttcap%
\pgfsetroundjoin%
\definecolor{currentfill}{rgb}{0.000000,0.000000,0.000000}%
\pgfsetfillcolor{currentfill}%
\pgfsetlinewidth{0.803000pt}%
\definecolor{currentstroke}{rgb}{0.000000,0.000000,0.000000}%
\pgfsetstrokecolor{currentstroke}%
\pgfsetdash{}{0pt}%
\pgfsys@defobject{currentmarker}{\pgfqpoint{0.000000in}{-0.048611in}}{\pgfqpoint{0.000000in}{0.000000in}}{%
\pgfpathmoveto{\pgfqpoint{0.000000in}{0.000000in}}%
\pgfpathlineto{\pgfqpoint{0.000000in}{-0.048611in}}%
\pgfusepath{stroke,fill}%
}%
\begin{pgfscope}%
\pgfsys@transformshift{1.976539in}{0.440000in}%
\pgfsys@useobject{currentmarker}{}%
\end{pgfscope}%
\end{pgfscope}%
\begin{pgfscope}%
\definecolor{textcolor}{rgb}{0.000000,0.000000,0.000000}%
\pgfsetstrokecolor{textcolor}%
\pgfsetfillcolor{textcolor}%
\pgftext[x=1.976539in,y=0.342778in,,top]{\color{textcolor}\rmfamily\fontsize{10.000000}{12.000000}\selectfont \(\displaystyle {10}\)}%
\end{pgfscope}%
\begin{pgfscope}%
\pgfsetbuttcap%
\pgfsetroundjoin%
\definecolor{currentfill}{rgb}{0.000000,0.000000,0.000000}%
\pgfsetfillcolor{currentfill}%
\pgfsetlinewidth{0.803000pt}%
\definecolor{currentstroke}{rgb}{0.000000,0.000000,0.000000}%
\pgfsetstrokecolor{currentstroke}%
\pgfsetdash{}{0pt}%
\pgfsys@defobject{currentmarker}{\pgfqpoint{0.000000in}{-0.048611in}}{\pgfqpoint{0.000000in}{0.000000in}}{%
\pgfpathmoveto{\pgfqpoint{0.000000in}{0.000000in}}%
\pgfpathlineto{\pgfqpoint{0.000000in}{-0.048611in}}%
\pgfusepath{stroke,fill}%
}%
\begin{pgfscope}%
\pgfsys@transformshift{2.927623in}{0.440000in}%
\pgfsys@useobject{currentmarker}{}%
\end{pgfscope}%
\end{pgfscope}%
\begin{pgfscope}%
\definecolor{textcolor}{rgb}{0.000000,0.000000,0.000000}%
\pgfsetstrokecolor{textcolor}%
\pgfsetfillcolor{textcolor}%
\pgftext[x=2.927623in,y=0.342778in,,top]{\color{textcolor}\rmfamily\fontsize{10.000000}{12.000000}\selectfont \(\displaystyle {20}\)}%
\end{pgfscope}%
\begin{pgfscope}%
\pgfsetbuttcap%
\pgfsetroundjoin%
\definecolor{currentfill}{rgb}{0.000000,0.000000,0.000000}%
\pgfsetfillcolor{currentfill}%
\pgfsetlinewidth{0.803000pt}%
\definecolor{currentstroke}{rgb}{0.000000,0.000000,0.000000}%
\pgfsetstrokecolor{currentstroke}%
\pgfsetdash{}{0pt}%
\pgfsys@defobject{currentmarker}{\pgfqpoint{0.000000in}{-0.048611in}}{\pgfqpoint{0.000000in}{0.000000in}}{%
\pgfpathmoveto{\pgfqpoint{0.000000in}{0.000000in}}%
\pgfpathlineto{\pgfqpoint{0.000000in}{-0.048611in}}%
\pgfusepath{stroke,fill}%
}%
\begin{pgfscope}%
\pgfsys@transformshift{3.878708in}{0.440000in}%
\pgfsys@useobject{currentmarker}{}%
\end{pgfscope}%
\end{pgfscope}%
\begin{pgfscope}%
\definecolor{textcolor}{rgb}{0.000000,0.000000,0.000000}%
\pgfsetstrokecolor{textcolor}%
\pgfsetfillcolor{textcolor}%
\pgftext[x=3.878708in,y=0.342778in,,top]{\color{textcolor}\rmfamily\fontsize{10.000000}{12.000000}\selectfont \(\displaystyle {30}\)}%
\end{pgfscope}%
\begin{pgfscope}%
\pgfsetbuttcap%
\pgfsetroundjoin%
\definecolor{currentfill}{rgb}{0.000000,0.000000,0.000000}%
\pgfsetfillcolor{currentfill}%
\pgfsetlinewidth{0.803000pt}%
\definecolor{currentstroke}{rgb}{0.000000,0.000000,0.000000}%
\pgfsetstrokecolor{currentstroke}%
\pgfsetdash{}{0pt}%
\pgfsys@defobject{currentmarker}{\pgfqpoint{0.000000in}{-0.048611in}}{\pgfqpoint{0.000000in}{0.000000in}}{%
\pgfpathmoveto{\pgfqpoint{0.000000in}{0.000000in}}%
\pgfpathlineto{\pgfqpoint{0.000000in}{-0.048611in}}%
\pgfusepath{stroke,fill}%
}%
\begin{pgfscope}%
\pgfsys@transformshift{4.829792in}{0.440000in}%
\pgfsys@useobject{currentmarker}{}%
\end{pgfscope}%
\end{pgfscope}%
\begin{pgfscope}%
\definecolor{textcolor}{rgb}{0.000000,0.000000,0.000000}%
\pgfsetstrokecolor{textcolor}%
\pgfsetfillcolor{textcolor}%
\pgftext[x=4.829792in,y=0.342778in,,top]{\color{textcolor}\rmfamily\fontsize{10.000000}{12.000000}\selectfont \(\displaystyle {40}\)}%
\end{pgfscope}%
\begin{pgfscope}%
\definecolor{textcolor}{rgb}{0.000000,0.000000,0.000000}%
\pgfsetstrokecolor{textcolor}%
\pgfsetfillcolor{textcolor}%
\pgftext[x=3.280000in,y=0.163766in,,top]{\color{textcolor}\rmfamily\fontsize{10.000000}{12.000000}\selectfont Time \(\displaystyle t\), \(\displaystyle t =\sec\)}%
\end{pgfscope}%
\begin{pgfscope}%
\pgfsetbuttcap%
\pgfsetroundjoin%
\definecolor{currentfill}{rgb}{0.000000,0.000000,0.000000}%
\pgfsetfillcolor{currentfill}%
\pgfsetlinewidth{0.803000pt}%
\definecolor{currentstroke}{rgb}{0.000000,0.000000,0.000000}%
\pgfsetstrokecolor{currentstroke}%
\pgfsetdash{}{0pt}%
\pgfsys@defobject{currentmarker}{\pgfqpoint{-0.048611in}{0.000000in}}{\pgfqpoint{-0.000000in}{0.000000in}}{%
\pgfpathmoveto{\pgfqpoint{-0.000000in}{0.000000in}}%
\pgfpathlineto{\pgfqpoint{-0.048611in}{0.000000in}}%
\pgfusepath{stroke,fill}%
}%
\begin{pgfscope}%
\pgfsys@transformshift{0.800000in}{0.988969in}%
\pgfsys@useobject{currentmarker}{}%
\end{pgfscope}%
\end{pgfscope}%
\begin{pgfscope}%
\definecolor{textcolor}{rgb}{0.000000,0.000000,0.000000}%
\pgfsetstrokecolor{textcolor}%
\pgfsetfillcolor{textcolor}%
\pgftext[x=0.414775in, y=0.940744in, left, base]{\color{textcolor}\rmfamily\fontsize{10.000000}{12.000000}\selectfont \(\displaystyle {10^{-2}}\)}%
\end{pgfscope}%
\begin{pgfscope}%
\pgfsetbuttcap%
\pgfsetroundjoin%
\definecolor{currentfill}{rgb}{0.000000,0.000000,0.000000}%
\pgfsetfillcolor{currentfill}%
\pgfsetlinewidth{0.803000pt}%
\definecolor{currentstroke}{rgb}{0.000000,0.000000,0.000000}%
\pgfsetstrokecolor{currentstroke}%
\pgfsetdash{}{0pt}%
\pgfsys@defobject{currentmarker}{\pgfqpoint{-0.048611in}{0.000000in}}{\pgfqpoint{-0.000000in}{0.000000in}}{%
\pgfpathmoveto{\pgfqpoint{-0.000000in}{0.000000in}}%
\pgfpathlineto{\pgfqpoint{-0.048611in}{0.000000in}}%
\pgfusepath{stroke,fill}%
}%
\begin{pgfscope}%
\pgfsys@transformshift{0.800000in}{2.183308in}%
\pgfsys@useobject{currentmarker}{}%
\end{pgfscope}%
\end{pgfscope}%
\begin{pgfscope}%
\definecolor{textcolor}{rgb}{0.000000,0.000000,0.000000}%
\pgfsetstrokecolor{textcolor}%
\pgfsetfillcolor{textcolor}%
\pgftext[x=0.414775in, y=2.135083in, left, base]{\color{textcolor}\rmfamily\fontsize{10.000000}{12.000000}\selectfont \(\displaystyle {10^{-1}}\)}%
\end{pgfscope}%
\begin{pgfscope}%
\pgfsetbuttcap%
\pgfsetroundjoin%
\definecolor{currentfill}{rgb}{0.000000,0.000000,0.000000}%
\pgfsetfillcolor{currentfill}%
\pgfsetlinewidth{0.803000pt}%
\definecolor{currentstroke}{rgb}{0.000000,0.000000,0.000000}%
\pgfsetstrokecolor{currentstroke}%
\pgfsetdash{}{0pt}%
\pgfsys@defobject{currentmarker}{\pgfqpoint{-0.048611in}{0.000000in}}{\pgfqpoint{-0.000000in}{0.000000in}}{%
\pgfpathmoveto{\pgfqpoint{-0.000000in}{0.000000in}}%
\pgfpathlineto{\pgfqpoint{-0.048611in}{0.000000in}}%
\pgfusepath{stroke,fill}%
}%
\begin{pgfscope}%
\pgfsys@transformshift{0.800000in}{3.377648in}%
\pgfsys@useobject{currentmarker}{}%
\end{pgfscope}%
\end{pgfscope}%
\begin{pgfscope}%
\definecolor{textcolor}{rgb}{0.000000,0.000000,0.000000}%
\pgfsetstrokecolor{textcolor}%
\pgfsetfillcolor{textcolor}%
\pgftext[x=0.501581in, y=3.329422in, left, base]{\color{textcolor}\rmfamily\fontsize{10.000000}{12.000000}\selectfont \(\displaystyle {10^{0}}\)}%
\end{pgfscope}%
\begin{pgfscope}%
\pgfsetbuttcap%
\pgfsetroundjoin%
\definecolor{currentfill}{rgb}{0.000000,0.000000,0.000000}%
\pgfsetfillcolor{currentfill}%
\pgfsetlinewidth{0.602250pt}%
\definecolor{currentstroke}{rgb}{0.000000,0.000000,0.000000}%
\pgfsetstrokecolor{currentstroke}%
\pgfsetdash{}{0pt}%
\pgfsys@defobject{currentmarker}{\pgfqpoint{-0.027778in}{0.000000in}}{\pgfqpoint{-0.000000in}{0.000000in}}{%
\pgfpathmoveto{\pgfqpoint{-0.000000in}{0.000000in}}%
\pgfpathlineto{\pgfqpoint{-0.027778in}{0.000000in}}%
\pgfusepath{stroke,fill}%
}%
\begin{pgfscope}%
\pgfsys@transformshift{0.800000in}{0.513693in}%
\pgfsys@useobject{currentmarker}{}%
\end{pgfscope}%
\end{pgfscope}%
\begin{pgfscope}%
\pgfsetbuttcap%
\pgfsetroundjoin%
\definecolor{currentfill}{rgb}{0.000000,0.000000,0.000000}%
\pgfsetfillcolor{currentfill}%
\pgfsetlinewidth{0.602250pt}%
\definecolor{currentstroke}{rgb}{0.000000,0.000000,0.000000}%
\pgfsetstrokecolor{currentstroke}%
\pgfsetdash{}{0pt}%
\pgfsys@defobject{currentmarker}{\pgfqpoint{-0.027778in}{0.000000in}}{\pgfqpoint{-0.000000in}{0.000000in}}{%
\pgfpathmoveto{\pgfqpoint{-0.000000in}{0.000000in}}%
\pgfpathlineto{\pgfqpoint{-0.027778in}{0.000000in}}%
\pgfusepath{stroke,fill}%
}%
\begin{pgfscope}%
\pgfsys@transformshift{0.800000in}{0.629437in}%
\pgfsys@useobject{currentmarker}{}%
\end{pgfscope}%
\end{pgfscope}%
\begin{pgfscope}%
\pgfsetbuttcap%
\pgfsetroundjoin%
\definecolor{currentfill}{rgb}{0.000000,0.000000,0.000000}%
\pgfsetfillcolor{currentfill}%
\pgfsetlinewidth{0.602250pt}%
\definecolor{currentstroke}{rgb}{0.000000,0.000000,0.000000}%
\pgfsetstrokecolor{currentstroke}%
\pgfsetdash{}{0pt}%
\pgfsys@defobject{currentmarker}{\pgfqpoint{-0.027778in}{0.000000in}}{\pgfqpoint{-0.000000in}{0.000000in}}{%
\pgfpathmoveto{\pgfqpoint{-0.000000in}{0.000000in}}%
\pgfpathlineto{\pgfqpoint{-0.027778in}{0.000000in}}%
\pgfusepath{stroke,fill}%
}%
\begin{pgfscope}%
\pgfsys@transformshift{0.800000in}{0.724006in}%
\pgfsys@useobject{currentmarker}{}%
\end{pgfscope}%
\end{pgfscope}%
\begin{pgfscope}%
\pgfsetbuttcap%
\pgfsetroundjoin%
\definecolor{currentfill}{rgb}{0.000000,0.000000,0.000000}%
\pgfsetfillcolor{currentfill}%
\pgfsetlinewidth{0.602250pt}%
\definecolor{currentstroke}{rgb}{0.000000,0.000000,0.000000}%
\pgfsetstrokecolor{currentstroke}%
\pgfsetdash{}{0pt}%
\pgfsys@defobject{currentmarker}{\pgfqpoint{-0.027778in}{0.000000in}}{\pgfqpoint{-0.000000in}{0.000000in}}{%
\pgfpathmoveto{\pgfqpoint{-0.000000in}{0.000000in}}%
\pgfpathlineto{\pgfqpoint{-0.027778in}{0.000000in}}%
\pgfusepath{stroke,fill}%
}%
\begin{pgfscope}%
\pgfsys@transformshift{0.800000in}{0.803963in}%
\pgfsys@useobject{currentmarker}{}%
\end{pgfscope}%
\end{pgfscope}%
\begin{pgfscope}%
\pgfsetbuttcap%
\pgfsetroundjoin%
\definecolor{currentfill}{rgb}{0.000000,0.000000,0.000000}%
\pgfsetfillcolor{currentfill}%
\pgfsetlinewidth{0.602250pt}%
\definecolor{currentstroke}{rgb}{0.000000,0.000000,0.000000}%
\pgfsetstrokecolor{currentstroke}%
\pgfsetdash{}{0pt}%
\pgfsys@defobject{currentmarker}{\pgfqpoint{-0.027778in}{0.000000in}}{\pgfqpoint{-0.000000in}{0.000000in}}{%
\pgfpathmoveto{\pgfqpoint{-0.000000in}{0.000000in}}%
\pgfpathlineto{\pgfqpoint{-0.027778in}{0.000000in}}%
\pgfusepath{stroke,fill}%
}%
\begin{pgfscope}%
\pgfsys@transformshift{0.800000in}{0.873225in}%
\pgfsys@useobject{currentmarker}{}%
\end{pgfscope}%
\end{pgfscope}%
\begin{pgfscope}%
\pgfsetbuttcap%
\pgfsetroundjoin%
\definecolor{currentfill}{rgb}{0.000000,0.000000,0.000000}%
\pgfsetfillcolor{currentfill}%
\pgfsetlinewidth{0.602250pt}%
\definecolor{currentstroke}{rgb}{0.000000,0.000000,0.000000}%
\pgfsetstrokecolor{currentstroke}%
\pgfsetdash{}{0pt}%
\pgfsys@defobject{currentmarker}{\pgfqpoint{-0.027778in}{0.000000in}}{\pgfqpoint{-0.000000in}{0.000000in}}{%
\pgfpathmoveto{\pgfqpoint{-0.000000in}{0.000000in}}%
\pgfpathlineto{\pgfqpoint{-0.027778in}{0.000000in}}%
\pgfusepath{stroke,fill}%
}%
\begin{pgfscope}%
\pgfsys@transformshift{0.800000in}{0.934319in}%
\pgfsys@useobject{currentmarker}{}%
\end{pgfscope}%
\end{pgfscope}%
\begin{pgfscope}%
\pgfsetbuttcap%
\pgfsetroundjoin%
\definecolor{currentfill}{rgb}{0.000000,0.000000,0.000000}%
\pgfsetfillcolor{currentfill}%
\pgfsetlinewidth{0.602250pt}%
\definecolor{currentstroke}{rgb}{0.000000,0.000000,0.000000}%
\pgfsetstrokecolor{currentstroke}%
\pgfsetdash{}{0pt}%
\pgfsys@defobject{currentmarker}{\pgfqpoint{-0.027778in}{0.000000in}}{\pgfqpoint{-0.000000in}{0.000000in}}{%
\pgfpathmoveto{\pgfqpoint{-0.000000in}{0.000000in}}%
\pgfpathlineto{\pgfqpoint{-0.027778in}{0.000000in}}%
\pgfusepath{stroke,fill}%
}%
\begin{pgfscope}%
\pgfsys@transformshift{0.800000in}{1.348501in}%
\pgfsys@useobject{currentmarker}{}%
\end{pgfscope}%
\end{pgfscope}%
\begin{pgfscope}%
\pgfsetbuttcap%
\pgfsetroundjoin%
\definecolor{currentfill}{rgb}{0.000000,0.000000,0.000000}%
\pgfsetfillcolor{currentfill}%
\pgfsetlinewidth{0.602250pt}%
\definecolor{currentstroke}{rgb}{0.000000,0.000000,0.000000}%
\pgfsetstrokecolor{currentstroke}%
\pgfsetdash{}{0pt}%
\pgfsys@defobject{currentmarker}{\pgfqpoint{-0.027778in}{0.000000in}}{\pgfqpoint{-0.000000in}{0.000000in}}{%
\pgfpathmoveto{\pgfqpoint{-0.000000in}{0.000000in}}%
\pgfpathlineto{\pgfqpoint{-0.027778in}{0.000000in}}%
\pgfusepath{stroke,fill}%
}%
\begin{pgfscope}%
\pgfsys@transformshift{0.800000in}{1.558814in}%
\pgfsys@useobject{currentmarker}{}%
\end{pgfscope}%
\end{pgfscope}%
\begin{pgfscope}%
\pgfsetbuttcap%
\pgfsetroundjoin%
\definecolor{currentfill}{rgb}{0.000000,0.000000,0.000000}%
\pgfsetfillcolor{currentfill}%
\pgfsetlinewidth{0.602250pt}%
\definecolor{currentstroke}{rgb}{0.000000,0.000000,0.000000}%
\pgfsetstrokecolor{currentstroke}%
\pgfsetdash{}{0pt}%
\pgfsys@defobject{currentmarker}{\pgfqpoint{-0.027778in}{0.000000in}}{\pgfqpoint{-0.000000in}{0.000000in}}{%
\pgfpathmoveto{\pgfqpoint{-0.000000in}{0.000000in}}%
\pgfpathlineto{\pgfqpoint{-0.027778in}{0.000000in}}%
\pgfusepath{stroke,fill}%
}%
\begin{pgfscope}%
\pgfsys@transformshift{0.800000in}{1.708033in}%
\pgfsys@useobject{currentmarker}{}%
\end{pgfscope}%
\end{pgfscope}%
\begin{pgfscope}%
\pgfsetbuttcap%
\pgfsetroundjoin%
\definecolor{currentfill}{rgb}{0.000000,0.000000,0.000000}%
\pgfsetfillcolor{currentfill}%
\pgfsetlinewidth{0.602250pt}%
\definecolor{currentstroke}{rgb}{0.000000,0.000000,0.000000}%
\pgfsetstrokecolor{currentstroke}%
\pgfsetdash{}{0pt}%
\pgfsys@defobject{currentmarker}{\pgfqpoint{-0.027778in}{0.000000in}}{\pgfqpoint{-0.000000in}{0.000000in}}{%
\pgfpathmoveto{\pgfqpoint{-0.000000in}{0.000000in}}%
\pgfpathlineto{\pgfqpoint{-0.027778in}{0.000000in}}%
\pgfusepath{stroke,fill}%
}%
\begin{pgfscope}%
\pgfsys@transformshift{0.800000in}{1.823776in}%
\pgfsys@useobject{currentmarker}{}%
\end{pgfscope}%
\end{pgfscope}%
\begin{pgfscope}%
\pgfsetbuttcap%
\pgfsetroundjoin%
\definecolor{currentfill}{rgb}{0.000000,0.000000,0.000000}%
\pgfsetfillcolor{currentfill}%
\pgfsetlinewidth{0.602250pt}%
\definecolor{currentstroke}{rgb}{0.000000,0.000000,0.000000}%
\pgfsetstrokecolor{currentstroke}%
\pgfsetdash{}{0pt}%
\pgfsys@defobject{currentmarker}{\pgfqpoint{-0.027778in}{0.000000in}}{\pgfqpoint{-0.000000in}{0.000000in}}{%
\pgfpathmoveto{\pgfqpoint{-0.000000in}{0.000000in}}%
\pgfpathlineto{\pgfqpoint{-0.027778in}{0.000000in}}%
\pgfusepath{stroke,fill}%
}%
\begin{pgfscope}%
\pgfsys@transformshift{0.800000in}{1.918346in}%
\pgfsys@useobject{currentmarker}{}%
\end{pgfscope}%
\end{pgfscope}%
\begin{pgfscope}%
\pgfsetbuttcap%
\pgfsetroundjoin%
\definecolor{currentfill}{rgb}{0.000000,0.000000,0.000000}%
\pgfsetfillcolor{currentfill}%
\pgfsetlinewidth{0.602250pt}%
\definecolor{currentstroke}{rgb}{0.000000,0.000000,0.000000}%
\pgfsetstrokecolor{currentstroke}%
\pgfsetdash{}{0pt}%
\pgfsys@defobject{currentmarker}{\pgfqpoint{-0.027778in}{0.000000in}}{\pgfqpoint{-0.000000in}{0.000000in}}{%
\pgfpathmoveto{\pgfqpoint{-0.000000in}{0.000000in}}%
\pgfpathlineto{\pgfqpoint{-0.027778in}{0.000000in}}%
\pgfusepath{stroke,fill}%
}%
\begin{pgfscope}%
\pgfsys@transformshift{0.800000in}{1.998303in}%
\pgfsys@useobject{currentmarker}{}%
\end{pgfscope}%
\end{pgfscope}%
\begin{pgfscope}%
\pgfsetbuttcap%
\pgfsetroundjoin%
\definecolor{currentfill}{rgb}{0.000000,0.000000,0.000000}%
\pgfsetfillcolor{currentfill}%
\pgfsetlinewidth{0.602250pt}%
\definecolor{currentstroke}{rgb}{0.000000,0.000000,0.000000}%
\pgfsetstrokecolor{currentstroke}%
\pgfsetdash{}{0pt}%
\pgfsys@defobject{currentmarker}{\pgfqpoint{-0.027778in}{0.000000in}}{\pgfqpoint{-0.000000in}{0.000000in}}{%
\pgfpathmoveto{\pgfqpoint{-0.000000in}{0.000000in}}%
\pgfpathlineto{\pgfqpoint{-0.027778in}{0.000000in}}%
\pgfusepath{stroke,fill}%
}%
\begin{pgfscope}%
\pgfsys@transformshift{0.800000in}{2.067565in}%
\pgfsys@useobject{currentmarker}{}%
\end{pgfscope}%
\end{pgfscope}%
\begin{pgfscope}%
\pgfsetbuttcap%
\pgfsetroundjoin%
\definecolor{currentfill}{rgb}{0.000000,0.000000,0.000000}%
\pgfsetfillcolor{currentfill}%
\pgfsetlinewidth{0.602250pt}%
\definecolor{currentstroke}{rgb}{0.000000,0.000000,0.000000}%
\pgfsetstrokecolor{currentstroke}%
\pgfsetdash{}{0pt}%
\pgfsys@defobject{currentmarker}{\pgfqpoint{-0.027778in}{0.000000in}}{\pgfqpoint{-0.000000in}{0.000000in}}{%
\pgfpathmoveto{\pgfqpoint{-0.000000in}{0.000000in}}%
\pgfpathlineto{\pgfqpoint{-0.027778in}{0.000000in}}%
\pgfusepath{stroke,fill}%
}%
\begin{pgfscope}%
\pgfsys@transformshift{0.800000in}{2.128658in}%
\pgfsys@useobject{currentmarker}{}%
\end{pgfscope}%
\end{pgfscope}%
\begin{pgfscope}%
\pgfsetbuttcap%
\pgfsetroundjoin%
\definecolor{currentfill}{rgb}{0.000000,0.000000,0.000000}%
\pgfsetfillcolor{currentfill}%
\pgfsetlinewidth{0.602250pt}%
\definecolor{currentstroke}{rgb}{0.000000,0.000000,0.000000}%
\pgfsetstrokecolor{currentstroke}%
\pgfsetdash{}{0pt}%
\pgfsys@defobject{currentmarker}{\pgfqpoint{-0.027778in}{0.000000in}}{\pgfqpoint{-0.000000in}{0.000000in}}{%
\pgfpathmoveto{\pgfqpoint{-0.000000in}{0.000000in}}%
\pgfpathlineto{\pgfqpoint{-0.027778in}{0.000000in}}%
\pgfusepath{stroke,fill}%
}%
\begin{pgfscope}%
\pgfsys@transformshift{0.800000in}{2.542840in}%
\pgfsys@useobject{currentmarker}{}%
\end{pgfscope}%
\end{pgfscope}%
\begin{pgfscope}%
\pgfsetbuttcap%
\pgfsetroundjoin%
\definecolor{currentfill}{rgb}{0.000000,0.000000,0.000000}%
\pgfsetfillcolor{currentfill}%
\pgfsetlinewidth{0.602250pt}%
\definecolor{currentstroke}{rgb}{0.000000,0.000000,0.000000}%
\pgfsetstrokecolor{currentstroke}%
\pgfsetdash{}{0pt}%
\pgfsys@defobject{currentmarker}{\pgfqpoint{-0.027778in}{0.000000in}}{\pgfqpoint{-0.000000in}{0.000000in}}{%
\pgfpathmoveto{\pgfqpoint{-0.000000in}{0.000000in}}%
\pgfpathlineto{\pgfqpoint{-0.027778in}{0.000000in}}%
\pgfusepath{stroke,fill}%
}%
\begin{pgfscope}%
\pgfsys@transformshift{0.800000in}{2.753153in}%
\pgfsys@useobject{currentmarker}{}%
\end{pgfscope}%
\end{pgfscope}%
\begin{pgfscope}%
\pgfsetbuttcap%
\pgfsetroundjoin%
\definecolor{currentfill}{rgb}{0.000000,0.000000,0.000000}%
\pgfsetfillcolor{currentfill}%
\pgfsetlinewidth{0.602250pt}%
\definecolor{currentstroke}{rgb}{0.000000,0.000000,0.000000}%
\pgfsetstrokecolor{currentstroke}%
\pgfsetdash{}{0pt}%
\pgfsys@defobject{currentmarker}{\pgfqpoint{-0.027778in}{0.000000in}}{\pgfqpoint{-0.000000in}{0.000000in}}{%
\pgfpathmoveto{\pgfqpoint{-0.000000in}{0.000000in}}%
\pgfpathlineto{\pgfqpoint{-0.027778in}{0.000000in}}%
\pgfusepath{stroke,fill}%
}%
\begin{pgfscope}%
\pgfsys@transformshift{0.800000in}{2.902372in}%
\pgfsys@useobject{currentmarker}{}%
\end{pgfscope}%
\end{pgfscope}%
\begin{pgfscope}%
\pgfsetbuttcap%
\pgfsetroundjoin%
\definecolor{currentfill}{rgb}{0.000000,0.000000,0.000000}%
\pgfsetfillcolor{currentfill}%
\pgfsetlinewidth{0.602250pt}%
\definecolor{currentstroke}{rgb}{0.000000,0.000000,0.000000}%
\pgfsetstrokecolor{currentstroke}%
\pgfsetdash{}{0pt}%
\pgfsys@defobject{currentmarker}{\pgfqpoint{-0.027778in}{0.000000in}}{\pgfqpoint{-0.000000in}{0.000000in}}{%
\pgfpathmoveto{\pgfqpoint{-0.000000in}{0.000000in}}%
\pgfpathlineto{\pgfqpoint{-0.027778in}{0.000000in}}%
\pgfusepath{stroke,fill}%
}%
\begin{pgfscope}%
\pgfsys@transformshift{0.800000in}{3.018116in}%
\pgfsys@useobject{currentmarker}{}%
\end{pgfscope}%
\end{pgfscope}%
\begin{pgfscope}%
\pgfsetbuttcap%
\pgfsetroundjoin%
\definecolor{currentfill}{rgb}{0.000000,0.000000,0.000000}%
\pgfsetfillcolor{currentfill}%
\pgfsetlinewidth{0.602250pt}%
\definecolor{currentstroke}{rgb}{0.000000,0.000000,0.000000}%
\pgfsetstrokecolor{currentstroke}%
\pgfsetdash{}{0pt}%
\pgfsys@defobject{currentmarker}{\pgfqpoint{-0.027778in}{0.000000in}}{\pgfqpoint{-0.000000in}{0.000000in}}{%
\pgfpathmoveto{\pgfqpoint{-0.000000in}{0.000000in}}%
\pgfpathlineto{\pgfqpoint{-0.027778in}{0.000000in}}%
\pgfusepath{stroke,fill}%
}%
\begin{pgfscope}%
\pgfsys@transformshift{0.800000in}{3.112685in}%
\pgfsys@useobject{currentmarker}{}%
\end{pgfscope}%
\end{pgfscope}%
\begin{pgfscope}%
\pgfsetbuttcap%
\pgfsetroundjoin%
\definecolor{currentfill}{rgb}{0.000000,0.000000,0.000000}%
\pgfsetfillcolor{currentfill}%
\pgfsetlinewidth{0.602250pt}%
\definecolor{currentstroke}{rgb}{0.000000,0.000000,0.000000}%
\pgfsetstrokecolor{currentstroke}%
\pgfsetdash{}{0pt}%
\pgfsys@defobject{currentmarker}{\pgfqpoint{-0.027778in}{0.000000in}}{\pgfqpoint{-0.000000in}{0.000000in}}{%
\pgfpathmoveto{\pgfqpoint{-0.000000in}{0.000000in}}%
\pgfpathlineto{\pgfqpoint{-0.027778in}{0.000000in}}%
\pgfusepath{stroke,fill}%
}%
\begin{pgfscope}%
\pgfsys@transformshift{0.800000in}{3.192642in}%
\pgfsys@useobject{currentmarker}{}%
\end{pgfscope}%
\end{pgfscope}%
\begin{pgfscope}%
\pgfsetbuttcap%
\pgfsetroundjoin%
\definecolor{currentfill}{rgb}{0.000000,0.000000,0.000000}%
\pgfsetfillcolor{currentfill}%
\pgfsetlinewidth{0.602250pt}%
\definecolor{currentstroke}{rgb}{0.000000,0.000000,0.000000}%
\pgfsetstrokecolor{currentstroke}%
\pgfsetdash{}{0pt}%
\pgfsys@defobject{currentmarker}{\pgfqpoint{-0.027778in}{0.000000in}}{\pgfqpoint{-0.000000in}{0.000000in}}{%
\pgfpathmoveto{\pgfqpoint{-0.000000in}{0.000000in}}%
\pgfpathlineto{\pgfqpoint{-0.027778in}{0.000000in}}%
\pgfusepath{stroke,fill}%
}%
\begin{pgfscope}%
\pgfsys@transformshift{0.800000in}{3.261904in}%
\pgfsys@useobject{currentmarker}{}%
\end{pgfscope}%
\end{pgfscope}%
\begin{pgfscope}%
\pgfsetbuttcap%
\pgfsetroundjoin%
\definecolor{currentfill}{rgb}{0.000000,0.000000,0.000000}%
\pgfsetfillcolor{currentfill}%
\pgfsetlinewidth{0.602250pt}%
\definecolor{currentstroke}{rgb}{0.000000,0.000000,0.000000}%
\pgfsetstrokecolor{currentstroke}%
\pgfsetdash{}{0pt}%
\pgfsys@defobject{currentmarker}{\pgfqpoint{-0.027778in}{0.000000in}}{\pgfqpoint{-0.000000in}{0.000000in}}{%
\pgfpathmoveto{\pgfqpoint{-0.000000in}{0.000000in}}%
\pgfpathlineto{\pgfqpoint{-0.027778in}{0.000000in}}%
\pgfusepath{stroke,fill}%
}%
\begin{pgfscope}%
\pgfsys@transformshift{0.800000in}{3.322998in}%
\pgfsys@useobject{currentmarker}{}%
\end{pgfscope}%
\end{pgfscope}%
\begin{pgfscope}%
\definecolor{textcolor}{rgb}{0.000000,0.000000,0.000000}%
\pgfsetstrokecolor{textcolor}%
\pgfsetfillcolor{textcolor}%
\pgftext[x=0.359220in,y=1.980000in,,bottom,rotate=90.000000]{\color{textcolor}\rmfamily\fontsize{10.000000}{12.000000}\selectfont Dampening of amplitude \(\displaystyle A/A_0\), \(\displaystyle [A] =  \deg\)}%
\end{pgfscope}%
\begin{pgfscope}%
\pgfpathrectangle{\pgfqpoint{0.800000in}{0.440000in}}{\pgfqpoint{4.960000in}{3.080000in}}%
\pgfusepath{clip}%
\pgfsetbuttcap%
\pgfsetroundjoin%
\pgfsetlinewidth{1.003750pt}%
\definecolor{currentstroke}{rgb}{1.000000,0.000000,0.000000}%
\pgfsetstrokecolor{currentstroke}%
\pgfsetdash{}{0pt}%
\pgfpathmoveto{\pgfqpoint{1.025455in}{3.375285in}}%
\pgfpathlineto{\pgfqpoint{1.025455in}{3.380000in}}%
\pgfusepath{stroke}%
\end{pgfscope}%
\begin{pgfscope}%
\pgfpathrectangle{\pgfqpoint{0.800000in}{0.440000in}}{\pgfqpoint{4.960000in}{3.080000in}}%
\pgfusepath{clip}%
\pgfsetbuttcap%
\pgfsetroundjoin%
\pgfsetlinewidth{1.003750pt}%
\definecolor{currentstroke}{rgb}{1.000000,0.000000,0.000000}%
\pgfsetstrokecolor{currentstroke}%
\pgfsetdash{}{0pt}%
\pgfpathmoveto{\pgfqpoint{1.589091in}{3.225275in}}%
\pgfpathlineto{\pgfqpoint{1.589091in}{3.249980in}}%
\pgfusepath{stroke}%
\end{pgfscope}%
\begin{pgfscope}%
\pgfpathrectangle{\pgfqpoint{0.800000in}{0.440000in}}{\pgfqpoint{4.960000in}{3.080000in}}%
\pgfusepath{clip}%
\pgfsetbuttcap%
\pgfsetroundjoin%
\pgfsetlinewidth{1.003750pt}%
\definecolor{currentstroke}{rgb}{1.000000,0.000000,0.000000}%
\pgfsetstrokecolor{currentstroke}%
\pgfsetdash{}{0pt}%
\pgfpathmoveto{\pgfqpoint{2.152727in}{3.041173in}}%
\pgfpathlineto{\pgfqpoint{2.152727in}{3.076056in}}%
\pgfusepath{stroke}%
\end{pgfscope}%
\begin{pgfscope}%
\pgfpathrectangle{\pgfqpoint{0.800000in}{0.440000in}}{\pgfqpoint{4.960000in}{3.080000in}}%
\pgfusepath{clip}%
\pgfsetbuttcap%
\pgfsetroundjoin%
\pgfsetlinewidth{1.003750pt}%
\definecolor{currentstroke}{rgb}{1.000000,0.000000,0.000000}%
\pgfsetstrokecolor{currentstroke}%
\pgfsetdash{}{0pt}%
\pgfpathmoveto{\pgfqpoint{2.716364in}{2.839803in}}%
\pgfpathlineto{\pgfqpoint{2.716364in}{2.890448in}}%
\pgfusepath{stroke}%
\end{pgfscope}%
\begin{pgfscope}%
\pgfpathrectangle{\pgfqpoint{0.800000in}{0.440000in}}{\pgfqpoint{4.960000in}{3.080000in}}%
\pgfusepath{clip}%
\pgfsetbuttcap%
\pgfsetroundjoin%
\pgfsetlinewidth{1.003750pt}%
\definecolor{currentstroke}{rgb}{1.000000,0.000000,0.000000}%
\pgfsetstrokecolor{currentstroke}%
\pgfsetdash{}{0pt}%
\pgfpathmoveto{\pgfqpoint{3.280000in}{2.629490in}}%
\pgfpathlineto{\pgfqpoint{3.280000in}{2.703716in}}%
\pgfusepath{stroke}%
\end{pgfscope}%
\begin{pgfscope}%
\pgfpathrectangle{\pgfqpoint{0.800000in}{0.440000in}}{\pgfqpoint{4.960000in}{3.080000in}}%
\pgfusepath{clip}%
\pgfsetbuttcap%
\pgfsetroundjoin%
\pgfsetlinewidth{1.003750pt}%
\definecolor{currentstroke}{rgb}{1.000000,0.000000,0.000000}%
\pgfsetstrokecolor{currentstroke}%
\pgfsetdash{}{0pt}%
\pgfpathmoveto{\pgfqpoint{3.843636in}{2.409106in}}%
\pgfpathlineto{\pgfqpoint{3.843636in}{2.518711in}}%
\pgfusepath{stroke}%
\end{pgfscope}%
\begin{pgfscope}%
\pgfpathrectangle{\pgfqpoint{0.800000in}{0.440000in}}{\pgfqpoint{4.960000in}{3.080000in}}%
\pgfusepath{clip}%
\pgfsetbuttcap%
\pgfsetroundjoin%
\pgfsetlinewidth{1.003750pt}%
\definecolor{currentstroke}{rgb}{1.000000,0.000000,0.000000}%
\pgfsetstrokecolor{currentstroke}%
\pgfsetdash{}{0pt}%
\pgfpathmoveto{\pgfqpoint{4.407273in}{2.107266in}}%
\pgfpathlineto{\pgfqpoint{4.407273in}{2.289534in}}%
\pgfusepath{stroke}%
\end{pgfscope}%
\begin{pgfscope}%
\pgfpathrectangle{\pgfqpoint{0.800000in}{0.440000in}}{\pgfqpoint{4.960000in}{3.080000in}}%
\pgfusepath{clip}%
\pgfsetbuttcap%
\pgfsetroundjoin%
\pgfsetlinewidth{1.003750pt}%
\definecolor{currentstroke}{rgb}{1.000000,0.000000,0.000000}%
\pgfsetstrokecolor{currentstroke}%
\pgfsetdash{}{0pt}%
\pgfpathmoveto{\pgfqpoint{4.970909in}{1.719689in}}%
\pgfpathlineto{\pgfqpoint{4.970909in}{2.049574in}}%
\pgfusepath{stroke}%
\end{pgfscope}%
\begin{pgfscope}%
\pgfpathrectangle{\pgfqpoint{0.800000in}{0.440000in}}{\pgfqpoint{4.960000in}{3.080000in}}%
\pgfusepath{clip}%
\pgfsetbuttcap%
\pgfsetroundjoin%
\pgfsetlinewidth{1.003750pt}%
\definecolor{currentstroke}{rgb}{1.000000,0.000000,0.000000}%
\pgfsetstrokecolor{currentstroke}%
\pgfsetdash{}{0pt}%
\pgfpathmoveto{\pgfqpoint{5.534545in}{0.580000in}}%
\pgfpathlineto{\pgfqpoint{5.534545in}{1.719689in}}%
\pgfusepath{stroke}%
\end{pgfscope}%
\begin{pgfscope}%
\pgfpathrectangle{\pgfqpoint{0.800000in}{0.440000in}}{\pgfqpoint{4.960000in}{3.080000in}}%
\pgfusepath{clip}%
\pgfsetbuttcap%
\pgfsetroundjoin%
\pgfsetlinewidth{1.003750pt}%
\definecolor{currentstroke}{rgb}{0.000000,0.000000,1.000000}%
\pgfsetstrokecolor{currentstroke}%
\pgfsetdash{}{0pt}%
\pgfpathmoveto{\pgfqpoint{1.025455in}{3.375285in}}%
\pgfpathlineto{\pgfqpoint{1.025455in}{3.380000in}}%
\pgfusepath{stroke}%
\end{pgfscope}%
\begin{pgfscope}%
\pgfpathrectangle{\pgfqpoint{0.800000in}{0.440000in}}{\pgfqpoint{4.960000in}{3.080000in}}%
\pgfusepath{clip}%
\pgfsetbuttcap%
\pgfsetroundjoin%
\pgfsetlinewidth{1.003750pt}%
\definecolor{currentstroke}{rgb}{0.000000,0.000000,1.000000}%
\pgfsetstrokecolor{currentstroke}%
\pgfsetdash{}{0pt}%
\pgfpathmoveto{\pgfqpoint{1.212954in}{3.301605in}}%
\pgfpathlineto{\pgfqpoint{1.212954in}{3.322998in}}%
\pgfusepath{stroke}%
\end{pgfscope}%
\begin{pgfscope}%
\pgfpathrectangle{\pgfqpoint{0.800000in}{0.440000in}}{\pgfqpoint{4.960000in}{3.080000in}}%
\pgfusepath{clip}%
\pgfsetbuttcap%
\pgfsetroundjoin%
\pgfsetlinewidth{1.003750pt}%
\definecolor{currentstroke}{rgb}{0.000000,0.000000,1.000000}%
\pgfsetstrokecolor{currentstroke}%
\pgfsetdash{}{0pt}%
\pgfpathmoveto{\pgfqpoint{1.400454in}{3.199335in}}%
\pgfpathlineto{\pgfqpoint{1.400454in}{3.225275in}}%
\pgfusepath{stroke}%
\end{pgfscope}%
\begin{pgfscope}%
\pgfpathrectangle{\pgfqpoint{0.800000in}{0.440000in}}{\pgfqpoint{4.960000in}{3.080000in}}%
\pgfusepath{clip}%
\pgfsetbuttcap%
\pgfsetroundjoin%
\pgfsetlinewidth{1.003750pt}%
\definecolor{currentstroke}{rgb}{0.000000,0.000000,1.000000}%
\pgfsetstrokecolor{currentstroke}%
\pgfsetdash{}{0pt}%
\pgfpathmoveto{\pgfqpoint{1.775452in}{3.008598in}}%
\pgfpathlineto{\pgfqpoint{1.775452in}{3.045663in}}%
\pgfusepath{stroke}%
\end{pgfscope}%
\begin{pgfscope}%
\pgfpathrectangle{\pgfqpoint{0.800000in}{0.440000in}}{\pgfqpoint{4.960000in}{3.080000in}}%
\pgfusepath{clip}%
\pgfsetbuttcap%
\pgfsetroundjoin%
\pgfsetlinewidth{1.003750pt}%
\definecolor{currentstroke}{rgb}{0.000000,0.000000,1.000000}%
\pgfsetstrokecolor{currentstroke}%
\pgfsetdash{}{0pt}%
\pgfpathmoveto{\pgfqpoint{2.150451in}{2.798285in}}%
\pgfpathlineto{\pgfqpoint{2.150451in}{2.852935in}}%
\pgfusepath{stroke}%
\end{pgfscope}%
\begin{pgfscope}%
\pgfpathrectangle{\pgfqpoint{0.800000in}{0.440000in}}{\pgfqpoint{4.960000in}{3.080000in}}%
\pgfusepath{clip}%
\pgfsetbuttcap%
\pgfsetroundjoin%
\pgfsetlinewidth{1.003750pt}%
\definecolor{currentstroke}{rgb}{0.000000,0.000000,1.000000}%
\pgfsetstrokecolor{currentstroke}%
\pgfsetdash{}{0pt}%
\pgfpathmoveto{\pgfqpoint{2.525450in}{2.565897in}}%
\pgfpathlineto{\pgfqpoint{2.525450in}{2.649066in}}%
\pgfusepath{stroke}%
\end{pgfscope}%
\begin{pgfscope}%
\pgfpathrectangle{\pgfqpoint{0.800000in}{0.440000in}}{\pgfqpoint{4.960000in}{3.080000in}}%
\pgfusepath{clip}%
\pgfsetbuttcap%
\pgfsetroundjoin%
\pgfsetlinewidth{1.003750pt}%
\definecolor{currentstroke}{rgb}{0.000000,0.000000,1.000000}%
\pgfsetstrokecolor{currentstroke}%
\pgfsetdash{}{0pt}%
\pgfpathmoveto{\pgfqpoint{2.900449in}{2.269958in}}%
\pgfpathlineto{\pgfqpoint{2.900449in}{2.409106in}}%
\pgfusepath{stroke}%
\end{pgfscope}%
\begin{pgfscope}%
\pgfpathrectangle{\pgfqpoint{0.800000in}{0.440000in}}{\pgfqpoint{4.960000in}{3.080000in}}%
\pgfusepath{clip}%
\pgfsetbuttcap%
\pgfsetroundjoin%
\pgfsetlinewidth{1.003750pt}%
\definecolor{currentstroke}{rgb}{0.000000,0.000000,1.000000}%
\pgfsetstrokecolor{currentstroke}%
\pgfsetdash{}{0pt}%
\pgfpathmoveto{\pgfqpoint{3.275448in}{2.018128in}}%
\pgfpathlineto{\pgfqpoint{3.275448in}{2.228441in}}%
\pgfusepath{stroke}%
\end{pgfscope}%
\begin{pgfscope}%
\pgfpathrectangle{\pgfqpoint{0.800000in}{0.440000in}}{\pgfqpoint{4.960000in}{3.080000in}}%
\pgfusepath{clip}%
\pgfsetbuttcap%
\pgfsetroundjoin%
\pgfsetlinewidth{1.003750pt}%
\definecolor{currentstroke}{rgb}{0.000000,0.000000,1.000000}%
\pgfsetstrokecolor{currentstroke}%
\pgfsetdash{}{0pt}%
\pgfpathmoveto{\pgfqpoint{3.650447in}{1.658596in}}%
\pgfpathlineto{\pgfqpoint{3.650447in}{2.018128in}}%
\pgfusepath{stroke}%
\end{pgfscope}%
\begin{pgfscope}%
\pgfpathrectangle{\pgfqpoint{0.800000in}{0.440000in}}{\pgfqpoint{4.960000in}{3.080000in}}%
\pgfusepath{clip}%
\pgfsetbuttcap%
\pgfsetroundjoin%
\pgfsetlinewidth{1.003750pt}%
\definecolor{currentstroke}{rgb}{0.000000,0.501961,0.000000}%
\pgfsetstrokecolor{currentstroke}%
\pgfsetdash{}{0pt}%
\pgfpathmoveto{\pgfqpoint{1.025455in}{3.375285in}}%
\pgfpathlineto{\pgfqpoint{1.025455in}{3.380000in}}%
\pgfusepath{stroke}%
\end{pgfscope}%
\begin{pgfscope}%
\pgfpathrectangle{\pgfqpoint{0.800000in}{0.440000in}}{\pgfqpoint{4.960000in}{3.080000in}}%
\pgfusepath{clip}%
\pgfsetbuttcap%
\pgfsetroundjoin%
\pgfsetlinewidth{1.003750pt}%
\definecolor{currentstroke}{rgb}{0.000000,0.501961,0.000000}%
\pgfsetstrokecolor{currentstroke}%
\pgfsetdash{}{0pt}%
\pgfpathmoveto{\pgfqpoint{1.215315in}{3.231563in}}%
\pgfpathlineto{\pgfqpoint{1.215315in}{3.255976in}}%
\pgfusepath{stroke}%
\end{pgfscope}%
\begin{pgfscope}%
\pgfpathrectangle{\pgfqpoint{0.800000in}{0.440000in}}{\pgfqpoint{4.960000in}{3.080000in}}%
\pgfusepath{clip}%
\pgfsetbuttcap%
\pgfsetroundjoin%
\pgfsetlinewidth{1.003750pt}%
\definecolor{currentstroke}{rgb}{0.000000,0.501961,0.000000}%
\pgfsetstrokecolor{currentstroke}%
\pgfsetdash{}{0pt}%
\pgfpathmoveto{\pgfqpoint{1.405175in}{3.080256in}}%
\pgfpathlineto{\pgfqpoint{1.405175in}{3.112685in}}%
\pgfusepath{stroke}%
\end{pgfscope}%
\begin{pgfscope}%
\pgfpathrectangle{\pgfqpoint{0.800000in}{0.440000in}}{\pgfqpoint{4.960000in}{3.080000in}}%
\pgfusepath{clip}%
\pgfsetbuttcap%
\pgfsetroundjoin%
\pgfsetlinewidth{1.003750pt}%
\definecolor{currentstroke}{rgb}{0.000000,0.501961,0.000000}%
\pgfsetstrokecolor{currentstroke}%
\pgfsetdash{}{0pt}%
\pgfpathmoveto{\pgfqpoint{1.595035in}{2.925429in}}%
\pgfpathlineto{\pgfqpoint{1.595035in}{2.968679in}}%
\pgfusepath{stroke}%
\end{pgfscope}%
\begin{pgfscope}%
\pgfpathrectangle{\pgfqpoint{0.800000in}{0.440000in}}{\pgfqpoint{4.960000in}{3.080000in}}%
\pgfusepath{clip}%
\pgfsetbuttcap%
\pgfsetroundjoin%
\pgfsetlinewidth{1.003750pt}%
\definecolor{currentstroke}{rgb}{0.000000,0.501961,0.000000}%
\pgfsetstrokecolor{currentstroke}%
\pgfsetdash{}{0pt}%
\pgfpathmoveto{\pgfqpoint{1.784895in}{2.768638in}}%
\pgfpathlineto{\pgfqpoint{1.784895in}{2.826330in}}%
\pgfusepath{stroke}%
\end{pgfscope}%
\begin{pgfscope}%
\pgfpathrectangle{\pgfqpoint{0.800000in}{0.440000in}}{\pgfqpoint{4.960000in}{3.080000in}}%
\pgfusepath{clip}%
\pgfsetbuttcap%
\pgfsetroundjoin%
\pgfsetlinewidth{1.003750pt}%
\definecolor{currentstroke}{rgb}{0.000000,0.501961,0.000000}%
\pgfsetstrokecolor{currentstroke}%
\pgfsetdash{}{0pt}%
\pgfpathmoveto{\pgfqpoint{1.974756in}{2.587973in}}%
\pgfpathlineto{\pgfqpoint{1.974756in}{2.667930in}}%
\pgfusepath{stroke}%
\end{pgfscope}%
\begin{pgfscope}%
\pgfpathrectangle{\pgfqpoint{0.800000in}{0.440000in}}{\pgfqpoint{4.960000in}{3.080000in}}%
\pgfusepath{clip}%
\pgfsetbuttcap%
\pgfsetroundjoin%
\pgfsetlinewidth{1.003750pt}%
\definecolor{currentstroke}{rgb}{0.000000,0.501961,0.000000}%
\pgfsetstrokecolor{currentstroke}%
\pgfsetdash{}{0pt}%
\pgfpathmoveto{\pgfqpoint{2.164616in}{2.409106in}}%
\pgfpathlineto{\pgfqpoint{2.164616in}{2.518711in}}%
\pgfusepath{stroke}%
\end{pgfscope}%
\begin{pgfscope}%
\pgfpathrectangle{\pgfqpoint{0.800000in}{0.440000in}}{\pgfqpoint{4.960000in}{3.080000in}}%
\pgfusepath{clip}%
\pgfsetbuttcap%
\pgfsetroundjoin%
\pgfsetlinewidth{1.003750pt}%
\definecolor{currentstroke}{rgb}{0.000000,0.501961,0.000000}%
\pgfsetstrokecolor{currentstroke}%
\pgfsetdash{}{0pt}%
\pgfpathmoveto{\pgfqpoint{2.354476in}{2.228441in}}%
\pgfpathlineto{\pgfqpoint{2.354476in}{2.377660in}}%
\pgfusepath{stroke}%
\end{pgfscope}%
\begin{pgfscope}%
\pgfpathrectangle{\pgfqpoint{0.800000in}{0.440000in}}{\pgfqpoint{4.960000in}{3.080000in}}%
\pgfusepath{clip}%
\pgfsetbuttcap%
\pgfsetroundjoin%
\pgfsetlinewidth{1.003750pt}%
\definecolor{currentstroke}{rgb}{0.000000,0.501961,0.000000}%
\pgfsetstrokecolor{currentstroke}%
\pgfsetdash{}{0pt}%
\pgfpathmoveto{\pgfqpoint{2.544336in}{2.018128in}}%
\pgfpathlineto{\pgfqpoint{2.544336in}{2.228441in}}%
\pgfusepath{stroke}%
\end{pgfscope}%
\begin{pgfscope}%
\pgfpathrectangle{\pgfqpoint{0.800000in}{0.440000in}}{\pgfqpoint{4.960000in}{3.080000in}}%
\pgfusepath{clip}%
\pgfsetbuttcap%
\pgfsetroundjoin%
\definecolor{currentfill}{rgb}{1.000000,0.000000,0.000000}%
\pgfsetfillcolor{currentfill}%
\pgfsetlinewidth{1.003750pt}%
\definecolor{currentstroke}{rgb}{1.000000,0.000000,0.000000}%
\pgfsetstrokecolor{currentstroke}%
\pgfsetdash{}{0pt}%
\pgfsys@defobject{currentmarker}{\pgfqpoint{-0.041667in}{-0.000000in}}{\pgfqpoint{0.041667in}{0.000000in}}{%
\pgfpathmoveto{\pgfqpoint{0.041667in}{-0.000000in}}%
\pgfpathlineto{\pgfqpoint{-0.041667in}{0.000000in}}%
\pgfusepath{stroke,fill}%
}%
\begin{pgfscope}%
\pgfsys@transformshift{1.025455in}{3.375285in}%
\pgfsys@useobject{currentmarker}{}%
\end{pgfscope}%
\begin{pgfscope}%
\pgfsys@transformshift{1.589091in}{3.225275in}%
\pgfsys@useobject{currentmarker}{}%
\end{pgfscope}%
\begin{pgfscope}%
\pgfsys@transformshift{2.152727in}{3.041173in}%
\pgfsys@useobject{currentmarker}{}%
\end{pgfscope}%
\begin{pgfscope}%
\pgfsys@transformshift{2.716364in}{2.839803in}%
\pgfsys@useobject{currentmarker}{}%
\end{pgfscope}%
\begin{pgfscope}%
\pgfsys@transformshift{3.280000in}{2.629490in}%
\pgfsys@useobject{currentmarker}{}%
\end{pgfscope}%
\begin{pgfscope}%
\pgfsys@transformshift{3.843636in}{2.409106in}%
\pgfsys@useobject{currentmarker}{}%
\end{pgfscope}%
\begin{pgfscope}%
\pgfsys@transformshift{4.407273in}{2.107266in}%
\pgfsys@useobject{currentmarker}{}%
\end{pgfscope}%
\begin{pgfscope}%
\pgfsys@transformshift{4.970909in}{1.719689in}%
\pgfsys@useobject{currentmarker}{}%
\end{pgfscope}%
\begin{pgfscope}%
\pgfsys@transformshift{5.534545in}{0.580000in}%
\pgfsys@useobject{currentmarker}{}%
\end{pgfscope}%
\end{pgfscope}%
\begin{pgfscope}%
\pgfpathrectangle{\pgfqpoint{0.800000in}{0.440000in}}{\pgfqpoint{4.960000in}{3.080000in}}%
\pgfusepath{clip}%
\pgfsetbuttcap%
\pgfsetroundjoin%
\definecolor{currentfill}{rgb}{1.000000,0.000000,0.000000}%
\pgfsetfillcolor{currentfill}%
\pgfsetlinewidth{1.003750pt}%
\definecolor{currentstroke}{rgb}{1.000000,0.000000,0.000000}%
\pgfsetstrokecolor{currentstroke}%
\pgfsetdash{}{0pt}%
\pgfsys@defobject{currentmarker}{\pgfqpoint{-0.041667in}{-0.000000in}}{\pgfqpoint{0.041667in}{0.000000in}}{%
\pgfpathmoveto{\pgfqpoint{0.041667in}{-0.000000in}}%
\pgfpathlineto{\pgfqpoint{-0.041667in}{0.000000in}}%
\pgfusepath{stroke,fill}%
}%
\begin{pgfscope}%
\pgfsys@transformshift{1.025455in}{3.380000in}%
\pgfsys@useobject{currentmarker}{}%
\end{pgfscope}%
\begin{pgfscope}%
\pgfsys@transformshift{1.589091in}{3.249980in}%
\pgfsys@useobject{currentmarker}{}%
\end{pgfscope}%
\begin{pgfscope}%
\pgfsys@transformshift{2.152727in}{3.076056in}%
\pgfsys@useobject{currentmarker}{}%
\end{pgfscope}%
\begin{pgfscope}%
\pgfsys@transformshift{2.716364in}{2.890448in}%
\pgfsys@useobject{currentmarker}{}%
\end{pgfscope}%
\begin{pgfscope}%
\pgfsys@transformshift{3.280000in}{2.703716in}%
\pgfsys@useobject{currentmarker}{}%
\end{pgfscope}%
\begin{pgfscope}%
\pgfsys@transformshift{3.843636in}{2.518711in}%
\pgfsys@useobject{currentmarker}{}%
\end{pgfscope}%
\begin{pgfscope}%
\pgfsys@transformshift{4.407273in}{2.289534in}%
\pgfsys@useobject{currentmarker}{}%
\end{pgfscope}%
\begin{pgfscope}%
\pgfsys@transformshift{4.970909in}{2.049574in}%
\pgfsys@useobject{currentmarker}{}%
\end{pgfscope}%
\begin{pgfscope}%
\pgfsys@transformshift{5.534545in}{1.719689in}%
\pgfsys@useobject{currentmarker}{}%
\end{pgfscope}%
\end{pgfscope}%
\begin{pgfscope}%
\pgfpathrectangle{\pgfqpoint{0.800000in}{0.440000in}}{\pgfqpoint{4.960000in}{3.080000in}}%
\pgfusepath{clip}%
\pgfsetbuttcap%
\pgfsetroundjoin%
\pgfsetlinewidth{1.003750pt}%
\definecolor{currentstroke}{rgb}{1.000000,0.000000,0.000000}%
\pgfsetstrokecolor{currentstroke}%
\pgfsetdash{{3.700000pt}{1.600000pt}}{0.000000pt}%
\pgfpathmoveto{\pgfqpoint{1.025455in}{3.377648in}}%
\pgfpathlineto{\pgfqpoint{1.071001in}{3.363458in}}%
\pgfpathlineto{\pgfqpoint{1.116547in}{3.349268in}}%
\pgfpathlineto{\pgfqpoint{1.162094in}{3.335078in}}%
\pgfpathlineto{\pgfqpoint{1.207640in}{3.320889in}}%
\pgfpathlineto{\pgfqpoint{1.253186in}{3.306699in}}%
\pgfpathlineto{\pgfqpoint{1.298733in}{3.292509in}}%
\pgfpathlineto{\pgfqpoint{1.344279in}{3.278319in}}%
\pgfpathlineto{\pgfqpoint{1.389826in}{3.264129in}}%
\pgfpathlineto{\pgfqpoint{1.435372in}{3.249940in}}%
\pgfpathlineto{\pgfqpoint{1.480918in}{3.235750in}}%
\pgfpathlineto{\pgfqpoint{1.526465in}{3.221560in}}%
\pgfpathlineto{\pgfqpoint{1.572011in}{3.207370in}}%
\pgfpathlineto{\pgfqpoint{1.617557in}{3.193181in}}%
\pgfpathlineto{\pgfqpoint{1.663104in}{3.178991in}}%
\pgfpathlineto{\pgfqpoint{1.708650in}{3.164801in}}%
\pgfpathlineto{\pgfqpoint{1.754197in}{3.150611in}}%
\pgfpathlineto{\pgfqpoint{1.799743in}{3.136421in}}%
\pgfpathlineto{\pgfqpoint{1.845289in}{3.122232in}}%
\pgfpathlineto{\pgfqpoint{1.890836in}{3.108042in}}%
\pgfpathlineto{\pgfqpoint{1.936382in}{3.093852in}}%
\pgfpathlineto{\pgfqpoint{1.981928in}{3.079662in}}%
\pgfpathlineto{\pgfqpoint{2.027475in}{3.065473in}}%
\pgfpathlineto{\pgfqpoint{2.073021in}{3.051283in}}%
\pgfpathlineto{\pgfqpoint{2.118567in}{3.037093in}}%
\pgfpathlineto{\pgfqpoint{2.164114in}{3.022903in}}%
\pgfpathlineto{\pgfqpoint{2.209660in}{3.008713in}}%
\pgfpathlineto{\pgfqpoint{2.255207in}{2.994524in}}%
\pgfpathlineto{\pgfqpoint{2.300753in}{2.980334in}}%
\pgfpathlineto{\pgfqpoint{2.346299in}{2.966144in}}%
\pgfpathlineto{\pgfqpoint{2.391846in}{2.951954in}}%
\pgfpathlineto{\pgfqpoint{2.437392in}{2.937765in}}%
\pgfpathlineto{\pgfqpoint{2.482938in}{2.923575in}}%
\pgfpathlineto{\pgfqpoint{2.528485in}{2.909385in}}%
\pgfpathlineto{\pgfqpoint{2.574031in}{2.895195in}}%
\pgfpathlineto{\pgfqpoint{2.619578in}{2.881005in}}%
\pgfpathlineto{\pgfqpoint{2.665124in}{2.866816in}}%
\pgfpathlineto{\pgfqpoint{2.710670in}{2.852626in}}%
\pgfpathlineto{\pgfqpoint{2.756217in}{2.838436in}}%
\pgfpathlineto{\pgfqpoint{2.801763in}{2.824246in}}%
\pgfpathlineto{\pgfqpoint{2.847309in}{2.810057in}}%
\pgfpathlineto{\pgfqpoint{2.892856in}{2.795867in}}%
\pgfpathlineto{\pgfqpoint{2.938402in}{2.781677in}}%
\pgfpathlineto{\pgfqpoint{2.983949in}{2.767487in}}%
\pgfpathlineto{\pgfqpoint{3.029495in}{2.753297in}}%
\pgfpathlineto{\pgfqpoint{3.075041in}{2.739108in}}%
\pgfpathlineto{\pgfqpoint{3.120588in}{2.724918in}}%
\pgfpathlineto{\pgfqpoint{3.166134in}{2.710728in}}%
\pgfpathlineto{\pgfqpoint{3.211680in}{2.696538in}}%
\pgfpathlineto{\pgfqpoint{3.257227in}{2.682349in}}%
\pgfpathlineto{\pgfqpoint{3.302773in}{2.668159in}}%
\pgfpathlineto{\pgfqpoint{3.348320in}{2.653969in}}%
\pgfpathlineto{\pgfqpoint{3.393866in}{2.639779in}}%
\pgfpathlineto{\pgfqpoint{3.439412in}{2.625589in}}%
\pgfpathlineto{\pgfqpoint{3.484959in}{2.611400in}}%
\pgfpathlineto{\pgfqpoint{3.530505in}{2.597210in}}%
\pgfpathlineto{\pgfqpoint{3.576051in}{2.583020in}}%
\pgfpathlineto{\pgfqpoint{3.621598in}{2.568830in}}%
\pgfpathlineto{\pgfqpoint{3.667144in}{2.554641in}}%
\pgfpathlineto{\pgfqpoint{3.712691in}{2.540451in}}%
\pgfpathlineto{\pgfqpoint{3.758237in}{2.526261in}}%
\pgfpathlineto{\pgfqpoint{3.803783in}{2.512071in}}%
\pgfpathlineto{\pgfqpoint{3.849330in}{2.497882in}}%
\pgfpathlineto{\pgfqpoint{3.894876in}{2.483692in}}%
\pgfpathlineto{\pgfqpoint{3.940422in}{2.469502in}}%
\pgfpathlineto{\pgfqpoint{3.985969in}{2.455312in}}%
\pgfpathlineto{\pgfqpoint{4.031515in}{2.441122in}}%
\pgfpathlineto{\pgfqpoint{4.077062in}{2.426933in}}%
\pgfpathlineto{\pgfqpoint{4.122608in}{2.412743in}}%
\pgfpathlineto{\pgfqpoint{4.168154in}{2.398553in}}%
\pgfpathlineto{\pgfqpoint{4.213701in}{2.384363in}}%
\pgfpathlineto{\pgfqpoint{4.259247in}{2.370174in}}%
\pgfpathlineto{\pgfqpoint{4.304793in}{2.355984in}}%
\pgfpathlineto{\pgfqpoint{4.350340in}{2.341794in}}%
\pgfpathlineto{\pgfqpoint{4.395886in}{2.327604in}}%
\pgfpathlineto{\pgfqpoint{4.441433in}{2.313414in}}%
\pgfpathlineto{\pgfqpoint{4.486979in}{2.299225in}}%
\pgfpathlineto{\pgfqpoint{4.532525in}{2.285035in}}%
\pgfpathlineto{\pgfqpoint{4.578072in}{2.270845in}}%
\pgfpathlineto{\pgfqpoint{4.623618in}{2.256655in}}%
\pgfpathlineto{\pgfqpoint{4.669164in}{2.242466in}}%
\pgfpathlineto{\pgfqpoint{4.714711in}{2.228276in}}%
\pgfpathlineto{\pgfqpoint{4.760257in}{2.214086in}}%
\pgfpathlineto{\pgfqpoint{4.805803in}{2.199896in}}%
\pgfpathlineto{\pgfqpoint{4.851350in}{2.185706in}}%
\pgfpathlineto{\pgfqpoint{4.896896in}{2.171517in}}%
\pgfpathlineto{\pgfqpoint{4.942443in}{2.157327in}}%
\pgfpathlineto{\pgfqpoint{4.987989in}{2.143137in}}%
\pgfpathlineto{\pgfqpoint{5.033535in}{2.128947in}}%
\pgfpathlineto{\pgfqpoint{5.079082in}{2.114758in}}%
\pgfpathlineto{\pgfqpoint{5.124628in}{2.100568in}}%
\pgfpathlineto{\pgfqpoint{5.170174in}{2.086378in}}%
\pgfpathlineto{\pgfqpoint{5.215721in}{2.072188in}}%
\pgfpathlineto{\pgfqpoint{5.261267in}{2.057998in}}%
\pgfpathlineto{\pgfqpoint{5.306814in}{2.043809in}}%
\pgfpathlineto{\pgfqpoint{5.352360in}{2.029619in}}%
\pgfpathlineto{\pgfqpoint{5.397906in}{2.015429in}}%
\pgfpathlineto{\pgfqpoint{5.443453in}{2.001239in}}%
\pgfpathlineto{\pgfqpoint{5.488999in}{1.987050in}}%
\pgfpathlineto{\pgfqpoint{5.534545in}{1.972860in}}%
\pgfusepath{stroke}%
\end{pgfscope}%
\begin{pgfscope}%
\pgfpathrectangle{\pgfqpoint{0.800000in}{0.440000in}}{\pgfqpoint{4.960000in}{3.080000in}}%
\pgfusepath{clip}%
\pgfsetbuttcap%
\pgfsetroundjoin%
\definecolor{currentfill}{rgb}{0.000000,0.000000,1.000000}%
\pgfsetfillcolor{currentfill}%
\pgfsetlinewidth{1.003750pt}%
\definecolor{currentstroke}{rgb}{0.000000,0.000000,1.000000}%
\pgfsetstrokecolor{currentstroke}%
\pgfsetdash{}{0pt}%
\pgfsys@defobject{currentmarker}{\pgfqpoint{-0.041667in}{-0.000000in}}{\pgfqpoint{0.041667in}{0.000000in}}{%
\pgfpathmoveto{\pgfqpoint{0.041667in}{-0.000000in}}%
\pgfpathlineto{\pgfqpoint{-0.041667in}{0.000000in}}%
\pgfusepath{stroke,fill}%
}%
\begin{pgfscope}%
\pgfsys@transformshift{1.025455in}{3.375285in}%
\pgfsys@useobject{currentmarker}{}%
\end{pgfscope}%
\begin{pgfscope}%
\pgfsys@transformshift{1.212954in}{3.301605in}%
\pgfsys@useobject{currentmarker}{}%
\end{pgfscope}%
\begin{pgfscope}%
\pgfsys@transformshift{1.400454in}{3.199335in}%
\pgfsys@useobject{currentmarker}{}%
\end{pgfscope}%
\begin{pgfscope}%
\pgfsys@transformshift{1.775452in}{3.008598in}%
\pgfsys@useobject{currentmarker}{}%
\end{pgfscope}%
\begin{pgfscope}%
\pgfsys@transformshift{2.150451in}{2.798285in}%
\pgfsys@useobject{currentmarker}{}%
\end{pgfscope}%
\begin{pgfscope}%
\pgfsys@transformshift{2.525450in}{2.565897in}%
\pgfsys@useobject{currentmarker}{}%
\end{pgfscope}%
\begin{pgfscope}%
\pgfsys@transformshift{2.900449in}{2.269958in}%
\pgfsys@useobject{currentmarker}{}%
\end{pgfscope}%
\begin{pgfscope}%
\pgfsys@transformshift{3.275448in}{2.018128in}%
\pgfsys@useobject{currentmarker}{}%
\end{pgfscope}%
\begin{pgfscope}%
\pgfsys@transformshift{3.650447in}{1.658596in}%
\pgfsys@useobject{currentmarker}{}%
\end{pgfscope}%
\end{pgfscope}%
\begin{pgfscope}%
\pgfpathrectangle{\pgfqpoint{0.800000in}{0.440000in}}{\pgfqpoint{4.960000in}{3.080000in}}%
\pgfusepath{clip}%
\pgfsetbuttcap%
\pgfsetroundjoin%
\definecolor{currentfill}{rgb}{0.000000,0.000000,1.000000}%
\pgfsetfillcolor{currentfill}%
\pgfsetlinewidth{1.003750pt}%
\definecolor{currentstroke}{rgb}{0.000000,0.000000,1.000000}%
\pgfsetstrokecolor{currentstroke}%
\pgfsetdash{}{0pt}%
\pgfsys@defobject{currentmarker}{\pgfqpoint{-0.041667in}{-0.000000in}}{\pgfqpoint{0.041667in}{0.000000in}}{%
\pgfpathmoveto{\pgfqpoint{0.041667in}{-0.000000in}}%
\pgfpathlineto{\pgfqpoint{-0.041667in}{0.000000in}}%
\pgfusepath{stroke,fill}%
}%
\begin{pgfscope}%
\pgfsys@transformshift{1.025455in}{3.380000in}%
\pgfsys@useobject{currentmarker}{}%
\end{pgfscope}%
\begin{pgfscope}%
\pgfsys@transformshift{1.212954in}{3.322998in}%
\pgfsys@useobject{currentmarker}{}%
\end{pgfscope}%
\begin{pgfscope}%
\pgfsys@transformshift{1.400454in}{3.225275in}%
\pgfsys@useobject{currentmarker}{}%
\end{pgfscope}%
\begin{pgfscope}%
\pgfsys@transformshift{1.775452in}{3.045663in}%
\pgfsys@useobject{currentmarker}{}%
\end{pgfscope}%
\begin{pgfscope}%
\pgfsys@transformshift{2.150451in}{2.852935in}%
\pgfsys@useobject{currentmarker}{}%
\end{pgfscope}%
\begin{pgfscope}%
\pgfsys@transformshift{2.525450in}{2.649066in}%
\pgfsys@useobject{currentmarker}{}%
\end{pgfscope}%
\begin{pgfscope}%
\pgfsys@transformshift{2.900449in}{2.409106in}%
\pgfsys@useobject{currentmarker}{}%
\end{pgfscope}%
\begin{pgfscope}%
\pgfsys@transformshift{3.275448in}{2.228441in}%
\pgfsys@useobject{currentmarker}{}%
\end{pgfscope}%
\begin{pgfscope}%
\pgfsys@transformshift{3.650447in}{2.018128in}%
\pgfsys@useobject{currentmarker}{}%
\end{pgfscope}%
\end{pgfscope}%
\begin{pgfscope}%
\pgfpathrectangle{\pgfqpoint{0.800000in}{0.440000in}}{\pgfqpoint{4.960000in}{3.080000in}}%
\pgfusepath{clip}%
\pgfsetbuttcap%
\pgfsetroundjoin%
\pgfsetlinewidth{1.003750pt}%
\definecolor{currentstroke}{rgb}{0.000000,0.000000,1.000000}%
\pgfsetstrokecolor{currentstroke}%
\pgfsetdash{{3.700000pt}{1.600000pt}}{0.000000pt}%
\pgfpathmoveto{\pgfqpoint{1.025455in}{3.377648in}}%
\pgfpathlineto{\pgfqpoint{1.051970in}{3.364467in}}%
\pgfpathlineto{\pgfqpoint{1.078485in}{3.351286in}}%
\pgfpathlineto{\pgfqpoint{1.105000in}{3.338106in}}%
\pgfpathlineto{\pgfqpoint{1.131515in}{3.324925in}}%
\pgfpathlineto{\pgfqpoint{1.158030in}{3.311744in}}%
\pgfpathlineto{\pgfqpoint{1.184545in}{3.298563in}}%
\pgfpathlineto{\pgfqpoint{1.211060in}{3.285383in}}%
\pgfpathlineto{\pgfqpoint{1.237575in}{3.272202in}}%
\pgfpathlineto{\pgfqpoint{1.264090in}{3.259021in}}%
\pgfpathlineto{\pgfqpoint{1.290605in}{3.245841in}}%
\pgfpathlineto{\pgfqpoint{1.317120in}{3.232660in}}%
\pgfpathlineto{\pgfqpoint{1.343635in}{3.219479in}}%
\pgfpathlineto{\pgfqpoint{1.370151in}{3.206299in}}%
\pgfpathlineto{\pgfqpoint{1.396666in}{3.193118in}}%
\pgfpathlineto{\pgfqpoint{1.423181in}{3.179937in}}%
\pgfpathlineto{\pgfqpoint{1.449696in}{3.166757in}}%
\pgfpathlineto{\pgfqpoint{1.476211in}{3.153576in}}%
\pgfpathlineto{\pgfqpoint{1.502726in}{3.140395in}}%
\pgfpathlineto{\pgfqpoint{1.529241in}{3.127214in}}%
\pgfpathlineto{\pgfqpoint{1.555756in}{3.114034in}}%
\pgfpathlineto{\pgfqpoint{1.582271in}{3.100853in}}%
\pgfpathlineto{\pgfqpoint{1.608786in}{3.087672in}}%
\pgfpathlineto{\pgfqpoint{1.635301in}{3.074492in}}%
\pgfpathlineto{\pgfqpoint{1.661816in}{3.061311in}}%
\pgfpathlineto{\pgfqpoint{1.688332in}{3.048130in}}%
\pgfpathlineto{\pgfqpoint{1.714847in}{3.034950in}}%
\pgfpathlineto{\pgfqpoint{1.741362in}{3.021769in}}%
\pgfpathlineto{\pgfqpoint{1.767877in}{3.008588in}}%
\pgfpathlineto{\pgfqpoint{1.794392in}{2.995407in}}%
\pgfpathlineto{\pgfqpoint{1.820907in}{2.982227in}}%
\pgfpathlineto{\pgfqpoint{1.847422in}{2.969046in}}%
\pgfpathlineto{\pgfqpoint{1.873937in}{2.955865in}}%
\pgfpathlineto{\pgfqpoint{1.900452in}{2.942685in}}%
\pgfpathlineto{\pgfqpoint{1.926967in}{2.929504in}}%
\pgfpathlineto{\pgfqpoint{1.953482in}{2.916323in}}%
\pgfpathlineto{\pgfqpoint{1.979997in}{2.903143in}}%
\pgfpathlineto{\pgfqpoint{2.006512in}{2.889962in}}%
\pgfpathlineto{\pgfqpoint{2.033028in}{2.876781in}}%
\pgfpathlineto{\pgfqpoint{2.059543in}{2.863601in}}%
\pgfpathlineto{\pgfqpoint{2.086058in}{2.850420in}}%
\pgfpathlineto{\pgfqpoint{2.112573in}{2.837239in}}%
\pgfpathlineto{\pgfqpoint{2.139088in}{2.824058in}}%
\pgfpathlineto{\pgfqpoint{2.165603in}{2.810878in}}%
\pgfpathlineto{\pgfqpoint{2.192118in}{2.797697in}}%
\pgfpathlineto{\pgfqpoint{2.218633in}{2.784516in}}%
\pgfpathlineto{\pgfqpoint{2.245148in}{2.771336in}}%
\pgfpathlineto{\pgfqpoint{2.271663in}{2.758155in}}%
\pgfpathlineto{\pgfqpoint{2.298178in}{2.744974in}}%
\pgfpathlineto{\pgfqpoint{2.324693in}{2.731794in}}%
\pgfpathlineto{\pgfqpoint{2.351208in}{2.718613in}}%
\pgfpathlineto{\pgfqpoint{2.377724in}{2.705432in}}%
\pgfpathlineto{\pgfqpoint{2.404239in}{2.692252in}}%
\pgfpathlineto{\pgfqpoint{2.430754in}{2.679071in}}%
\pgfpathlineto{\pgfqpoint{2.457269in}{2.665890in}}%
\pgfpathlineto{\pgfqpoint{2.483784in}{2.652709in}}%
\pgfpathlineto{\pgfqpoint{2.510299in}{2.639529in}}%
\pgfpathlineto{\pgfqpoint{2.536814in}{2.626348in}}%
\pgfpathlineto{\pgfqpoint{2.563329in}{2.613167in}}%
\pgfpathlineto{\pgfqpoint{2.589844in}{2.599987in}}%
\pgfpathlineto{\pgfqpoint{2.616359in}{2.586806in}}%
\pgfpathlineto{\pgfqpoint{2.642874in}{2.573625in}}%
\pgfpathlineto{\pgfqpoint{2.669389in}{2.560445in}}%
\pgfpathlineto{\pgfqpoint{2.695905in}{2.547264in}}%
\pgfpathlineto{\pgfqpoint{2.722420in}{2.534083in}}%
\pgfpathlineto{\pgfqpoint{2.748935in}{2.520902in}}%
\pgfpathlineto{\pgfqpoint{2.775450in}{2.507722in}}%
\pgfpathlineto{\pgfqpoint{2.801965in}{2.494541in}}%
\pgfpathlineto{\pgfqpoint{2.828480in}{2.481360in}}%
\pgfpathlineto{\pgfqpoint{2.854995in}{2.468180in}}%
\pgfpathlineto{\pgfqpoint{2.881510in}{2.454999in}}%
\pgfpathlineto{\pgfqpoint{2.908025in}{2.441818in}}%
\pgfpathlineto{\pgfqpoint{2.934540in}{2.428638in}}%
\pgfpathlineto{\pgfqpoint{2.961055in}{2.415457in}}%
\pgfpathlineto{\pgfqpoint{2.987570in}{2.402276in}}%
\pgfpathlineto{\pgfqpoint{3.014085in}{2.389096in}}%
\pgfpathlineto{\pgfqpoint{3.040601in}{2.375915in}}%
\pgfpathlineto{\pgfqpoint{3.067116in}{2.362734in}}%
\pgfpathlineto{\pgfqpoint{3.093631in}{2.349553in}}%
\pgfpathlineto{\pgfqpoint{3.120146in}{2.336373in}}%
\pgfpathlineto{\pgfqpoint{3.146661in}{2.323192in}}%
\pgfpathlineto{\pgfqpoint{3.173176in}{2.310011in}}%
\pgfpathlineto{\pgfqpoint{3.199691in}{2.296831in}}%
\pgfpathlineto{\pgfqpoint{3.226206in}{2.283650in}}%
\pgfpathlineto{\pgfqpoint{3.252721in}{2.270469in}}%
\pgfpathlineto{\pgfqpoint{3.279236in}{2.257289in}}%
\pgfpathlineto{\pgfqpoint{3.305751in}{2.244108in}}%
\pgfpathlineto{\pgfqpoint{3.332266in}{2.230927in}}%
\pgfpathlineto{\pgfqpoint{3.358781in}{2.217747in}}%
\pgfpathlineto{\pgfqpoint{3.385297in}{2.204566in}}%
\pgfpathlineto{\pgfqpoint{3.411812in}{2.191385in}}%
\pgfpathlineto{\pgfqpoint{3.438327in}{2.178204in}}%
\pgfpathlineto{\pgfqpoint{3.464842in}{2.165024in}}%
\pgfpathlineto{\pgfqpoint{3.491357in}{2.151843in}}%
\pgfpathlineto{\pgfqpoint{3.517872in}{2.138662in}}%
\pgfpathlineto{\pgfqpoint{3.544387in}{2.125482in}}%
\pgfpathlineto{\pgfqpoint{3.570902in}{2.112301in}}%
\pgfpathlineto{\pgfqpoint{3.597417in}{2.099120in}}%
\pgfpathlineto{\pgfqpoint{3.623932in}{2.085940in}}%
\pgfpathlineto{\pgfqpoint{3.650447in}{2.072759in}}%
\pgfusepath{stroke}%
\end{pgfscope}%
\begin{pgfscope}%
\pgfpathrectangle{\pgfqpoint{0.800000in}{0.440000in}}{\pgfqpoint{4.960000in}{3.080000in}}%
\pgfusepath{clip}%
\pgfsetbuttcap%
\pgfsetroundjoin%
\definecolor{currentfill}{rgb}{0.000000,0.501961,0.000000}%
\pgfsetfillcolor{currentfill}%
\pgfsetlinewidth{1.003750pt}%
\definecolor{currentstroke}{rgb}{0.000000,0.501961,0.000000}%
\pgfsetstrokecolor{currentstroke}%
\pgfsetdash{}{0pt}%
\pgfsys@defobject{currentmarker}{\pgfqpoint{-0.041667in}{-0.000000in}}{\pgfqpoint{0.041667in}{0.000000in}}{%
\pgfpathmoveto{\pgfqpoint{0.041667in}{-0.000000in}}%
\pgfpathlineto{\pgfqpoint{-0.041667in}{0.000000in}}%
\pgfusepath{stroke,fill}%
}%
\begin{pgfscope}%
\pgfsys@transformshift{1.025455in}{3.375285in}%
\pgfsys@useobject{currentmarker}{}%
\end{pgfscope}%
\begin{pgfscope}%
\pgfsys@transformshift{1.215315in}{3.231563in}%
\pgfsys@useobject{currentmarker}{}%
\end{pgfscope}%
\begin{pgfscope}%
\pgfsys@transformshift{1.405175in}{3.080256in}%
\pgfsys@useobject{currentmarker}{}%
\end{pgfscope}%
\begin{pgfscope}%
\pgfsys@transformshift{1.595035in}{2.925429in}%
\pgfsys@useobject{currentmarker}{}%
\end{pgfscope}%
\begin{pgfscope}%
\pgfsys@transformshift{1.784895in}{2.768638in}%
\pgfsys@useobject{currentmarker}{}%
\end{pgfscope}%
\begin{pgfscope}%
\pgfsys@transformshift{1.974756in}{2.587973in}%
\pgfsys@useobject{currentmarker}{}%
\end{pgfscope}%
\begin{pgfscope}%
\pgfsys@transformshift{2.164616in}{2.409106in}%
\pgfsys@useobject{currentmarker}{}%
\end{pgfscope}%
\begin{pgfscope}%
\pgfsys@transformshift{2.354476in}{2.228441in}%
\pgfsys@useobject{currentmarker}{}%
\end{pgfscope}%
\begin{pgfscope}%
\pgfsys@transformshift{2.544336in}{2.018128in}%
\pgfsys@useobject{currentmarker}{}%
\end{pgfscope}%
\end{pgfscope}%
\begin{pgfscope}%
\pgfpathrectangle{\pgfqpoint{0.800000in}{0.440000in}}{\pgfqpoint{4.960000in}{3.080000in}}%
\pgfusepath{clip}%
\pgfsetbuttcap%
\pgfsetroundjoin%
\definecolor{currentfill}{rgb}{0.000000,0.501961,0.000000}%
\pgfsetfillcolor{currentfill}%
\pgfsetlinewidth{1.003750pt}%
\definecolor{currentstroke}{rgb}{0.000000,0.501961,0.000000}%
\pgfsetstrokecolor{currentstroke}%
\pgfsetdash{}{0pt}%
\pgfsys@defobject{currentmarker}{\pgfqpoint{-0.041667in}{-0.000000in}}{\pgfqpoint{0.041667in}{0.000000in}}{%
\pgfpathmoveto{\pgfqpoint{0.041667in}{-0.000000in}}%
\pgfpathlineto{\pgfqpoint{-0.041667in}{0.000000in}}%
\pgfusepath{stroke,fill}%
}%
\begin{pgfscope}%
\pgfsys@transformshift{1.025455in}{3.380000in}%
\pgfsys@useobject{currentmarker}{}%
\end{pgfscope}%
\begin{pgfscope}%
\pgfsys@transformshift{1.215315in}{3.255976in}%
\pgfsys@useobject{currentmarker}{}%
\end{pgfscope}%
\begin{pgfscope}%
\pgfsys@transformshift{1.405175in}{3.112685in}%
\pgfsys@useobject{currentmarker}{}%
\end{pgfscope}%
\begin{pgfscope}%
\pgfsys@transformshift{1.595035in}{2.968679in}%
\pgfsys@useobject{currentmarker}{}%
\end{pgfscope}%
\begin{pgfscope}%
\pgfsys@transformshift{1.784895in}{2.826330in}%
\pgfsys@useobject{currentmarker}{}%
\end{pgfscope}%
\begin{pgfscope}%
\pgfsys@transformshift{1.974756in}{2.667930in}%
\pgfsys@useobject{currentmarker}{}%
\end{pgfscope}%
\begin{pgfscope}%
\pgfsys@transformshift{2.164616in}{2.518711in}%
\pgfsys@useobject{currentmarker}{}%
\end{pgfscope}%
\begin{pgfscope}%
\pgfsys@transformshift{2.354476in}{2.377660in}%
\pgfsys@useobject{currentmarker}{}%
\end{pgfscope}%
\begin{pgfscope}%
\pgfsys@transformshift{2.544336in}{2.228441in}%
\pgfsys@useobject{currentmarker}{}%
\end{pgfscope}%
\end{pgfscope}%
\begin{pgfscope}%
\pgfpathrectangle{\pgfqpoint{0.800000in}{0.440000in}}{\pgfqpoint{4.960000in}{3.080000in}}%
\pgfusepath{clip}%
\pgfsetbuttcap%
\pgfsetroundjoin%
\pgfsetlinewidth{1.003750pt}%
\definecolor{currentstroke}{rgb}{0.000000,0.501961,0.000000}%
\pgfsetstrokecolor{currentstroke}%
\pgfsetdash{{3.700000pt}{1.600000pt}}{0.000000pt}%
\pgfpathmoveto{\pgfqpoint{1.025455in}{3.377648in}}%
\pgfpathlineto{\pgfqpoint{1.040797in}{3.365817in}}%
\pgfpathlineto{\pgfqpoint{1.056139in}{3.353986in}}%
\pgfpathlineto{\pgfqpoint{1.071481in}{3.342156in}}%
\pgfpathlineto{\pgfqpoint{1.086824in}{3.330325in}}%
\pgfpathlineto{\pgfqpoint{1.102166in}{3.318494in}}%
\pgfpathlineto{\pgfqpoint{1.117508in}{3.306664in}}%
\pgfpathlineto{\pgfqpoint{1.132850in}{3.294833in}}%
\pgfpathlineto{\pgfqpoint{1.148192in}{3.283002in}}%
\pgfpathlineto{\pgfqpoint{1.163535in}{3.271172in}}%
\pgfpathlineto{\pgfqpoint{1.178877in}{3.259341in}}%
\pgfpathlineto{\pgfqpoint{1.194219in}{3.247510in}}%
\pgfpathlineto{\pgfqpoint{1.209561in}{3.235680in}}%
\pgfpathlineto{\pgfqpoint{1.224904in}{3.223849in}}%
\pgfpathlineto{\pgfqpoint{1.240246in}{3.212018in}}%
\pgfpathlineto{\pgfqpoint{1.255588in}{3.200188in}}%
\pgfpathlineto{\pgfqpoint{1.270930in}{3.188357in}}%
\pgfpathlineto{\pgfqpoint{1.286273in}{3.176526in}}%
\pgfpathlineto{\pgfqpoint{1.301615in}{3.164696in}}%
\pgfpathlineto{\pgfqpoint{1.316957in}{3.152865in}}%
\pgfpathlineto{\pgfqpoint{1.332299in}{3.141034in}}%
\pgfpathlineto{\pgfqpoint{1.347642in}{3.129204in}}%
\pgfpathlineto{\pgfqpoint{1.362984in}{3.117373in}}%
\pgfpathlineto{\pgfqpoint{1.378326in}{3.105542in}}%
\pgfpathlineto{\pgfqpoint{1.393668in}{3.093712in}}%
\pgfpathlineto{\pgfqpoint{1.409011in}{3.081881in}}%
\pgfpathlineto{\pgfqpoint{1.424353in}{3.070050in}}%
\pgfpathlineto{\pgfqpoint{1.439695in}{3.058220in}}%
\pgfpathlineto{\pgfqpoint{1.455037in}{3.046389in}}%
\pgfpathlineto{\pgfqpoint{1.470379in}{3.034558in}}%
\pgfpathlineto{\pgfqpoint{1.485722in}{3.022728in}}%
\pgfpathlineto{\pgfqpoint{1.501064in}{3.010897in}}%
\pgfpathlineto{\pgfqpoint{1.516406in}{2.999066in}}%
\pgfpathlineto{\pgfqpoint{1.531748in}{2.987236in}}%
\pgfpathlineto{\pgfqpoint{1.547091in}{2.975405in}}%
\pgfpathlineto{\pgfqpoint{1.562433in}{2.963574in}}%
\pgfpathlineto{\pgfqpoint{1.577775in}{2.951744in}}%
\pgfpathlineto{\pgfqpoint{1.593117in}{2.939913in}}%
\pgfpathlineto{\pgfqpoint{1.608460in}{2.928082in}}%
\pgfpathlineto{\pgfqpoint{1.623802in}{2.916252in}}%
\pgfpathlineto{\pgfqpoint{1.639144in}{2.904421in}}%
\pgfpathlineto{\pgfqpoint{1.654486in}{2.892590in}}%
\pgfpathlineto{\pgfqpoint{1.669829in}{2.880760in}}%
\pgfpathlineto{\pgfqpoint{1.685171in}{2.868929in}}%
\pgfpathlineto{\pgfqpoint{1.700513in}{2.857099in}}%
\pgfpathlineto{\pgfqpoint{1.715855in}{2.845268in}}%
\pgfpathlineto{\pgfqpoint{1.731198in}{2.833437in}}%
\pgfpathlineto{\pgfqpoint{1.746540in}{2.821607in}}%
\pgfpathlineto{\pgfqpoint{1.761882in}{2.809776in}}%
\pgfpathlineto{\pgfqpoint{1.777224in}{2.797945in}}%
\pgfpathlineto{\pgfqpoint{1.792567in}{2.786115in}}%
\pgfpathlineto{\pgfqpoint{1.807909in}{2.774284in}}%
\pgfpathlineto{\pgfqpoint{1.823251in}{2.762453in}}%
\pgfpathlineto{\pgfqpoint{1.838593in}{2.750623in}}%
\pgfpathlineto{\pgfqpoint{1.853935in}{2.738792in}}%
\pgfpathlineto{\pgfqpoint{1.869278in}{2.726961in}}%
\pgfpathlineto{\pgfqpoint{1.884620in}{2.715131in}}%
\pgfpathlineto{\pgfqpoint{1.899962in}{2.703300in}}%
\pgfpathlineto{\pgfqpoint{1.915304in}{2.691469in}}%
\pgfpathlineto{\pgfqpoint{1.930647in}{2.679639in}}%
\pgfpathlineto{\pgfqpoint{1.945989in}{2.667808in}}%
\pgfpathlineto{\pgfqpoint{1.961331in}{2.655977in}}%
\pgfpathlineto{\pgfqpoint{1.976673in}{2.644147in}}%
\pgfpathlineto{\pgfqpoint{1.992016in}{2.632316in}}%
\pgfpathlineto{\pgfqpoint{2.007358in}{2.620485in}}%
\pgfpathlineto{\pgfqpoint{2.022700in}{2.608655in}}%
\pgfpathlineto{\pgfqpoint{2.038042in}{2.596824in}}%
\pgfpathlineto{\pgfqpoint{2.053385in}{2.584993in}}%
\pgfpathlineto{\pgfqpoint{2.068727in}{2.573163in}}%
\pgfpathlineto{\pgfqpoint{2.084069in}{2.561332in}}%
\pgfpathlineto{\pgfqpoint{2.099411in}{2.549501in}}%
\pgfpathlineto{\pgfqpoint{2.114754in}{2.537671in}}%
\pgfpathlineto{\pgfqpoint{2.130096in}{2.525840in}}%
\pgfpathlineto{\pgfqpoint{2.145438in}{2.514009in}}%
\pgfpathlineto{\pgfqpoint{2.160780in}{2.502179in}}%
\pgfpathlineto{\pgfqpoint{2.176123in}{2.490348in}}%
\pgfpathlineto{\pgfqpoint{2.191465in}{2.478517in}}%
\pgfpathlineto{\pgfqpoint{2.206807in}{2.466687in}}%
\pgfpathlineto{\pgfqpoint{2.222149in}{2.454856in}}%
\pgfpathlineto{\pgfqpoint{2.237491in}{2.443025in}}%
\pgfpathlineto{\pgfqpoint{2.252834in}{2.431195in}}%
\pgfpathlineto{\pgfqpoint{2.268176in}{2.419364in}}%
\pgfpathlineto{\pgfqpoint{2.283518in}{2.407533in}}%
\pgfpathlineto{\pgfqpoint{2.298860in}{2.395703in}}%
\pgfpathlineto{\pgfqpoint{2.314203in}{2.383872in}}%
\pgfpathlineto{\pgfqpoint{2.329545in}{2.372041in}}%
\pgfpathlineto{\pgfqpoint{2.344887in}{2.360211in}}%
\pgfpathlineto{\pgfqpoint{2.360229in}{2.348380in}}%
\pgfpathlineto{\pgfqpoint{2.375572in}{2.336549in}}%
\pgfpathlineto{\pgfqpoint{2.390914in}{2.324719in}}%
\pgfpathlineto{\pgfqpoint{2.406256in}{2.312888in}}%
\pgfpathlineto{\pgfqpoint{2.421598in}{2.301057in}}%
\pgfpathlineto{\pgfqpoint{2.436941in}{2.289227in}}%
\pgfpathlineto{\pgfqpoint{2.452283in}{2.277396in}}%
\pgfpathlineto{\pgfqpoint{2.467625in}{2.265565in}}%
\pgfpathlineto{\pgfqpoint{2.482967in}{2.253735in}}%
\pgfpathlineto{\pgfqpoint{2.498310in}{2.241904in}}%
\pgfpathlineto{\pgfqpoint{2.513652in}{2.230073in}}%
\pgfpathlineto{\pgfqpoint{2.528994in}{2.218243in}}%
\pgfpathlineto{\pgfqpoint{2.544336in}{2.206412in}}%
\pgfusepath{stroke}%
\end{pgfscope}%
\begin{pgfscope}%
\pgfpathrectangle{\pgfqpoint{0.800000in}{0.440000in}}{\pgfqpoint{4.960000in}{3.080000in}}%
\pgfusepath{clip}%
\pgfsetbuttcap%
\pgfsetroundjoin%
\definecolor{currentfill}{rgb}{1.000000,0.000000,0.000000}%
\pgfsetfillcolor{currentfill}%
\pgfsetlinewidth{1.003750pt}%
\definecolor{currentstroke}{rgb}{1.000000,0.000000,0.000000}%
\pgfsetstrokecolor{currentstroke}%
\pgfsetdash{}{0pt}%
\pgfsys@defobject{currentmarker}{\pgfqpoint{-0.010417in}{-0.010417in}}{\pgfqpoint{0.010417in}{0.010417in}}{%
\pgfpathmoveto{\pgfqpoint{0.000000in}{-0.010417in}}%
\pgfpathcurveto{\pgfqpoint{0.002763in}{-0.010417in}}{\pgfqpoint{0.005412in}{-0.009319in}}{\pgfqpoint{0.007366in}{-0.007366in}}%
\pgfpathcurveto{\pgfqpoint{0.009319in}{-0.005412in}}{\pgfqpoint{0.010417in}{-0.002763in}}{\pgfqpoint{0.010417in}{0.000000in}}%
\pgfpathcurveto{\pgfqpoint{0.010417in}{0.002763in}}{\pgfqpoint{0.009319in}{0.005412in}}{\pgfqpoint{0.007366in}{0.007366in}}%
\pgfpathcurveto{\pgfqpoint{0.005412in}{0.009319in}}{\pgfqpoint{0.002763in}{0.010417in}}{\pgfqpoint{0.000000in}{0.010417in}}%
\pgfpathcurveto{\pgfqpoint{-0.002763in}{0.010417in}}{\pgfqpoint{-0.005412in}{0.009319in}}{\pgfqpoint{-0.007366in}{0.007366in}}%
\pgfpathcurveto{\pgfqpoint{-0.009319in}{0.005412in}}{\pgfqpoint{-0.010417in}{0.002763in}}{\pgfqpoint{-0.010417in}{0.000000in}}%
\pgfpathcurveto{\pgfqpoint{-0.010417in}{-0.002763in}}{\pgfqpoint{-0.009319in}{-0.005412in}}{\pgfqpoint{-0.007366in}{-0.007366in}}%
\pgfpathcurveto{\pgfqpoint{-0.005412in}{-0.009319in}}{\pgfqpoint{-0.002763in}{-0.010417in}}{\pgfqpoint{0.000000in}{-0.010417in}}%
\pgfpathclose%
\pgfusepath{stroke,fill}%
}%
\begin{pgfscope}%
\pgfsys@transformshift{1.025455in}{3.377648in}%
\pgfsys@useobject{currentmarker}{}%
\end{pgfscope}%
\begin{pgfscope}%
\pgfsys@transformshift{1.589091in}{3.237774in}%
\pgfsys@useobject{currentmarker}{}%
\end{pgfscope}%
\begin{pgfscope}%
\pgfsys@transformshift{2.152727in}{3.058907in}%
\pgfsys@useobject{currentmarker}{}%
\end{pgfscope}%
\begin{pgfscope}%
\pgfsys@transformshift{2.716364in}{2.865743in}%
\pgfsys@useobject{currentmarker}{}%
\end{pgfscope}%
\begin{pgfscope}%
\pgfsys@transformshift{3.280000in}{2.667930in}%
\pgfsys@useobject{currentmarker}{}%
\end{pgfscope}%
\begin{pgfscope}%
\pgfsys@transformshift{3.843636in}{2.466798in}%
\pgfsys@useobject{currentmarker}{}%
\end{pgfscope}%
\begin{pgfscope}%
\pgfsys@transformshift{4.407273in}{2.206365in}%
\pgfsys@useobject{currentmarker}{}%
\end{pgfscope}%
\begin{pgfscope}%
\pgfsys@transformshift{4.970909in}{1.910426in}%
\pgfsys@useobject{currentmarker}{}%
\end{pgfscope}%
\begin{pgfscope}%
\pgfsys@transformshift{5.534545in}{1.414807in}%
\pgfsys@useobject{currentmarker}{}%
\end{pgfscope}%
\end{pgfscope}%
\begin{pgfscope}%
\pgfpathrectangle{\pgfqpoint{0.800000in}{0.440000in}}{\pgfqpoint{4.960000in}{3.080000in}}%
\pgfusepath{clip}%
\pgfsetbuttcap%
\pgfsetroundjoin%
\definecolor{currentfill}{rgb}{0.000000,0.000000,1.000000}%
\pgfsetfillcolor{currentfill}%
\pgfsetlinewidth{1.003750pt}%
\definecolor{currentstroke}{rgb}{0.000000,0.000000,1.000000}%
\pgfsetstrokecolor{currentstroke}%
\pgfsetdash{}{0pt}%
\pgfsys@defobject{currentmarker}{\pgfqpoint{-0.010417in}{-0.010417in}}{\pgfqpoint{0.010417in}{0.010417in}}{%
\pgfpathmoveto{\pgfqpoint{0.000000in}{-0.010417in}}%
\pgfpathcurveto{\pgfqpoint{0.002763in}{-0.010417in}}{\pgfqpoint{0.005412in}{-0.009319in}}{\pgfqpoint{0.007366in}{-0.007366in}}%
\pgfpathcurveto{\pgfqpoint{0.009319in}{-0.005412in}}{\pgfqpoint{0.010417in}{-0.002763in}}{\pgfqpoint{0.010417in}{0.000000in}}%
\pgfpathcurveto{\pgfqpoint{0.010417in}{0.002763in}}{\pgfqpoint{0.009319in}{0.005412in}}{\pgfqpoint{0.007366in}{0.007366in}}%
\pgfpathcurveto{\pgfqpoint{0.005412in}{0.009319in}}{\pgfqpoint{0.002763in}{0.010417in}}{\pgfqpoint{0.000000in}{0.010417in}}%
\pgfpathcurveto{\pgfqpoint{-0.002763in}{0.010417in}}{\pgfqpoint{-0.005412in}{0.009319in}}{\pgfqpoint{-0.007366in}{0.007366in}}%
\pgfpathcurveto{\pgfqpoint{-0.009319in}{0.005412in}}{\pgfqpoint{-0.010417in}{0.002763in}}{\pgfqpoint{-0.010417in}{0.000000in}}%
\pgfpathcurveto{\pgfqpoint{-0.010417in}{-0.002763in}}{\pgfqpoint{-0.009319in}{-0.005412in}}{\pgfqpoint{-0.007366in}{-0.007366in}}%
\pgfpathcurveto{\pgfqpoint{-0.005412in}{-0.009319in}}{\pgfqpoint{-0.002763in}{-0.010417in}}{\pgfqpoint{0.000000in}{-0.010417in}}%
\pgfpathclose%
\pgfusepath{stroke,fill}%
}%
\begin{pgfscope}%
\pgfsys@transformshift{1.025455in}{3.377648in}%
\pgfsys@useobject{currentmarker}{}%
\end{pgfscope}%
\begin{pgfscope}%
\pgfsys@transformshift{1.212954in}{3.312412in}%
\pgfsys@useobject{currentmarker}{}%
\end{pgfscope}%
\begin{pgfscope}%
\pgfsys@transformshift{1.400454in}{3.212467in}%
\pgfsys@useobject{currentmarker}{}%
\end{pgfscope}%
\begin{pgfscope}%
\pgfsys@transformshift{1.775452in}{3.027462in}%
\pgfsys@useobject{currentmarker}{}%
\end{pgfscope}%
\begin{pgfscope}%
\pgfsys@transformshift{2.150451in}{2.826330in}%
\pgfsys@useobject{currentmarker}{}%
\end{pgfscope}%
\begin{pgfscope}%
\pgfsys@transformshift{2.525450in}{2.609147in}%
\pgfsys@useobject{currentmarker}{}%
\end{pgfscope}%
\begin{pgfscope}%
\pgfsys@transformshift{2.900449in}{2.344184in}%
\pgfsys@useobject{currentmarker}{}%
\end{pgfscope}%
\begin{pgfscope}%
\pgfsys@transformshift{3.275448in}{2.133871in}%
\pgfsys@useobject{currentmarker}{}%
\end{pgfscope}%
\begin{pgfscope}%
\pgfsys@transformshift{3.650447in}{1.868909in}%
\pgfsys@useobject{currentmarker}{}%
\end{pgfscope}%
\end{pgfscope}%
\begin{pgfscope}%
\pgfpathrectangle{\pgfqpoint{0.800000in}{0.440000in}}{\pgfqpoint{4.960000in}{3.080000in}}%
\pgfusepath{clip}%
\pgfsetbuttcap%
\pgfsetroundjoin%
\definecolor{currentfill}{rgb}{0.000000,0.501961,0.000000}%
\pgfsetfillcolor{currentfill}%
\pgfsetlinewidth{1.003750pt}%
\definecolor{currentstroke}{rgb}{0.000000,0.501961,0.000000}%
\pgfsetstrokecolor{currentstroke}%
\pgfsetdash{}{0pt}%
\pgfsys@defobject{currentmarker}{\pgfqpoint{-0.010417in}{-0.010417in}}{\pgfqpoint{0.010417in}{0.010417in}}{%
\pgfpathmoveto{\pgfqpoint{0.000000in}{-0.010417in}}%
\pgfpathcurveto{\pgfqpoint{0.002763in}{-0.010417in}}{\pgfqpoint{0.005412in}{-0.009319in}}{\pgfqpoint{0.007366in}{-0.007366in}}%
\pgfpathcurveto{\pgfqpoint{0.009319in}{-0.005412in}}{\pgfqpoint{0.010417in}{-0.002763in}}{\pgfqpoint{0.010417in}{0.000000in}}%
\pgfpathcurveto{\pgfqpoint{0.010417in}{0.002763in}}{\pgfqpoint{0.009319in}{0.005412in}}{\pgfqpoint{0.007366in}{0.007366in}}%
\pgfpathcurveto{\pgfqpoint{0.005412in}{0.009319in}}{\pgfqpoint{0.002763in}{0.010417in}}{\pgfqpoint{0.000000in}{0.010417in}}%
\pgfpathcurveto{\pgfqpoint{-0.002763in}{0.010417in}}{\pgfqpoint{-0.005412in}{0.009319in}}{\pgfqpoint{-0.007366in}{0.007366in}}%
\pgfpathcurveto{\pgfqpoint{-0.009319in}{0.005412in}}{\pgfqpoint{-0.010417in}{0.002763in}}{\pgfqpoint{-0.010417in}{0.000000in}}%
\pgfpathcurveto{\pgfqpoint{-0.010417in}{-0.002763in}}{\pgfqpoint{-0.009319in}{-0.005412in}}{\pgfqpoint{-0.007366in}{-0.007366in}}%
\pgfpathcurveto{\pgfqpoint{-0.005412in}{-0.009319in}}{\pgfqpoint{-0.002763in}{-0.010417in}}{\pgfqpoint{0.000000in}{-0.010417in}}%
\pgfpathclose%
\pgfusepath{stroke,fill}%
}%
\begin{pgfscope}%
\pgfsys@transformshift{1.025455in}{3.377648in}%
\pgfsys@useobject{currentmarker}{}%
\end{pgfscope}%
\begin{pgfscope}%
\pgfsys@transformshift{1.215315in}{3.243913in}%
\pgfsys@useobject{currentmarker}{}%
\end{pgfscope}%
\begin{pgfscope}%
\pgfsys@transformshift{1.405175in}{3.096724in}%
\pgfsys@useobject{currentmarker}{}%
\end{pgfscope}%
\begin{pgfscope}%
\pgfsys@transformshift{1.595035in}{2.947505in}%
\pgfsys@useobject{currentmarker}{}%
\end{pgfscope}%
\begin{pgfscope}%
\pgfsys@transformshift{1.784895in}{2.798285in}%
\pgfsys@useobject{currentmarker}{}%
\end{pgfscope}%
\begin{pgfscope}%
\pgfsys@transformshift{1.974756in}{2.629490in}%
\pgfsys@useobject{currentmarker}{}%
\end{pgfscope}%
\begin{pgfscope}%
\pgfsys@transformshift{2.164616in}{2.466798in}%
\pgfsys@useobject{currentmarker}{}%
\end{pgfscope}%
\begin{pgfscope}%
\pgfsys@transformshift{2.354476in}{2.308398in}%
\pgfsys@useobject{currentmarker}{}%
\end{pgfscope}%
\begin{pgfscope}%
\pgfsys@transformshift{2.544336in}{2.133871in}%
\pgfsys@useobject{currentmarker}{}%
\end{pgfscope}%
\end{pgfscope}%
\begin{pgfscope}%
\pgfsetrectcap%
\pgfsetmiterjoin%
\pgfsetlinewidth{0.803000pt}%
\definecolor{currentstroke}{rgb}{0.000000,0.000000,0.000000}%
\pgfsetstrokecolor{currentstroke}%
\pgfsetdash{}{0pt}%
\pgfpathmoveto{\pgfqpoint{0.800000in}{0.440000in}}%
\pgfpathlineto{\pgfqpoint{0.800000in}{3.520000in}}%
\pgfusepath{stroke}%
\end{pgfscope}%
\begin{pgfscope}%
\pgfsetrectcap%
\pgfsetmiterjoin%
\pgfsetlinewidth{0.803000pt}%
\definecolor{currentstroke}{rgb}{0.000000,0.000000,0.000000}%
\pgfsetstrokecolor{currentstroke}%
\pgfsetdash{}{0pt}%
\pgfpathmoveto{\pgfqpoint{5.760000in}{0.440000in}}%
\pgfpathlineto{\pgfqpoint{5.760000in}{3.520000in}}%
\pgfusepath{stroke}%
\end{pgfscope}%
\begin{pgfscope}%
\pgfsetrectcap%
\pgfsetmiterjoin%
\pgfsetlinewidth{0.803000pt}%
\definecolor{currentstroke}{rgb}{0.000000,0.000000,0.000000}%
\pgfsetstrokecolor{currentstroke}%
\pgfsetdash{}{0pt}%
\pgfpathmoveto{\pgfqpoint{0.800000in}{0.440000in}}%
\pgfpathlineto{\pgfqpoint{5.760000in}{0.440000in}}%
\pgfusepath{stroke}%
\end{pgfscope}%
\begin{pgfscope}%
\pgfsetrectcap%
\pgfsetmiterjoin%
\pgfsetlinewidth{0.803000pt}%
\definecolor{currentstroke}{rgb}{0.000000,0.000000,0.000000}%
\pgfsetstrokecolor{currentstroke}%
\pgfsetdash{}{0pt}%
\pgfpathmoveto{\pgfqpoint{0.800000in}{3.520000in}}%
\pgfpathlineto{\pgfqpoint{5.760000in}{3.520000in}}%
\pgfusepath{stroke}%
\end{pgfscope}%
\begin{pgfscope}%
\pgfsetbuttcap%
\pgfsetmiterjoin%
\definecolor{currentfill}{rgb}{1.000000,1.000000,1.000000}%
\pgfsetfillcolor{currentfill}%
\pgfsetfillopacity{0.800000}%
\pgfsetlinewidth{1.003750pt}%
\definecolor{currentstroke}{rgb}{0.800000,0.800000,0.800000}%
\pgfsetstrokecolor{currentstroke}%
\pgfsetstrokeopacity{0.800000}%
\pgfsetdash{}{0pt}%
\pgfpathmoveto{\pgfqpoint{0.877778in}{0.495556in}}%
\pgfpathlineto{\pgfqpoint{3.088306in}{0.495556in}}%
\pgfpathquadraticcurveto{\pgfqpoint{3.110529in}{0.495556in}}{\pgfqpoint{3.110529in}{0.517778in}}%
\pgfpathlineto{\pgfqpoint{3.110529in}{1.947453in}}%
\pgfpathquadraticcurveto{\pgfqpoint{3.110529in}{1.969675in}}{\pgfqpoint{3.088306in}{1.969675in}}%
\pgfpathlineto{\pgfqpoint{0.877778in}{1.969675in}}%
\pgfpathquadraticcurveto{\pgfqpoint{0.855556in}{1.969675in}}{\pgfqpoint{0.855556in}{1.947453in}}%
\pgfpathlineto{\pgfqpoint{0.855556in}{0.517778in}}%
\pgfpathquadraticcurveto{\pgfqpoint{0.855556in}{0.495556in}}{\pgfqpoint{0.877778in}{0.495556in}}%
\pgfpathclose%
\pgfusepath{stroke,fill}%
\end{pgfscope}%
\begin{pgfscope}%
\pgfsetbuttcap%
\pgfsetroundjoin%
\pgfsetlinewidth{1.003750pt}%
\definecolor{currentstroke}{rgb}{1.000000,0.000000,0.000000}%
\pgfsetstrokecolor{currentstroke}%
\pgfsetdash{{3.700000pt}{1.600000pt}}{0.000000pt}%
\pgfpathmoveto{\pgfqpoint{0.900000in}{1.870895in}}%
\pgfpathlineto{\pgfqpoint{1.122222in}{1.870895in}}%
\pgfusepath{stroke}%
\end{pgfscope}%
\begin{pgfscope}%
\definecolor{textcolor}{rgb}{0.000000,0.000000,0.000000}%
\pgfsetstrokecolor{textcolor}%
\pgfsetfillcolor{textcolor}%
\pgftext[x=1.211111in,y=1.832006in,left,base]{\color{textcolor}\rmfamily\fontsize{8.000000}{9.600000}\selectfont \(\displaystyle A_0 e^{-\alpha_1 t}\) with \(\displaystyle \alpha_1\) = 0.057}%
\end{pgfscope}%
\begin{pgfscope}%
\pgfsetbuttcap%
\pgfsetroundjoin%
\pgfsetlinewidth{1.003750pt}%
\definecolor{currentstroke}{rgb}{0.000000,0.000000,1.000000}%
\pgfsetstrokecolor{currentstroke}%
\pgfsetdash{{3.700000pt}{1.600000pt}}{0.000000pt}%
\pgfpathmoveto{\pgfqpoint{0.900000in}{1.700509in}}%
\pgfpathlineto{\pgfqpoint{1.122222in}{1.700509in}}%
\pgfusepath{stroke}%
\end{pgfscope}%
\begin{pgfscope}%
\definecolor{textcolor}{rgb}{0.000000,0.000000,0.000000}%
\pgfsetstrokecolor{textcolor}%
\pgfsetfillcolor{textcolor}%
\pgftext[x=1.211111in,y=1.661620in,left,base]{\color{textcolor}\rmfamily\fontsize{8.000000}{9.600000}\selectfont \(\displaystyle A_0 e^{-\alpha_2 t}\) with \(\displaystyle \alpha_2\) = 0.091}%
\end{pgfscope}%
\begin{pgfscope}%
\pgfsetbuttcap%
\pgfsetroundjoin%
\pgfsetlinewidth{1.003750pt}%
\definecolor{currentstroke}{rgb}{0.000000,0.501961,0.000000}%
\pgfsetstrokecolor{currentstroke}%
\pgfsetdash{{3.700000pt}{1.600000pt}}{0.000000pt}%
\pgfpathmoveto{\pgfqpoint{0.900000in}{1.530124in}}%
\pgfpathlineto{\pgfqpoint{1.122222in}{1.530124in}}%
\pgfusepath{stroke}%
\end{pgfscope}%
\begin{pgfscope}%
\definecolor{textcolor}{rgb}{0.000000,0.000000,0.000000}%
\pgfsetstrokecolor{textcolor}%
\pgfsetfillcolor{textcolor}%
\pgftext[x=1.211111in,y=1.491235in,left,base]{\color{textcolor}\rmfamily\fontsize{8.000000}{9.600000}\selectfont \(\displaystyle A_0 e^{-\alpha_3 t}\) with \(\displaystyle \alpha_3\) = 0.141}%
\end{pgfscope}%
\begin{pgfscope}%
\pgfsetbuttcap%
\pgfsetmiterjoin%
\definecolor{currentfill}{rgb}{1.000000,0.000000,0.000000}%
\pgfsetfillcolor{currentfill}%
\pgfsetfillopacity{0.100000}%
\pgfsetlinewidth{1.003750pt}%
\definecolor{currentstroke}{rgb}{1.000000,0.000000,0.000000}%
\pgfsetstrokecolor{currentstroke}%
\pgfsetstrokeopacity{0.100000}%
\pgfsetdash{}{0pt}%
\pgfpathmoveto{\pgfqpoint{0.900000in}{1.336297in}}%
\pgfpathlineto{\pgfqpoint{1.122222in}{1.336297in}}%
\pgfpathlineto{\pgfqpoint{1.122222in}{1.414074in}}%
\pgfpathlineto{\pgfqpoint{0.900000in}{1.414074in}}%
\pgfpathclose%
\pgfusepath{stroke,fill}%
\end{pgfscope}%
\begin{pgfscope}%
\definecolor{textcolor}{rgb}{0.000000,0.000000,0.000000}%
\pgfsetstrokecolor{textcolor}%
\pgfsetfillcolor{textcolor}%
\pgftext[x=1.211111in,y=1.336297in,left,base]{\color{textcolor}\rmfamily\fontsize{8.000000}{9.600000}\selectfont \(\displaystyle \alpha_{1min}\) = 0.06,  \(\displaystyle \alpha_{1max}\) = 0.054}%
\end{pgfscope}%
\begin{pgfscope}%
\pgfsetbuttcap%
\pgfsetmiterjoin%
\definecolor{currentfill}{rgb}{0.000000,0.000000,1.000000}%
\pgfsetfillcolor{currentfill}%
\pgfsetfillopacity{0.100000}%
\pgfsetlinewidth{1.003750pt}%
\definecolor{currentstroke}{rgb}{0.000000,0.000000,1.000000}%
\pgfsetstrokecolor{currentstroke}%
\pgfsetstrokeopacity{0.100000}%
\pgfsetdash{}{0pt}%
\pgfpathmoveto{\pgfqpoint{0.900000in}{1.181358in}}%
\pgfpathlineto{\pgfqpoint{1.122222in}{1.181358in}}%
\pgfpathlineto{\pgfqpoint{1.122222in}{1.259136in}}%
\pgfpathlineto{\pgfqpoint{0.900000in}{1.259136in}}%
\pgfpathclose%
\pgfusepath{stroke,fill}%
\end{pgfscope}%
\begin{pgfscope}%
\definecolor{textcolor}{rgb}{0.000000,0.000000,0.000000}%
\pgfsetstrokecolor{textcolor}%
\pgfsetfillcolor{textcolor}%
\pgftext[x=1.211111in,y=1.181358in,left,base]{\color{textcolor}\rmfamily\fontsize{8.000000}{9.600000}\selectfont \(\displaystyle \alpha_{2min}\) = 0.097,  \(\displaystyle \alpha_{2max}\) = 0.086}%
\end{pgfscope}%
\begin{pgfscope}%
\pgfsetbuttcap%
\pgfsetmiterjoin%
\definecolor{currentfill}{rgb}{0.000000,0.501961,0.000000}%
\pgfsetfillcolor{currentfill}%
\pgfsetfillopacity{0.100000}%
\pgfsetlinewidth{1.003750pt}%
\definecolor{currentstroke}{rgb}{0.000000,0.501961,0.000000}%
\pgfsetstrokecolor{currentstroke}%
\pgfsetstrokeopacity{0.100000}%
\pgfsetdash{}{0pt}%
\pgfpathmoveto{\pgfqpoint{0.900000in}{1.026420in}}%
\pgfpathlineto{\pgfqpoint{1.122222in}{1.026420in}}%
\pgfpathlineto{\pgfqpoint{1.122222in}{1.104198in}}%
\pgfpathlineto{\pgfqpoint{0.900000in}{1.104198in}}%
\pgfpathclose%
\pgfusepath{stroke,fill}%
\end{pgfscope}%
\begin{pgfscope}%
\definecolor{textcolor}{rgb}{0.000000,0.000000,0.000000}%
\pgfsetstrokecolor{textcolor}%
\pgfsetfillcolor{textcolor}%
\pgftext[x=1.211111in,y=1.026420in,left,base]{\color{textcolor}\rmfamily\fontsize{8.000000}{9.600000}\selectfont \(\displaystyle \alpha_{3min}\) = 0.15,  \(\displaystyle \alpha_{3max}\) = 0.134}%
\end{pgfscope}%
\begin{pgfscope}%
\pgfsetbuttcap%
\pgfsetroundjoin%
\pgfsetlinewidth{1.003750pt}%
\definecolor{currentstroke}{rgb}{1.000000,0.000000,0.000000}%
\pgfsetstrokecolor{currentstroke}%
\pgfsetdash{}{0pt}%
\pgfpathmoveto{\pgfqpoint{1.011111in}{0.854815in}}%
\pgfpathlineto{\pgfqpoint{1.011111in}{0.965926in}}%
\pgfusepath{stroke}%
\end{pgfscope}%
\begin{pgfscope}%
\pgfsetbuttcap%
\pgfsetroundjoin%
\definecolor{currentfill}{rgb}{1.000000,0.000000,0.000000}%
\pgfsetfillcolor{currentfill}%
\pgfsetlinewidth{1.003750pt}%
\definecolor{currentstroke}{rgb}{1.000000,0.000000,0.000000}%
\pgfsetstrokecolor{currentstroke}%
\pgfsetdash{}{0pt}%
\pgfsys@defobject{currentmarker}{\pgfqpoint{-0.041667in}{-0.000000in}}{\pgfqpoint{0.041667in}{0.000000in}}{%
\pgfpathmoveto{\pgfqpoint{0.041667in}{-0.000000in}}%
\pgfpathlineto{\pgfqpoint{-0.041667in}{0.000000in}}%
\pgfusepath{stroke,fill}%
}%
\begin{pgfscope}%
\pgfsys@transformshift{1.011111in}{0.854815in}%
\pgfsys@useobject{currentmarker}{}%
\end{pgfscope}%
\end{pgfscope}%
\begin{pgfscope}%
\pgfsetbuttcap%
\pgfsetroundjoin%
\definecolor{currentfill}{rgb}{1.000000,0.000000,0.000000}%
\pgfsetfillcolor{currentfill}%
\pgfsetlinewidth{1.003750pt}%
\definecolor{currentstroke}{rgb}{1.000000,0.000000,0.000000}%
\pgfsetstrokecolor{currentstroke}%
\pgfsetdash{}{0pt}%
\pgfsys@defobject{currentmarker}{\pgfqpoint{-0.041667in}{-0.000000in}}{\pgfqpoint{0.041667in}{0.000000in}}{%
\pgfpathmoveto{\pgfqpoint{0.041667in}{-0.000000in}}%
\pgfpathlineto{\pgfqpoint{-0.041667in}{0.000000in}}%
\pgfusepath{stroke,fill}%
}%
\begin{pgfscope}%
\pgfsys@transformshift{1.011111in}{0.965926in}%
\pgfsys@useobject{currentmarker}{}%
\end{pgfscope}%
\end{pgfscope}%
\begin{pgfscope}%
\pgfsetbuttcap%
\pgfsetroundjoin%
\definecolor{currentfill}{rgb}{1.000000,0.000000,0.000000}%
\pgfsetfillcolor{currentfill}%
\pgfsetlinewidth{1.003750pt}%
\definecolor{currentstroke}{rgb}{1.000000,0.000000,0.000000}%
\pgfsetstrokecolor{currentstroke}%
\pgfsetdash{}{0pt}%
\pgfsys@defobject{currentmarker}{\pgfqpoint{-0.010417in}{-0.010417in}}{\pgfqpoint{0.010417in}{0.010417in}}{%
\pgfpathmoveto{\pgfqpoint{0.000000in}{-0.010417in}}%
\pgfpathcurveto{\pgfqpoint{0.002763in}{-0.010417in}}{\pgfqpoint{0.005412in}{-0.009319in}}{\pgfqpoint{0.007366in}{-0.007366in}}%
\pgfpathcurveto{\pgfqpoint{0.009319in}{-0.005412in}}{\pgfqpoint{0.010417in}{-0.002763in}}{\pgfqpoint{0.010417in}{0.000000in}}%
\pgfpathcurveto{\pgfqpoint{0.010417in}{0.002763in}}{\pgfqpoint{0.009319in}{0.005412in}}{\pgfqpoint{0.007366in}{0.007366in}}%
\pgfpathcurveto{\pgfqpoint{0.005412in}{0.009319in}}{\pgfqpoint{0.002763in}{0.010417in}}{\pgfqpoint{0.000000in}{0.010417in}}%
\pgfpathcurveto{\pgfqpoint{-0.002763in}{0.010417in}}{\pgfqpoint{-0.005412in}{0.009319in}}{\pgfqpoint{-0.007366in}{0.007366in}}%
\pgfpathcurveto{\pgfqpoint{-0.009319in}{0.005412in}}{\pgfqpoint{-0.010417in}{0.002763in}}{\pgfqpoint{-0.010417in}{0.000000in}}%
\pgfpathcurveto{\pgfqpoint{-0.010417in}{-0.002763in}}{\pgfqpoint{-0.009319in}{-0.005412in}}{\pgfqpoint{-0.007366in}{-0.007366in}}%
\pgfpathcurveto{\pgfqpoint{-0.005412in}{-0.009319in}}{\pgfqpoint{-0.002763in}{-0.010417in}}{\pgfqpoint{0.000000in}{-0.010417in}}%
\pgfpathclose%
\pgfusepath{stroke,fill}%
}%
\begin{pgfscope}%
\pgfsys@transformshift{1.011111in}{0.910371in}%
\pgfsys@useobject{currentmarker}{}%
\end{pgfscope}%
\end{pgfscope}%
\begin{pgfscope}%
\definecolor{textcolor}{rgb}{0.000000,0.000000,0.000000}%
\pgfsetstrokecolor{textcolor}%
\pgfsetfillcolor{textcolor}%
\pgftext[x=1.211111in,y=0.871482in,left,base]{\color{textcolor}\rmfamily\fontsize{8.000000}{9.600000}\selectfont Measured amplitudes dampening \(\displaystyle I_2\)}%
\end{pgfscope}%
\begin{pgfscope}%
\pgfsetbuttcap%
\pgfsetroundjoin%
\pgfsetlinewidth{1.003750pt}%
\definecolor{currentstroke}{rgb}{0.000000,0.000000,1.000000}%
\pgfsetstrokecolor{currentstroke}%
\pgfsetdash{}{0pt}%
\pgfpathmoveto{\pgfqpoint{1.011111in}{0.699877in}}%
\pgfpathlineto{\pgfqpoint{1.011111in}{0.810988in}}%
\pgfusepath{stroke}%
\end{pgfscope}%
\begin{pgfscope}%
\pgfsetbuttcap%
\pgfsetroundjoin%
\definecolor{currentfill}{rgb}{0.000000,0.000000,1.000000}%
\pgfsetfillcolor{currentfill}%
\pgfsetlinewidth{1.003750pt}%
\definecolor{currentstroke}{rgb}{0.000000,0.000000,1.000000}%
\pgfsetstrokecolor{currentstroke}%
\pgfsetdash{}{0pt}%
\pgfsys@defobject{currentmarker}{\pgfqpoint{-0.041667in}{-0.000000in}}{\pgfqpoint{0.041667in}{0.000000in}}{%
\pgfpathmoveto{\pgfqpoint{0.041667in}{-0.000000in}}%
\pgfpathlineto{\pgfqpoint{-0.041667in}{0.000000in}}%
\pgfusepath{stroke,fill}%
}%
\begin{pgfscope}%
\pgfsys@transformshift{1.011111in}{0.699877in}%
\pgfsys@useobject{currentmarker}{}%
\end{pgfscope}%
\end{pgfscope}%
\begin{pgfscope}%
\pgfsetbuttcap%
\pgfsetroundjoin%
\definecolor{currentfill}{rgb}{0.000000,0.000000,1.000000}%
\pgfsetfillcolor{currentfill}%
\pgfsetlinewidth{1.003750pt}%
\definecolor{currentstroke}{rgb}{0.000000,0.000000,1.000000}%
\pgfsetstrokecolor{currentstroke}%
\pgfsetdash{}{0pt}%
\pgfsys@defobject{currentmarker}{\pgfqpoint{-0.041667in}{-0.000000in}}{\pgfqpoint{0.041667in}{0.000000in}}{%
\pgfpathmoveto{\pgfqpoint{0.041667in}{-0.000000in}}%
\pgfpathlineto{\pgfqpoint{-0.041667in}{0.000000in}}%
\pgfusepath{stroke,fill}%
}%
\begin{pgfscope}%
\pgfsys@transformshift{1.011111in}{0.810988in}%
\pgfsys@useobject{currentmarker}{}%
\end{pgfscope}%
\end{pgfscope}%
\begin{pgfscope}%
\pgfsetbuttcap%
\pgfsetroundjoin%
\definecolor{currentfill}{rgb}{0.000000,0.000000,1.000000}%
\pgfsetfillcolor{currentfill}%
\pgfsetlinewidth{1.003750pt}%
\definecolor{currentstroke}{rgb}{0.000000,0.000000,1.000000}%
\pgfsetstrokecolor{currentstroke}%
\pgfsetdash{}{0pt}%
\pgfsys@defobject{currentmarker}{\pgfqpoint{-0.010417in}{-0.010417in}}{\pgfqpoint{0.010417in}{0.010417in}}{%
\pgfpathmoveto{\pgfqpoint{0.000000in}{-0.010417in}}%
\pgfpathcurveto{\pgfqpoint{0.002763in}{-0.010417in}}{\pgfqpoint{0.005412in}{-0.009319in}}{\pgfqpoint{0.007366in}{-0.007366in}}%
\pgfpathcurveto{\pgfqpoint{0.009319in}{-0.005412in}}{\pgfqpoint{0.010417in}{-0.002763in}}{\pgfqpoint{0.010417in}{0.000000in}}%
\pgfpathcurveto{\pgfqpoint{0.010417in}{0.002763in}}{\pgfqpoint{0.009319in}{0.005412in}}{\pgfqpoint{0.007366in}{0.007366in}}%
\pgfpathcurveto{\pgfqpoint{0.005412in}{0.009319in}}{\pgfqpoint{0.002763in}{0.010417in}}{\pgfqpoint{0.000000in}{0.010417in}}%
\pgfpathcurveto{\pgfqpoint{-0.002763in}{0.010417in}}{\pgfqpoint{-0.005412in}{0.009319in}}{\pgfqpoint{-0.007366in}{0.007366in}}%
\pgfpathcurveto{\pgfqpoint{-0.009319in}{0.005412in}}{\pgfqpoint{-0.010417in}{0.002763in}}{\pgfqpoint{-0.010417in}{0.000000in}}%
\pgfpathcurveto{\pgfqpoint{-0.010417in}{-0.002763in}}{\pgfqpoint{-0.009319in}{-0.005412in}}{\pgfqpoint{-0.007366in}{-0.007366in}}%
\pgfpathcurveto{\pgfqpoint{-0.005412in}{-0.009319in}}{\pgfqpoint{-0.002763in}{-0.010417in}}{\pgfqpoint{0.000000in}{-0.010417in}}%
\pgfpathclose%
\pgfusepath{stroke,fill}%
}%
\begin{pgfscope}%
\pgfsys@transformshift{1.011111in}{0.755432in}%
\pgfsys@useobject{currentmarker}{}%
\end{pgfscope}%
\end{pgfscope}%
\begin{pgfscope}%
\definecolor{textcolor}{rgb}{0.000000,0.000000,0.000000}%
\pgfsetstrokecolor{textcolor}%
\pgfsetfillcolor{textcolor}%
\pgftext[x=1.211111in,y=0.716543in,left,base]{\color{textcolor}\rmfamily\fontsize{8.000000}{9.600000}\selectfont Measured amplitudes dampening \(\displaystyle I_2\)}%
\end{pgfscope}%
\begin{pgfscope}%
\pgfsetbuttcap%
\pgfsetroundjoin%
\pgfsetlinewidth{1.003750pt}%
\definecolor{currentstroke}{rgb}{0.000000,0.501961,0.000000}%
\pgfsetstrokecolor{currentstroke}%
\pgfsetdash{}{0pt}%
\pgfpathmoveto{\pgfqpoint{1.011111in}{0.544938in}}%
\pgfpathlineto{\pgfqpoint{1.011111in}{0.656049in}}%
\pgfusepath{stroke}%
\end{pgfscope}%
\begin{pgfscope}%
\pgfsetbuttcap%
\pgfsetroundjoin%
\definecolor{currentfill}{rgb}{0.000000,0.501961,0.000000}%
\pgfsetfillcolor{currentfill}%
\pgfsetlinewidth{1.003750pt}%
\definecolor{currentstroke}{rgb}{0.000000,0.501961,0.000000}%
\pgfsetstrokecolor{currentstroke}%
\pgfsetdash{}{0pt}%
\pgfsys@defobject{currentmarker}{\pgfqpoint{-0.041667in}{-0.000000in}}{\pgfqpoint{0.041667in}{0.000000in}}{%
\pgfpathmoveto{\pgfqpoint{0.041667in}{-0.000000in}}%
\pgfpathlineto{\pgfqpoint{-0.041667in}{0.000000in}}%
\pgfusepath{stroke,fill}%
}%
\begin{pgfscope}%
\pgfsys@transformshift{1.011111in}{0.544938in}%
\pgfsys@useobject{currentmarker}{}%
\end{pgfscope}%
\end{pgfscope}%
\begin{pgfscope}%
\pgfsetbuttcap%
\pgfsetroundjoin%
\definecolor{currentfill}{rgb}{0.000000,0.501961,0.000000}%
\pgfsetfillcolor{currentfill}%
\pgfsetlinewidth{1.003750pt}%
\definecolor{currentstroke}{rgb}{0.000000,0.501961,0.000000}%
\pgfsetstrokecolor{currentstroke}%
\pgfsetdash{}{0pt}%
\pgfsys@defobject{currentmarker}{\pgfqpoint{-0.041667in}{-0.000000in}}{\pgfqpoint{0.041667in}{0.000000in}}{%
\pgfpathmoveto{\pgfqpoint{0.041667in}{-0.000000in}}%
\pgfpathlineto{\pgfqpoint{-0.041667in}{0.000000in}}%
\pgfusepath{stroke,fill}%
}%
\begin{pgfscope}%
\pgfsys@transformshift{1.011111in}{0.656049in}%
\pgfsys@useobject{currentmarker}{}%
\end{pgfscope}%
\end{pgfscope}%
\begin{pgfscope}%
\pgfsetbuttcap%
\pgfsetroundjoin%
\definecolor{currentfill}{rgb}{0.000000,0.501961,0.000000}%
\pgfsetfillcolor{currentfill}%
\pgfsetlinewidth{1.003750pt}%
\definecolor{currentstroke}{rgb}{0.000000,0.501961,0.000000}%
\pgfsetstrokecolor{currentstroke}%
\pgfsetdash{}{0pt}%
\pgfsys@defobject{currentmarker}{\pgfqpoint{-0.010417in}{-0.010417in}}{\pgfqpoint{0.010417in}{0.010417in}}{%
\pgfpathmoveto{\pgfqpoint{0.000000in}{-0.010417in}}%
\pgfpathcurveto{\pgfqpoint{0.002763in}{-0.010417in}}{\pgfqpoint{0.005412in}{-0.009319in}}{\pgfqpoint{0.007366in}{-0.007366in}}%
\pgfpathcurveto{\pgfqpoint{0.009319in}{-0.005412in}}{\pgfqpoint{0.010417in}{-0.002763in}}{\pgfqpoint{0.010417in}{0.000000in}}%
\pgfpathcurveto{\pgfqpoint{0.010417in}{0.002763in}}{\pgfqpoint{0.009319in}{0.005412in}}{\pgfqpoint{0.007366in}{0.007366in}}%
\pgfpathcurveto{\pgfqpoint{0.005412in}{0.009319in}}{\pgfqpoint{0.002763in}{0.010417in}}{\pgfqpoint{0.000000in}{0.010417in}}%
\pgfpathcurveto{\pgfqpoint{-0.002763in}{0.010417in}}{\pgfqpoint{-0.005412in}{0.009319in}}{\pgfqpoint{-0.007366in}{0.007366in}}%
\pgfpathcurveto{\pgfqpoint{-0.009319in}{0.005412in}}{\pgfqpoint{-0.010417in}{0.002763in}}{\pgfqpoint{-0.010417in}{0.000000in}}%
\pgfpathcurveto{\pgfqpoint{-0.010417in}{-0.002763in}}{\pgfqpoint{-0.009319in}{-0.005412in}}{\pgfqpoint{-0.007366in}{-0.007366in}}%
\pgfpathcurveto{\pgfqpoint{-0.005412in}{-0.009319in}}{\pgfqpoint{-0.002763in}{-0.010417in}}{\pgfqpoint{0.000000in}{-0.010417in}}%
\pgfpathclose%
\pgfusepath{stroke,fill}%
}%
\begin{pgfscope}%
\pgfsys@transformshift{1.011111in}{0.600494in}%
\pgfsys@useobject{currentmarker}{}%
\end{pgfscope}%
\end{pgfscope}%
\begin{pgfscope}%
\definecolor{textcolor}{rgb}{0.000000,0.000000,0.000000}%
\pgfsetstrokecolor{textcolor}%
\pgfsetfillcolor{textcolor}%
\pgftext[x=1.211111in,y=0.561605in,left,base]{\color{textcolor}\rmfamily\fontsize{8.000000}{9.600000}\selectfont Measured amplitudes dampening \(\displaystyle I_3\)}%
\end{pgfscope}%
\end{pgfpicture}%
\makeatother%
\endgroup%
}
	\caption{Measured amplitudes with an estimated error of $\Delta A = \pm2 \degree$  plotted on a logarithmic scale. Shown next to the exponential decay with $\alpha_{1, 2, 3}$ as decay rates and $A_0=110  \degree$  as starting value of the measurement. To every $\alpha$ value a calculated maximum and minimum error band is shown.}
	\label{fig::log}
\end{figure}

The uncertainty $\Delta t$ of the measured time is in the magnitude of $0,4$s with the stop watch.
Since this is barly visible in the plot it was removed.