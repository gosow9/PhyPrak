\section{Discussion and Conclusion}
The values we get for $\alpha$ with the two different methods are pretty close to each other.
Nonetheless, we could improve the results of both methods quite a bit. 
In the first part...

In the second part we did too many measurements in regions we later did not use at all.
To get better results, we would try to find the important values $\A_{max}, \sigma_1$ and $\sigma_2$ as perfect as possible, so we would perform all the measurements around the expected values for those.
This should definitely help us get closer to the correct value.
We also experienced problems calculating the correct error, specially when doing (linear) approximations.
If we had more measurements in the regions we look at, we could probably get a more sophisticated approximation then a linear one, as the real resonance curve is by no means linear.







So far you have discussed how you have obtained your data and the
quantities you derived from it. In this section you should discuss
the results in the context of physical laws. Depending on the
experiment you want to compare your result and its uncertainty with
the literature value. If you want to confirm a physical model that
explains a certain phenomenon you want to assess if this model
describes the data well within the confidence intervals, or whether
a simpler model describes the data just as well.