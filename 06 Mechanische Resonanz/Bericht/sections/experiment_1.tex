\subsection{Experiment}
In this part of the experiment we want to use the dampened natural oscillation of our system to determine the dampening constants $\alpha_{1, 2, 3}$.
The formula for the amplitude of this system with a given starting value $A_0$ is  as follows.
\begin{align}
	A(t) = A_0 e^{-\alpha t}
	\label{eq::model}
\end{align} 
To determine $\alpha$, we can measure the amplitude of the system for different times $t$, and then fit the model (\ref{eq::model}) to our measurement points numerically.

We always start with an amplitude $A_0$ of $110 \pm 0.5 \degree$.
For the first dampening $\alpha_1$ with $I_1 = 0.64 \pm \SI{0.01}{\ampere}$, we measure the amplitudes $A$ at every third period, for $n_1 = 24$ periods in total.
As a higher dampening results in a faster decay time, we measure $n_2 = 14$ periods with $I_2 = \SI{0.9}{\ampere}$, measuring every second period, and $n_3 = 8$ periods for $I_3 = \SI{1.2}{\ampere}$, measuring every single period.
For every dampening, we also measure the total time $t_i$ of those $n_i$ periods with a stopwatch. We estimate the error of every measurement for $t$ with $\Delta t = \SI{0.4}{\second}$ because of the human reaction time.


The Gaussian error propagation for a function $R(A, B, \dots)$, where $A \pm \Delta A, B \pm \Delta B, \dots$ is as follows.
\begin{align}
	\Delta R = \sqrt{\left(\frac{\partial R}{\partial A} \Delta A\right)^2 + \left(\frac{\partial R}{\partial B} \Delta B\right)^2 + \dots}
	\label{eq::gauss}
\end{align}
