\subsection{Calibration}

To perform our experiment we need to know the exact angular frequency $\omega$ of our motor.
As we are only able to read the output voltage of the motor, we have to calibrate it before actually performing the experiment.
In order to do that, we measure the time $t$ that it takes the motor to do $N$ rotations at a voltage $V_{Tacho}$.
This gives us the period $T = t/N$.
From that we know the angular frequency  $\omega = 2 \pi/T$  of the motor at $V_{Tacho}$.
The relation between $V_{Tacho}$ and $\omega$ is given by $C_T = \omega/V_{Tacho}$.

In our case, we do this measurement three times for three different voltages.
Again we use $\Delta t = \SI{0.4}{\second}$.
As our voltmeter reads to an accuracy of $\SI{0.001}{\volt}$, we use an uncertainty for the voltage of $\Delta V_{Tacho} = \SI{0.0005}{\volt}$.
Using Gaussian error propagation (\ref{eq::gauss}) we get a final value of $C_T = 2.45 \pm \SI{0.01}{\volt^{-1}\second^{-1}}$.
This lets us translate a voltage measurement $V$ into the angular frequency $\omega$ with the formula 
\begin{align*}
	\omega = C_T V.
	\label{eq::calibration}
\end{align*}


