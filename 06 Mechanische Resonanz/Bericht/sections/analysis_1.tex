\subsection{Data Analysis}
For the first experiment we followed the steps described in the subsection \ref{sec::exp}.
To determine the value $\alpha_{1}$, we taken all the measured amplitudes with the corresponding times and performed a least square fit on the expected model \ref{eq::model}.
The least square fit for $\alpha_1$ on the measured values is shown in figure \ref{fig::dampening} as an red dashed line.
Since this curve gets not very close to the measured values the measuring technique is probably not the most exact method used.
We expected an error of $\pm 2\degree$ while reading the scale in motion. 
But since the values changed very fast and the reading was done by eye an even grater uncertainty can be expected.

To calculate an error on the calculated decaying rate $\alpha_{1}$ another two fits were done.
For the first one $\alpha_{1max}$ we used the same set of data but added the estimated errors to the original values before performing the fit.
On the second one $\alpha_{1min}$ we subtracted the estimated error.
Having the maximum and minimum $\alpha_{1}$ we calculate the error 
\[
\Delta\alpha_1 = \frac{1}{2}(\alpha_{1max} - \alpha_{1min})
\]
This error is shown in figure \ref{fig::dampening} and figure \ref{fig::log} as the error band.

\begin{figure} [ht]
	%% Creator: Matplotlib, PGF backend
%%
%% To include the figure in your LaTeX document, write
%%   \input{<filename>.pgf}
%%
%% Make sure the required packages are loaded in your preamble
%%   \usepackage{pgf}
%%
%% and, on pdftex
%%   \usepackage[utf8]{inputenc}\DeclareUnicodeCharacter{2212}{-}
%%
%% or, on luatex and xetex
%%   \usepackage{unicode-math}
%%
%% Figures using additional raster images can only be included by \input if
%% they are in the same directory as the main LaTeX file. For loading figures
%% from other directories you can use the `import` package
%%   \usepackage{import}
%%
%% and then include the figures with
%%   \import{<path to file>}{<filename>.pgf}
%%
%% Matplotlib used the following preamble
%%
\begingroup%
\makeatletter%
\begin{pgfpicture}%
\pgfpathrectangle{\pgfpointorigin}{\pgfqpoint{6.400000in}{4.000000in}}%
\pgfusepath{use as bounding box, clip}%
\begin{pgfscope}%
\pgfsetbuttcap%
\pgfsetmiterjoin%
\definecolor{currentfill}{rgb}{1.000000,1.000000,1.000000}%
\pgfsetfillcolor{currentfill}%
\pgfsetlinewidth{0.000000pt}%
\definecolor{currentstroke}{rgb}{1.000000,1.000000,1.000000}%
\pgfsetstrokecolor{currentstroke}%
\pgfsetdash{}{0pt}%
\pgfpathmoveto{\pgfqpoint{0.000000in}{0.000000in}}%
\pgfpathlineto{\pgfqpoint{6.400000in}{0.000000in}}%
\pgfpathlineto{\pgfqpoint{6.400000in}{4.000000in}}%
\pgfpathlineto{\pgfqpoint{0.000000in}{4.000000in}}%
\pgfpathclose%
\pgfusepath{fill}%
\end{pgfscope}%
\begin{pgfscope}%
\pgfsetbuttcap%
\pgfsetmiterjoin%
\definecolor{currentfill}{rgb}{1.000000,1.000000,1.000000}%
\pgfsetfillcolor{currentfill}%
\pgfsetlinewidth{0.000000pt}%
\definecolor{currentstroke}{rgb}{0.000000,0.000000,0.000000}%
\pgfsetstrokecolor{currentstroke}%
\pgfsetstrokeopacity{0.000000}%
\pgfsetdash{}{0pt}%
\pgfpathmoveto{\pgfqpoint{0.800000in}{0.440000in}}%
\pgfpathlineto{\pgfqpoint{5.760000in}{0.440000in}}%
\pgfpathlineto{\pgfqpoint{5.760000in}{3.520000in}}%
\pgfpathlineto{\pgfqpoint{0.800000in}{3.520000in}}%
\pgfpathclose%
\pgfusepath{fill}%
\end{pgfscope}%
\begin{pgfscope}%
\pgfpathrectangle{\pgfqpoint{0.800000in}{0.440000in}}{\pgfqpoint{4.960000in}{3.080000in}}%
\pgfusepath{clip}%
\pgfsetbuttcap%
\pgfsetroundjoin%
\definecolor{currentfill}{rgb}{1.000000,0.000000,0.000000}%
\pgfsetfillcolor{currentfill}%
\pgfsetfillopacity{0.100000}%
\pgfsetlinewidth{1.003750pt}%
\definecolor{currentstroke}{rgb}{1.000000,0.000000,0.000000}%
\pgfsetstrokecolor{currentstroke}%
\pgfsetstrokeopacity{0.100000}%
\pgfsetdash{}{0pt}%
\pgfsys@defobject{currentmarker}{\pgfqpoint{1.025455in}{0.726462in}}{\pgfqpoint{5.534545in}{3.367273in}}{%
\pgfpathmoveto{\pgfqpoint{1.025455in}{3.367273in}}%
\pgfpathlineto{\pgfqpoint{1.025455in}{3.367273in}}%
\pgfpathlineto{\pgfqpoint{1.071001in}{3.287341in}}%
\pgfpathlineto{\pgfqpoint{1.116547in}{3.209691in}}%
\pgfpathlineto{\pgfqpoint{1.162094in}{3.134258in}}%
\pgfpathlineto{\pgfqpoint{1.207640in}{3.060978in}}%
\pgfpathlineto{\pgfqpoint{1.253186in}{2.989790in}}%
\pgfpathlineto{\pgfqpoint{1.298733in}{2.920635in}}%
\pgfpathlineto{\pgfqpoint{1.344279in}{2.853453in}}%
\pgfpathlineto{\pgfqpoint{1.389826in}{2.788190in}}%
\pgfpathlineto{\pgfqpoint{1.435372in}{2.724789in}}%
\pgfpathlineto{\pgfqpoint{1.480918in}{2.663199in}}%
\pgfpathlineto{\pgfqpoint{1.526465in}{2.603366in}}%
\pgfpathlineto{\pgfqpoint{1.572011in}{2.545242in}}%
\pgfpathlineto{\pgfqpoint{1.617557in}{2.488777in}}%
\pgfpathlineto{\pgfqpoint{1.663104in}{2.433923in}}%
\pgfpathlineto{\pgfqpoint{1.708650in}{2.380636in}}%
\pgfpathlineto{\pgfqpoint{1.754197in}{2.328870in}}%
\pgfpathlineto{\pgfqpoint{1.799743in}{2.278582in}}%
\pgfpathlineto{\pgfqpoint{1.845289in}{2.229729in}}%
\pgfpathlineto{\pgfqpoint{1.890836in}{2.182271in}}%
\pgfpathlineto{\pgfqpoint{1.936382in}{2.136167in}}%
\pgfpathlineto{\pgfqpoint{1.981928in}{2.091380in}}%
\pgfpathlineto{\pgfqpoint{2.027475in}{2.047871in}}%
\pgfpathlineto{\pgfqpoint{2.073021in}{2.005605in}}%
\pgfpathlineto{\pgfqpoint{2.118567in}{1.964545in}}%
\pgfpathlineto{\pgfqpoint{2.164114in}{1.924657in}}%
\pgfpathlineto{\pgfqpoint{2.209660in}{1.885907in}}%
\pgfpathlineto{\pgfqpoint{2.255207in}{1.848264in}}%
\pgfpathlineto{\pgfqpoint{2.300753in}{1.811696in}}%
\pgfpathlineto{\pgfqpoint{2.346299in}{1.776171in}}%
\pgfpathlineto{\pgfqpoint{2.391846in}{1.741661in}}%
\pgfpathlineto{\pgfqpoint{2.437392in}{1.708136in}}%
\pgfpathlineto{\pgfqpoint{2.482938in}{1.675567in}}%
\pgfpathlineto{\pgfqpoint{2.528485in}{1.643929in}}%
\pgfpathlineto{\pgfqpoint{2.574031in}{1.613194in}}%
\pgfpathlineto{\pgfqpoint{2.619578in}{1.583336in}}%
\pgfpathlineto{\pgfqpoint{2.665124in}{1.554330in}}%
\pgfpathlineto{\pgfqpoint{2.710670in}{1.526153in}}%
\pgfpathlineto{\pgfqpoint{2.756217in}{1.498779in}}%
\pgfpathlineto{\pgfqpoint{2.801763in}{1.472188in}}%
\pgfpathlineto{\pgfqpoint{2.847309in}{1.446355in}}%
\pgfpathlineto{\pgfqpoint{2.892856in}{1.421260in}}%
\pgfpathlineto{\pgfqpoint{2.938402in}{1.396881in}}%
\pgfpathlineto{\pgfqpoint{2.983949in}{1.373198in}}%
\pgfpathlineto{\pgfqpoint{3.029495in}{1.350192in}}%
\pgfpathlineto{\pgfqpoint{3.075041in}{1.327842in}}%
\pgfpathlineto{\pgfqpoint{3.120588in}{1.306130in}}%
\pgfpathlineto{\pgfqpoint{3.166134in}{1.285037in}}%
\pgfpathlineto{\pgfqpoint{3.211680in}{1.264547in}}%
\pgfpathlineto{\pgfqpoint{3.257227in}{1.244642in}}%
\pgfpathlineto{\pgfqpoint{3.302773in}{1.225305in}}%
\pgfpathlineto{\pgfqpoint{3.348320in}{1.206521in}}%
\pgfpathlineto{\pgfqpoint{3.393866in}{1.188272in}}%
\pgfpathlineto{\pgfqpoint{3.439412in}{1.170544in}}%
\pgfpathlineto{\pgfqpoint{3.484959in}{1.153323in}}%
\pgfpathlineto{\pgfqpoint{3.530505in}{1.136593in}}%
\pgfpathlineto{\pgfqpoint{3.576051in}{1.120340in}}%
\pgfpathlineto{\pgfqpoint{3.621598in}{1.104552in}}%
\pgfpathlineto{\pgfqpoint{3.667144in}{1.089214in}}%
\pgfpathlineto{\pgfqpoint{3.712691in}{1.074314in}}%
\pgfpathlineto{\pgfqpoint{3.758237in}{1.059840in}}%
\pgfpathlineto{\pgfqpoint{3.803783in}{1.045778in}}%
\pgfpathlineto{\pgfqpoint{3.849330in}{1.032119in}}%
\pgfpathlineto{\pgfqpoint{3.894876in}{1.018849in}}%
\pgfpathlineto{\pgfqpoint{3.940422in}{1.005957in}}%
\pgfpathlineto{\pgfqpoint{3.985969in}{0.993434in}}%
\pgfpathlineto{\pgfqpoint{4.031515in}{0.981269in}}%
\pgfpathlineto{\pgfqpoint{4.077062in}{0.969450in}}%
\pgfpathlineto{\pgfqpoint{4.122608in}{0.957969in}}%
\pgfpathlineto{\pgfqpoint{4.168154in}{0.946816in}}%
\pgfpathlineto{\pgfqpoint{4.213701in}{0.935981in}}%
\pgfpathlineto{\pgfqpoint{4.259247in}{0.925456in}}%
\pgfpathlineto{\pgfqpoint{4.304793in}{0.915231in}}%
\pgfpathlineto{\pgfqpoint{4.350340in}{0.905298in}}%
\pgfpathlineto{\pgfqpoint{4.395886in}{0.895648in}}%
\pgfpathlineto{\pgfqpoint{4.441433in}{0.886274in}}%
\pgfpathlineto{\pgfqpoint{4.486979in}{0.877167in}}%
\pgfpathlineto{\pgfqpoint{4.532525in}{0.868321in}}%
\pgfpathlineto{\pgfqpoint{4.578072in}{0.859727in}}%
\pgfpathlineto{\pgfqpoint{4.623618in}{0.851378in}}%
\pgfpathlineto{\pgfqpoint{4.669164in}{0.843268in}}%
\pgfpathlineto{\pgfqpoint{4.714711in}{0.835389in}}%
\pgfpathlineto{\pgfqpoint{4.760257in}{0.827735in}}%
\pgfpathlineto{\pgfqpoint{4.805803in}{0.820299in}}%
\pgfpathlineto{\pgfqpoint{4.851350in}{0.813076in}}%
\pgfpathlineto{\pgfqpoint{4.896896in}{0.806059in}}%
\pgfpathlineto{\pgfqpoint{4.942443in}{0.799243in}}%
\pgfpathlineto{\pgfqpoint{4.987989in}{0.792621in}}%
\pgfpathlineto{\pgfqpoint{5.033535in}{0.786188in}}%
\pgfpathlineto{\pgfqpoint{5.079082in}{0.779938in}}%
\pgfpathlineto{\pgfqpoint{5.124628in}{0.773867in}}%
\pgfpathlineto{\pgfqpoint{5.170174in}{0.767970in}}%
\pgfpathlineto{\pgfqpoint{5.215721in}{0.762240in}}%
\pgfpathlineto{\pgfqpoint{5.261267in}{0.756675in}}%
\pgfpathlineto{\pgfqpoint{5.306814in}{0.751268in}}%
\pgfpathlineto{\pgfqpoint{5.352360in}{0.746015in}}%
\pgfpathlineto{\pgfqpoint{5.397906in}{0.740913in}}%
\pgfpathlineto{\pgfqpoint{5.443453in}{0.735956in}}%
\pgfpathlineto{\pgfqpoint{5.488999in}{0.731140in}}%
\pgfpathlineto{\pgfqpoint{5.534545in}{0.726462in}}%
\pgfpathlineto{\pgfqpoint{5.534545in}{0.783940in}}%
\pgfpathlineto{\pgfqpoint{5.534545in}{0.783940in}}%
\pgfpathlineto{\pgfqpoint{5.488999in}{0.789614in}}%
\pgfpathlineto{\pgfqpoint{5.443453in}{0.795436in}}%
\pgfpathlineto{\pgfqpoint{5.397906in}{0.801410in}}%
\pgfpathlineto{\pgfqpoint{5.352360in}{0.807541in}}%
\pgfpathlineto{\pgfqpoint{5.306814in}{0.813833in}}%
\pgfpathlineto{\pgfqpoint{5.261267in}{0.820289in}}%
\pgfpathlineto{\pgfqpoint{5.215721in}{0.826915in}}%
\pgfpathlineto{\pgfqpoint{5.170174in}{0.833713in}}%
\pgfpathlineto{\pgfqpoint{5.124628in}{0.840690in}}%
\pgfpathlineto{\pgfqpoint{5.079082in}{0.847850in}}%
\pgfpathlineto{\pgfqpoint{5.033535in}{0.855197in}}%
\pgfpathlineto{\pgfqpoint{4.987989in}{0.862736in}}%
\pgfpathlineto{\pgfqpoint{4.942443in}{0.870473in}}%
\pgfpathlineto{\pgfqpoint{4.896896in}{0.878413in}}%
\pgfpathlineto{\pgfqpoint{4.851350in}{0.886560in}}%
\pgfpathlineto{\pgfqpoint{4.805803in}{0.894921in}}%
\pgfpathlineto{\pgfqpoint{4.760257in}{0.903501in}}%
\pgfpathlineto{\pgfqpoint{4.714711in}{0.912305in}}%
\pgfpathlineto{\pgfqpoint{4.669164in}{0.921340in}}%
\pgfpathlineto{\pgfqpoint{4.623618in}{0.930611in}}%
\pgfpathlineto{\pgfqpoint{4.578072in}{0.940125in}}%
\pgfpathlineto{\pgfqpoint{4.532525in}{0.949889in}}%
\pgfpathlineto{\pgfqpoint{4.486979in}{0.959908in}}%
\pgfpathlineto{\pgfqpoint{4.441433in}{0.970189in}}%
\pgfpathlineto{\pgfqpoint{4.395886in}{0.980740in}}%
\pgfpathlineto{\pgfqpoint{4.350340in}{0.991567in}}%
\pgfpathlineto{\pgfqpoint{4.304793in}{1.002677in}}%
\pgfpathlineto{\pgfqpoint{4.259247in}{1.014078in}}%
\pgfpathlineto{\pgfqpoint{4.213701in}{1.025778in}}%
\pgfpathlineto{\pgfqpoint{4.168154in}{1.037784in}}%
\pgfpathlineto{\pgfqpoint{4.122608in}{1.050105in}}%
\pgfpathlineto{\pgfqpoint{4.077062in}{1.062748in}}%
\pgfpathlineto{\pgfqpoint{4.031515in}{1.075723in}}%
\pgfpathlineto{\pgfqpoint{3.985969in}{1.089037in}}%
\pgfpathlineto{\pgfqpoint{3.940422in}{1.102699in}}%
\pgfpathlineto{\pgfqpoint{3.894876in}{1.116720in}}%
\pgfpathlineto{\pgfqpoint{3.849330in}{1.131107in}}%
\pgfpathlineto{\pgfqpoint{3.803783in}{1.145872in}}%
\pgfpathlineto{\pgfqpoint{3.758237in}{1.161023in}}%
\pgfpathlineto{\pgfqpoint{3.712691in}{1.176570in}}%
\pgfpathlineto{\pgfqpoint{3.667144in}{1.192525in}}%
\pgfpathlineto{\pgfqpoint{3.621598in}{1.208898in}}%
\pgfpathlineto{\pgfqpoint{3.576051in}{1.225699in}}%
\pgfpathlineto{\pgfqpoint{3.530505in}{1.242940in}}%
\pgfpathlineto{\pgfqpoint{3.484959in}{1.260633in}}%
\pgfpathlineto{\pgfqpoint{3.439412in}{1.278789in}}%
\pgfpathlineto{\pgfqpoint{3.393866in}{1.297421in}}%
\pgfpathlineto{\pgfqpoint{3.348320in}{1.316540in}}%
\pgfpathlineto{\pgfqpoint{3.302773in}{1.336160in}}%
\pgfpathlineto{\pgfqpoint{3.257227in}{1.356294in}}%
\pgfpathlineto{\pgfqpoint{3.211680in}{1.376955in}}%
\pgfpathlineto{\pgfqpoint{3.166134in}{1.398157in}}%
\pgfpathlineto{\pgfqpoint{3.120588in}{1.419914in}}%
\pgfpathlineto{\pgfqpoint{3.075041in}{1.442241in}}%
\pgfpathlineto{\pgfqpoint{3.029495in}{1.465152in}}%
\pgfpathlineto{\pgfqpoint{2.983949in}{1.488664in}}%
\pgfpathlineto{\pgfqpoint{2.938402in}{1.512791in}}%
\pgfpathlineto{\pgfqpoint{2.892856in}{1.537550in}}%
\pgfpathlineto{\pgfqpoint{2.847309in}{1.562957in}}%
\pgfpathlineto{\pgfqpoint{2.801763in}{1.589030in}}%
\pgfpathlineto{\pgfqpoint{2.756217in}{1.615785in}}%
\pgfpathlineto{\pgfqpoint{2.710670in}{1.643241in}}%
\pgfpathlineto{\pgfqpoint{2.665124in}{1.671416in}}%
\pgfpathlineto{\pgfqpoint{2.619578in}{1.700328in}}%
\pgfpathlineto{\pgfqpoint{2.574031in}{1.729998in}}%
\pgfpathlineto{\pgfqpoint{2.528485in}{1.760445in}}%
\pgfpathlineto{\pgfqpoint{2.482938in}{1.791689in}}%
\pgfpathlineto{\pgfqpoint{2.437392in}{1.823751in}}%
\pgfpathlineto{\pgfqpoint{2.391846in}{1.856652in}}%
\pgfpathlineto{\pgfqpoint{2.346299in}{1.890416in}}%
\pgfpathlineto{\pgfqpoint{2.300753in}{1.925063in}}%
\pgfpathlineto{\pgfqpoint{2.255207in}{1.960617in}}%
\pgfpathlineto{\pgfqpoint{2.209660in}{1.997103in}}%
\pgfpathlineto{\pgfqpoint{2.164114in}{2.034544in}}%
\pgfpathlineto{\pgfqpoint{2.118567in}{2.072965in}}%
\pgfpathlineto{\pgfqpoint{2.073021in}{2.112393in}}%
\pgfpathlineto{\pgfqpoint{2.027475in}{2.152852in}}%
\pgfpathlineto{\pgfqpoint{1.981928in}{2.194372in}}%
\pgfpathlineto{\pgfqpoint{1.936382in}{2.236978in}}%
\pgfpathlineto{\pgfqpoint{1.890836in}{2.280700in}}%
\pgfpathlineto{\pgfqpoint{1.845289in}{2.325568in}}%
\pgfpathlineto{\pgfqpoint{1.799743in}{2.371610in}}%
\pgfpathlineto{\pgfqpoint{1.754197in}{2.418857in}}%
\pgfpathlineto{\pgfqpoint{1.708650in}{2.467342in}}%
\pgfpathlineto{\pgfqpoint{1.663104in}{2.517096in}}%
\pgfpathlineto{\pgfqpoint{1.617557in}{2.568154in}}%
\pgfpathlineto{\pgfqpoint{1.572011in}{2.620548in}}%
\pgfpathlineto{\pgfqpoint{1.526465in}{2.674314in}}%
\pgfpathlineto{\pgfqpoint{1.480918in}{2.729488in}}%
\pgfpathlineto{\pgfqpoint{1.435372in}{2.786107in}}%
\pgfpathlineto{\pgfqpoint{1.389826in}{2.844208in}}%
\pgfpathlineto{\pgfqpoint{1.344279in}{2.903831in}}%
\pgfpathlineto{\pgfqpoint{1.298733in}{2.965015in}}%
\pgfpathlineto{\pgfqpoint{1.253186in}{3.027802in}}%
\pgfpathlineto{\pgfqpoint{1.207640in}{3.092232in}}%
\pgfpathlineto{\pgfqpoint{1.162094in}{3.158350in}}%
\pgfpathlineto{\pgfqpoint{1.116547in}{3.226199in}}%
\pgfpathlineto{\pgfqpoint{1.071001in}{3.295824in}}%
\pgfpathlineto{\pgfqpoint{1.025455in}{3.367273in}}%
\pgfpathclose%
\pgfusepath{stroke,fill}%
}%
\begin{pgfscope}%
\pgfsys@transformshift{0.000000in}{0.000000in}%
\pgfsys@useobject{currentmarker}{}%
\end{pgfscope}%
\end{pgfscope}%
\begin{pgfscope}%
\pgfpathrectangle{\pgfqpoint{0.800000in}{0.440000in}}{\pgfqpoint{4.960000in}{3.080000in}}%
\pgfusepath{clip}%
\pgfsetbuttcap%
\pgfsetroundjoin%
\definecolor{currentfill}{rgb}{0.000000,0.000000,1.000000}%
\pgfsetfillcolor{currentfill}%
\pgfsetfillopacity{0.100000}%
\pgfsetlinewidth{1.003750pt}%
\definecolor{currentstroke}{rgb}{0.000000,0.000000,1.000000}%
\pgfsetstrokecolor{currentstroke}%
\pgfsetstrokeopacity{0.100000}%
\pgfsetdash{}{0pt}%
\pgfsys@defobject{currentmarker}{\pgfqpoint{1.025455in}{0.761606in}}{\pgfqpoint{3.650447in}{3.367273in}}{%
\pgfpathmoveto{\pgfqpoint{1.025455in}{3.367273in}}%
\pgfpathlineto{\pgfqpoint{1.025455in}{3.367273in}}%
\pgfpathlineto{\pgfqpoint{1.051970in}{3.292827in}}%
\pgfpathlineto{\pgfqpoint{1.078485in}{3.220361in}}%
\pgfpathlineto{\pgfqpoint{1.105000in}{3.149822in}}%
\pgfpathlineto{\pgfqpoint{1.131515in}{3.081158in}}%
\pgfpathlineto{\pgfqpoint{1.158030in}{3.014320in}}%
\pgfpathlineto{\pgfqpoint{1.184545in}{2.949259in}}%
\pgfpathlineto{\pgfqpoint{1.211060in}{2.885928in}}%
\pgfpathlineto{\pgfqpoint{1.237575in}{2.824280in}}%
\pgfpathlineto{\pgfqpoint{1.264090in}{2.764272in}}%
\pgfpathlineto{\pgfqpoint{1.290605in}{2.705859in}}%
\pgfpathlineto{\pgfqpoint{1.317120in}{2.648999in}}%
\pgfpathlineto{\pgfqpoint{1.343635in}{2.593651in}}%
\pgfpathlineto{\pgfqpoint{1.370151in}{2.539774in}}%
\pgfpathlineto{\pgfqpoint{1.396666in}{2.487330in}}%
\pgfpathlineto{\pgfqpoint{1.423181in}{2.436280in}}%
\pgfpathlineto{\pgfqpoint{1.449696in}{2.386588in}}%
\pgfpathlineto{\pgfqpoint{1.476211in}{2.338217in}}%
\pgfpathlineto{\pgfqpoint{1.502726in}{2.291131in}}%
\pgfpathlineto{\pgfqpoint{1.529241in}{2.245298in}}%
\pgfpathlineto{\pgfqpoint{1.555756in}{2.200683in}}%
\pgfpathlineto{\pgfqpoint{1.582271in}{2.157255in}}%
\pgfpathlineto{\pgfqpoint{1.608786in}{2.114981in}}%
\pgfpathlineto{\pgfqpoint{1.635301in}{2.073831in}}%
\pgfpathlineto{\pgfqpoint{1.661816in}{2.033775in}}%
\pgfpathlineto{\pgfqpoint{1.688332in}{1.994785in}}%
\pgfpathlineto{\pgfqpoint{1.714847in}{1.956831in}}%
\pgfpathlineto{\pgfqpoint{1.741362in}{1.919886in}}%
\pgfpathlineto{\pgfqpoint{1.767877in}{1.883923in}}%
\pgfpathlineto{\pgfqpoint{1.794392in}{1.848916in}}%
\pgfpathlineto{\pgfqpoint{1.820907in}{1.814840in}}%
\pgfpathlineto{\pgfqpoint{1.847422in}{1.781670in}}%
\pgfpathlineto{\pgfqpoint{1.873937in}{1.749383in}}%
\pgfpathlineto{\pgfqpoint{1.900452in}{1.717953in}}%
\pgfpathlineto{\pgfqpoint{1.926967in}{1.687359in}}%
\pgfpathlineto{\pgfqpoint{1.953482in}{1.657579in}}%
\pgfpathlineto{\pgfqpoint{1.979997in}{1.628590in}}%
\pgfpathlineto{\pgfqpoint{2.006512in}{1.600372in}}%
\pgfpathlineto{\pgfqpoint{2.033028in}{1.572905in}}%
\pgfpathlineto{\pgfqpoint{2.059543in}{1.546167in}}%
\pgfpathlineto{\pgfqpoint{2.086058in}{1.520141in}}%
\pgfpathlineto{\pgfqpoint{2.112573in}{1.494806in}}%
\pgfpathlineto{\pgfqpoint{2.139088in}{1.470145in}}%
\pgfpathlineto{\pgfqpoint{2.165603in}{1.446140in}}%
\pgfpathlineto{\pgfqpoint{2.192118in}{1.422773in}}%
\pgfpathlineto{\pgfqpoint{2.218633in}{1.400027in}}%
\pgfpathlineto{\pgfqpoint{2.245148in}{1.377887in}}%
\pgfpathlineto{\pgfqpoint{2.271663in}{1.356334in}}%
\pgfpathlineto{\pgfqpoint{2.298178in}{1.335355in}}%
\pgfpathlineto{\pgfqpoint{2.324693in}{1.314934in}}%
\pgfpathlineto{\pgfqpoint{2.351208in}{1.295055in}}%
\pgfpathlineto{\pgfqpoint{2.377724in}{1.275705in}}%
\pgfpathlineto{\pgfqpoint{2.404239in}{1.256869in}}%
\pgfpathlineto{\pgfqpoint{2.430754in}{1.238535in}}%
\pgfpathlineto{\pgfqpoint{2.457269in}{1.220687in}}%
\pgfpathlineto{\pgfqpoint{2.483784in}{1.203315in}}%
\pgfpathlineto{\pgfqpoint{2.510299in}{1.186404in}}%
\pgfpathlineto{\pgfqpoint{2.536814in}{1.169943in}}%
\pgfpathlineto{\pgfqpoint{2.563329in}{1.153919in}}%
\pgfpathlineto{\pgfqpoint{2.589844in}{1.138322in}}%
\pgfpathlineto{\pgfqpoint{2.616359in}{1.123139in}}%
\pgfpathlineto{\pgfqpoint{2.642874in}{1.108360in}}%
\pgfpathlineto{\pgfqpoint{2.669389in}{1.093973in}}%
\pgfpathlineto{\pgfqpoint{2.695905in}{1.079970in}}%
\pgfpathlineto{\pgfqpoint{2.722420in}{1.066338in}}%
\pgfpathlineto{\pgfqpoint{2.748935in}{1.053069in}}%
\pgfpathlineto{\pgfqpoint{2.775450in}{1.040153in}}%
\pgfpathlineto{\pgfqpoint{2.801965in}{1.027580in}}%
\pgfpathlineto{\pgfqpoint{2.828480in}{1.015342in}}%
\pgfpathlineto{\pgfqpoint{2.854995in}{1.003429in}}%
\pgfpathlineto{\pgfqpoint{2.881510in}{0.991832in}}%
\pgfpathlineto{\pgfqpoint{2.908025in}{0.980544in}}%
\pgfpathlineto{\pgfqpoint{2.934540in}{0.969557in}}%
\pgfpathlineto{\pgfqpoint{2.961055in}{0.958861in}}%
\pgfpathlineto{\pgfqpoint{2.987570in}{0.948449in}}%
\pgfpathlineto{\pgfqpoint{3.014085in}{0.938315in}}%
\pgfpathlineto{\pgfqpoint{3.040601in}{0.928450in}}%
\pgfpathlineto{\pgfqpoint{3.067116in}{0.918847in}}%
\pgfpathlineto{\pgfqpoint{3.093631in}{0.909499in}}%
\pgfpathlineto{\pgfqpoint{3.120146in}{0.900400in}}%
\pgfpathlineto{\pgfqpoint{3.146661in}{0.891543in}}%
\pgfpathlineto{\pgfqpoint{3.173176in}{0.882922in}}%
\pgfpathlineto{\pgfqpoint{3.199691in}{0.874529in}}%
\pgfpathlineto{\pgfqpoint{3.226206in}{0.866360in}}%
\pgfpathlineto{\pgfqpoint{3.252721in}{0.858408in}}%
\pgfpathlineto{\pgfqpoint{3.279236in}{0.850668in}}%
\pgfpathlineto{\pgfqpoint{3.305751in}{0.843133in}}%
\pgfpathlineto{\pgfqpoint{3.332266in}{0.835798in}}%
\pgfpathlineto{\pgfqpoint{3.358781in}{0.828659in}}%
\pgfpathlineto{\pgfqpoint{3.385297in}{0.821709in}}%
\pgfpathlineto{\pgfqpoint{3.411812in}{0.814944in}}%
\pgfpathlineto{\pgfqpoint{3.438327in}{0.808359in}}%
\pgfpathlineto{\pgfqpoint{3.464842in}{0.801949in}}%
\pgfpathlineto{\pgfqpoint{3.491357in}{0.795710in}}%
\pgfpathlineto{\pgfqpoint{3.517872in}{0.789636in}}%
\pgfpathlineto{\pgfqpoint{3.544387in}{0.783724in}}%
\pgfpathlineto{\pgfqpoint{3.570902in}{0.777969in}}%
\pgfpathlineto{\pgfqpoint{3.597417in}{0.772367in}}%
\pgfpathlineto{\pgfqpoint{3.623932in}{0.766914in}}%
\pgfpathlineto{\pgfqpoint{3.650447in}{0.761606in}}%
\pgfpathlineto{\pgfqpoint{3.650447in}{0.828077in}}%
\pgfpathlineto{\pgfqpoint{3.650447in}{0.828077in}}%
\pgfpathlineto{\pgfqpoint{3.623932in}{0.834405in}}%
\pgfpathlineto{\pgfqpoint{3.597417in}{0.840887in}}%
\pgfpathlineto{\pgfqpoint{3.570902in}{0.847527in}}%
\pgfpathlineto{\pgfqpoint{3.544387in}{0.854327in}}%
\pgfpathlineto{\pgfqpoint{3.517872in}{0.861293in}}%
\pgfpathlineto{\pgfqpoint{3.491357in}{0.868427in}}%
\pgfpathlineto{\pgfqpoint{3.464842in}{0.875735in}}%
\pgfpathlineto{\pgfqpoint{3.438327in}{0.883220in}}%
\pgfpathlineto{\pgfqpoint{3.411812in}{0.890887in}}%
\pgfpathlineto{\pgfqpoint{3.385297in}{0.898739in}}%
\pgfpathlineto{\pgfqpoint{3.358781in}{0.906782in}}%
\pgfpathlineto{\pgfqpoint{3.332266in}{0.915021in}}%
\pgfpathlineto{\pgfqpoint{3.305751in}{0.923459in}}%
\pgfpathlineto{\pgfqpoint{3.279236in}{0.932102in}}%
\pgfpathlineto{\pgfqpoint{3.252721in}{0.940955in}}%
\pgfpathlineto{\pgfqpoint{3.226206in}{0.950023in}}%
\pgfpathlineto{\pgfqpoint{3.199691in}{0.959310in}}%
\pgfpathlineto{\pgfqpoint{3.173176in}{0.968823in}}%
\pgfpathlineto{\pgfqpoint{3.146661in}{0.978567in}}%
\pgfpathlineto{\pgfqpoint{3.120146in}{0.988547in}}%
\pgfpathlineto{\pgfqpoint{3.093631in}{0.998770in}}%
\pgfpathlineto{\pgfqpoint{3.067116in}{1.009240in}}%
\pgfpathlineto{\pgfqpoint{3.040601in}{1.019965in}}%
\pgfpathlineto{\pgfqpoint{3.014085in}{1.030950in}}%
\pgfpathlineto{\pgfqpoint{2.987570in}{1.042201in}}%
\pgfpathlineto{\pgfqpoint{2.961055in}{1.053726in}}%
\pgfpathlineto{\pgfqpoint{2.934540in}{1.065530in}}%
\pgfpathlineto{\pgfqpoint{2.908025in}{1.077620in}}%
\pgfpathlineto{\pgfqpoint{2.881510in}{1.090004in}}%
\pgfpathlineto{\pgfqpoint{2.854995in}{1.102688in}}%
\pgfpathlineto{\pgfqpoint{2.828480in}{1.115680in}}%
\pgfpathlineto{\pgfqpoint{2.801965in}{1.128988in}}%
\pgfpathlineto{\pgfqpoint{2.775450in}{1.142618in}}%
\pgfpathlineto{\pgfqpoint{2.748935in}{1.156579in}}%
\pgfpathlineto{\pgfqpoint{2.722420in}{1.170879in}}%
\pgfpathlineto{\pgfqpoint{2.695905in}{1.185526in}}%
\pgfpathlineto{\pgfqpoint{2.669389in}{1.200528in}}%
\pgfpathlineto{\pgfqpoint{2.642874in}{1.215894in}}%
\pgfpathlineto{\pgfqpoint{2.616359in}{1.231634in}}%
\pgfpathlineto{\pgfqpoint{2.589844in}{1.247755in}}%
\pgfpathlineto{\pgfqpoint{2.563329in}{1.264267in}}%
\pgfpathlineto{\pgfqpoint{2.536814in}{1.281180in}}%
\pgfpathlineto{\pgfqpoint{2.510299in}{1.298503in}}%
\pgfpathlineto{\pgfqpoint{2.483784in}{1.316247in}}%
\pgfpathlineto{\pgfqpoint{2.457269in}{1.334421in}}%
\pgfpathlineto{\pgfqpoint{2.430754in}{1.353037in}}%
\pgfpathlineto{\pgfqpoint{2.404239in}{1.372104in}}%
\pgfpathlineto{\pgfqpoint{2.377724in}{1.391633in}}%
\pgfpathlineto{\pgfqpoint{2.351208in}{1.411637in}}%
\pgfpathlineto{\pgfqpoint{2.324693in}{1.432126in}}%
\pgfpathlineto{\pgfqpoint{2.298178in}{1.453112in}}%
\pgfpathlineto{\pgfqpoint{2.271663in}{1.474607in}}%
\pgfpathlineto{\pgfqpoint{2.245148in}{1.496624in}}%
\pgfpathlineto{\pgfqpoint{2.218633in}{1.519175in}}%
\pgfpathlineto{\pgfqpoint{2.192118in}{1.542274in}}%
\pgfpathlineto{\pgfqpoint{2.165603in}{1.565933in}}%
\pgfpathlineto{\pgfqpoint{2.139088in}{1.590166in}}%
\pgfpathlineto{\pgfqpoint{2.112573in}{1.614987in}}%
\pgfpathlineto{\pgfqpoint{2.086058in}{1.640410in}}%
\pgfpathlineto{\pgfqpoint{2.059543in}{1.666451in}}%
\pgfpathlineto{\pgfqpoint{2.033028in}{1.693123in}}%
\pgfpathlineto{\pgfqpoint{2.006512in}{1.720442in}}%
\pgfpathlineto{\pgfqpoint{1.979997in}{1.748424in}}%
\pgfpathlineto{\pgfqpoint{1.953482in}{1.777086in}}%
\pgfpathlineto{\pgfqpoint{1.926967in}{1.806442in}}%
\pgfpathlineto{\pgfqpoint{1.900452in}{1.836512in}}%
\pgfpathlineto{\pgfqpoint{1.873937in}{1.867310in}}%
\pgfpathlineto{\pgfqpoint{1.847422in}{1.898856in}}%
\pgfpathlineto{\pgfqpoint{1.820907in}{1.931168in}}%
\pgfpathlineto{\pgfqpoint{1.794392in}{1.964264in}}%
\pgfpathlineto{\pgfqpoint{1.767877in}{1.998163in}}%
\pgfpathlineto{\pgfqpoint{1.741362in}{2.032884in}}%
\pgfpathlineto{\pgfqpoint{1.714847in}{2.068448in}}%
\pgfpathlineto{\pgfqpoint{1.688332in}{2.104875in}}%
\pgfpathlineto{\pgfqpoint{1.661816in}{2.142185in}}%
\pgfpathlineto{\pgfqpoint{1.635301in}{2.180401in}}%
\pgfpathlineto{\pgfqpoint{1.608786in}{2.219545in}}%
\pgfpathlineto{\pgfqpoint{1.582271in}{2.259638in}}%
\pgfpathlineto{\pgfqpoint{1.555756in}{2.300704in}}%
\pgfpathlineto{\pgfqpoint{1.529241in}{2.342767in}}%
\pgfpathlineto{\pgfqpoint{1.502726in}{2.385850in}}%
\pgfpathlineto{\pgfqpoint{1.476211in}{2.429979in}}%
\pgfpathlineto{\pgfqpoint{1.449696in}{2.475179in}}%
\pgfpathlineto{\pgfqpoint{1.423181in}{2.521475in}}%
\pgfpathlineto{\pgfqpoint{1.396666in}{2.568895in}}%
\pgfpathlineto{\pgfqpoint{1.370151in}{2.617465in}}%
\pgfpathlineto{\pgfqpoint{1.343635in}{2.667214in}}%
\pgfpathlineto{\pgfqpoint{1.317120in}{2.718171in}}%
\pgfpathlineto{\pgfqpoint{1.290605in}{2.770363in}}%
\pgfpathlineto{\pgfqpoint{1.264090in}{2.823823in}}%
\pgfpathlineto{\pgfqpoint{1.237575in}{2.878579in}}%
\pgfpathlineto{\pgfqpoint{1.211060in}{2.934664in}}%
\pgfpathlineto{\pgfqpoint{1.184545in}{2.992110in}}%
\pgfpathlineto{\pgfqpoint{1.158030in}{3.050950in}}%
\pgfpathlineto{\pgfqpoint{1.131515in}{3.111218in}}%
\pgfpathlineto{\pgfqpoint{1.105000in}{3.172948in}}%
\pgfpathlineto{\pgfqpoint{1.078485in}{3.236176in}}%
\pgfpathlineto{\pgfqpoint{1.051970in}{3.300939in}}%
\pgfpathlineto{\pgfqpoint{1.025455in}{3.367273in}}%
\pgfpathclose%
\pgfusepath{stroke,fill}%
}%
\begin{pgfscope}%
\pgfsys@transformshift{0.000000in}{0.000000in}%
\pgfsys@useobject{currentmarker}{}%
\end{pgfscope}%
\end{pgfscope}%
\begin{pgfscope}%
\pgfpathrectangle{\pgfqpoint{0.800000in}{0.440000in}}{\pgfqpoint{4.960000in}{3.080000in}}%
\pgfusepath{clip}%
\pgfsetbuttcap%
\pgfsetroundjoin%
\definecolor{currentfill}{rgb}{0.000000,0.501961,0.000000}%
\pgfsetfillcolor{currentfill}%
\pgfsetfillopacity{0.100000}%
\pgfsetlinewidth{1.003750pt}%
\definecolor{currentstroke}{rgb}{0.000000,0.501961,0.000000}%
\pgfsetstrokecolor{currentstroke}%
\pgfsetstrokeopacity{0.100000}%
\pgfsetdash{}{0pt}%
\pgfsys@defobject{currentmarker}{\pgfqpoint{1.025455in}{0.823573in}}{\pgfqpoint{2.544336in}{3.367273in}}{%
\pgfpathmoveto{\pgfqpoint{1.025455in}{3.367273in}}%
\pgfpathlineto{\pgfqpoint{1.025455in}{3.367273in}}%
\pgfpathlineto{\pgfqpoint{1.040797in}{3.300458in}}%
\pgfpathlineto{\pgfqpoint{1.056139in}{3.235237in}}%
\pgfpathlineto{\pgfqpoint{1.071481in}{3.171573in}}%
\pgfpathlineto{\pgfqpoint{1.086824in}{3.109428in}}%
\pgfpathlineto{\pgfqpoint{1.102166in}{3.048766in}}%
\pgfpathlineto{\pgfqpoint{1.117508in}{2.989552in}}%
\pgfpathlineto{\pgfqpoint{1.132850in}{2.931750in}}%
\pgfpathlineto{\pgfqpoint{1.148192in}{2.875328in}}%
\pgfpathlineto{\pgfqpoint{1.163535in}{2.820252in}}%
\pgfpathlineto{\pgfqpoint{1.178877in}{2.766491in}}%
\pgfpathlineto{\pgfqpoint{1.194219in}{2.714012in}}%
\pgfpathlineto{\pgfqpoint{1.209561in}{2.662786in}}%
\pgfpathlineto{\pgfqpoint{1.224904in}{2.612782in}}%
\pgfpathlineto{\pgfqpoint{1.240246in}{2.563971in}}%
\pgfpathlineto{\pgfqpoint{1.255588in}{2.516325in}}%
\pgfpathlineto{\pgfqpoint{1.270930in}{2.469815in}}%
\pgfpathlineto{\pgfqpoint{1.286273in}{2.424416in}}%
\pgfpathlineto{\pgfqpoint{1.301615in}{2.380100in}}%
\pgfpathlineto{\pgfqpoint{1.316957in}{2.336842in}}%
\pgfpathlineto{\pgfqpoint{1.332299in}{2.294615in}}%
\pgfpathlineto{\pgfqpoint{1.347642in}{2.253397in}}%
\pgfpathlineto{\pgfqpoint{1.362984in}{2.213162in}}%
\pgfpathlineto{\pgfqpoint{1.378326in}{2.173887in}}%
\pgfpathlineto{\pgfqpoint{1.393668in}{2.135549in}}%
\pgfpathlineto{\pgfqpoint{1.409011in}{2.098126in}}%
\pgfpathlineto{\pgfqpoint{1.424353in}{2.061596in}}%
\pgfpathlineto{\pgfqpoint{1.439695in}{2.025938in}}%
\pgfpathlineto{\pgfqpoint{1.455037in}{1.991131in}}%
\pgfpathlineto{\pgfqpoint{1.470379in}{1.957154in}}%
\pgfpathlineto{\pgfqpoint{1.485722in}{1.923988in}}%
\pgfpathlineto{\pgfqpoint{1.501064in}{1.891614in}}%
\pgfpathlineto{\pgfqpoint{1.516406in}{1.860012in}}%
\pgfpathlineto{\pgfqpoint{1.531748in}{1.829164in}}%
\pgfpathlineto{\pgfqpoint{1.547091in}{1.799052in}}%
\pgfpathlineto{\pgfqpoint{1.562433in}{1.769659in}}%
\pgfpathlineto{\pgfqpoint{1.577775in}{1.740967in}}%
\pgfpathlineto{\pgfqpoint{1.593117in}{1.712960in}}%
\pgfpathlineto{\pgfqpoint{1.608460in}{1.685621in}}%
\pgfpathlineto{\pgfqpoint{1.623802in}{1.658934in}}%
\pgfpathlineto{\pgfqpoint{1.639144in}{1.632885in}}%
\pgfpathlineto{\pgfqpoint{1.654486in}{1.607456in}}%
\pgfpathlineto{\pgfqpoint{1.669829in}{1.582635in}}%
\pgfpathlineto{\pgfqpoint{1.685171in}{1.558406in}}%
\pgfpathlineto{\pgfqpoint{1.700513in}{1.534755in}}%
\pgfpathlineto{\pgfqpoint{1.715855in}{1.511669in}}%
\pgfpathlineto{\pgfqpoint{1.731198in}{1.489133in}}%
\pgfpathlineto{\pgfqpoint{1.746540in}{1.467135in}}%
\pgfpathlineto{\pgfqpoint{1.761882in}{1.445662in}}%
\pgfpathlineto{\pgfqpoint{1.777224in}{1.424702in}}%
\pgfpathlineto{\pgfqpoint{1.792567in}{1.404242in}}%
\pgfpathlineto{\pgfqpoint{1.807909in}{1.384269in}}%
\pgfpathlineto{\pgfqpoint{1.823251in}{1.364774in}}%
\pgfpathlineto{\pgfqpoint{1.838593in}{1.345744in}}%
\pgfpathlineto{\pgfqpoint{1.853935in}{1.327167in}}%
\pgfpathlineto{\pgfqpoint{1.869278in}{1.309034in}}%
\pgfpathlineto{\pgfqpoint{1.884620in}{1.291334in}}%
\pgfpathlineto{\pgfqpoint{1.899962in}{1.274056in}}%
\pgfpathlineto{\pgfqpoint{1.915304in}{1.257191in}}%
\pgfpathlineto{\pgfqpoint{1.930647in}{1.240728in}}%
\pgfpathlineto{\pgfqpoint{1.945989in}{1.224657in}}%
\pgfpathlineto{\pgfqpoint{1.961331in}{1.208970in}}%
\pgfpathlineto{\pgfqpoint{1.976673in}{1.193658in}}%
\pgfpathlineto{\pgfqpoint{1.992016in}{1.178711in}}%
\pgfpathlineto{\pgfqpoint{2.007358in}{1.164121in}}%
\pgfpathlineto{\pgfqpoint{2.022700in}{1.149878in}}%
\pgfpathlineto{\pgfqpoint{2.038042in}{1.135976in}}%
\pgfpathlineto{\pgfqpoint{2.053385in}{1.122405in}}%
\pgfpathlineto{\pgfqpoint{2.068727in}{1.109158in}}%
\pgfpathlineto{\pgfqpoint{2.084069in}{1.096228in}}%
\pgfpathlineto{\pgfqpoint{2.099411in}{1.083606in}}%
\pgfpathlineto{\pgfqpoint{2.114754in}{1.071285in}}%
\pgfpathlineto{\pgfqpoint{2.130096in}{1.059258in}}%
\pgfpathlineto{\pgfqpoint{2.145438in}{1.047518in}}%
\pgfpathlineto{\pgfqpoint{2.160780in}{1.036058in}}%
\pgfpathlineto{\pgfqpoint{2.176123in}{1.024872in}}%
\pgfpathlineto{\pgfqpoint{2.191465in}{1.013952in}}%
\pgfpathlineto{\pgfqpoint{2.206807in}{1.003293in}}%
\pgfpathlineto{\pgfqpoint{2.222149in}{0.992889in}}%
\pgfpathlineto{\pgfqpoint{2.237491in}{0.982732in}}%
\pgfpathlineto{\pgfqpoint{2.252834in}{0.972819in}}%
\pgfpathlineto{\pgfqpoint{2.268176in}{0.963141in}}%
\pgfpathlineto{\pgfqpoint{2.283518in}{0.953695in}}%
\pgfpathlineto{\pgfqpoint{2.298860in}{0.944474in}}%
\pgfpathlineto{\pgfqpoint{2.314203in}{0.935473in}}%
\pgfpathlineto{\pgfqpoint{2.329545in}{0.926687in}}%
\pgfpathlineto{\pgfqpoint{2.344887in}{0.918110in}}%
\pgfpathlineto{\pgfqpoint{2.360229in}{0.909738in}}%
\pgfpathlineto{\pgfqpoint{2.375572in}{0.901566in}}%
\pgfpathlineto{\pgfqpoint{2.390914in}{0.893589in}}%
\pgfpathlineto{\pgfqpoint{2.406256in}{0.885803in}}%
\pgfpathlineto{\pgfqpoint{2.421598in}{0.878202in}}%
\pgfpathlineto{\pgfqpoint{2.436941in}{0.870782in}}%
\pgfpathlineto{\pgfqpoint{2.452283in}{0.863540in}}%
\pgfpathlineto{\pgfqpoint{2.467625in}{0.856470in}}%
\pgfpathlineto{\pgfqpoint{2.482967in}{0.849569in}}%
\pgfpathlineto{\pgfqpoint{2.498310in}{0.842833in}}%
\pgfpathlineto{\pgfqpoint{2.513652in}{0.836257in}}%
\pgfpathlineto{\pgfqpoint{2.528994in}{0.829839in}}%
\pgfpathlineto{\pgfqpoint{2.544336in}{0.823573in}}%
\pgfpathlineto{\pgfqpoint{2.544336in}{0.898810in}}%
\pgfpathlineto{\pgfqpoint{2.544336in}{0.898810in}}%
\pgfpathlineto{\pgfqpoint{2.528994in}{0.906033in}}%
\pgfpathlineto{\pgfqpoint{2.513652in}{0.913413in}}%
\pgfpathlineto{\pgfqpoint{2.498310in}{0.920954in}}%
\pgfpathlineto{\pgfqpoint{2.482967in}{0.928660in}}%
\pgfpathlineto{\pgfqpoint{2.467625in}{0.936533in}}%
\pgfpathlineto{\pgfqpoint{2.452283in}{0.944577in}}%
\pgfpathlineto{\pgfqpoint{2.436941in}{0.952797in}}%
\pgfpathlineto{\pgfqpoint{2.421598in}{0.961196in}}%
\pgfpathlineto{\pgfqpoint{2.406256in}{0.969778in}}%
\pgfpathlineto{\pgfqpoint{2.390914in}{0.978547in}}%
\pgfpathlineto{\pgfqpoint{2.375572in}{0.987507in}}%
\pgfpathlineto{\pgfqpoint{2.360229in}{0.996662in}}%
\pgfpathlineto{\pgfqpoint{2.344887in}{1.006017in}}%
\pgfpathlineto{\pgfqpoint{2.329545in}{1.015575in}}%
\pgfpathlineto{\pgfqpoint{2.314203in}{1.025342in}}%
\pgfpathlineto{\pgfqpoint{2.298860in}{1.035321in}}%
\pgfpathlineto{\pgfqpoint{2.283518in}{1.045518in}}%
\pgfpathlineto{\pgfqpoint{2.268176in}{1.055937in}}%
\pgfpathlineto{\pgfqpoint{2.252834in}{1.066583in}}%
\pgfpathlineto{\pgfqpoint{2.237491in}{1.077460in}}%
\pgfpathlineto{\pgfqpoint{2.222149in}{1.088575in}}%
\pgfpathlineto{\pgfqpoint{2.206807in}{1.099932in}}%
\pgfpathlineto{\pgfqpoint{2.191465in}{1.111537in}}%
\pgfpathlineto{\pgfqpoint{2.176123in}{1.123394in}}%
\pgfpathlineto{\pgfqpoint{2.160780in}{1.135509in}}%
\pgfpathlineto{\pgfqpoint{2.145438in}{1.147889in}}%
\pgfpathlineto{\pgfqpoint{2.130096in}{1.160538in}}%
\pgfpathlineto{\pgfqpoint{2.114754in}{1.173463in}}%
\pgfpathlineto{\pgfqpoint{2.099411in}{1.186669in}}%
\pgfpathlineto{\pgfqpoint{2.084069in}{1.200163in}}%
\pgfpathlineto{\pgfqpoint{2.068727in}{1.213951in}}%
\pgfpathlineto{\pgfqpoint{2.053385in}{1.228039in}}%
\pgfpathlineto{\pgfqpoint{2.038042in}{1.242435in}}%
\pgfpathlineto{\pgfqpoint{2.022700in}{1.257144in}}%
\pgfpathlineto{\pgfqpoint{2.007358in}{1.272173in}}%
\pgfpathlineto{\pgfqpoint{1.992016in}{1.287530in}}%
\pgfpathlineto{\pgfqpoint{1.976673in}{1.303221in}}%
\pgfpathlineto{\pgfqpoint{1.961331in}{1.319254in}}%
\pgfpathlineto{\pgfqpoint{1.945989in}{1.335637in}}%
\pgfpathlineto{\pgfqpoint{1.930647in}{1.352376in}}%
\pgfpathlineto{\pgfqpoint{1.915304in}{1.369480in}}%
\pgfpathlineto{\pgfqpoint{1.899962in}{1.386957in}}%
\pgfpathlineto{\pgfqpoint{1.884620in}{1.404815in}}%
\pgfpathlineto{\pgfqpoint{1.869278in}{1.423061in}}%
\pgfpathlineto{\pgfqpoint{1.853935in}{1.441705in}}%
\pgfpathlineto{\pgfqpoint{1.838593in}{1.460755in}}%
\pgfpathlineto{\pgfqpoint{1.823251in}{1.480220in}}%
\pgfpathlineto{\pgfqpoint{1.807909in}{1.500110in}}%
\pgfpathlineto{\pgfqpoint{1.792567in}{1.520432in}}%
\pgfpathlineto{\pgfqpoint{1.777224in}{1.541198in}}%
\pgfpathlineto{\pgfqpoint{1.761882in}{1.562415in}}%
\pgfpathlineto{\pgfqpoint{1.746540in}{1.584095in}}%
\pgfpathlineto{\pgfqpoint{1.731198in}{1.606247in}}%
\pgfpathlineto{\pgfqpoint{1.715855in}{1.628882in}}%
\pgfpathlineto{\pgfqpoint{1.700513in}{1.652010in}}%
\pgfpathlineto{\pgfqpoint{1.685171in}{1.675642in}}%
\pgfpathlineto{\pgfqpoint{1.669829in}{1.699789in}}%
\pgfpathlineto{\pgfqpoint{1.654486in}{1.724462in}}%
\pgfpathlineto{\pgfqpoint{1.639144in}{1.749672in}}%
\pgfpathlineto{\pgfqpoint{1.623802in}{1.775431in}}%
\pgfpathlineto{\pgfqpoint{1.608460in}{1.801752in}}%
\pgfpathlineto{\pgfqpoint{1.593117in}{1.828646in}}%
\pgfpathlineto{\pgfqpoint{1.577775in}{1.856126in}}%
\pgfpathlineto{\pgfqpoint{1.562433in}{1.884205in}}%
\pgfpathlineto{\pgfqpoint{1.547091in}{1.912895in}}%
\pgfpathlineto{\pgfqpoint{1.531748in}{1.942210in}}%
\pgfpathlineto{\pgfqpoint{1.516406in}{1.972164in}}%
\pgfpathlineto{\pgfqpoint{1.501064in}{2.002771in}}%
\pgfpathlineto{\pgfqpoint{1.485722in}{2.034045in}}%
\pgfpathlineto{\pgfqpoint{1.470379in}{2.065999in}}%
\pgfpathlineto{\pgfqpoint{1.455037in}{2.098650in}}%
\pgfpathlineto{\pgfqpoint{1.439695in}{2.132012in}}%
\pgfpathlineto{\pgfqpoint{1.424353in}{2.166102in}}%
\pgfpathlineto{\pgfqpoint{1.409011in}{2.200933in}}%
\pgfpathlineto{\pgfqpoint{1.393668in}{2.236524in}}%
\pgfpathlineto{\pgfqpoint{1.378326in}{2.272890in}}%
\pgfpathlineto{\pgfqpoint{1.362984in}{2.310048in}}%
\pgfpathlineto{\pgfqpoint{1.347642in}{2.348016in}}%
\pgfpathlineto{\pgfqpoint{1.332299in}{2.386810in}}%
\pgfpathlineto{\pgfqpoint{1.316957in}{2.426450in}}%
\pgfpathlineto{\pgfqpoint{1.301615in}{2.466954in}}%
\pgfpathlineto{\pgfqpoint{1.286273in}{2.508340in}}%
\pgfpathlineto{\pgfqpoint{1.270930in}{2.550628in}}%
\pgfpathlineto{\pgfqpoint{1.255588in}{2.593837in}}%
\pgfpathlineto{\pgfqpoint{1.240246in}{2.637987in}}%
\pgfpathlineto{\pgfqpoint{1.224904in}{2.683099in}}%
\pgfpathlineto{\pgfqpoint{1.209561in}{2.729194in}}%
\pgfpathlineto{\pgfqpoint{1.194219in}{2.776293in}}%
\pgfpathlineto{\pgfqpoint{1.178877in}{2.824418in}}%
\pgfpathlineto{\pgfqpoint{1.163535in}{2.873592in}}%
\pgfpathlineto{\pgfqpoint{1.148192in}{2.923837in}}%
\pgfpathlineto{\pgfqpoint{1.132850in}{2.975176in}}%
\pgfpathlineto{\pgfqpoint{1.117508in}{3.027634in}}%
\pgfpathlineto{\pgfqpoint{1.102166in}{3.081235in}}%
\pgfpathlineto{\pgfqpoint{1.086824in}{3.136004in}}%
\pgfpathlineto{\pgfqpoint{1.071481in}{3.191966in}}%
\pgfpathlineto{\pgfqpoint{1.056139in}{3.249147in}}%
\pgfpathlineto{\pgfqpoint{1.040797in}{3.307573in}}%
\pgfpathlineto{\pgfqpoint{1.025455in}{3.367273in}}%
\pgfpathclose%
\pgfusepath{stroke,fill}%
}%
\begin{pgfscope}%
\pgfsys@transformshift{0.000000in}{0.000000in}%
\pgfsys@useobject{currentmarker}{}%
\end{pgfscope}%
\end{pgfscope}%
\begin{pgfscope}%
\pgfsetbuttcap%
\pgfsetroundjoin%
\definecolor{currentfill}{rgb}{0.000000,0.000000,0.000000}%
\pgfsetfillcolor{currentfill}%
\pgfsetlinewidth{0.803000pt}%
\definecolor{currentstroke}{rgb}{0.000000,0.000000,0.000000}%
\pgfsetstrokecolor{currentstroke}%
\pgfsetdash{}{0pt}%
\pgfsys@defobject{currentmarker}{\pgfqpoint{0.000000in}{-0.048611in}}{\pgfqpoint{0.000000in}{0.000000in}}{%
\pgfpathmoveto{\pgfqpoint{0.000000in}{0.000000in}}%
\pgfpathlineto{\pgfqpoint{0.000000in}{-0.048611in}}%
\pgfusepath{stroke,fill}%
}%
\begin{pgfscope}%
\pgfsys@transformshift{1.025455in}{0.440000in}%
\pgfsys@useobject{currentmarker}{}%
\end{pgfscope}%
\end{pgfscope}%
\begin{pgfscope}%
\definecolor{textcolor}{rgb}{0.000000,0.000000,0.000000}%
\pgfsetstrokecolor{textcolor}%
\pgfsetfillcolor{textcolor}%
\pgftext[x=1.025455in,y=0.342778in,,top]{\color{textcolor}\rmfamily\fontsize{10.000000}{12.000000}\selectfont \(\displaystyle {0}\)}%
\end{pgfscope}%
\begin{pgfscope}%
\pgfsetbuttcap%
\pgfsetroundjoin%
\definecolor{currentfill}{rgb}{0.000000,0.000000,0.000000}%
\pgfsetfillcolor{currentfill}%
\pgfsetlinewidth{0.803000pt}%
\definecolor{currentstroke}{rgb}{0.000000,0.000000,0.000000}%
\pgfsetstrokecolor{currentstroke}%
\pgfsetdash{}{0pt}%
\pgfsys@defobject{currentmarker}{\pgfqpoint{0.000000in}{-0.048611in}}{\pgfqpoint{0.000000in}{0.000000in}}{%
\pgfpathmoveto{\pgfqpoint{0.000000in}{0.000000in}}%
\pgfpathlineto{\pgfqpoint{0.000000in}{-0.048611in}}%
\pgfusepath{stroke,fill}%
}%
\begin{pgfscope}%
\pgfsys@transformshift{1.976539in}{0.440000in}%
\pgfsys@useobject{currentmarker}{}%
\end{pgfscope}%
\end{pgfscope}%
\begin{pgfscope}%
\definecolor{textcolor}{rgb}{0.000000,0.000000,0.000000}%
\pgfsetstrokecolor{textcolor}%
\pgfsetfillcolor{textcolor}%
\pgftext[x=1.976539in,y=0.342778in,,top]{\color{textcolor}\rmfamily\fontsize{10.000000}{12.000000}\selectfont \(\displaystyle {10}\)}%
\end{pgfscope}%
\begin{pgfscope}%
\pgfsetbuttcap%
\pgfsetroundjoin%
\definecolor{currentfill}{rgb}{0.000000,0.000000,0.000000}%
\pgfsetfillcolor{currentfill}%
\pgfsetlinewidth{0.803000pt}%
\definecolor{currentstroke}{rgb}{0.000000,0.000000,0.000000}%
\pgfsetstrokecolor{currentstroke}%
\pgfsetdash{}{0pt}%
\pgfsys@defobject{currentmarker}{\pgfqpoint{0.000000in}{-0.048611in}}{\pgfqpoint{0.000000in}{0.000000in}}{%
\pgfpathmoveto{\pgfqpoint{0.000000in}{0.000000in}}%
\pgfpathlineto{\pgfqpoint{0.000000in}{-0.048611in}}%
\pgfusepath{stroke,fill}%
}%
\begin{pgfscope}%
\pgfsys@transformshift{2.927623in}{0.440000in}%
\pgfsys@useobject{currentmarker}{}%
\end{pgfscope}%
\end{pgfscope}%
\begin{pgfscope}%
\definecolor{textcolor}{rgb}{0.000000,0.000000,0.000000}%
\pgfsetstrokecolor{textcolor}%
\pgfsetfillcolor{textcolor}%
\pgftext[x=2.927623in,y=0.342778in,,top]{\color{textcolor}\rmfamily\fontsize{10.000000}{12.000000}\selectfont \(\displaystyle {20}\)}%
\end{pgfscope}%
\begin{pgfscope}%
\pgfsetbuttcap%
\pgfsetroundjoin%
\definecolor{currentfill}{rgb}{0.000000,0.000000,0.000000}%
\pgfsetfillcolor{currentfill}%
\pgfsetlinewidth{0.803000pt}%
\definecolor{currentstroke}{rgb}{0.000000,0.000000,0.000000}%
\pgfsetstrokecolor{currentstroke}%
\pgfsetdash{}{0pt}%
\pgfsys@defobject{currentmarker}{\pgfqpoint{0.000000in}{-0.048611in}}{\pgfqpoint{0.000000in}{0.000000in}}{%
\pgfpathmoveto{\pgfqpoint{0.000000in}{0.000000in}}%
\pgfpathlineto{\pgfqpoint{0.000000in}{-0.048611in}}%
\pgfusepath{stroke,fill}%
}%
\begin{pgfscope}%
\pgfsys@transformshift{3.878708in}{0.440000in}%
\pgfsys@useobject{currentmarker}{}%
\end{pgfscope}%
\end{pgfscope}%
\begin{pgfscope}%
\definecolor{textcolor}{rgb}{0.000000,0.000000,0.000000}%
\pgfsetstrokecolor{textcolor}%
\pgfsetfillcolor{textcolor}%
\pgftext[x=3.878708in,y=0.342778in,,top]{\color{textcolor}\rmfamily\fontsize{10.000000}{12.000000}\selectfont \(\displaystyle {30}\)}%
\end{pgfscope}%
\begin{pgfscope}%
\pgfsetbuttcap%
\pgfsetroundjoin%
\definecolor{currentfill}{rgb}{0.000000,0.000000,0.000000}%
\pgfsetfillcolor{currentfill}%
\pgfsetlinewidth{0.803000pt}%
\definecolor{currentstroke}{rgb}{0.000000,0.000000,0.000000}%
\pgfsetstrokecolor{currentstroke}%
\pgfsetdash{}{0pt}%
\pgfsys@defobject{currentmarker}{\pgfqpoint{0.000000in}{-0.048611in}}{\pgfqpoint{0.000000in}{0.000000in}}{%
\pgfpathmoveto{\pgfqpoint{0.000000in}{0.000000in}}%
\pgfpathlineto{\pgfqpoint{0.000000in}{-0.048611in}}%
\pgfusepath{stroke,fill}%
}%
\begin{pgfscope}%
\pgfsys@transformshift{4.829792in}{0.440000in}%
\pgfsys@useobject{currentmarker}{}%
\end{pgfscope}%
\end{pgfscope}%
\begin{pgfscope}%
\definecolor{textcolor}{rgb}{0.000000,0.000000,0.000000}%
\pgfsetstrokecolor{textcolor}%
\pgfsetfillcolor{textcolor}%
\pgftext[x=4.829792in,y=0.342778in,,top]{\color{textcolor}\rmfamily\fontsize{10.000000}{12.000000}\selectfont \(\displaystyle {40}\)}%
\end{pgfscope}%
\begin{pgfscope}%
\definecolor{textcolor}{rgb}{0.000000,0.000000,0.000000}%
\pgfsetstrokecolor{textcolor}%
\pgfsetfillcolor{textcolor}%
\pgftext[x=3.280000in,y=0.163766in,,top]{\color{textcolor}\rmfamily\fontsize{10.000000}{12.000000}\selectfont Time \(\displaystyle t\), \(\displaystyle t = \sec\)}%
\end{pgfscope}%
\begin{pgfscope}%
\pgfsetbuttcap%
\pgfsetroundjoin%
\definecolor{currentfill}{rgb}{0.000000,0.000000,0.000000}%
\pgfsetfillcolor{currentfill}%
\pgfsetlinewidth{0.803000pt}%
\definecolor{currentstroke}{rgb}{0.000000,0.000000,0.000000}%
\pgfsetstrokecolor{currentstroke}%
\pgfsetdash{}{0pt}%
\pgfsys@defobject{currentmarker}{\pgfqpoint{-0.048611in}{0.000000in}}{\pgfqpoint{-0.000000in}{0.000000in}}{%
\pgfpathmoveto{\pgfqpoint{-0.000000in}{0.000000in}}%
\pgfpathlineto{\pgfqpoint{-0.048611in}{0.000000in}}%
\pgfusepath{stroke,fill}%
}%
\begin{pgfscope}%
\pgfsys@transformshift{0.800000in}{0.567273in}%
\pgfsys@useobject{currentmarker}{}%
\end{pgfscope}%
\end{pgfscope}%
\begin{pgfscope}%
\definecolor{textcolor}{rgb}{0.000000,0.000000,0.000000}%
\pgfsetstrokecolor{textcolor}%
\pgfsetfillcolor{textcolor}%
\pgftext[x=0.633333in, y=0.519047in, left, base]{\color{textcolor}\rmfamily\fontsize{10.000000}{12.000000}\selectfont \(\displaystyle {0}\)}%
\end{pgfscope}%
\begin{pgfscope}%
\pgfsetbuttcap%
\pgfsetroundjoin%
\definecolor{currentfill}{rgb}{0.000000,0.000000,0.000000}%
\pgfsetfillcolor{currentfill}%
\pgfsetlinewidth{0.803000pt}%
\definecolor{currentstroke}{rgb}{0.000000,0.000000,0.000000}%
\pgfsetstrokecolor{currentstroke}%
\pgfsetdash{}{0pt}%
\pgfsys@defobject{currentmarker}{\pgfqpoint{-0.048611in}{0.000000in}}{\pgfqpoint{-0.000000in}{0.000000in}}{%
\pgfpathmoveto{\pgfqpoint{-0.000000in}{0.000000in}}%
\pgfpathlineto{\pgfqpoint{-0.048611in}{0.000000in}}%
\pgfusepath{stroke,fill}%
}%
\begin{pgfscope}%
\pgfsys@transformshift{0.800000in}{1.076364in}%
\pgfsys@useobject{currentmarker}{}%
\end{pgfscope}%
\end{pgfscope}%
\begin{pgfscope}%
\definecolor{textcolor}{rgb}{0.000000,0.000000,0.000000}%
\pgfsetstrokecolor{textcolor}%
\pgfsetfillcolor{textcolor}%
\pgftext[x=0.563888in, y=1.028138in, left, base]{\color{textcolor}\rmfamily\fontsize{10.000000}{12.000000}\selectfont \(\displaystyle {20}\)}%
\end{pgfscope}%
\begin{pgfscope}%
\pgfsetbuttcap%
\pgfsetroundjoin%
\definecolor{currentfill}{rgb}{0.000000,0.000000,0.000000}%
\pgfsetfillcolor{currentfill}%
\pgfsetlinewidth{0.803000pt}%
\definecolor{currentstroke}{rgb}{0.000000,0.000000,0.000000}%
\pgfsetstrokecolor{currentstroke}%
\pgfsetdash{}{0pt}%
\pgfsys@defobject{currentmarker}{\pgfqpoint{-0.048611in}{0.000000in}}{\pgfqpoint{-0.000000in}{0.000000in}}{%
\pgfpathmoveto{\pgfqpoint{-0.000000in}{0.000000in}}%
\pgfpathlineto{\pgfqpoint{-0.048611in}{0.000000in}}%
\pgfusepath{stroke,fill}%
}%
\begin{pgfscope}%
\pgfsys@transformshift{0.800000in}{1.585455in}%
\pgfsys@useobject{currentmarker}{}%
\end{pgfscope}%
\end{pgfscope}%
\begin{pgfscope}%
\definecolor{textcolor}{rgb}{0.000000,0.000000,0.000000}%
\pgfsetstrokecolor{textcolor}%
\pgfsetfillcolor{textcolor}%
\pgftext[x=0.563888in, y=1.537229in, left, base]{\color{textcolor}\rmfamily\fontsize{10.000000}{12.000000}\selectfont \(\displaystyle {40}\)}%
\end{pgfscope}%
\begin{pgfscope}%
\pgfsetbuttcap%
\pgfsetroundjoin%
\definecolor{currentfill}{rgb}{0.000000,0.000000,0.000000}%
\pgfsetfillcolor{currentfill}%
\pgfsetlinewidth{0.803000pt}%
\definecolor{currentstroke}{rgb}{0.000000,0.000000,0.000000}%
\pgfsetstrokecolor{currentstroke}%
\pgfsetdash{}{0pt}%
\pgfsys@defobject{currentmarker}{\pgfqpoint{-0.048611in}{0.000000in}}{\pgfqpoint{-0.000000in}{0.000000in}}{%
\pgfpathmoveto{\pgfqpoint{-0.000000in}{0.000000in}}%
\pgfpathlineto{\pgfqpoint{-0.048611in}{0.000000in}}%
\pgfusepath{stroke,fill}%
}%
\begin{pgfscope}%
\pgfsys@transformshift{0.800000in}{2.094545in}%
\pgfsys@useobject{currentmarker}{}%
\end{pgfscope}%
\end{pgfscope}%
\begin{pgfscope}%
\definecolor{textcolor}{rgb}{0.000000,0.000000,0.000000}%
\pgfsetstrokecolor{textcolor}%
\pgfsetfillcolor{textcolor}%
\pgftext[x=0.563888in, y=2.046320in, left, base]{\color{textcolor}\rmfamily\fontsize{10.000000}{12.000000}\selectfont \(\displaystyle {60}\)}%
\end{pgfscope}%
\begin{pgfscope}%
\pgfsetbuttcap%
\pgfsetroundjoin%
\definecolor{currentfill}{rgb}{0.000000,0.000000,0.000000}%
\pgfsetfillcolor{currentfill}%
\pgfsetlinewidth{0.803000pt}%
\definecolor{currentstroke}{rgb}{0.000000,0.000000,0.000000}%
\pgfsetstrokecolor{currentstroke}%
\pgfsetdash{}{0pt}%
\pgfsys@defobject{currentmarker}{\pgfqpoint{-0.048611in}{0.000000in}}{\pgfqpoint{-0.000000in}{0.000000in}}{%
\pgfpathmoveto{\pgfqpoint{-0.000000in}{0.000000in}}%
\pgfpathlineto{\pgfqpoint{-0.048611in}{0.000000in}}%
\pgfusepath{stroke,fill}%
}%
\begin{pgfscope}%
\pgfsys@transformshift{0.800000in}{2.603636in}%
\pgfsys@useobject{currentmarker}{}%
\end{pgfscope}%
\end{pgfscope}%
\begin{pgfscope}%
\definecolor{textcolor}{rgb}{0.000000,0.000000,0.000000}%
\pgfsetstrokecolor{textcolor}%
\pgfsetfillcolor{textcolor}%
\pgftext[x=0.563888in, y=2.555411in, left, base]{\color{textcolor}\rmfamily\fontsize{10.000000}{12.000000}\selectfont \(\displaystyle {80}\)}%
\end{pgfscope}%
\begin{pgfscope}%
\pgfsetbuttcap%
\pgfsetroundjoin%
\definecolor{currentfill}{rgb}{0.000000,0.000000,0.000000}%
\pgfsetfillcolor{currentfill}%
\pgfsetlinewidth{0.803000pt}%
\definecolor{currentstroke}{rgb}{0.000000,0.000000,0.000000}%
\pgfsetstrokecolor{currentstroke}%
\pgfsetdash{}{0pt}%
\pgfsys@defobject{currentmarker}{\pgfqpoint{-0.048611in}{0.000000in}}{\pgfqpoint{-0.000000in}{0.000000in}}{%
\pgfpathmoveto{\pgfqpoint{-0.000000in}{0.000000in}}%
\pgfpathlineto{\pgfqpoint{-0.048611in}{0.000000in}}%
\pgfusepath{stroke,fill}%
}%
\begin{pgfscope}%
\pgfsys@transformshift{0.800000in}{3.112727in}%
\pgfsys@useobject{currentmarker}{}%
\end{pgfscope}%
\end{pgfscope}%
\begin{pgfscope}%
\definecolor{textcolor}{rgb}{0.000000,0.000000,0.000000}%
\pgfsetstrokecolor{textcolor}%
\pgfsetfillcolor{textcolor}%
\pgftext[x=0.494444in, y=3.064502in, left, base]{\color{textcolor}\rmfamily\fontsize{10.000000}{12.000000}\selectfont \(\displaystyle {100}\)}%
\end{pgfscope}%
\begin{pgfscope}%
\definecolor{textcolor}{rgb}{0.000000,0.000000,0.000000}%
\pgfsetstrokecolor{textcolor}%
\pgfsetfillcolor{textcolor}%
\pgftext[x=0.438888in,y=1.980000in,,bottom,rotate=90.000000]{\color{textcolor}\rmfamily\fontsize{10.000000}{12.000000}\selectfont Amplitude \(\displaystyle A\), \(\displaystyle [A] =  \deg\)}%
\end{pgfscope}%
\begin{pgfscope}%
\pgfpathrectangle{\pgfqpoint{0.800000in}{0.440000in}}{\pgfqpoint{4.960000in}{3.080000in}}%
\pgfusepath{clip}%
\pgfsetbuttcap%
\pgfsetroundjoin%
\pgfsetlinewidth{1.003750pt}%
\definecolor{currentstroke}{rgb}{1.000000,0.000000,0.000000}%
\pgfsetstrokecolor{currentstroke}%
\pgfsetdash{}{0pt}%
\pgfpathmoveto{\pgfqpoint{1.025455in}{3.354545in}}%
\pgfpathlineto{\pgfqpoint{1.025455in}{3.380000in}}%
\pgfusepath{stroke}%
\end{pgfscope}%
\begin{pgfscope}%
\pgfpathrectangle{\pgfqpoint{0.800000in}{0.440000in}}{\pgfqpoint{4.960000in}{3.080000in}}%
\pgfusepath{clip}%
\pgfsetbuttcap%
\pgfsetroundjoin%
\pgfsetlinewidth{1.003750pt}%
\definecolor{currentstroke}{rgb}{1.000000,0.000000,0.000000}%
\pgfsetstrokecolor{currentstroke}%
\pgfsetdash{}{0pt}%
\pgfpathmoveto{\pgfqpoint{1.589091in}{2.654545in}}%
\pgfpathlineto{\pgfqpoint{1.589091in}{2.756364in}}%
\pgfusepath{stroke}%
\end{pgfscope}%
\begin{pgfscope}%
\pgfpathrectangle{\pgfqpoint{0.800000in}{0.440000in}}{\pgfqpoint{4.960000in}{3.080000in}}%
\pgfusepath{clip}%
\pgfsetbuttcap%
\pgfsetroundjoin%
\pgfsetlinewidth{1.003750pt}%
\definecolor{currentstroke}{rgb}{1.000000,0.000000,0.000000}%
\pgfsetstrokecolor{currentstroke}%
\pgfsetdash{}{0pt}%
\pgfpathmoveto{\pgfqpoint{2.152727in}{2.030909in}}%
\pgfpathlineto{\pgfqpoint{2.152727in}{2.132727in}}%
\pgfusepath{stroke}%
\end{pgfscope}%
\begin{pgfscope}%
\pgfpathrectangle{\pgfqpoint{0.800000in}{0.440000in}}{\pgfqpoint{4.960000in}{3.080000in}}%
\pgfusepath{clip}%
\pgfsetbuttcap%
\pgfsetroundjoin%
\pgfsetlinewidth{1.003750pt}%
\definecolor{currentstroke}{rgb}{1.000000,0.000000,0.000000}%
\pgfsetstrokecolor{currentstroke}%
\pgfsetdash{}{0pt}%
\pgfpathmoveto{\pgfqpoint{2.716364in}{1.560000in}}%
\pgfpathlineto{\pgfqpoint{2.716364in}{1.661818in}}%
\pgfusepath{stroke}%
\end{pgfscope}%
\begin{pgfscope}%
\pgfpathrectangle{\pgfqpoint{0.800000in}{0.440000in}}{\pgfqpoint{4.960000in}{3.080000in}}%
\pgfusepath{clip}%
\pgfsetbuttcap%
\pgfsetroundjoin%
\pgfsetlinewidth{1.003750pt}%
\definecolor{currentstroke}{rgb}{1.000000,0.000000,0.000000}%
\pgfsetstrokecolor{currentstroke}%
\pgfsetdash{}{0pt}%
\pgfpathmoveto{\pgfqpoint{3.280000in}{1.229091in}}%
\pgfpathlineto{\pgfqpoint{3.280000in}{1.330909in}}%
\pgfusepath{stroke}%
\end{pgfscope}%
\begin{pgfscope}%
\pgfpathrectangle{\pgfqpoint{0.800000in}{0.440000in}}{\pgfqpoint{4.960000in}{3.080000in}}%
\pgfusepath{clip}%
\pgfsetbuttcap%
\pgfsetroundjoin%
\pgfsetlinewidth{1.003750pt}%
\definecolor{currentstroke}{rgb}{1.000000,0.000000,0.000000}%
\pgfsetstrokecolor{currentstroke}%
\pgfsetdash{}{0pt}%
\pgfpathmoveto{\pgfqpoint{3.843636in}{1.000000in}}%
\pgfpathlineto{\pgfqpoint{3.843636in}{1.101818in}}%
\pgfusepath{stroke}%
\end{pgfscope}%
\begin{pgfscope}%
\pgfpathrectangle{\pgfqpoint{0.800000in}{0.440000in}}{\pgfqpoint{4.960000in}{3.080000in}}%
\pgfusepath{clip}%
\pgfsetbuttcap%
\pgfsetroundjoin%
\pgfsetlinewidth{1.003750pt}%
\definecolor{currentstroke}{rgb}{1.000000,0.000000,0.000000}%
\pgfsetstrokecolor{currentstroke}%
\pgfsetdash{}{0pt}%
\pgfpathmoveto{\pgfqpoint{4.407273in}{0.809091in}}%
\pgfpathlineto{\pgfqpoint{4.407273in}{0.910909in}}%
\pgfusepath{stroke}%
\end{pgfscope}%
\begin{pgfscope}%
\pgfpathrectangle{\pgfqpoint{0.800000in}{0.440000in}}{\pgfqpoint{4.960000in}{3.080000in}}%
\pgfusepath{clip}%
\pgfsetbuttcap%
\pgfsetroundjoin%
\pgfsetlinewidth{1.003750pt}%
\definecolor{currentstroke}{rgb}{1.000000,0.000000,0.000000}%
\pgfsetstrokecolor{currentstroke}%
\pgfsetdash{}{0pt}%
\pgfpathmoveto{\pgfqpoint{4.970909in}{0.681818in}}%
\pgfpathlineto{\pgfqpoint{4.970909in}{0.783636in}}%
\pgfusepath{stroke}%
\end{pgfscope}%
\begin{pgfscope}%
\pgfpathrectangle{\pgfqpoint{0.800000in}{0.440000in}}{\pgfqpoint{4.960000in}{3.080000in}}%
\pgfusepath{clip}%
\pgfsetbuttcap%
\pgfsetroundjoin%
\pgfsetlinewidth{1.003750pt}%
\definecolor{currentstroke}{rgb}{1.000000,0.000000,0.000000}%
\pgfsetstrokecolor{currentstroke}%
\pgfsetdash{}{0pt}%
\pgfpathmoveto{\pgfqpoint{5.534545in}{0.580000in}}%
\pgfpathlineto{\pgfqpoint{5.534545in}{0.681818in}}%
\pgfusepath{stroke}%
\end{pgfscope}%
\begin{pgfscope}%
\pgfpathrectangle{\pgfqpoint{0.800000in}{0.440000in}}{\pgfqpoint{4.960000in}{3.080000in}}%
\pgfusepath{clip}%
\pgfsetbuttcap%
\pgfsetroundjoin%
\pgfsetlinewidth{1.003750pt}%
\definecolor{currentstroke}{rgb}{0.000000,0.000000,1.000000}%
\pgfsetstrokecolor{currentstroke}%
\pgfsetdash{}{0pt}%
\pgfpathmoveto{\pgfqpoint{1.025455in}{3.354545in}}%
\pgfpathlineto{\pgfqpoint{1.025455in}{3.380000in}}%
\pgfusepath{stroke}%
\end{pgfscope}%
\begin{pgfscope}%
\pgfpathrectangle{\pgfqpoint{0.800000in}{0.440000in}}{\pgfqpoint{4.960000in}{3.080000in}}%
\pgfusepath{clip}%
\pgfsetbuttcap%
\pgfsetroundjoin%
\pgfsetlinewidth{1.003750pt}%
\definecolor{currentstroke}{rgb}{0.000000,0.000000,1.000000}%
\pgfsetstrokecolor{currentstroke}%
\pgfsetdash{}{0pt}%
\pgfpathmoveto{\pgfqpoint{1.212954in}{2.985455in}}%
\pgfpathlineto{\pgfqpoint{1.212954in}{3.087273in}}%
\pgfusepath{stroke}%
\end{pgfscope}%
\begin{pgfscope}%
\pgfpathrectangle{\pgfqpoint{0.800000in}{0.440000in}}{\pgfqpoint{4.960000in}{3.080000in}}%
\pgfusepath{clip}%
\pgfsetbuttcap%
\pgfsetroundjoin%
\pgfsetlinewidth{1.003750pt}%
\definecolor{currentstroke}{rgb}{0.000000,0.000000,1.000000}%
\pgfsetstrokecolor{currentstroke}%
\pgfsetdash{}{0pt}%
\pgfpathmoveto{\pgfqpoint{1.400454in}{2.552727in}}%
\pgfpathlineto{\pgfqpoint{1.400454in}{2.654545in}}%
\pgfusepath{stroke}%
\end{pgfscope}%
\begin{pgfscope}%
\pgfpathrectangle{\pgfqpoint{0.800000in}{0.440000in}}{\pgfqpoint{4.960000in}{3.080000in}}%
\pgfusepath{clip}%
\pgfsetbuttcap%
\pgfsetroundjoin%
\pgfsetlinewidth{1.003750pt}%
\definecolor{currentstroke}{rgb}{0.000000,0.000000,1.000000}%
\pgfsetstrokecolor{currentstroke}%
\pgfsetdash{}{0pt}%
\pgfpathmoveto{\pgfqpoint{1.775452in}{1.941818in}}%
\pgfpathlineto{\pgfqpoint{1.775452in}{2.043636in}}%
\pgfusepath{stroke}%
\end{pgfscope}%
\begin{pgfscope}%
\pgfpathrectangle{\pgfqpoint{0.800000in}{0.440000in}}{\pgfqpoint{4.960000in}{3.080000in}}%
\pgfusepath{clip}%
\pgfsetbuttcap%
\pgfsetroundjoin%
\pgfsetlinewidth{1.003750pt}%
\definecolor{currentstroke}{rgb}{0.000000,0.000000,1.000000}%
\pgfsetstrokecolor{currentstroke}%
\pgfsetdash{}{0pt}%
\pgfpathmoveto{\pgfqpoint{2.150451in}{1.483636in}}%
\pgfpathlineto{\pgfqpoint{2.150451in}{1.585455in}}%
\pgfusepath{stroke}%
\end{pgfscope}%
\begin{pgfscope}%
\pgfpathrectangle{\pgfqpoint{0.800000in}{0.440000in}}{\pgfqpoint{4.960000in}{3.080000in}}%
\pgfusepath{clip}%
\pgfsetbuttcap%
\pgfsetroundjoin%
\pgfsetlinewidth{1.003750pt}%
\definecolor{currentstroke}{rgb}{0.000000,0.000000,1.000000}%
\pgfsetstrokecolor{currentstroke}%
\pgfsetdash{}{0pt}%
\pgfpathmoveto{\pgfqpoint{2.525450in}{1.152727in}}%
\pgfpathlineto{\pgfqpoint{2.525450in}{1.254545in}}%
\pgfusepath{stroke}%
\end{pgfscope}%
\begin{pgfscope}%
\pgfpathrectangle{\pgfqpoint{0.800000in}{0.440000in}}{\pgfqpoint{4.960000in}{3.080000in}}%
\pgfusepath{clip}%
\pgfsetbuttcap%
\pgfsetroundjoin%
\pgfsetlinewidth{1.003750pt}%
\definecolor{currentstroke}{rgb}{0.000000,0.000000,1.000000}%
\pgfsetstrokecolor{currentstroke}%
\pgfsetdash{}{0pt}%
\pgfpathmoveto{\pgfqpoint{2.900449in}{0.898182in}}%
\pgfpathlineto{\pgfqpoint{2.900449in}{1.000000in}}%
\pgfusepath{stroke}%
\end{pgfscope}%
\begin{pgfscope}%
\pgfpathrectangle{\pgfqpoint{0.800000in}{0.440000in}}{\pgfqpoint{4.960000in}{3.080000in}}%
\pgfusepath{clip}%
\pgfsetbuttcap%
\pgfsetroundjoin%
\pgfsetlinewidth{1.003750pt}%
\definecolor{currentstroke}{rgb}{0.000000,0.000000,1.000000}%
\pgfsetstrokecolor{currentstroke}%
\pgfsetdash{}{0pt}%
\pgfpathmoveto{\pgfqpoint{3.275448in}{0.770909in}}%
\pgfpathlineto{\pgfqpoint{3.275448in}{0.872727in}}%
\pgfusepath{stroke}%
\end{pgfscope}%
\begin{pgfscope}%
\pgfpathrectangle{\pgfqpoint{0.800000in}{0.440000in}}{\pgfqpoint{4.960000in}{3.080000in}}%
\pgfusepath{clip}%
\pgfsetbuttcap%
\pgfsetroundjoin%
\pgfsetlinewidth{1.003750pt}%
\definecolor{currentstroke}{rgb}{0.000000,0.000000,1.000000}%
\pgfsetstrokecolor{currentstroke}%
\pgfsetdash{}{0pt}%
\pgfpathmoveto{\pgfqpoint{3.650447in}{0.669091in}}%
\pgfpathlineto{\pgfqpoint{3.650447in}{0.770909in}}%
\pgfusepath{stroke}%
\end{pgfscope}%
\begin{pgfscope}%
\pgfpathrectangle{\pgfqpoint{0.800000in}{0.440000in}}{\pgfqpoint{4.960000in}{3.080000in}}%
\pgfusepath{clip}%
\pgfsetbuttcap%
\pgfsetroundjoin%
\pgfsetlinewidth{1.003750pt}%
\definecolor{currentstroke}{rgb}{0.000000,0.501961,0.000000}%
\pgfsetstrokecolor{currentstroke}%
\pgfsetdash{}{0pt}%
\pgfpathmoveto{\pgfqpoint{1.025455in}{3.354545in}}%
\pgfpathlineto{\pgfqpoint{1.025455in}{3.380000in}}%
\pgfusepath{stroke}%
\end{pgfscope}%
\begin{pgfscope}%
\pgfpathrectangle{\pgfqpoint{0.800000in}{0.440000in}}{\pgfqpoint{4.960000in}{3.080000in}}%
\pgfusepath{clip}%
\pgfsetbuttcap%
\pgfsetroundjoin%
\pgfsetlinewidth{1.003750pt}%
\definecolor{currentstroke}{rgb}{0.000000,0.501961,0.000000}%
\pgfsetstrokecolor{currentstroke}%
\pgfsetdash{}{0pt}%
\pgfpathmoveto{\pgfqpoint{1.215315in}{2.680000in}}%
\pgfpathlineto{\pgfqpoint{1.215315in}{2.781818in}}%
\pgfusepath{stroke}%
\end{pgfscope}%
\begin{pgfscope}%
\pgfpathrectangle{\pgfqpoint{0.800000in}{0.440000in}}{\pgfqpoint{4.960000in}{3.080000in}}%
\pgfusepath{clip}%
\pgfsetbuttcap%
\pgfsetroundjoin%
\pgfsetlinewidth{1.003750pt}%
\definecolor{currentstroke}{rgb}{0.000000,0.501961,0.000000}%
\pgfsetstrokecolor{currentstroke}%
\pgfsetdash{}{0pt}%
\pgfpathmoveto{\pgfqpoint{1.405175in}{2.145455in}}%
\pgfpathlineto{\pgfqpoint{1.405175in}{2.247273in}}%
\pgfusepath{stroke}%
\end{pgfscope}%
\begin{pgfscope}%
\pgfpathrectangle{\pgfqpoint{0.800000in}{0.440000in}}{\pgfqpoint{4.960000in}{3.080000in}}%
\pgfusepath{clip}%
\pgfsetbuttcap%
\pgfsetroundjoin%
\pgfsetlinewidth{1.003750pt}%
\definecolor{currentstroke}{rgb}{0.000000,0.501961,0.000000}%
\pgfsetstrokecolor{currentstroke}%
\pgfsetdash{}{0pt}%
\pgfpathmoveto{\pgfqpoint{1.595035in}{1.738182in}}%
\pgfpathlineto{\pgfqpoint{1.595035in}{1.840000in}}%
\pgfusepath{stroke}%
\end{pgfscope}%
\begin{pgfscope}%
\pgfpathrectangle{\pgfqpoint{0.800000in}{0.440000in}}{\pgfqpoint{4.960000in}{3.080000in}}%
\pgfusepath{clip}%
\pgfsetbuttcap%
\pgfsetroundjoin%
\pgfsetlinewidth{1.003750pt}%
\definecolor{currentstroke}{rgb}{0.000000,0.501961,0.000000}%
\pgfsetstrokecolor{currentstroke}%
\pgfsetdash{}{0pt}%
\pgfpathmoveto{\pgfqpoint{1.784895in}{1.432727in}}%
\pgfpathlineto{\pgfqpoint{1.784895in}{1.534545in}}%
\pgfusepath{stroke}%
\end{pgfscope}%
\begin{pgfscope}%
\pgfpathrectangle{\pgfqpoint{0.800000in}{0.440000in}}{\pgfqpoint{4.960000in}{3.080000in}}%
\pgfusepath{clip}%
\pgfsetbuttcap%
\pgfsetroundjoin%
\pgfsetlinewidth{1.003750pt}%
\definecolor{currentstroke}{rgb}{0.000000,0.501961,0.000000}%
\pgfsetstrokecolor{currentstroke}%
\pgfsetdash{}{0pt}%
\pgfpathmoveto{\pgfqpoint{1.974756in}{1.178182in}}%
\pgfpathlineto{\pgfqpoint{1.974756in}{1.280000in}}%
\pgfusepath{stroke}%
\end{pgfscope}%
\begin{pgfscope}%
\pgfpathrectangle{\pgfqpoint{0.800000in}{0.440000in}}{\pgfqpoint{4.960000in}{3.080000in}}%
\pgfusepath{clip}%
\pgfsetbuttcap%
\pgfsetroundjoin%
\pgfsetlinewidth{1.003750pt}%
\definecolor{currentstroke}{rgb}{0.000000,0.501961,0.000000}%
\pgfsetstrokecolor{currentstroke}%
\pgfsetdash{}{0pt}%
\pgfpathmoveto{\pgfqpoint{2.164616in}{1.000000in}}%
\pgfpathlineto{\pgfqpoint{2.164616in}{1.101818in}}%
\pgfusepath{stroke}%
\end{pgfscope}%
\begin{pgfscope}%
\pgfpathrectangle{\pgfqpoint{0.800000in}{0.440000in}}{\pgfqpoint{4.960000in}{3.080000in}}%
\pgfusepath{clip}%
\pgfsetbuttcap%
\pgfsetroundjoin%
\pgfsetlinewidth{1.003750pt}%
\definecolor{currentstroke}{rgb}{0.000000,0.501961,0.000000}%
\pgfsetstrokecolor{currentstroke}%
\pgfsetdash{}{0pt}%
\pgfpathmoveto{\pgfqpoint{2.354476in}{0.872727in}}%
\pgfpathlineto{\pgfqpoint{2.354476in}{0.974545in}}%
\pgfusepath{stroke}%
\end{pgfscope}%
\begin{pgfscope}%
\pgfpathrectangle{\pgfqpoint{0.800000in}{0.440000in}}{\pgfqpoint{4.960000in}{3.080000in}}%
\pgfusepath{clip}%
\pgfsetbuttcap%
\pgfsetroundjoin%
\pgfsetlinewidth{1.003750pt}%
\definecolor{currentstroke}{rgb}{0.000000,0.501961,0.000000}%
\pgfsetstrokecolor{currentstroke}%
\pgfsetdash{}{0pt}%
\pgfpathmoveto{\pgfqpoint{2.544336in}{0.770909in}}%
\pgfpathlineto{\pgfqpoint{2.544336in}{0.872727in}}%
\pgfusepath{stroke}%
\end{pgfscope}%
\begin{pgfscope}%
\pgfpathrectangle{\pgfqpoint{0.800000in}{0.440000in}}{\pgfqpoint{4.960000in}{3.080000in}}%
\pgfusepath{clip}%
\pgfsetbuttcap%
\pgfsetroundjoin%
\definecolor{currentfill}{rgb}{1.000000,0.000000,0.000000}%
\pgfsetfillcolor{currentfill}%
\pgfsetlinewidth{1.003750pt}%
\definecolor{currentstroke}{rgb}{1.000000,0.000000,0.000000}%
\pgfsetstrokecolor{currentstroke}%
\pgfsetdash{}{0pt}%
\pgfsys@defobject{currentmarker}{\pgfqpoint{-0.041667in}{-0.000000in}}{\pgfqpoint{0.041667in}{0.000000in}}{%
\pgfpathmoveto{\pgfqpoint{0.041667in}{-0.000000in}}%
\pgfpathlineto{\pgfqpoint{-0.041667in}{0.000000in}}%
\pgfusepath{stroke,fill}%
}%
\begin{pgfscope}%
\pgfsys@transformshift{1.025455in}{3.354545in}%
\pgfsys@useobject{currentmarker}{}%
\end{pgfscope}%
\begin{pgfscope}%
\pgfsys@transformshift{1.589091in}{2.654545in}%
\pgfsys@useobject{currentmarker}{}%
\end{pgfscope}%
\begin{pgfscope}%
\pgfsys@transformshift{2.152727in}{2.030909in}%
\pgfsys@useobject{currentmarker}{}%
\end{pgfscope}%
\begin{pgfscope}%
\pgfsys@transformshift{2.716364in}{1.560000in}%
\pgfsys@useobject{currentmarker}{}%
\end{pgfscope}%
\begin{pgfscope}%
\pgfsys@transformshift{3.280000in}{1.229091in}%
\pgfsys@useobject{currentmarker}{}%
\end{pgfscope}%
\begin{pgfscope}%
\pgfsys@transformshift{3.843636in}{1.000000in}%
\pgfsys@useobject{currentmarker}{}%
\end{pgfscope}%
\begin{pgfscope}%
\pgfsys@transformshift{4.407273in}{0.809091in}%
\pgfsys@useobject{currentmarker}{}%
\end{pgfscope}%
\begin{pgfscope}%
\pgfsys@transformshift{4.970909in}{0.681818in}%
\pgfsys@useobject{currentmarker}{}%
\end{pgfscope}%
\begin{pgfscope}%
\pgfsys@transformshift{5.534545in}{0.580000in}%
\pgfsys@useobject{currentmarker}{}%
\end{pgfscope}%
\end{pgfscope}%
\begin{pgfscope}%
\pgfpathrectangle{\pgfqpoint{0.800000in}{0.440000in}}{\pgfqpoint{4.960000in}{3.080000in}}%
\pgfusepath{clip}%
\pgfsetbuttcap%
\pgfsetroundjoin%
\definecolor{currentfill}{rgb}{1.000000,0.000000,0.000000}%
\pgfsetfillcolor{currentfill}%
\pgfsetlinewidth{1.003750pt}%
\definecolor{currentstroke}{rgb}{1.000000,0.000000,0.000000}%
\pgfsetstrokecolor{currentstroke}%
\pgfsetdash{}{0pt}%
\pgfsys@defobject{currentmarker}{\pgfqpoint{-0.041667in}{-0.000000in}}{\pgfqpoint{0.041667in}{0.000000in}}{%
\pgfpathmoveto{\pgfqpoint{0.041667in}{-0.000000in}}%
\pgfpathlineto{\pgfqpoint{-0.041667in}{0.000000in}}%
\pgfusepath{stroke,fill}%
}%
\begin{pgfscope}%
\pgfsys@transformshift{1.025455in}{3.380000in}%
\pgfsys@useobject{currentmarker}{}%
\end{pgfscope}%
\begin{pgfscope}%
\pgfsys@transformshift{1.589091in}{2.756364in}%
\pgfsys@useobject{currentmarker}{}%
\end{pgfscope}%
\begin{pgfscope}%
\pgfsys@transformshift{2.152727in}{2.132727in}%
\pgfsys@useobject{currentmarker}{}%
\end{pgfscope}%
\begin{pgfscope}%
\pgfsys@transformshift{2.716364in}{1.661818in}%
\pgfsys@useobject{currentmarker}{}%
\end{pgfscope}%
\begin{pgfscope}%
\pgfsys@transformshift{3.280000in}{1.330909in}%
\pgfsys@useobject{currentmarker}{}%
\end{pgfscope}%
\begin{pgfscope}%
\pgfsys@transformshift{3.843636in}{1.101818in}%
\pgfsys@useobject{currentmarker}{}%
\end{pgfscope}%
\begin{pgfscope}%
\pgfsys@transformshift{4.407273in}{0.910909in}%
\pgfsys@useobject{currentmarker}{}%
\end{pgfscope}%
\begin{pgfscope}%
\pgfsys@transformshift{4.970909in}{0.783636in}%
\pgfsys@useobject{currentmarker}{}%
\end{pgfscope}%
\begin{pgfscope}%
\pgfsys@transformshift{5.534545in}{0.681818in}%
\pgfsys@useobject{currentmarker}{}%
\end{pgfscope}%
\end{pgfscope}%
\begin{pgfscope}%
\pgfpathrectangle{\pgfqpoint{0.800000in}{0.440000in}}{\pgfqpoint{4.960000in}{3.080000in}}%
\pgfusepath{clip}%
\pgfsetbuttcap%
\pgfsetroundjoin%
\pgfsetlinewidth{1.003750pt}%
\definecolor{currentstroke}{rgb}{1.000000,0.000000,0.000000}%
\pgfsetstrokecolor{currentstroke}%
\pgfsetdash{{3.700000pt}{1.600000pt}}{0.000000pt}%
\pgfpathmoveto{\pgfqpoint{1.025455in}{3.367273in}}%
\pgfpathlineto{\pgfqpoint{1.071001in}{3.291712in}}%
\pgfpathlineto{\pgfqpoint{1.116547in}{3.218191in}}%
\pgfpathlineto{\pgfqpoint{1.162094in}{3.146653in}}%
\pgfpathlineto{\pgfqpoint{1.207640in}{3.077047in}}%
\pgfpathlineto{\pgfqpoint{1.253186in}{3.009318in}}%
\pgfpathlineto{\pgfqpoint{1.298733in}{2.943417in}}%
\pgfpathlineto{\pgfqpoint{1.344279in}{2.879295in}}%
\pgfpathlineto{\pgfqpoint{1.389826in}{2.816903in}}%
\pgfpathlineto{\pgfqpoint{1.435372in}{2.756195in}}%
\pgfpathlineto{\pgfqpoint{1.480918in}{2.697125in}}%
\pgfpathlineto{\pgfqpoint{1.526465in}{2.639649in}}%
\pgfpathlineto{\pgfqpoint{1.572011in}{2.583724in}}%
\pgfpathlineto{\pgfqpoint{1.617557in}{2.529308in}}%
\pgfpathlineto{\pgfqpoint{1.663104in}{2.476361in}}%
\pgfpathlineto{\pgfqpoint{1.708650in}{2.424843in}}%
\pgfpathlineto{\pgfqpoint{1.754197in}{2.374714in}}%
\pgfpathlineto{\pgfqpoint{1.799743in}{2.325939in}}%
\pgfpathlineto{\pgfqpoint{1.845289in}{2.278480in}}%
\pgfpathlineto{\pgfqpoint{1.890836in}{2.232301in}}%
\pgfpathlineto{\pgfqpoint{1.936382in}{2.187369in}}%
\pgfpathlineto{\pgfqpoint{1.981928in}{2.143649in}}%
\pgfpathlineto{\pgfqpoint{2.027475in}{2.101110in}}%
\pgfpathlineto{\pgfqpoint{2.073021in}{2.059718in}}%
\pgfpathlineto{\pgfqpoint{2.118567in}{2.019443in}}%
\pgfpathlineto{\pgfqpoint{2.164114in}{1.980255in}}%
\pgfpathlineto{\pgfqpoint{2.209660in}{1.942124in}}%
\pgfpathlineto{\pgfqpoint{2.255207in}{1.905022in}}%
\pgfpathlineto{\pgfqpoint{2.300753in}{1.868922in}}%
\pgfpathlineto{\pgfqpoint{2.346299in}{1.833796in}}%
\pgfpathlineto{\pgfqpoint{2.391846in}{1.799618in}}%
\pgfpathlineto{\pgfqpoint{2.437392in}{1.766362in}}%
\pgfpathlineto{\pgfqpoint{2.482938in}{1.734003in}}%
\pgfpathlineto{\pgfqpoint{2.528485in}{1.702518in}}%
\pgfpathlineto{\pgfqpoint{2.574031in}{1.671882in}}%
\pgfpathlineto{\pgfqpoint{2.619578in}{1.642074in}}%
\pgfpathlineto{\pgfqpoint{2.665124in}{1.613069in}}%
\pgfpathlineto{\pgfqpoint{2.710670in}{1.584847in}}%
\pgfpathlineto{\pgfqpoint{2.756217in}{1.557387in}}%
\pgfpathlineto{\pgfqpoint{2.801763in}{1.530668in}}%
\pgfpathlineto{\pgfqpoint{2.847309in}{1.504670in}}%
\pgfpathlineto{\pgfqpoint{2.892856in}{1.479374in}}%
\pgfpathlineto{\pgfqpoint{2.938402in}{1.454760in}}%
\pgfpathlineto{\pgfqpoint{2.983949in}{1.430810in}}%
\pgfpathlineto{\pgfqpoint{3.029495in}{1.407507in}}%
\pgfpathlineto{\pgfqpoint{3.075041in}{1.384832in}}%
\pgfpathlineto{\pgfqpoint{3.120588in}{1.362770in}}%
\pgfpathlineto{\pgfqpoint{3.166134in}{1.341303in}}%
\pgfpathlineto{\pgfqpoint{3.211680in}{1.320415in}}%
\pgfpathlineto{\pgfqpoint{3.257227in}{1.300090in}}%
\pgfpathlineto{\pgfqpoint{3.302773in}{1.280315in}}%
\pgfpathlineto{\pgfqpoint{3.348320in}{1.261073in}}%
\pgfpathlineto{\pgfqpoint{3.393866in}{1.242350in}}%
\pgfpathlineto{\pgfqpoint{3.439412in}{1.224132in}}%
\pgfpathlineto{\pgfqpoint{3.484959in}{1.206406in}}%
\pgfpathlineto{\pgfqpoint{3.530505in}{1.189159in}}%
\pgfpathlineto{\pgfqpoint{3.576051in}{1.172377in}}%
\pgfpathlineto{\pgfqpoint{3.621598in}{1.156047in}}%
\pgfpathlineto{\pgfqpoint{3.667144in}{1.140159in}}%
\pgfpathlineto{\pgfqpoint{3.712691in}{1.124699in}}%
\pgfpathlineto{\pgfqpoint{3.758237in}{1.109656in}}%
\pgfpathlineto{\pgfqpoint{3.803783in}{1.095020in}}%
\pgfpathlineto{\pgfqpoint{3.849330in}{1.080778in}}%
\pgfpathlineto{\pgfqpoint{3.894876in}{1.066921in}}%
\pgfpathlineto{\pgfqpoint{3.940422in}{1.053437in}}%
\pgfpathlineto{\pgfqpoint{3.985969in}{1.040318in}}%
\pgfpathlineto{\pgfqpoint{4.031515in}{1.027552in}}%
\pgfpathlineto{\pgfqpoint{4.077062in}{1.015131in}}%
\pgfpathlineto{\pgfqpoint{4.122608in}{1.003045in}}%
\pgfpathlineto{\pgfqpoint{4.168154in}{0.991285in}}%
\pgfpathlineto{\pgfqpoint{4.213701in}{0.979843in}}%
\pgfpathlineto{\pgfqpoint{4.259247in}{0.968709in}}%
\pgfpathlineto{\pgfqpoint{4.304793in}{0.957876in}}%
\pgfpathlineto{\pgfqpoint{4.350340in}{0.947336in}}%
\pgfpathlineto{\pgfqpoint{4.395886in}{0.937079in}}%
\pgfpathlineto{\pgfqpoint{4.441433in}{0.927100in}}%
\pgfpathlineto{\pgfqpoint{4.486979in}{0.917389in}}%
\pgfpathlineto{\pgfqpoint{4.532525in}{0.907941in}}%
\pgfpathlineto{\pgfqpoint{4.578072in}{0.898748in}}%
\pgfpathlineto{\pgfqpoint{4.623618in}{0.889803in}}%
\pgfpathlineto{\pgfqpoint{4.669164in}{0.881099in}}%
\pgfpathlineto{\pgfqpoint{4.714711in}{0.872630in}}%
\pgfpathlineto{\pgfqpoint{4.760257in}{0.864390in}}%
\pgfpathlineto{\pgfqpoint{4.805803in}{0.856372in}}%
\pgfpathlineto{\pgfqpoint{4.851350in}{0.848570in}}%
\pgfpathlineto{\pgfqpoint{4.896896in}{0.840979in}}%
\pgfpathlineto{\pgfqpoint{4.942443in}{0.833593in}}%
\pgfpathlineto{\pgfqpoint{4.987989in}{0.826406in}}%
\pgfpathlineto{\pgfqpoint{5.033535in}{0.819413in}}%
\pgfpathlineto{\pgfqpoint{5.079082in}{0.812609in}}%
\pgfpathlineto{\pgfqpoint{5.124628in}{0.805988in}}%
\pgfpathlineto{\pgfqpoint{5.170174in}{0.799546in}}%
\pgfpathlineto{\pgfqpoint{5.215721in}{0.793278in}}%
\pgfpathlineto{\pgfqpoint{5.261267in}{0.787179in}}%
\pgfpathlineto{\pgfqpoint{5.306814in}{0.781245in}}%
\pgfpathlineto{\pgfqpoint{5.352360in}{0.775471in}}%
\pgfpathlineto{\pgfqpoint{5.397906in}{0.769852in}}%
\pgfpathlineto{\pgfqpoint{5.443453in}{0.764386in}}%
\pgfpathlineto{\pgfqpoint{5.488999in}{0.759066in}}%
\pgfpathlineto{\pgfqpoint{5.534545in}{0.753891in}}%
\pgfusepath{stroke}%
\end{pgfscope}%
\begin{pgfscope}%
\pgfpathrectangle{\pgfqpoint{0.800000in}{0.440000in}}{\pgfqpoint{4.960000in}{3.080000in}}%
\pgfusepath{clip}%
\pgfsetbuttcap%
\pgfsetroundjoin%
\definecolor{currentfill}{rgb}{0.000000,0.000000,1.000000}%
\pgfsetfillcolor{currentfill}%
\pgfsetlinewidth{1.003750pt}%
\definecolor{currentstroke}{rgb}{0.000000,0.000000,1.000000}%
\pgfsetstrokecolor{currentstroke}%
\pgfsetdash{}{0pt}%
\pgfsys@defobject{currentmarker}{\pgfqpoint{-0.041667in}{-0.000000in}}{\pgfqpoint{0.041667in}{0.000000in}}{%
\pgfpathmoveto{\pgfqpoint{0.041667in}{-0.000000in}}%
\pgfpathlineto{\pgfqpoint{-0.041667in}{0.000000in}}%
\pgfusepath{stroke,fill}%
}%
\begin{pgfscope}%
\pgfsys@transformshift{1.025455in}{3.354545in}%
\pgfsys@useobject{currentmarker}{}%
\end{pgfscope}%
\begin{pgfscope}%
\pgfsys@transformshift{1.212954in}{2.985455in}%
\pgfsys@useobject{currentmarker}{}%
\end{pgfscope}%
\begin{pgfscope}%
\pgfsys@transformshift{1.400454in}{2.552727in}%
\pgfsys@useobject{currentmarker}{}%
\end{pgfscope}%
\begin{pgfscope}%
\pgfsys@transformshift{1.775452in}{1.941818in}%
\pgfsys@useobject{currentmarker}{}%
\end{pgfscope}%
\begin{pgfscope}%
\pgfsys@transformshift{2.150451in}{1.483636in}%
\pgfsys@useobject{currentmarker}{}%
\end{pgfscope}%
\begin{pgfscope}%
\pgfsys@transformshift{2.525450in}{1.152727in}%
\pgfsys@useobject{currentmarker}{}%
\end{pgfscope}%
\begin{pgfscope}%
\pgfsys@transformshift{2.900449in}{0.898182in}%
\pgfsys@useobject{currentmarker}{}%
\end{pgfscope}%
\begin{pgfscope}%
\pgfsys@transformshift{3.275448in}{0.770909in}%
\pgfsys@useobject{currentmarker}{}%
\end{pgfscope}%
\begin{pgfscope}%
\pgfsys@transformshift{3.650447in}{0.669091in}%
\pgfsys@useobject{currentmarker}{}%
\end{pgfscope}%
\end{pgfscope}%
\begin{pgfscope}%
\pgfpathrectangle{\pgfqpoint{0.800000in}{0.440000in}}{\pgfqpoint{4.960000in}{3.080000in}}%
\pgfusepath{clip}%
\pgfsetbuttcap%
\pgfsetroundjoin%
\definecolor{currentfill}{rgb}{0.000000,0.000000,1.000000}%
\pgfsetfillcolor{currentfill}%
\pgfsetlinewidth{1.003750pt}%
\definecolor{currentstroke}{rgb}{0.000000,0.000000,1.000000}%
\pgfsetstrokecolor{currentstroke}%
\pgfsetdash{}{0pt}%
\pgfsys@defobject{currentmarker}{\pgfqpoint{-0.041667in}{-0.000000in}}{\pgfqpoint{0.041667in}{0.000000in}}{%
\pgfpathmoveto{\pgfqpoint{0.041667in}{-0.000000in}}%
\pgfpathlineto{\pgfqpoint{-0.041667in}{0.000000in}}%
\pgfusepath{stroke,fill}%
}%
\begin{pgfscope}%
\pgfsys@transformshift{1.025455in}{3.380000in}%
\pgfsys@useobject{currentmarker}{}%
\end{pgfscope}%
\begin{pgfscope}%
\pgfsys@transformshift{1.212954in}{3.087273in}%
\pgfsys@useobject{currentmarker}{}%
\end{pgfscope}%
\begin{pgfscope}%
\pgfsys@transformshift{1.400454in}{2.654545in}%
\pgfsys@useobject{currentmarker}{}%
\end{pgfscope}%
\begin{pgfscope}%
\pgfsys@transformshift{1.775452in}{2.043636in}%
\pgfsys@useobject{currentmarker}{}%
\end{pgfscope}%
\begin{pgfscope}%
\pgfsys@transformshift{2.150451in}{1.585455in}%
\pgfsys@useobject{currentmarker}{}%
\end{pgfscope}%
\begin{pgfscope}%
\pgfsys@transformshift{2.525450in}{1.254545in}%
\pgfsys@useobject{currentmarker}{}%
\end{pgfscope}%
\begin{pgfscope}%
\pgfsys@transformshift{2.900449in}{1.000000in}%
\pgfsys@useobject{currentmarker}{}%
\end{pgfscope}%
\begin{pgfscope}%
\pgfsys@transformshift{3.275448in}{0.872727in}%
\pgfsys@useobject{currentmarker}{}%
\end{pgfscope}%
\begin{pgfscope}%
\pgfsys@transformshift{3.650447in}{0.770909in}%
\pgfsys@useobject{currentmarker}{}%
\end{pgfscope}%
\end{pgfscope}%
\begin{pgfscope}%
\pgfpathrectangle{\pgfqpoint{0.800000in}{0.440000in}}{\pgfqpoint{4.960000in}{3.080000in}}%
\pgfusepath{clip}%
\pgfsetbuttcap%
\pgfsetroundjoin%
\pgfsetlinewidth{1.003750pt}%
\definecolor{currentstroke}{rgb}{0.000000,0.000000,1.000000}%
\pgfsetstrokecolor{currentstroke}%
\pgfsetdash{{3.700000pt}{1.600000pt}}{0.000000pt}%
\pgfpathmoveto{\pgfqpoint{1.025455in}{3.367273in}}%
\pgfpathlineto{\pgfqpoint{1.051970in}{3.297018in}}%
\pgfpathlineto{\pgfqpoint{1.078485in}{3.228525in}}%
\pgfpathlineto{\pgfqpoint{1.105000in}{3.161751in}}%
\pgfpathlineto{\pgfqpoint{1.131515in}{3.096653in}}%
\pgfpathlineto{\pgfqpoint{1.158030in}{3.033188in}}%
\pgfpathlineto{\pgfqpoint{1.184545in}{2.971316in}}%
\pgfpathlineto{\pgfqpoint{1.211060in}{2.910995in}}%
\pgfpathlineto{\pgfqpoint{1.237575in}{2.852189in}}%
\pgfpathlineto{\pgfqpoint{1.264090in}{2.794858in}}%
\pgfpathlineto{\pgfqpoint{1.290605in}{2.738965in}}%
\pgfpathlineto{\pgfqpoint{1.317120in}{2.684475in}}%
\pgfpathlineto{\pgfqpoint{1.343635in}{2.631352in}}%
\pgfpathlineto{\pgfqpoint{1.370151in}{2.579562in}}%
\pgfpathlineto{\pgfqpoint{1.396666in}{2.529071in}}%
\pgfpathlineto{\pgfqpoint{1.423181in}{2.479848in}}%
\pgfpathlineto{\pgfqpoint{1.449696in}{2.431859in}}%
\pgfpathlineto{\pgfqpoint{1.476211in}{2.385075in}}%
\pgfpathlineto{\pgfqpoint{1.502726in}{2.339464in}}%
\pgfpathlineto{\pgfqpoint{1.529241in}{2.294998in}}%
\pgfpathlineto{\pgfqpoint{1.555756in}{2.251647in}}%
\pgfpathlineto{\pgfqpoint{1.582271in}{2.209384in}}%
\pgfpathlineto{\pgfqpoint{1.608786in}{2.168182in}}%
\pgfpathlineto{\pgfqpoint{1.635301in}{2.128013in}}%
\pgfpathlineto{\pgfqpoint{1.661816in}{2.088853in}}%
\pgfpathlineto{\pgfqpoint{1.688332in}{2.050674in}}%
\pgfpathlineto{\pgfqpoint{1.714847in}{2.013454in}}%
\pgfpathlineto{\pgfqpoint{1.741362in}{1.977168in}}%
\pgfpathlineto{\pgfqpoint{1.767877in}{1.941792in}}%
\pgfpathlineto{\pgfqpoint{1.794392in}{1.907304in}}%
\pgfpathlineto{\pgfqpoint{1.820907in}{1.873681in}}%
\pgfpathlineto{\pgfqpoint{1.847422in}{1.840902in}}%
\pgfpathlineto{\pgfqpoint{1.873937in}{1.808945in}}%
\pgfpathlineto{\pgfqpoint{1.900452in}{1.777790in}}%
\pgfpathlineto{\pgfqpoint{1.926967in}{1.747417in}}%
\pgfpathlineto{\pgfqpoint{1.953482in}{1.717806in}}%
\pgfpathlineto{\pgfqpoint{1.979997in}{1.688937in}}%
\pgfpathlineto{\pgfqpoint{2.006512in}{1.660794in}}%
\pgfpathlineto{\pgfqpoint{2.033028in}{1.633356in}}%
\pgfpathlineto{\pgfqpoint{2.059543in}{1.606607in}}%
\pgfpathlineto{\pgfqpoint{2.086058in}{1.580529in}}%
\pgfpathlineto{\pgfqpoint{2.112573in}{1.555105in}}%
\pgfpathlineto{\pgfqpoint{2.139088in}{1.530319in}}%
\pgfpathlineto{\pgfqpoint{2.165603in}{1.506155in}}%
\pgfpathlineto{\pgfqpoint{2.192118in}{1.482598in}}%
\pgfpathlineto{\pgfqpoint{2.218633in}{1.459631in}}%
\pgfpathlineto{\pgfqpoint{2.245148in}{1.437241in}}%
\pgfpathlineto{\pgfqpoint{2.271663in}{1.415412in}}%
\pgfpathlineto{\pgfqpoint{2.298178in}{1.394132in}}%
\pgfpathlineto{\pgfqpoint{2.324693in}{1.373385in}}%
\pgfpathlineto{\pgfqpoint{2.351208in}{1.353159in}}%
\pgfpathlineto{\pgfqpoint{2.377724in}{1.333440in}}%
\pgfpathlineto{\pgfqpoint{2.404239in}{1.314216in}}%
\pgfpathlineto{\pgfqpoint{2.430754in}{1.295474in}}%
\pgfpathlineto{\pgfqpoint{2.457269in}{1.277203in}}%
\pgfpathlineto{\pgfqpoint{2.483784in}{1.259390in}}%
\pgfpathlineto{\pgfqpoint{2.510299in}{1.242024in}}%
\pgfpathlineto{\pgfqpoint{2.536814in}{1.225094in}}%
\pgfpathlineto{\pgfqpoint{2.563329in}{1.208588in}}%
\pgfpathlineto{\pgfqpoint{2.589844in}{1.192497in}}%
\pgfpathlineto{\pgfqpoint{2.616359in}{1.176809in}}%
\pgfpathlineto{\pgfqpoint{2.642874in}{1.161515in}}%
\pgfpathlineto{\pgfqpoint{2.669389in}{1.146605in}}%
\pgfpathlineto{\pgfqpoint{2.695905in}{1.132069in}}%
\pgfpathlineto{\pgfqpoint{2.722420in}{1.117898in}}%
\pgfpathlineto{\pgfqpoint{2.748935in}{1.104082in}}%
\pgfpathlineto{\pgfqpoint{2.775450in}{1.090613in}}%
\pgfpathlineto{\pgfqpoint{2.801965in}{1.077482in}}%
\pgfpathlineto{\pgfqpoint{2.828480in}{1.064680in}}%
\pgfpathlineto{\pgfqpoint{2.854995in}{1.052199in}}%
\pgfpathlineto{\pgfqpoint{2.881510in}{1.040032in}}%
\pgfpathlineto{\pgfqpoint{2.908025in}{1.028170in}}%
\pgfpathlineto{\pgfqpoint{2.934540in}{1.016606in}}%
\pgfpathlineto{\pgfqpoint{2.961055in}{1.005331in}}%
\pgfpathlineto{\pgfqpoint{2.987570in}{0.994340in}}%
\pgfpathlineto{\pgfqpoint{3.014085in}{0.983624in}}%
\pgfpathlineto{\pgfqpoint{3.040601in}{0.973178in}}%
\pgfpathlineto{\pgfqpoint{3.067116in}{0.962993in}}%
\pgfpathlineto{\pgfqpoint{3.093631in}{0.953064in}}%
\pgfpathlineto{\pgfqpoint{3.120146in}{0.943384in}}%
\pgfpathlineto{\pgfqpoint{3.146661in}{0.933947in}}%
\pgfpathlineto{\pgfqpoint{3.173176in}{0.924747in}}%
\pgfpathlineto{\pgfqpoint{3.199691in}{0.915777in}}%
\pgfpathlineto{\pgfqpoint{3.226206in}{0.907033in}}%
\pgfpathlineto{\pgfqpoint{3.252721in}{0.898508in}}%
\pgfpathlineto{\pgfqpoint{3.279236in}{0.890197in}}%
\pgfpathlineto{\pgfqpoint{3.305751in}{0.882094in}}%
\pgfpathlineto{\pgfqpoint{3.332266in}{0.874195in}}%
\pgfpathlineto{\pgfqpoint{3.358781in}{0.866494in}}%
\pgfpathlineto{\pgfqpoint{3.385297in}{0.858986in}}%
\pgfpathlineto{\pgfqpoint{3.411812in}{0.851667in}}%
\pgfpathlineto{\pgfqpoint{3.438327in}{0.844531in}}%
\pgfpathlineto{\pgfqpoint{3.464842in}{0.837574in}}%
\pgfpathlineto{\pgfqpoint{3.491357in}{0.830792in}}%
\pgfpathlineto{\pgfqpoint{3.517872in}{0.824180in}}%
\pgfpathlineto{\pgfqpoint{3.544387in}{0.817734in}}%
\pgfpathlineto{\pgfqpoint{3.570902in}{0.811450in}}%
\pgfpathlineto{\pgfqpoint{3.597417in}{0.805323in}}%
\pgfpathlineto{\pgfqpoint{3.623932in}{0.799350in}}%
\pgfpathlineto{\pgfqpoint{3.650447in}{0.793527in}}%
\pgfusepath{stroke}%
\end{pgfscope}%
\begin{pgfscope}%
\pgfpathrectangle{\pgfqpoint{0.800000in}{0.440000in}}{\pgfqpoint{4.960000in}{3.080000in}}%
\pgfusepath{clip}%
\pgfsetbuttcap%
\pgfsetroundjoin%
\definecolor{currentfill}{rgb}{0.000000,0.501961,0.000000}%
\pgfsetfillcolor{currentfill}%
\pgfsetlinewidth{1.003750pt}%
\definecolor{currentstroke}{rgb}{0.000000,0.501961,0.000000}%
\pgfsetstrokecolor{currentstroke}%
\pgfsetdash{}{0pt}%
\pgfsys@defobject{currentmarker}{\pgfqpoint{-0.041667in}{-0.000000in}}{\pgfqpoint{0.041667in}{0.000000in}}{%
\pgfpathmoveto{\pgfqpoint{0.041667in}{-0.000000in}}%
\pgfpathlineto{\pgfqpoint{-0.041667in}{0.000000in}}%
\pgfusepath{stroke,fill}%
}%
\begin{pgfscope}%
\pgfsys@transformshift{1.025455in}{3.354545in}%
\pgfsys@useobject{currentmarker}{}%
\end{pgfscope}%
\begin{pgfscope}%
\pgfsys@transformshift{1.215315in}{2.680000in}%
\pgfsys@useobject{currentmarker}{}%
\end{pgfscope}%
\begin{pgfscope}%
\pgfsys@transformshift{1.405175in}{2.145455in}%
\pgfsys@useobject{currentmarker}{}%
\end{pgfscope}%
\begin{pgfscope}%
\pgfsys@transformshift{1.595035in}{1.738182in}%
\pgfsys@useobject{currentmarker}{}%
\end{pgfscope}%
\begin{pgfscope}%
\pgfsys@transformshift{1.784895in}{1.432727in}%
\pgfsys@useobject{currentmarker}{}%
\end{pgfscope}%
\begin{pgfscope}%
\pgfsys@transformshift{1.974756in}{1.178182in}%
\pgfsys@useobject{currentmarker}{}%
\end{pgfscope}%
\begin{pgfscope}%
\pgfsys@transformshift{2.164616in}{1.000000in}%
\pgfsys@useobject{currentmarker}{}%
\end{pgfscope}%
\begin{pgfscope}%
\pgfsys@transformshift{2.354476in}{0.872727in}%
\pgfsys@useobject{currentmarker}{}%
\end{pgfscope}%
\begin{pgfscope}%
\pgfsys@transformshift{2.544336in}{0.770909in}%
\pgfsys@useobject{currentmarker}{}%
\end{pgfscope}%
\end{pgfscope}%
\begin{pgfscope}%
\pgfpathrectangle{\pgfqpoint{0.800000in}{0.440000in}}{\pgfqpoint{4.960000in}{3.080000in}}%
\pgfusepath{clip}%
\pgfsetbuttcap%
\pgfsetroundjoin%
\definecolor{currentfill}{rgb}{0.000000,0.501961,0.000000}%
\pgfsetfillcolor{currentfill}%
\pgfsetlinewidth{1.003750pt}%
\definecolor{currentstroke}{rgb}{0.000000,0.501961,0.000000}%
\pgfsetstrokecolor{currentstroke}%
\pgfsetdash{}{0pt}%
\pgfsys@defobject{currentmarker}{\pgfqpoint{-0.041667in}{-0.000000in}}{\pgfqpoint{0.041667in}{0.000000in}}{%
\pgfpathmoveto{\pgfqpoint{0.041667in}{-0.000000in}}%
\pgfpathlineto{\pgfqpoint{-0.041667in}{0.000000in}}%
\pgfusepath{stroke,fill}%
}%
\begin{pgfscope}%
\pgfsys@transformshift{1.025455in}{3.380000in}%
\pgfsys@useobject{currentmarker}{}%
\end{pgfscope}%
\begin{pgfscope}%
\pgfsys@transformshift{1.215315in}{2.781818in}%
\pgfsys@useobject{currentmarker}{}%
\end{pgfscope}%
\begin{pgfscope}%
\pgfsys@transformshift{1.405175in}{2.247273in}%
\pgfsys@useobject{currentmarker}{}%
\end{pgfscope}%
\begin{pgfscope}%
\pgfsys@transformshift{1.595035in}{1.840000in}%
\pgfsys@useobject{currentmarker}{}%
\end{pgfscope}%
\begin{pgfscope}%
\pgfsys@transformshift{1.784895in}{1.534545in}%
\pgfsys@useobject{currentmarker}{}%
\end{pgfscope}%
\begin{pgfscope}%
\pgfsys@transformshift{1.974756in}{1.280000in}%
\pgfsys@useobject{currentmarker}{}%
\end{pgfscope}%
\begin{pgfscope}%
\pgfsys@transformshift{2.164616in}{1.101818in}%
\pgfsys@useobject{currentmarker}{}%
\end{pgfscope}%
\begin{pgfscope}%
\pgfsys@transformshift{2.354476in}{0.974545in}%
\pgfsys@useobject{currentmarker}{}%
\end{pgfscope}%
\begin{pgfscope}%
\pgfsys@transformshift{2.544336in}{0.872727in}%
\pgfsys@useobject{currentmarker}{}%
\end{pgfscope}%
\end{pgfscope}%
\begin{pgfscope}%
\pgfpathrectangle{\pgfqpoint{0.800000in}{0.440000in}}{\pgfqpoint{4.960000in}{3.080000in}}%
\pgfusepath{clip}%
\pgfsetbuttcap%
\pgfsetroundjoin%
\pgfsetlinewidth{1.003750pt}%
\definecolor{currentstroke}{rgb}{0.000000,0.501961,0.000000}%
\pgfsetstrokecolor{currentstroke}%
\pgfsetdash{{3.700000pt}{1.600000pt}}{0.000000pt}%
\pgfpathmoveto{\pgfqpoint{1.025455in}{3.367273in}}%
\pgfpathlineto{\pgfqpoint{1.040797in}{3.304132in}}%
\pgfpathlineto{\pgfqpoint{1.056139in}{3.242415in}}%
\pgfpathlineto{\pgfqpoint{1.071481in}{3.182089in}}%
\pgfpathlineto{\pgfqpoint{1.086824in}{3.123124in}}%
\pgfpathlineto{\pgfqpoint{1.102166in}{3.065489in}}%
\pgfpathlineto{\pgfqpoint{1.117508in}{3.009153in}}%
\pgfpathlineto{\pgfqpoint{1.132850in}{2.954088in}}%
\pgfpathlineto{\pgfqpoint{1.148192in}{2.900264in}}%
\pgfpathlineto{\pgfqpoint{1.163535in}{2.847654in}}%
\pgfpathlineto{\pgfqpoint{1.178877in}{2.796231in}}%
\pgfpathlineto{\pgfqpoint{1.194219in}{2.745967in}}%
\pgfpathlineto{\pgfqpoint{1.209561in}{2.696837in}}%
\pgfpathlineto{\pgfqpoint{1.224904in}{2.648814in}}%
\pgfpathlineto{\pgfqpoint{1.240246in}{2.601875in}}%
\pgfpathlineto{\pgfqpoint{1.255588in}{2.555994in}}%
\pgfpathlineto{\pgfqpoint{1.270930in}{2.511147in}}%
\pgfpathlineto{\pgfqpoint{1.286273in}{2.467312in}}%
\pgfpathlineto{\pgfqpoint{1.301615in}{2.424466in}}%
\pgfpathlineto{\pgfqpoint{1.316957in}{2.382585in}}%
\pgfpathlineto{\pgfqpoint{1.332299in}{2.341649in}}%
\pgfpathlineto{\pgfqpoint{1.347642in}{2.301636in}}%
\pgfpathlineto{\pgfqpoint{1.362984in}{2.262526in}}%
\pgfpathlineto{\pgfqpoint{1.378326in}{2.224297in}}%
\pgfpathlineto{\pgfqpoint{1.393668in}{2.186931in}}%
\pgfpathlineto{\pgfqpoint{1.409011in}{2.150407in}}%
\pgfpathlineto{\pgfqpoint{1.424353in}{2.114707in}}%
\pgfpathlineto{\pgfqpoint{1.439695in}{2.079812in}}%
\pgfpathlineto{\pgfqpoint{1.455037in}{2.045703in}}%
\pgfpathlineto{\pgfqpoint{1.470379in}{2.012364in}}%
\pgfpathlineto{\pgfqpoint{1.485722in}{1.979777in}}%
\pgfpathlineto{\pgfqpoint{1.501064in}{1.947924in}}%
\pgfpathlineto{\pgfqpoint{1.516406in}{1.916790in}}%
\pgfpathlineto{\pgfqpoint{1.531748in}{1.886358in}}%
\pgfpathlineto{\pgfqpoint{1.547091in}{1.856612in}}%
\pgfpathlineto{\pgfqpoint{1.562433in}{1.827537in}}%
\pgfpathlineto{\pgfqpoint{1.577775in}{1.799118in}}%
\pgfpathlineto{\pgfqpoint{1.593117in}{1.771339in}}%
\pgfpathlineto{\pgfqpoint{1.608460in}{1.744187in}}%
\pgfpathlineto{\pgfqpoint{1.623802in}{1.717647in}}%
\pgfpathlineto{\pgfqpoint{1.639144in}{1.691706in}}%
\pgfpathlineto{\pgfqpoint{1.654486in}{1.666349in}}%
\pgfpathlineto{\pgfqpoint{1.669829in}{1.641565in}}%
\pgfpathlineto{\pgfqpoint{1.685171in}{1.617339in}}%
\pgfpathlineto{\pgfqpoint{1.700513in}{1.593660in}}%
\pgfpathlineto{\pgfqpoint{1.715855in}{1.570514in}}%
\pgfpathlineto{\pgfqpoint{1.731198in}{1.547891in}}%
\pgfpathlineto{\pgfqpoint{1.746540in}{1.525778in}}%
\pgfpathlineto{\pgfqpoint{1.761882in}{1.504163in}}%
\pgfpathlineto{\pgfqpoint{1.777224in}{1.483036in}}%
\pgfpathlineto{\pgfqpoint{1.792567in}{1.462385in}}%
\pgfpathlineto{\pgfqpoint{1.807909in}{1.442200in}}%
\pgfpathlineto{\pgfqpoint{1.823251in}{1.422470in}}%
\pgfpathlineto{\pgfqpoint{1.838593in}{1.403185in}}%
\pgfpathlineto{\pgfqpoint{1.853935in}{1.384335in}}%
\pgfpathlineto{\pgfqpoint{1.869278in}{1.365910in}}%
\pgfpathlineto{\pgfqpoint{1.884620in}{1.347900in}}%
\pgfpathlineto{\pgfqpoint{1.899962in}{1.330297in}}%
\pgfpathlineto{\pgfqpoint{1.915304in}{1.313090in}}%
\pgfpathlineto{\pgfqpoint{1.930647in}{1.296272in}}%
\pgfpathlineto{\pgfqpoint{1.945989in}{1.279833in}}%
\pgfpathlineto{\pgfqpoint{1.961331in}{1.263764in}}%
\pgfpathlineto{\pgfqpoint{1.976673in}{1.248058in}}%
\pgfpathlineto{\pgfqpoint{1.992016in}{1.232706in}}%
\pgfpathlineto{\pgfqpoint{2.007358in}{1.217700in}}%
\pgfpathlineto{\pgfqpoint{2.022700in}{1.203033in}}%
\pgfpathlineto{\pgfqpoint{2.038042in}{1.188696in}}%
\pgfpathlineto{\pgfqpoint{2.053385in}{1.174683in}}%
\pgfpathlineto{\pgfqpoint{2.068727in}{1.160986in}}%
\pgfpathlineto{\pgfqpoint{2.084069in}{1.147597in}}%
\pgfpathlineto{\pgfqpoint{2.099411in}{1.134511in}}%
\pgfpathlineto{\pgfqpoint{2.114754in}{1.121719in}}%
\pgfpathlineto{\pgfqpoint{2.130096in}{1.109216in}}%
\pgfpathlineto{\pgfqpoint{2.145438in}{1.096995in}}%
\pgfpathlineto{\pgfqpoint{2.160780in}{1.085050in}}%
\pgfpathlineto{\pgfqpoint{2.176123in}{1.073374in}}%
\pgfpathlineto{\pgfqpoint{2.191465in}{1.061961in}}%
\pgfpathlineto{\pgfqpoint{2.206807in}{1.050806in}}%
\pgfpathlineto{\pgfqpoint{2.222149in}{1.039902in}}%
\pgfpathlineto{\pgfqpoint{2.237491in}{1.029244in}}%
\pgfpathlineto{\pgfqpoint{2.252834in}{1.018826in}}%
\pgfpathlineto{\pgfqpoint{2.268176in}{1.008643in}}%
\pgfpathlineto{\pgfqpoint{2.283518in}{0.998690in}}%
\pgfpathlineto{\pgfqpoint{2.298860in}{0.988962in}}%
\pgfpathlineto{\pgfqpoint{2.314203in}{0.979452in}}%
\pgfpathlineto{\pgfqpoint{2.329545in}{0.970158in}}%
\pgfpathlineto{\pgfqpoint{2.344887in}{0.961072in}}%
\pgfpathlineto{\pgfqpoint{2.360229in}{0.952192in}}%
\pgfpathlineto{\pgfqpoint{2.375572in}{0.943512in}}%
\pgfpathlineto{\pgfqpoint{2.390914in}{0.935028in}}%
\pgfpathlineto{\pgfqpoint{2.406256in}{0.926735in}}%
\pgfpathlineto{\pgfqpoint{2.421598in}{0.918629in}}%
\pgfpathlineto{\pgfqpoint{2.436941in}{0.910706in}}%
\pgfpathlineto{\pgfqpoint{2.452283in}{0.902961in}}%
\pgfpathlineto{\pgfqpoint{2.467625in}{0.895391in}}%
\pgfpathlineto{\pgfqpoint{2.482967in}{0.887992in}}%
\pgfpathlineto{\pgfqpoint{2.498310in}{0.880760in}}%
\pgfpathlineto{\pgfqpoint{2.513652in}{0.873690in}}%
\pgfpathlineto{\pgfqpoint{2.528994in}{0.866780in}}%
\pgfpathlineto{\pgfqpoint{2.544336in}{0.860026in}}%
\pgfusepath{stroke}%
\end{pgfscope}%
\begin{pgfscope}%
\pgfpathrectangle{\pgfqpoint{0.800000in}{0.440000in}}{\pgfqpoint{4.960000in}{3.080000in}}%
\pgfusepath{clip}%
\pgfsetbuttcap%
\pgfsetroundjoin%
\definecolor{currentfill}{rgb}{1.000000,0.000000,0.000000}%
\pgfsetfillcolor{currentfill}%
\pgfsetlinewidth{1.003750pt}%
\definecolor{currentstroke}{rgb}{1.000000,0.000000,0.000000}%
\pgfsetstrokecolor{currentstroke}%
\pgfsetdash{}{0pt}%
\pgfsys@defobject{currentmarker}{\pgfqpoint{-0.010417in}{-0.010417in}}{\pgfqpoint{0.010417in}{0.010417in}}{%
\pgfpathmoveto{\pgfqpoint{0.000000in}{-0.010417in}}%
\pgfpathcurveto{\pgfqpoint{0.002763in}{-0.010417in}}{\pgfqpoint{0.005412in}{-0.009319in}}{\pgfqpoint{0.007366in}{-0.007366in}}%
\pgfpathcurveto{\pgfqpoint{0.009319in}{-0.005412in}}{\pgfqpoint{0.010417in}{-0.002763in}}{\pgfqpoint{0.010417in}{0.000000in}}%
\pgfpathcurveto{\pgfqpoint{0.010417in}{0.002763in}}{\pgfqpoint{0.009319in}{0.005412in}}{\pgfqpoint{0.007366in}{0.007366in}}%
\pgfpathcurveto{\pgfqpoint{0.005412in}{0.009319in}}{\pgfqpoint{0.002763in}{0.010417in}}{\pgfqpoint{0.000000in}{0.010417in}}%
\pgfpathcurveto{\pgfqpoint{-0.002763in}{0.010417in}}{\pgfqpoint{-0.005412in}{0.009319in}}{\pgfqpoint{-0.007366in}{0.007366in}}%
\pgfpathcurveto{\pgfqpoint{-0.009319in}{0.005412in}}{\pgfqpoint{-0.010417in}{0.002763in}}{\pgfqpoint{-0.010417in}{0.000000in}}%
\pgfpathcurveto{\pgfqpoint{-0.010417in}{-0.002763in}}{\pgfqpoint{-0.009319in}{-0.005412in}}{\pgfqpoint{-0.007366in}{-0.007366in}}%
\pgfpathcurveto{\pgfqpoint{-0.005412in}{-0.009319in}}{\pgfqpoint{-0.002763in}{-0.010417in}}{\pgfqpoint{0.000000in}{-0.010417in}}%
\pgfpathclose%
\pgfusepath{stroke,fill}%
}%
\begin{pgfscope}%
\pgfsys@transformshift{1.025455in}{3.367273in}%
\pgfsys@useobject{currentmarker}{}%
\end{pgfscope}%
\begin{pgfscope}%
\pgfsys@transformshift{1.589091in}{2.705455in}%
\pgfsys@useobject{currentmarker}{}%
\end{pgfscope}%
\begin{pgfscope}%
\pgfsys@transformshift{2.152727in}{2.081818in}%
\pgfsys@useobject{currentmarker}{}%
\end{pgfscope}%
\begin{pgfscope}%
\pgfsys@transformshift{2.716364in}{1.610909in}%
\pgfsys@useobject{currentmarker}{}%
\end{pgfscope}%
\begin{pgfscope}%
\pgfsys@transformshift{3.280000in}{1.280000in}%
\pgfsys@useobject{currentmarker}{}%
\end{pgfscope}%
\begin{pgfscope}%
\pgfsys@transformshift{3.843636in}{1.050909in}%
\pgfsys@useobject{currentmarker}{}%
\end{pgfscope}%
\begin{pgfscope}%
\pgfsys@transformshift{4.407273in}{0.860000in}%
\pgfsys@useobject{currentmarker}{}%
\end{pgfscope}%
\begin{pgfscope}%
\pgfsys@transformshift{4.970909in}{0.732727in}%
\pgfsys@useobject{currentmarker}{}%
\end{pgfscope}%
\begin{pgfscope}%
\pgfsys@transformshift{5.534545in}{0.630909in}%
\pgfsys@useobject{currentmarker}{}%
\end{pgfscope}%
\end{pgfscope}%
\begin{pgfscope}%
\pgfpathrectangle{\pgfqpoint{0.800000in}{0.440000in}}{\pgfqpoint{4.960000in}{3.080000in}}%
\pgfusepath{clip}%
\pgfsetbuttcap%
\pgfsetroundjoin%
\definecolor{currentfill}{rgb}{0.000000,0.000000,1.000000}%
\pgfsetfillcolor{currentfill}%
\pgfsetlinewidth{1.003750pt}%
\definecolor{currentstroke}{rgb}{0.000000,0.000000,1.000000}%
\pgfsetstrokecolor{currentstroke}%
\pgfsetdash{}{0pt}%
\pgfsys@defobject{currentmarker}{\pgfqpoint{-0.010417in}{-0.010417in}}{\pgfqpoint{0.010417in}{0.010417in}}{%
\pgfpathmoveto{\pgfqpoint{0.000000in}{-0.010417in}}%
\pgfpathcurveto{\pgfqpoint{0.002763in}{-0.010417in}}{\pgfqpoint{0.005412in}{-0.009319in}}{\pgfqpoint{0.007366in}{-0.007366in}}%
\pgfpathcurveto{\pgfqpoint{0.009319in}{-0.005412in}}{\pgfqpoint{0.010417in}{-0.002763in}}{\pgfqpoint{0.010417in}{0.000000in}}%
\pgfpathcurveto{\pgfqpoint{0.010417in}{0.002763in}}{\pgfqpoint{0.009319in}{0.005412in}}{\pgfqpoint{0.007366in}{0.007366in}}%
\pgfpathcurveto{\pgfqpoint{0.005412in}{0.009319in}}{\pgfqpoint{0.002763in}{0.010417in}}{\pgfqpoint{0.000000in}{0.010417in}}%
\pgfpathcurveto{\pgfqpoint{-0.002763in}{0.010417in}}{\pgfqpoint{-0.005412in}{0.009319in}}{\pgfqpoint{-0.007366in}{0.007366in}}%
\pgfpathcurveto{\pgfqpoint{-0.009319in}{0.005412in}}{\pgfqpoint{-0.010417in}{0.002763in}}{\pgfqpoint{-0.010417in}{0.000000in}}%
\pgfpathcurveto{\pgfqpoint{-0.010417in}{-0.002763in}}{\pgfqpoint{-0.009319in}{-0.005412in}}{\pgfqpoint{-0.007366in}{-0.007366in}}%
\pgfpathcurveto{\pgfqpoint{-0.005412in}{-0.009319in}}{\pgfqpoint{-0.002763in}{-0.010417in}}{\pgfqpoint{0.000000in}{-0.010417in}}%
\pgfpathclose%
\pgfusepath{stroke,fill}%
}%
\begin{pgfscope}%
\pgfsys@transformshift{1.025455in}{3.367273in}%
\pgfsys@useobject{currentmarker}{}%
\end{pgfscope}%
\begin{pgfscope}%
\pgfsys@transformshift{1.212954in}{3.036364in}%
\pgfsys@useobject{currentmarker}{}%
\end{pgfscope}%
\begin{pgfscope}%
\pgfsys@transformshift{1.400454in}{2.603636in}%
\pgfsys@useobject{currentmarker}{}%
\end{pgfscope}%
\begin{pgfscope}%
\pgfsys@transformshift{1.775452in}{1.992727in}%
\pgfsys@useobject{currentmarker}{}%
\end{pgfscope}%
\begin{pgfscope}%
\pgfsys@transformshift{2.150451in}{1.534545in}%
\pgfsys@useobject{currentmarker}{}%
\end{pgfscope}%
\begin{pgfscope}%
\pgfsys@transformshift{2.525450in}{1.203636in}%
\pgfsys@useobject{currentmarker}{}%
\end{pgfscope}%
\begin{pgfscope}%
\pgfsys@transformshift{2.900449in}{0.949091in}%
\pgfsys@useobject{currentmarker}{}%
\end{pgfscope}%
\begin{pgfscope}%
\pgfsys@transformshift{3.275448in}{0.821818in}%
\pgfsys@useobject{currentmarker}{}%
\end{pgfscope}%
\begin{pgfscope}%
\pgfsys@transformshift{3.650447in}{0.720000in}%
\pgfsys@useobject{currentmarker}{}%
\end{pgfscope}%
\end{pgfscope}%
\begin{pgfscope}%
\pgfpathrectangle{\pgfqpoint{0.800000in}{0.440000in}}{\pgfqpoint{4.960000in}{3.080000in}}%
\pgfusepath{clip}%
\pgfsetbuttcap%
\pgfsetroundjoin%
\definecolor{currentfill}{rgb}{0.000000,0.501961,0.000000}%
\pgfsetfillcolor{currentfill}%
\pgfsetlinewidth{1.003750pt}%
\definecolor{currentstroke}{rgb}{0.000000,0.501961,0.000000}%
\pgfsetstrokecolor{currentstroke}%
\pgfsetdash{}{0pt}%
\pgfsys@defobject{currentmarker}{\pgfqpoint{-0.010417in}{-0.010417in}}{\pgfqpoint{0.010417in}{0.010417in}}{%
\pgfpathmoveto{\pgfqpoint{0.000000in}{-0.010417in}}%
\pgfpathcurveto{\pgfqpoint{0.002763in}{-0.010417in}}{\pgfqpoint{0.005412in}{-0.009319in}}{\pgfqpoint{0.007366in}{-0.007366in}}%
\pgfpathcurveto{\pgfqpoint{0.009319in}{-0.005412in}}{\pgfqpoint{0.010417in}{-0.002763in}}{\pgfqpoint{0.010417in}{0.000000in}}%
\pgfpathcurveto{\pgfqpoint{0.010417in}{0.002763in}}{\pgfqpoint{0.009319in}{0.005412in}}{\pgfqpoint{0.007366in}{0.007366in}}%
\pgfpathcurveto{\pgfqpoint{0.005412in}{0.009319in}}{\pgfqpoint{0.002763in}{0.010417in}}{\pgfqpoint{0.000000in}{0.010417in}}%
\pgfpathcurveto{\pgfqpoint{-0.002763in}{0.010417in}}{\pgfqpoint{-0.005412in}{0.009319in}}{\pgfqpoint{-0.007366in}{0.007366in}}%
\pgfpathcurveto{\pgfqpoint{-0.009319in}{0.005412in}}{\pgfqpoint{-0.010417in}{0.002763in}}{\pgfqpoint{-0.010417in}{0.000000in}}%
\pgfpathcurveto{\pgfqpoint{-0.010417in}{-0.002763in}}{\pgfqpoint{-0.009319in}{-0.005412in}}{\pgfqpoint{-0.007366in}{-0.007366in}}%
\pgfpathcurveto{\pgfqpoint{-0.005412in}{-0.009319in}}{\pgfqpoint{-0.002763in}{-0.010417in}}{\pgfqpoint{0.000000in}{-0.010417in}}%
\pgfpathclose%
\pgfusepath{stroke,fill}%
}%
\begin{pgfscope}%
\pgfsys@transformshift{1.025455in}{3.367273in}%
\pgfsys@useobject{currentmarker}{}%
\end{pgfscope}%
\begin{pgfscope}%
\pgfsys@transformshift{1.215315in}{2.730909in}%
\pgfsys@useobject{currentmarker}{}%
\end{pgfscope}%
\begin{pgfscope}%
\pgfsys@transformshift{1.405175in}{2.196364in}%
\pgfsys@useobject{currentmarker}{}%
\end{pgfscope}%
\begin{pgfscope}%
\pgfsys@transformshift{1.595035in}{1.789091in}%
\pgfsys@useobject{currentmarker}{}%
\end{pgfscope}%
\begin{pgfscope}%
\pgfsys@transformshift{1.784895in}{1.483636in}%
\pgfsys@useobject{currentmarker}{}%
\end{pgfscope}%
\begin{pgfscope}%
\pgfsys@transformshift{1.974756in}{1.229091in}%
\pgfsys@useobject{currentmarker}{}%
\end{pgfscope}%
\begin{pgfscope}%
\pgfsys@transformshift{2.164616in}{1.050909in}%
\pgfsys@useobject{currentmarker}{}%
\end{pgfscope}%
\begin{pgfscope}%
\pgfsys@transformshift{2.354476in}{0.923636in}%
\pgfsys@useobject{currentmarker}{}%
\end{pgfscope}%
\begin{pgfscope}%
\pgfsys@transformshift{2.544336in}{0.821818in}%
\pgfsys@useobject{currentmarker}{}%
\end{pgfscope}%
\end{pgfscope}%
\begin{pgfscope}%
\pgfsetrectcap%
\pgfsetmiterjoin%
\pgfsetlinewidth{0.803000pt}%
\definecolor{currentstroke}{rgb}{0.000000,0.000000,0.000000}%
\pgfsetstrokecolor{currentstroke}%
\pgfsetdash{}{0pt}%
\pgfpathmoveto{\pgfqpoint{0.800000in}{0.440000in}}%
\pgfpathlineto{\pgfqpoint{0.800000in}{3.520000in}}%
\pgfusepath{stroke}%
\end{pgfscope}%
\begin{pgfscope}%
\pgfsetrectcap%
\pgfsetmiterjoin%
\pgfsetlinewidth{0.803000pt}%
\definecolor{currentstroke}{rgb}{0.000000,0.000000,0.000000}%
\pgfsetstrokecolor{currentstroke}%
\pgfsetdash{}{0pt}%
\pgfpathmoveto{\pgfqpoint{5.760000in}{0.440000in}}%
\pgfpathlineto{\pgfqpoint{5.760000in}{3.520000in}}%
\pgfusepath{stroke}%
\end{pgfscope}%
\begin{pgfscope}%
\pgfsetrectcap%
\pgfsetmiterjoin%
\pgfsetlinewidth{0.803000pt}%
\definecolor{currentstroke}{rgb}{0.000000,0.000000,0.000000}%
\pgfsetstrokecolor{currentstroke}%
\pgfsetdash{}{0pt}%
\pgfpathmoveto{\pgfqpoint{0.800000in}{0.440000in}}%
\pgfpathlineto{\pgfqpoint{5.760000in}{0.440000in}}%
\pgfusepath{stroke}%
\end{pgfscope}%
\begin{pgfscope}%
\pgfsetrectcap%
\pgfsetmiterjoin%
\pgfsetlinewidth{0.803000pt}%
\definecolor{currentstroke}{rgb}{0.000000,0.000000,0.000000}%
\pgfsetstrokecolor{currentstroke}%
\pgfsetdash{}{0pt}%
\pgfpathmoveto{\pgfqpoint{0.800000in}{3.520000in}}%
\pgfpathlineto{\pgfqpoint{5.760000in}{3.520000in}}%
\pgfusepath{stroke}%
\end{pgfscope}%
\begin{pgfscope}%
\pgfsetbuttcap%
\pgfsetmiterjoin%
\definecolor{currentfill}{rgb}{1.000000,1.000000,1.000000}%
\pgfsetfillcolor{currentfill}%
\pgfsetfillopacity{0.800000}%
\pgfsetlinewidth{1.003750pt}%
\definecolor{currentstroke}{rgb}{0.800000,0.800000,0.800000}%
\pgfsetstrokecolor{currentstroke}%
\pgfsetstrokeopacity{0.800000}%
\pgfsetdash{}{0pt}%
\pgfpathmoveto{\pgfqpoint{3.471694in}{1.990325in}}%
\pgfpathlineto{\pgfqpoint{5.682222in}{1.990325in}}%
\pgfpathquadraticcurveto{\pgfqpoint{5.704444in}{1.990325in}}{\pgfqpoint{5.704444in}{2.012547in}}%
\pgfpathlineto{\pgfqpoint{5.704444in}{3.442222in}}%
\pgfpathquadraticcurveto{\pgfqpoint{5.704444in}{3.464444in}}{\pgfqpoint{5.682222in}{3.464444in}}%
\pgfpathlineto{\pgfqpoint{3.471694in}{3.464444in}}%
\pgfpathquadraticcurveto{\pgfqpoint{3.449471in}{3.464444in}}{\pgfqpoint{3.449471in}{3.442222in}}%
\pgfpathlineto{\pgfqpoint{3.449471in}{2.012547in}}%
\pgfpathquadraticcurveto{\pgfqpoint{3.449471in}{1.990325in}}{\pgfqpoint{3.471694in}{1.990325in}}%
\pgfpathclose%
\pgfusepath{stroke,fill}%
\end{pgfscope}%
\begin{pgfscope}%
\pgfsetbuttcap%
\pgfsetroundjoin%
\pgfsetlinewidth{1.003750pt}%
\definecolor{currentstroke}{rgb}{1.000000,0.000000,0.000000}%
\pgfsetstrokecolor{currentstroke}%
\pgfsetdash{{3.700000pt}{1.600000pt}}{0.000000pt}%
\pgfpathmoveto{\pgfqpoint{3.493916in}{3.365664in}}%
\pgfpathlineto{\pgfqpoint{3.716138in}{3.365664in}}%
\pgfusepath{stroke}%
\end{pgfscope}%
\begin{pgfscope}%
\definecolor{textcolor}{rgb}{0.000000,0.000000,0.000000}%
\pgfsetstrokecolor{textcolor}%
\pgfsetfillcolor{textcolor}%
\pgftext[x=3.805027in,y=3.326775in,left,base]{\color{textcolor}\rmfamily\fontsize{8.000000}{9.600000}\selectfont \(\displaystyle A_0 e^{-\alpha_1 t}\) with \(\displaystyle \alpha_1\) = 0.057}%
\end{pgfscope}%
\begin{pgfscope}%
\pgfsetbuttcap%
\pgfsetroundjoin%
\pgfsetlinewidth{1.003750pt}%
\definecolor{currentstroke}{rgb}{0.000000,0.000000,1.000000}%
\pgfsetstrokecolor{currentstroke}%
\pgfsetdash{{3.700000pt}{1.600000pt}}{0.000000pt}%
\pgfpathmoveto{\pgfqpoint{3.493916in}{3.195279in}}%
\pgfpathlineto{\pgfqpoint{3.716138in}{3.195279in}}%
\pgfusepath{stroke}%
\end{pgfscope}%
\begin{pgfscope}%
\definecolor{textcolor}{rgb}{0.000000,0.000000,0.000000}%
\pgfsetstrokecolor{textcolor}%
\pgfsetfillcolor{textcolor}%
\pgftext[x=3.805027in,y=3.156390in,left,base]{\color{textcolor}\rmfamily\fontsize{8.000000}{9.600000}\selectfont \(\displaystyle A_0 e^{-\alpha_2 t}\) with \(\displaystyle \alpha_2\) = 0.091}%
\end{pgfscope}%
\begin{pgfscope}%
\pgfsetbuttcap%
\pgfsetroundjoin%
\pgfsetlinewidth{1.003750pt}%
\definecolor{currentstroke}{rgb}{0.000000,0.501961,0.000000}%
\pgfsetstrokecolor{currentstroke}%
\pgfsetdash{{3.700000pt}{1.600000pt}}{0.000000pt}%
\pgfpathmoveto{\pgfqpoint{3.493916in}{3.024893in}}%
\pgfpathlineto{\pgfqpoint{3.716138in}{3.024893in}}%
\pgfusepath{stroke}%
\end{pgfscope}%
\begin{pgfscope}%
\definecolor{textcolor}{rgb}{0.000000,0.000000,0.000000}%
\pgfsetstrokecolor{textcolor}%
\pgfsetfillcolor{textcolor}%
\pgftext[x=3.805027in,y=2.986004in,left,base]{\color{textcolor}\rmfamily\fontsize{8.000000}{9.600000}\selectfont \(\displaystyle A_0 e^{-\alpha_3 t}\) with \(\displaystyle \alpha_3\) = 0.141}%
\end{pgfscope}%
\begin{pgfscope}%
\pgfsetbuttcap%
\pgfsetmiterjoin%
\definecolor{currentfill}{rgb}{1.000000,0.000000,0.000000}%
\pgfsetfillcolor{currentfill}%
\pgfsetfillopacity{0.100000}%
\pgfsetlinewidth{1.003750pt}%
\definecolor{currentstroke}{rgb}{1.000000,0.000000,0.000000}%
\pgfsetstrokecolor{currentstroke}%
\pgfsetstrokeopacity{0.100000}%
\pgfsetdash{}{0pt}%
\pgfpathmoveto{\pgfqpoint{3.493916in}{2.831066in}}%
\pgfpathlineto{\pgfqpoint{3.716138in}{2.831066in}}%
\pgfpathlineto{\pgfqpoint{3.716138in}{2.908844in}}%
\pgfpathlineto{\pgfqpoint{3.493916in}{2.908844in}}%
\pgfpathclose%
\pgfusepath{stroke,fill}%
\end{pgfscope}%
\begin{pgfscope}%
\definecolor{textcolor}{rgb}{0.000000,0.000000,0.000000}%
\pgfsetstrokecolor{textcolor}%
\pgfsetfillcolor{textcolor}%
\pgftext[x=3.805027in,y=2.831066in,left,base]{\color{textcolor}\rmfamily\fontsize{8.000000}{9.600000}\selectfont \(\displaystyle \alpha_{1min}\) = 0.06,  \(\displaystyle \alpha_{1max}\) = 0.054}%
\end{pgfscope}%
\begin{pgfscope}%
\pgfsetbuttcap%
\pgfsetmiterjoin%
\definecolor{currentfill}{rgb}{0.000000,0.000000,1.000000}%
\pgfsetfillcolor{currentfill}%
\pgfsetfillopacity{0.100000}%
\pgfsetlinewidth{1.003750pt}%
\definecolor{currentstroke}{rgb}{0.000000,0.000000,1.000000}%
\pgfsetstrokecolor{currentstroke}%
\pgfsetstrokeopacity{0.100000}%
\pgfsetdash{}{0pt}%
\pgfpathmoveto{\pgfqpoint{3.493916in}{2.676128in}}%
\pgfpathlineto{\pgfqpoint{3.716138in}{2.676128in}}%
\pgfpathlineto{\pgfqpoint{3.716138in}{2.753905in}}%
\pgfpathlineto{\pgfqpoint{3.493916in}{2.753905in}}%
\pgfpathclose%
\pgfusepath{stroke,fill}%
\end{pgfscope}%
\begin{pgfscope}%
\definecolor{textcolor}{rgb}{0.000000,0.000000,0.000000}%
\pgfsetstrokecolor{textcolor}%
\pgfsetfillcolor{textcolor}%
\pgftext[x=3.805027in,y=2.676128in,left,base]{\color{textcolor}\rmfamily\fontsize{8.000000}{9.600000}\selectfont \(\displaystyle \alpha_{2min}\) = 0.097,  \(\displaystyle \alpha_{2max}\) = 0.086}%
\end{pgfscope}%
\begin{pgfscope}%
\pgfsetbuttcap%
\pgfsetmiterjoin%
\definecolor{currentfill}{rgb}{0.000000,0.501961,0.000000}%
\pgfsetfillcolor{currentfill}%
\pgfsetfillopacity{0.100000}%
\pgfsetlinewidth{1.003750pt}%
\definecolor{currentstroke}{rgb}{0.000000,0.501961,0.000000}%
\pgfsetstrokecolor{currentstroke}%
\pgfsetstrokeopacity{0.100000}%
\pgfsetdash{}{0pt}%
\pgfpathmoveto{\pgfqpoint{3.493916in}{2.521189in}}%
\pgfpathlineto{\pgfqpoint{3.716138in}{2.521189in}}%
\pgfpathlineto{\pgfqpoint{3.716138in}{2.598967in}}%
\pgfpathlineto{\pgfqpoint{3.493916in}{2.598967in}}%
\pgfpathclose%
\pgfusepath{stroke,fill}%
\end{pgfscope}%
\begin{pgfscope}%
\definecolor{textcolor}{rgb}{0.000000,0.000000,0.000000}%
\pgfsetstrokecolor{textcolor}%
\pgfsetfillcolor{textcolor}%
\pgftext[x=3.805027in,y=2.521189in,left,base]{\color{textcolor}\rmfamily\fontsize{8.000000}{9.600000}\selectfont \(\displaystyle \alpha_{3min}\) = 0.15,  \(\displaystyle \alpha_{3max}\) = 0.134}%
\end{pgfscope}%
\begin{pgfscope}%
\pgfsetbuttcap%
\pgfsetroundjoin%
\pgfsetlinewidth{1.003750pt}%
\definecolor{currentstroke}{rgb}{1.000000,0.000000,0.000000}%
\pgfsetstrokecolor{currentstroke}%
\pgfsetdash{}{0pt}%
\pgfpathmoveto{\pgfqpoint{3.605027in}{2.349584in}}%
\pgfpathlineto{\pgfqpoint{3.605027in}{2.460695in}}%
\pgfusepath{stroke}%
\end{pgfscope}%
\begin{pgfscope}%
\pgfsetbuttcap%
\pgfsetroundjoin%
\definecolor{currentfill}{rgb}{1.000000,0.000000,0.000000}%
\pgfsetfillcolor{currentfill}%
\pgfsetlinewidth{1.003750pt}%
\definecolor{currentstroke}{rgb}{1.000000,0.000000,0.000000}%
\pgfsetstrokecolor{currentstroke}%
\pgfsetdash{}{0pt}%
\pgfsys@defobject{currentmarker}{\pgfqpoint{-0.041667in}{-0.000000in}}{\pgfqpoint{0.041667in}{0.000000in}}{%
\pgfpathmoveto{\pgfqpoint{0.041667in}{-0.000000in}}%
\pgfpathlineto{\pgfqpoint{-0.041667in}{0.000000in}}%
\pgfusepath{stroke,fill}%
}%
\begin{pgfscope}%
\pgfsys@transformshift{3.605027in}{2.349584in}%
\pgfsys@useobject{currentmarker}{}%
\end{pgfscope}%
\end{pgfscope}%
\begin{pgfscope}%
\pgfsetbuttcap%
\pgfsetroundjoin%
\definecolor{currentfill}{rgb}{1.000000,0.000000,0.000000}%
\pgfsetfillcolor{currentfill}%
\pgfsetlinewidth{1.003750pt}%
\definecolor{currentstroke}{rgb}{1.000000,0.000000,0.000000}%
\pgfsetstrokecolor{currentstroke}%
\pgfsetdash{}{0pt}%
\pgfsys@defobject{currentmarker}{\pgfqpoint{-0.041667in}{-0.000000in}}{\pgfqpoint{0.041667in}{0.000000in}}{%
\pgfpathmoveto{\pgfqpoint{0.041667in}{-0.000000in}}%
\pgfpathlineto{\pgfqpoint{-0.041667in}{0.000000in}}%
\pgfusepath{stroke,fill}%
}%
\begin{pgfscope}%
\pgfsys@transformshift{3.605027in}{2.460695in}%
\pgfsys@useobject{currentmarker}{}%
\end{pgfscope}%
\end{pgfscope}%
\begin{pgfscope}%
\pgfsetbuttcap%
\pgfsetroundjoin%
\definecolor{currentfill}{rgb}{1.000000,0.000000,0.000000}%
\pgfsetfillcolor{currentfill}%
\pgfsetlinewidth{1.003750pt}%
\definecolor{currentstroke}{rgb}{1.000000,0.000000,0.000000}%
\pgfsetstrokecolor{currentstroke}%
\pgfsetdash{}{0pt}%
\pgfsys@defobject{currentmarker}{\pgfqpoint{-0.010417in}{-0.010417in}}{\pgfqpoint{0.010417in}{0.010417in}}{%
\pgfpathmoveto{\pgfqpoint{0.000000in}{-0.010417in}}%
\pgfpathcurveto{\pgfqpoint{0.002763in}{-0.010417in}}{\pgfqpoint{0.005412in}{-0.009319in}}{\pgfqpoint{0.007366in}{-0.007366in}}%
\pgfpathcurveto{\pgfqpoint{0.009319in}{-0.005412in}}{\pgfqpoint{0.010417in}{-0.002763in}}{\pgfqpoint{0.010417in}{0.000000in}}%
\pgfpathcurveto{\pgfqpoint{0.010417in}{0.002763in}}{\pgfqpoint{0.009319in}{0.005412in}}{\pgfqpoint{0.007366in}{0.007366in}}%
\pgfpathcurveto{\pgfqpoint{0.005412in}{0.009319in}}{\pgfqpoint{0.002763in}{0.010417in}}{\pgfqpoint{0.000000in}{0.010417in}}%
\pgfpathcurveto{\pgfqpoint{-0.002763in}{0.010417in}}{\pgfqpoint{-0.005412in}{0.009319in}}{\pgfqpoint{-0.007366in}{0.007366in}}%
\pgfpathcurveto{\pgfqpoint{-0.009319in}{0.005412in}}{\pgfqpoint{-0.010417in}{0.002763in}}{\pgfqpoint{-0.010417in}{0.000000in}}%
\pgfpathcurveto{\pgfqpoint{-0.010417in}{-0.002763in}}{\pgfqpoint{-0.009319in}{-0.005412in}}{\pgfqpoint{-0.007366in}{-0.007366in}}%
\pgfpathcurveto{\pgfqpoint{-0.005412in}{-0.009319in}}{\pgfqpoint{-0.002763in}{-0.010417in}}{\pgfqpoint{0.000000in}{-0.010417in}}%
\pgfpathclose%
\pgfusepath{stroke,fill}%
}%
\begin{pgfscope}%
\pgfsys@transformshift{3.605027in}{2.405140in}%
\pgfsys@useobject{currentmarker}{}%
\end{pgfscope}%
\end{pgfscope}%
\begin{pgfscope}%
\definecolor{textcolor}{rgb}{0.000000,0.000000,0.000000}%
\pgfsetstrokecolor{textcolor}%
\pgfsetfillcolor{textcolor}%
\pgftext[x=3.805027in,y=2.366251in,left,base]{\color{textcolor}\rmfamily\fontsize{8.000000}{9.600000}\selectfont Measured amplitudes dampening \(\displaystyle I_1\)}%
\end{pgfscope}%
\begin{pgfscope}%
\pgfsetbuttcap%
\pgfsetroundjoin%
\pgfsetlinewidth{1.003750pt}%
\definecolor{currentstroke}{rgb}{0.000000,0.000000,1.000000}%
\pgfsetstrokecolor{currentstroke}%
\pgfsetdash{}{0pt}%
\pgfpathmoveto{\pgfqpoint{3.605027in}{2.194646in}}%
\pgfpathlineto{\pgfqpoint{3.605027in}{2.305757in}}%
\pgfusepath{stroke}%
\end{pgfscope}%
\begin{pgfscope}%
\pgfsetbuttcap%
\pgfsetroundjoin%
\definecolor{currentfill}{rgb}{0.000000,0.000000,1.000000}%
\pgfsetfillcolor{currentfill}%
\pgfsetlinewidth{1.003750pt}%
\definecolor{currentstroke}{rgb}{0.000000,0.000000,1.000000}%
\pgfsetstrokecolor{currentstroke}%
\pgfsetdash{}{0pt}%
\pgfsys@defobject{currentmarker}{\pgfqpoint{-0.041667in}{-0.000000in}}{\pgfqpoint{0.041667in}{0.000000in}}{%
\pgfpathmoveto{\pgfqpoint{0.041667in}{-0.000000in}}%
\pgfpathlineto{\pgfqpoint{-0.041667in}{0.000000in}}%
\pgfusepath{stroke,fill}%
}%
\begin{pgfscope}%
\pgfsys@transformshift{3.605027in}{2.194646in}%
\pgfsys@useobject{currentmarker}{}%
\end{pgfscope}%
\end{pgfscope}%
\begin{pgfscope}%
\pgfsetbuttcap%
\pgfsetroundjoin%
\definecolor{currentfill}{rgb}{0.000000,0.000000,1.000000}%
\pgfsetfillcolor{currentfill}%
\pgfsetlinewidth{1.003750pt}%
\definecolor{currentstroke}{rgb}{0.000000,0.000000,1.000000}%
\pgfsetstrokecolor{currentstroke}%
\pgfsetdash{}{0pt}%
\pgfsys@defobject{currentmarker}{\pgfqpoint{-0.041667in}{-0.000000in}}{\pgfqpoint{0.041667in}{0.000000in}}{%
\pgfpathmoveto{\pgfqpoint{0.041667in}{-0.000000in}}%
\pgfpathlineto{\pgfqpoint{-0.041667in}{0.000000in}}%
\pgfusepath{stroke,fill}%
}%
\begin{pgfscope}%
\pgfsys@transformshift{3.605027in}{2.305757in}%
\pgfsys@useobject{currentmarker}{}%
\end{pgfscope}%
\end{pgfscope}%
\begin{pgfscope}%
\pgfsetbuttcap%
\pgfsetroundjoin%
\definecolor{currentfill}{rgb}{0.000000,0.000000,1.000000}%
\pgfsetfillcolor{currentfill}%
\pgfsetlinewidth{1.003750pt}%
\definecolor{currentstroke}{rgb}{0.000000,0.000000,1.000000}%
\pgfsetstrokecolor{currentstroke}%
\pgfsetdash{}{0pt}%
\pgfsys@defobject{currentmarker}{\pgfqpoint{-0.010417in}{-0.010417in}}{\pgfqpoint{0.010417in}{0.010417in}}{%
\pgfpathmoveto{\pgfqpoint{0.000000in}{-0.010417in}}%
\pgfpathcurveto{\pgfqpoint{0.002763in}{-0.010417in}}{\pgfqpoint{0.005412in}{-0.009319in}}{\pgfqpoint{0.007366in}{-0.007366in}}%
\pgfpathcurveto{\pgfqpoint{0.009319in}{-0.005412in}}{\pgfqpoint{0.010417in}{-0.002763in}}{\pgfqpoint{0.010417in}{0.000000in}}%
\pgfpathcurveto{\pgfqpoint{0.010417in}{0.002763in}}{\pgfqpoint{0.009319in}{0.005412in}}{\pgfqpoint{0.007366in}{0.007366in}}%
\pgfpathcurveto{\pgfqpoint{0.005412in}{0.009319in}}{\pgfqpoint{0.002763in}{0.010417in}}{\pgfqpoint{0.000000in}{0.010417in}}%
\pgfpathcurveto{\pgfqpoint{-0.002763in}{0.010417in}}{\pgfqpoint{-0.005412in}{0.009319in}}{\pgfqpoint{-0.007366in}{0.007366in}}%
\pgfpathcurveto{\pgfqpoint{-0.009319in}{0.005412in}}{\pgfqpoint{-0.010417in}{0.002763in}}{\pgfqpoint{-0.010417in}{0.000000in}}%
\pgfpathcurveto{\pgfqpoint{-0.010417in}{-0.002763in}}{\pgfqpoint{-0.009319in}{-0.005412in}}{\pgfqpoint{-0.007366in}{-0.007366in}}%
\pgfpathcurveto{\pgfqpoint{-0.005412in}{-0.009319in}}{\pgfqpoint{-0.002763in}{-0.010417in}}{\pgfqpoint{0.000000in}{-0.010417in}}%
\pgfpathclose%
\pgfusepath{stroke,fill}%
}%
\begin{pgfscope}%
\pgfsys@transformshift{3.605027in}{2.250202in}%
\pgfsys@useobject{currentmarker}{}%
\end{pgfscope}%
\end{pgfscope}%
\begin{pgfscope}%
\definecolor{textcolor}{rgb}{0.000000,0.000000,0.000000}%
\pgfsetstrokecolor{textcolor}%
\pgfsetfillcolor{textcolor}%
\pgftext[x=3.805027in,y=2.211313in,left,base]{\color{textcolor}\rmfamily\fontsize{8.000000}{9.600000}\selectfont Measured amplitudes dampening \(\displaystyle I_2\)}%
\end{pgfscope}%
\begin{pgfscope}%
\pgfsetbuttcap%
\pgfsetroundjoin%
\pgfsetlinewidth{1.003750pt}%
\definecolor{currentstroke}{rgb}{0.000000,0.501961,0.000000}%
\pgfsetstrokecolor{currentstroke}%
\pgfsetdash{}{0pt}%
\pgfpathmoveto{\pgfqpoint{3.605027in}{2.039708in}}%
\pgfpathlineto{\pgfqpoint{3.605027in}{2.150819in}}%
\pgfusepath{stroke}%
\end{pgfscope}%
\begin{pgfscope}%
\pgfsetbuttcap%
\pgfsetroundjoin%
\definecolor{currentfill}{rgb}{0.000000,0.501961,0.000000}%
\pgfsetfillcolor{currentfill}%
\pgfsetlinewidth{1.003750pt}%
\definecolor{currentstroke}{rgb}{0.000000,0.501961,0.000000}%
\pgfsetstrokecolor{currentstroke}%
\pgfsetdash{}{0pt}%
\pgfsys@defobject{currentmarker}{\pgfqpoint{-0.041667in}{-0.000000in}}{\pgfqpoint{0.041667in}{0.000000in}}{%
\pgfpathmoveto{\pgfqpoint{0.041667in}{-0.000000in}}%
\pgfpathlineto{\pgfqpoint{-0.041667in}{0.000000in}}%
\pgfusepath{stroke,fill}%
}%
\begin{pgfscope}%
\pgfsys@transformshift{3.605027in}{2.039708in}%
\pgfsys@useobject{currentmarker}{}%
\end{pgfscope}%
\end{pgfscope}%
\begin{pgfscope}%
\pgfsetbuttcap%
\pgfsetroundjoin%
\definecolor{currentfill}{rgb}{0.000000,0.501961,0.000000}%
\pgfsetfillcolor{currentfill}%
\pgfsetlinewidth{1.003750pt}%
\definecolor{currentstroke}{rgb}{0.000000,0.501961,0.000000}%
\pgfsetstrokecolor{currentstroke}%
\pgfsetdash{}{0pt}%
\pgfsys@defobject{currentmarker}{\pgfqpoint{-0.041667in}{-0.000000in}}{\pgfqpoint{0.041667in}{0.000000in}}{%
\pgfpathmoveto{\pgfqpoint{0.041667in}{-0.000000in}}%
\pgfpathlineto{\pgfqpoint{-0.041667in}{0.000000in}}%
\pgfusepath{stroke,fill}%
}%
\begin{pgfscope}%
\pgfsys@transformshift{3.605027in}{2.150819in}%
\pgfsys@useobject{currentmarker}{}%
\end{pgfscope}%
\end{pgfscope}%
\begin{pgfscope}%
\pgfsetbuttcap%
\pgfsetroundjoin%
\definecolor{currentfill}{rgb}{0.000000,0.501961,0.000000}%
\pgfsetfillcolor{currentfill}%
\pgfsetlinewidth{1.003750pt}%
\definecolor{currentstroke}{rgb}{0.000000,0.501961,0.000000}%
\pgfsetstrokecolor{currentstroke}%
\pgfsetdash{}{0pt}%
\pgfsys@defobject{currentmarker}{\pgfqpoint{-0.010417in}{-0.010417in}}{\pgfqpoint{0.010417in}{0.010417in}}{%
\pgfpathmoveto{\pgfqpoint{0.000000in}{-0.010417in}}%
\pgfpathcurveto{\pgfqpoint{0.002763in}{-0.010417in}}{\pgfqpoint{0.005412in}{-0.009319in}}{\pgfqpoint{0.007366in}{-0.007366in}}%
\pgfpathcurveto{\pgfqpoint{0.009319in}{-0.005412in}}{\pgfqpoint{0.010417in}{-0.002763in}}{\pgfqpoint{0.010417in}{0.000000in}}%
\pgfpathcurveto{\pgfqpoint{0.010417in}{0.002763in}}{\pgfqpoint{0.009319in}{0.005412in}}{\pgfqpoint{0.007366in}{0.007366in}}%
\pgfpathcurveto{\pgfqpoint{0.005412in}{0.009319in}}{\pgfqpoint{0.002763in}{0.010417in}}{\pgfqpoint{0.000000in}{0.010417in}}%
\pgfpathcurveto{\pgfqpoint{-0.002763in}{0.010417in}}{\pgfqpoint{-0.005412in}{0.009319in}}{\pgfqpoint{-0.007366in}{0.007366in}}%
\pgfpathcurveto{\pgfqpoint{-0.009319in}{0.005412in}}{\pgfqpoint{-0.010417in}{0.002763in}}{\pgfqpoint{-0.010417in}{0.000000in}}%
\pgfpathcurveto{\pgfqpoint{-0.010417in}{-0.002763in}}{\pgfqpoint{-0.009319in}{-0.005412in}}{\pgfqpoint{-0.007366in}{-0.007366in}}%
\pgfpathcurveto{\pgfqpoint{-0.005412in}{-0.009319in}}{\pgfqpoint{-0.002763in}{-0.010417in}}{\pgfqpoint{0.000000in}{-0.010417in}}%
\pgfpathclose%
\pgfusepath{stroke,fill}%
}%
\begin{pgfscope}%
\pgfsys@transformshift{3.605027in}{2.095263in}%
\pgfsys@useobject{currentmarker}{}%
\end{pgfscope}%
\end{pgfscope}%
\begin{pgfscope}%
\definecolor{textcolor}{rgb}{0.000000,0.000000,0.000000}%
\pgfsetstrokecolor{textcolor}%
\pgfsetfillcolor{textcolor}%
\pgftext[x=3.805027in,y=2.056374in,left,base]{\color{textcolor}\rmfamily\fontsize{8.000000}{9.600000}\selectfont Measured amplitudes dampening \(\displaystyle I_3\)}%
\end{pgfscope}%
\end{pgfpicture}%
\makeatother%
\endgroup%

	\caption{Measured amplitudes with an estimated error of $\Delta A = \pm2 \degree$. Shown next to the exponential decay with $\alpha_{1, 2, 3}$ as decay rates and $A_0=110  \degree$  as starting value of the measurement. To every $\alpha$ value a calculated maximum and minimum error band is shown.}
	\label{fig::dampening}
\end{figure}

The same procedure as for the value $\alpha_1$ was done for $\alpha_{2,3}$.
resulting in the following values for $\alpha_{1, 2, 3}$:
\begin{align*}
&\alpha_1 = 0.057 \pm \SI{0.003}{\second^{-1}}\\
&\alpha_2 = 0.091 \pm \SI{0.006}{\second^{-1}}\\
&\alpha_3 = 0.141 \pm \SI{0.008}{\second^{-1}}
\end{align*}
