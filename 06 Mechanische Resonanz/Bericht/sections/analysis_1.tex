\subsection{Data Analysis}

In the data analysis section you describe the post processing of the
data. How did you obtain the data that you plot in the figures? The raw data does not necessarily need
to be presented in the report. An important part of this paragraph
are the measurement uncertainties. You should provide the
uncertainties of all experimental results, i.e. in the form of error
bars. Further, you should explain the origin of these uncertainties.

There are many way how you can include equations and mathematical terms in your report. The easiest is to write them inside the text like this: $\Gamma =\SI{1.5}{\micro\meter\per\square\second}$. If you need to write a long equation, it is recommended to use e.g. the align environment.
\begin{align}
    \Gamma = \frac{a}{4\kappa}\times ...
\end{align}

All figures or tables that are part of your report have to be
referenced somewhere in the text, ideally in order of their
appearance (``as shown in Fig. \ref{fig1}''). Figures have to
have axis labels with units and a sensible scale. If more than on
data set is plotted you need to provide a legend. This may be a sentence in the caption (``red dots denote data measured with \SI{1}{\milli\volt}, blue crosses were measured with \SI{10}{\milli\volt}'').

After you have presented the data you need to interpret it. To this
end you want to discuss the theoretical model that describes your
data and you will derive model parameters from your measurement data
(i.e. by fitting it to the data). Here, you will again elaborate on
the confidence interval of the derived values (error propagation).
This is an important part of the report and it will be the basis for
the next paragraph.