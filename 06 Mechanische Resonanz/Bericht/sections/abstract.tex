\begin{abstract}

As dampened oscillations are important all over physics, it is valuable to know how to determine the dampening constant of a given system.
In this experiment we measure three different dampening constants of a rotating pendulum with two different methods.
We first look at the dampened natural oscillation, and then add a motor and analyse the resulting forced oscillation.
The results we get are close to each other, but do not match perfectly.
We try to find optimizations to the methods to improve the results for further experiments.
\end{abstract}