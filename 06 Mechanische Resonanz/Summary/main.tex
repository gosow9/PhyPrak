\documentclass[12pt,a4paper]{article}



\usepackage{a4wide}
\usepackage{fancyhdr}
\usepackage{graphicx}
\usepackage{epsfig}
\usepackage{parskip}
\usepackage[utf8]{inputenc}
\usepackage{amsmath}
\usepackage{amssymb}
\usepackage{bm}
\usepackage{tikz}
\usepackage{graphicx}
\usepackage{pgf}
\usepackage{pgfplots}
\usepackage{pgfplotstable}
\usepackage{tabularx}
\usepackage{booktabs}
\usepackage{array}
\usepackage{pdfpages}
\newcolumntype{M}[1]{>{\centering\arraybackslash}m{#1}}
\usepackage[free-standing-units=true]{siunitx} % for consistent handling of SI units
\usepackage[colorlinks=true, pdfstartview=FitV, linkcolor=blue, citecolor=blue, urlcolor=blue]{hyperref} % enable links


\setlength{\parindent}{0pt}

\newcommand{\m}[1]
{\mathrm{#1}}

\title{Summery Experiment 05 Mechanische resonanz}
\author{Cedric Renda, Fritz Kurz}
\date{\today }

\begin{document}

\maketitle
In the first part of the experiment different damping constants have to be found. 
To achieve this the period of the rotating pendulum is measured giving us the eigenfrequency $\omega_0$.
Then the amplitudes of the rotating pendulum are measured which gives us the measured decay time $\tau$ from which we can calculate $\alpha=\frac{1}{\tau}$.


In the second part the corresponding resonance curves should be draw to the damping coefficients above.
This is done with a forced oscillation from outside the pendulum by changing its $\omega$ and comparing the resulting amplitudes.


We can determine all the dampening constants $\alpha$ by using the resonance curve or with the decay time.
 

 





\end{document}