\subsection{Experiment}
We use a grating with $15000$ lines per inch, which is equal to $g = 25.4/15000$ lines per mm.
In this part of the experiment, we will use a helium discharge tube and analyse its spectrum.

In order to do that, we place the tube in the middle of our scale and put the grating at a distance $a$ from the tube.
By looking through the middle of the grating we can see the diffraction maxima as lines projected to the scale.
As the lines should be at the same distance $b$ from the middle on both sides, we can adjust the setup until it is symmetrical.
We can then read $b$ off the scale.
As there are many lines on such a spectrum, we tried to find the brightest eight of them.
Here, we will only look at the first order maxima, so as soon as the pattern repeats itself, we will stop.