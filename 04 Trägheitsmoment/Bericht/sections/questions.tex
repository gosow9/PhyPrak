\subsection{Questions for Physics Students}
\begin{enumerate}
	\item \textbf{How is the moment of inertia defined?}
	
	The moment of inertia $I$ of a rigid body is specified by the torque needed to accelerate it to a certain angular acceleration. This is expressed by the formula
	\begin{align}
		I = \int_{V} (\vec{r}_\perp)^2 \rho (\vec{r}) dV.
		\label{eq::inertia}
	\end{align}
	Here, $\vec{r}_\perp$ is the part of $\vec{r}$ perpendicular to the axis of rotation and $\rho$ is the mass distribution.
	
	\begin{enumerate}
		\item 
		\textbf{Specify its value for a cylinder when the axis of rotation coincides with the cylinder axis itself.}
		
	 	With $R_2 > R_1$ being the radii, $L$ the length and $m$ the mass of the cylinder and assuming a constant mass distribution, formula (\ref{eq::inertia}) results in
		\begin{align*}
			I_{cyl}{}
			&=\int_{0}^{2 \pi} \int_{R_1}^{R_2} \int_{0}^{L} r^3 \frac{m}{\pi L ({R_2}^2 - {R_1}^2)} dl dr d\theta \\
			&= \frac{m}{\pi L ({R_2}^2 - {R_1}^2)} 2 \pi L \left[ \frac{r^4}{4}\right]_{R_1}^{R_2} \\
			&= \frac{m (R_2^4-R_1^4)}{2 ({R_2}^2 - {R_1}^2)} \\
			&= \frac{1}{2}m({R_2}^2 + {R_1}^2).
		\end{align*}

	 	
 		\item
 		\textbf{What happens to the value of the torque when the cylinder is not infinitely thin, but has a finite radius?}
 		
 		
	\end{enumerate}
\end{enumerate}